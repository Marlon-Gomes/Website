%%%%%%%%%%%%%%%%%%%%%%%%%%%%%%%%%%%%%%%%%%%%%%%%%%%%%%%%%%%%%%%%%%%%%%%%
%%%%%%%%%%%%%%%%%%%%%% Simple LaTeX CV Template %%%%%%%%%%%%%%%%%%%%%%%%
%%%%%%%%%%%%%%%%%%%%%%%%%%%%%%%%%%%%%%%%%%%%%%%%%%%%%%%%%%%%%%%%%%%%%%%%

%%%%%%%%%%%%%%%%%%%%%%%%%%%%%%%%%%%%%%%%%%%%%%%%%%%%%%%%%%%%%%%%%%%%%%%%
%% NOTE: If you find that it says                                     %%
%%                                                                    %%
%%                           1 of ??                                  %%
%%                                                                    %%
%% at the bottom of your first page, this means that the AUX file     %%
%% was not available when you ran LaTeX on this source. Simply RERUN  %%
%% LaTeX to get the ``??'' replaced with the number of the last page  %%
%% of the document. The AUX file will be generated on the first run   %%
%% of LaTeX and used on the second run to fill in all of the          %%
%% references.                                                        %%
%%%%%%%%%%%%%%%%%%%%%%%%%%%%%%%%%%%%%%%%%%%%%%%%%%%%%%%%%%%%%%%%%%%%%%%%

%%%%%%%%%%%%%%%%%%%%%%%%%%%% Document Setup %%%%%%%%%%%%%%%%%%%%%%%%%%%%

% Don't like 10pt? Try 11pt or 12pt
\documentclass[10pt]{article}

% The automated optical recognition software used to digitize resume
% information works best with fonts that do not have serifs. This
% command uses a sans serif font throughout. Uncomment both lines (or at
% least the second) to restore a Roman font (i.e., a font with serifs).
%\usepackage{times}
%\renewcommand{\familydefault}{\sfdefault}

% This is a helpful package that puts math inside length specifications
\usepackage{calc}
\usepackage{comment}

% Simpler bibsection for CV sections
% (thanks to natbib for inspiration)
\makeatletter
\newlength{\bibhang}
\setlength{\bibhang}{1em} %1em}
\newlength{\bibsep}
 {\@listi \global\bibsep\itemsep \global\advance\bibsep by\parsep}
\newenvironment{bibsection}%
        {\begin{enumerate}{}{%
%        {\begin{list}{}{%
       \setlength{\leftmargin}{\bibhang}%
       \setlength{\itemindent}{-\leftmargin}%
       \setlength{\itemsep}{\bibsep}%
       \setlength{\parsep}{\z@}%
        \setlength{\partopsep}{0pt}%
        \setlength{\topsep}{0pt}}}
        {\end{enumerate}\vspace{-.6\baselineskip}}
%        {\end{list}\vspace{-.6\baselineskip}}
\makeatother

% Layout: Puts the section titles on left side of page
\reversemarginpar

%
%         PAPER SIZE, PAGE NUMBER, AND DOCUMENT LAYOUT NOTES:
%
% The next \usepackage line changes the layout for CV style section
% headings as marginal notes. It also sets up the paper size as either
% letter or A4. By default, letter was used. If A4 paper is desired,
% comment out the letterpaper lines and uncomment the a4paper lines.
%
% As you can see, the margin widths and section title widths can be
% easily adjusted.
%
% ALSO: Notice that the includefoot option can be commented OUT in order
% to put the PAGE NUMBER *IN* the bottom margin. This will make the
% effective text area larger.
%
% IF YOU WISH TO REMOVE THE ``of LASTPAGE'' next to each page number,
% see the note about the +LP and -LP lines below. Comment out the +LP
% and uncomment the -LP.
%
% IF YOU WISH TO REMOVE PAGE NUMBERS, be sure that the includefoot line
% is uncommented and ALSO uncomment the \pagestyle{empty} a few lines
% below.
%

%% Use these lines for letter-sized paper
\usepackage[paper=letterpaper,
            %includefoot, % Uncomment to put page number above margin
            marginparwidth=1.2in,     % Length of section titles
            marginparsep=.05in,       % Space between titles and text
            margin=1in,               % 1 inch margins
            includemp]{geometry}

%% Use these lines for A4-sized paper
%\usepackage[paper=a4paper,
%            %includefoot, % Uncomment to put page number above margin
%            marginparwidth=30.5mm,    % Length of section titles
%            marginparsep=1.5mm,       % Space between titles and text
%            margin=25mm,              % 25mm margins
%            includemp]{geometry}

%% More layout: Get rid of indenting throughout entire document
\setlength{\parindent}{0in}

\usepackage[shortlabels]{enumitem}

%% Reference the last page in the page number
%
% NOTE: comment the +LP line and uncomment the -LP line to have page
%       numbers without the ``of ##'' last page reference)
%
% NOTE: uncomment the \pagestyle{empty} line to get rid of all page
%       numbers (make sure includefoot is commented out above)
%
\usepackage{fancyhdr,lastpage}
\pagestyle{fancy}
%\pagestyle{empty}      % Uncomment this to get rid of page numbers
\fancyhf{}\renewcommand{\headrulewidth}{0pt}
\fancyfootoffset{\marginparsep+\marginparwidth}
\newlength{\footpageshift}
\setlength{\footpageshift}
          {0.5\textwidth+0.5\marginparsep+0.5\marginparwidth-2in}
\lfoot{\hspace{\footpageshift}%
       \parbox{4in}{\, \hfill %
                    \arabic{page} of \protect\pageref*{LastPage} % +LP
%                    \arabic{page}                               % -LP
                    \hfill \,}}

% Finally, give us PDF bookmarks
\usepackage{color,hyperref}
\definecolor{darkblue}{rgb}{0.0,0.0,0.3}
\hypersetup{colorlinks,breaklinks,
            linkcolor=darkblue,urlcolor=darkblue,
            anchorcolor=darkblue,citecolor=darkblue}

%%%%%%%%%%%%%%%%%%%%%%%% End Document Setup %%%%%%%%%%%%%%%%%%%%%%%%%%%%


%%%%%%%%%%%%%%%%%%%%%%%%%%% Helper Commands %%%%%%%%%%%%%%%%%%%%%%%%%%%%

% The title (name) with a horizontal rule under it
% (optional argument typesets an object right-justified across from name
%  as well)
%
% Usage: \makeheading{name}
%        OR
%        \makeheading[right_object]{name}
%
% Place at top of document. It should be the first thing.
% If ``right_object'' is provided in the square-braced optional
% argument, it will be right justified on the same line as ``name'' at
% the top of the CV. For example:
%
%       \makeheading[\emph{Curriculum vitae}]{Your Name}
%
% will put an emphasized ``Curriculum vitae'' at the top of the document
% as a title. Likewise, a picture could be included:
%
%   \makeheading[\includegraphics[height=1.5in]{my_picutre}]{Your Name}
%
% the picture will be flush right across from the name.
\newcommand{\makeheading}[2][]%
        {\hspace*{-\marginparsep minus \marginparwidth}%
         \begin{minipage}[t]{\textwidth+\marginparwidth+\marginparsep}%
             {\large \bfseries #2 \hfill #1}\\[-0.15\baselineskip]%
                 \rule{\columnwidth}{1pt}%
         \end{minipage}}

% The section headings
%
% Usage: \section{section name}
\renewcommand{\section}[1]{\pagebreak[3]%
    \hyphenpenalty=10000%
    \vspace{1.3\baselineskip}%
    \phantomsection\addcontentsline{toc}{section}{#1}%
    \noindent\llap{\scshape\smash{\parbox[t]{\marginparwidth}{\raggedright #1}}}%
    \vspace{-\baselineskip}\par}

% An itemize-style list with lots of space between items
\newenvironment{outerlist}[1][\enskip\textbullet]%
        {\begin{itemize}[#1,leftmargin=*]}{\end{itemize}%
         \vspace{-.6\baselineskip}}

% An environment IDENTICAL to outerlist that has better pre-list spacing
% when used as the first thing in a \section
\newenvironment{lonelist}[1][\enskip\textbullet]%
        {\begin{list}{#1}{%
        \setlength{\partopsep}{0pt}%
        \setlength{\topsep}{0pt}}}
        {\end{list}\vspace{-.6\baselineskip}}

% An itemize-style list with little space between items
\newenvironment{innerlist}[1][\enskip\textbullet]%
        {\begin{itemize}[#1,leftmargin=*,parsep=0pt,itemsep=0pt,topsep=0pt,partopsep=0pt]}
        {\end{itemize}}

% An environment IDENTICAL to innerlist that has better pre-list spacing
% when used as the first thing in a \section
\newenvironment{loneinnerlist}[1][\enskip\textbullet]%
        {\begin{itemize}[#1,leftmargin=*,parsep=0pt,itemsep=0pt,topsep=0pt,partopsep=0pt]}
        {\end{itemize}\vspace{-.6\baselineskip}}

% To add some paragraph space between lines.
% This also tells LaTeX to preferably break a page on one of these gaps
% if there is a needed pagebreak nearby.
\newcommand{\blankline}{\quad\pagebreak[3]}
\newcommand{\halfblankline}{\quad\vspace{-0.5\baselineskip}\pagebreak[3]}

% Uses hyperref to link DOI
\newcommand\doilink[1]{\href{http://dx.doi.org/#1}{#1}}
\newcommand\doi[1]{doi:\doilink{#1}}

% For \url{SOME_URL}, links SOME_URL to the url SOME_URL
\providecommand*\url[1]{\href{#1}{#1}}
% Same as above, but pretty-prints SOME_URL in teletype fixed-width font
\renewcommand*\url[1]{\href{#1}{\texttt{#1}}}

% For \email{ADDRESS}, links ADDRESS to the url mailto:ADDRESS
\providecommand*\email[1]{\href{mailto:#1}{#1}}
% Same as above, but pretty-prints ADDRESS in teletype fixed-width font
%\renewcommand*\email[1]{\href{mailto:#1}{\texttt{#1}}}

%\providecommand\BibTeX{{\rm B\kern-.05em{\sc i\kern-.025em b}\kern-.08em
%    T\kern-.1667em\lower.7ex\hbox{E}\kern-.125emX}}
%\providecommand\BibTeX{{\rm B\kern-.05em{\sc i\kern-.025em b}\kern-.08em
%    \TeX}}
\providecommand\BibTeX{{B\kern-.05em{\sc i\kern-.025em b}\kern-.08em
    \TeX}}
\providecommand\Matlab{\textsc{Matlab}}

%%%%%%%%%%%%%%%%%%%%%%%% End Helper Commands %%%%%%%%%%%%%%%%%%%%%%%%%%%

%%%%%%%%%%%%%%%%%%%%%%%%% Begin CV Document %%%%%%%%%%%%%%%%%%%%%%%%%%%%

\begin{document}
\makeheading{Marlon de Oliveira Gomes}

\section{Contact Information}

% NOTE: Mind where the & separators and \\ breaks are in the following
%       table.
%
% ALSO: \rcollength is the width of the right column of the table
%       (adjust it to your liking; default is 1.85in).
%
\newlength{\rcollength}\setlength{\rcollength}{2.5in}%
%
\begin{tabular}[t]{@{}p{\textwidth-\rcollength}p{\rcollength}}
Port Jefferson, NY    & \\
E-mail: \ \email{mgomes@math.stonybrook.edu} & \hfill Phone: 631-406-3271 (mobile) \\
Linkedin: \url{www.linkedin.com/in/marlon-deoliveiragomes} \\
Github: \url{https://github.com/Marlon-Gomes}
\end{tabular}

%\section{Biographical Information}
%First name: Marlon\\
%Last name: de Oliveira Gomes\\
%Nationality: Brazilian\\

%\section{Objective}

%Insert text here if you want to
%\begin{innerlist}
%\item More information and auxiliary documents can be found at\\\url{http://www.tedpavlic.com/facjobsearch/}
%\end{innerlist}

\section{Education}

\href{http://www.stonybrook.edu/}{\textbf{Stony Brook University}},
Stony Brook, NY, U.S.A.

\begin{outerlist}
\item[] Ph.D.,
        \href{http://www.math.sunysb.edu/}
                     {Mathematics}, \hfill Aug 2014 - Present
        \begin{innerlist}
        \item \emph{Expected:} December 2020
        \item Thesis Topic: \emph{Anti-self-dual metrics from the geometry of plane conics.}
        \item Advisor:
              \href{http://www.math.stonybrook.edu/~claude/}
                   {Claude LeBrun}
        \item Area of Knowledge: Differential Geometry.
        \end{innerlist}
\end{outerlist}
%
\vspace{1cm}

\href{http://www.ufc.br/}{\textbf{Universidade Federal do Ceara}},
Fortaleza, CE, Brazil.
\begin{outerlist}

\item[] M.S.,
        \href{http://www.mat.ufc.br/}
             {Mathematics},\hfill Aug 2011 -- Aug 2013
        \begin{innerlist}
        \item \emph{Graduated with Honors}
        \item Thesis title: \emph{The Bernstein Problem}.
        \item Advisors:
              \href{http://www.mat.ufc.br/portal2/en/professor-staff/effective-professor/31-luquesio-petrola-de-melo-jorge}
                   {Luqu\'esio Petrola de Melo Jorge, Ph.D} and 
              \href{http://www.mat.ufc.br/portal2/en/professor-staff/visiting-professor/43-luciano-mari}
                   {Luciano Mari, Ph.D}
        \item Area of Knowledge: Differential Geometry.
        \end{innerlist}
\item[] B.S.,
        \href{http://www.mat.ufc.br/}
             {Mathematics } \hfill Mar 2009 -- Jun 2011

\end{outerlist}

\section{Skills}

\begin{outerlist}
\item Computing: Python, Pandas, Data Cleaning, Data Visualization, Machine Learning, MATLAB, LaTeX, HTML.
\item Languages: Portuguese (native), English (fluent), Spanish (intermediate).
\end{outerlist}

\section{Professional Experience}

\textbf{Stony Brook University, Stony Broook, United States}

\begin{outerlist}
\item[] Teaching Assistant, Department of Mathematics. \hfill {Aug 2014 - present}\\
Worked as a Grader, Recitation Instructor, and Instructor (for summer courses). Courses taught include:
\begin{innerlist}
                 \item Logical reasoning: ``Mathematical Thinking", ``Logic, Language and Proof", ``Fundamental Concepts of Mathematics" (graduate level).
                 \item Quantitative reasoning: Pre-Calculus, Single-variable Calculus sequences  (both in their regular and accelerated versions), Multivariable Calculus with Applications, Ordinary Differential Equations with Applications.
                 \item Qualitative reasonining: taught real analysis and measure theory (``Real Analysis", ``Foundations of Analysis"), as well as complex analysis ``Analysis for Teachers II" (graduate level).                  
\end{innerlist}
\end{outerlist}     
\vspace{1cm}
                                   
\textbf{Universidade Federal do Cear\'a, Fortaleza, Brazil} 

\begin{outerlist}
\item[] Substitute Professor, Department of Mathematics. \hfill {Aug 2013 -- Jun 2014}  \\
Worked as main Instructor for courses. Courses taught include
\begin{innerlist}
                 \item Quantitative reasoning: Single-variable Calculus, Multivariable Calculus, Linear Algebra, Differential Equations.
                 \item ``Supervised Training in Mathematics Teaching III", one of the courses in the Mathematics Education program. Supervised students as they developed projects for Math circles and competition Math teams in the schools they were interns.  
\end{innerlist}
\end{outerlist}

\vspace{1cm}
\textbf{Sete de Setembro School, Fortaleza, Brazil} \hfill {Mar 2013 -- Jul 2014}
\begin{innerlist}
\item[] Worked with Middle and High school students as a coach for Mathematics olympiads at State, National and International levels. 
\end{innerlist}

\vspace{1cm}

\textbf{Master School, Fortaleza, Brazil} \hfill{ Apr 2013 -- Dec 2013}
\begin{innerlist}
\item[] Worked with Middle and High school students as a coach for Mathematics and Informatics olympiads at State and National levels. 
\end{innerlist}

\vspace{1cm}

\textbf{Universidade da Integra\c{c}\~{a}o Internacional \\
da Lusofonia Afro-Brasileira,  Reden\c c\~ao, Brazil } \hfill{Mar 2012-- Jul 2012}
\begin{innerlist}
\item[] Worked as Instructor for courses in the Bachelor's degree in Mathematics and Natural Sciences.  
\end{innerlist}

%\section{Research Experience}
%
%\textbf{Research Assistant} \hfill {May 2011 to present}
%\begin{innerlist}
%
%\item[] Division of Biostatistics,\\
%        University of Minnesota\\
%        Supervisor: Bradley P. Carlin, Ph.D
%\end{innerlist}
%\textbf{Research Assistant} \hfill {June 2010 to May 2011}
%\begin{innerlist}
%
%\item[] Division of Epidemiology,\\
%        University of Minnesota\\
%        Supervisors: Traci L. Toomey, Ph.D and Bradley P. Carlin, Ph.D
%\end{innerlist}
%\textbf{Research Assistant} \hfill {Sept 2008 to Aug 2010}
%\begin{innerlist}
%
%\item[] Division of Biostatistics,\\
%        University of Minnesota\\
%        Supervisors: Katherine Huppler-Hullsiek, Ph.D and Jason V. Baker, M.D., M.S.
%\end{innerlist}

%%\section{Refereed Journal Publications}
%\vspace{-.1275in}
%\begin{bibsection}
%    \item Baker, J., Duprez, D., Rapkin, J., Huppler-Hullsiek, K., {\bf Quick, H.}, Grimm, R., Neaton, J.D., and Henry, K.  ``Untreated HIV infection and large and small artery elasticity." \emph{JAIDS}, 52(1):25--31, 2009.
%    \end{bibsection}
%

\section{Publications}
\vspace{-.125in}
\begin{bibsection}
      \item Birbrair, L., {\bf Gomes, M.}, and Pereira, W. ``Resonance sequences and recoverability". \emph{Inter. J. Number Theory}, 11 (2) pp.495-506, 2015. \doi{10.1142/S1793042115500256}     
\end{bibsection}

\section{Papers in Preparation}
\vspace{-.1in}
\begin{bibsection}
    \item \textbf{Gomes, M.} ``Anti-self-dual metrics from the geometry of plane conics."
\end{bibsection}
      
%\section{Submitted Journal Publications}
%\vspace{-.125in}
%\begin{bibsection}
%    \item Birbrair, L., {\bf Gomes, M.}, and Pereira, W. ``Resonance sequences and recoverability" 2013. Submitted to \emph{Inter. J. Number Theory}.
%\end{bibsection}

% Add a little space to nudge next ``Conference Publications'' marginpar
% down to make room for tall ``Submitted Journal Publications''
% marginpar. If there are enough submitted journal publications, this
% space will not be needed (and should be removed).
%\vspace{0.1in}

\section{Fellowships}

\begin{innerlist}
\item CAPES/LASPAU Science Without Borders Ph.D. Fellowship, \hfill Aug 2014 - Jul 2018
\item CAPES Masters Degree Fellowship \hfill Aug 2011 - Jul 2013
\item IMPA Summer Fellowship \hfill  Jan 2010 - Feb 2010 
\item CNPq Undergraduate Research Fellowship \hfill Mar 2009 - Jul 2011
\end{innerlist}

\section{Awards and Honors}

\begin{innerlist}
\item \emph{Chairman's Award for Excellence in Teaching by a }\hfill 2020 \\
\emph{Graduate Student Receiving a Ph.D.}\\
Department of Mathematics,\\
Stony Brook University, Stony Brook, USA.
\item \emph{Teaching Award:``Outstanding educator."} \hfill 2016\\
The Governor's School for Mathematics, Science and Technology,\\ Lynchburg College, Lynchburg, VA, USA. 
\item \emph{Master's Dissertation Approved with Honors}  \hfill 2013\\
Department of Mathematics,\\
Universidade Federal do Cear\'a, Fortaleza, Brazil.
\item \emph{Teaching Award, ``in recognition for the outstanding work in the preparation}  \hfill 2013\\ 
\emph{of Mathematics Olympiad teams''}\\
Coordination of the Cear\'a State Mathematics Olympiad,\\
Universidade Federal do Cear\'a, Fortaleza, Brazil.
\item \emph{Silver Medal - Brazilian National Physics Olympiad} \hfill 2008
\item \emph{Honorable Mention - Cear\'a State Mathematics Olympiad} \hfill 2008
\item \emph{Gold Medal - Cear\'a State Mathematics Olympiad for Public Schools} \hfill 2006
\item \emph{Bronze Medal - Brazilian National Mathematics Olympiad for Public Schools} \hfill 2006
\end{innerlist}


\section{Invited Talks}

\begin{outerlist}
\item[] \textit{Anti-self-dual metrics, Twistors, and Plane Conics} \hfill Nov 2019\\
CUNY Almost Complex Geometry Seminar, New York, NY. 

\item[] \textit{Conics, Twistors, and Anti-self-dual metrics.} \hfill Sep 2019 \\
Union College Mathematics Conference, Schenectady, NY.

\end{outerlist}

%\section{Conferences attended}
%\begin{innerlist}
%\item \textit{Geometry of Manifolds.} \hfill Oct 2017\\
%Simons Center for Geometry and Physics, Stony Brook, USA.
%\item \textit{Thematic Program on K\"{a}hler Geometry.} \hfill Jun 2017\\
%University of Notre Dame, Notre Dame, USA. 
%\item \textit{Conference in Differential Geometry: } \hfill Jul 2016\\
%\textit{in Honour of Claude LeBrun's 60th birthday.}\\
%Universit\'{e} du Qu\'{e}bec \`{a} Montr\'{e}al, Montreal, Canada.
%\item \textit{3rd Biannual Stony Brook Mini-school in Geometry:} \hfill Jan 2015\\
%\textit{An invitation to Gromov-Witten Theory.}\\
%Stony Brook University, Stony Brook, USA.
%\end{innerlist}

\section{Outreach activities}

\textbf{Institute for STEM Education, Stony Brook University.} 
	\begin{outerlist}
	\item[] Instructor for the Stony Brook Mathematics Summer Program. 
		\begin{innerlist}
		\item Mini-course: \textit{Two-person Games and the Minimax Theorem}. \hfill {Jul 2019}
		\item Mini-course: \textit{Planar Graphs, Euler's Formula, and Platonic Solids}. \hfill {Jul 2018}
		\item Mini-course: \textit{Coloring Problems in Graph Theory}. \hfill {Jul 2017}
		\item Mini-course: \textit{An Introduction to Graph Theory}. \hfill {Jul 2016}
		\end{innerlist}
	\item[] Instructor for the Math in Jeans program. 	
		\begin{innerlist}
			\item Mini-course: \textit{An Introduction to Algorithms}. \hfill {Mar - Apr 2017}
		\end{innerlist}
	\end{outerlist}

\section{Professional Service}

 \textbf{Graduate Student Representative.} \hfill {Aug 2017 - Jul 2018}
 \begin{outerlist}
 	\item[] Met with the Graduate Commitee to discuss final exams and written \\
 	qualifying exams. Represented students interests in Graduate Commitee meetings. 
\end{outerlist}
 \vspace{0.5cm}
 
 \textbf{Co-organizer, Student Differential Geometry Seminar}. \hfill {Aug 2018 - Present} 
\begin{outerlist}
	\item[] A seminar on topics related to Complex Differential Geometry, \\
 	Riemannian Geometry and Geometric Analysis.
 		\begin{innerlist}
 		    \item 	\textit{Spectral Theory of the Laplacian}. \hfill {Fall 2019}
 		    \item   \textit{Topics in Scalar Curvature}. \hfill {Spring 2019}
 			\item   \textit{Einstein Metrics and Special Holonomy}. \hfill {Fall 2018}
 		\end{innerlist}
 \end{outerlist}
 \vspace{0.5cm}
 
 \textbf{Co-organizer, RTG in Geometry Seminar}. \hfill {Aug 2017 - May 2018}
 \begin{outerlist}
 	\item[] A student seminar on topics on the interface of\\
 	 Algebraic, Differential and Symplectic Geometry. 
 		\begin{innerlist}
 		    \item \textit{Stability Conditions and Wall-crossing Phenomena}. \hfill {Spring 2018} 
 			\item \textit{Introduction to Mirror Symmetry}. \hfill {Fall 2017}
 	 	\end{innerlist}
 \end{outerlist}
%Recruiting Committee, Division of Biostatistics \hfill {May 2010 -- Present}
%\begin{innerlist}
%    \item Assist with planning of annual Division of Biostatistics Open House and Admitted Student Visit Days
%    \item Meet with prospective and admitted students %; answer questions from a student's perspective
%\end{innerlist}
%
%\halfblankline
%
%Student Member of Search Committee for the \hfill {June 2010 -- Aug 2010}\\
%SPH Coordinator of Recruitment and Student Leadership
%\begin{innerlist}
%    \item Assisted in job search for the SPH Coordinator of Recruitment and Student Leadership
%    \item Reviewed applications, conducted interviews
%\end{innerlist}

\section{References}

Claude LeBrun
\begin{innerlist}
\item[] Professor \hfill {Phone: +1 631 632 8254}\\
Mathematics Department \hfill{E-mail: claude@math.stonybrook.edu}\\
Stony Brook University, Stony Brook, NY, U.S.A.
\end{innerlist}

\halfblankline

H. Blaine Lawson, Jr. 
\begin{innerlist}
\item[] Distinguished Professor \hfill {Phone: +1 631 632 8285}\\
Mathematics Department \hfill{E-mail: blaine@math.stonybrook.edu}\\
Stony Brook University, Stony Brook, NY, U.S.A.
\end{innerlist}

\halfblankline

Xiuxiong Chen
\begin{innerlist}
\item[] Distinguished Professor, \hfill {Phone: +1 631 632 8327}\\
Mathematics Department \hfill{E-mail: xiu@math.stonybrook.edu}\\
Stony Brook University, Stony Brook, NY, U.S.A.
\end{innerlist}
\halfblankline


Lowell Jones
\begin{innerlist}
\item[] Professor, \hfill {Phone: +1 631 632 8248}\\
Mathematics Department \hfill{E-mail: lejones@math.stonybrook.edu}\\
Stony Brook University, Stony Brook, NY, U.S.A
\end{innerlist}


\end{document}

%%%%%%%%%%%%%%%%%%%%%%%%%% End CV Document %%%%%%%%%%%%%%%%%%%%%%%%%%%%%

%----------------------------------------------------------------------%
% The following is copyright and licensing information for
% redistribution of this LaTeX source code; it also includes a liability
% statement. If this source code is not being redistributed to others,
% it may be omitted. It has no effect on the function of the above code.
%----------------------------------------------------------------------%
% Copyright (c) 2007, 2008, 2009, 2010, 2011 by Theodore P. Pavlic
%
% Unless otherwise expressly stated, this work is licensed under the
% Creative Commons Attribution-Noncommercial 3.0 United States License. To
% view a copy of this license, visit
% http://creativecommons.org/licenses/by-nc/3.0/us/ or send a letter to
% Creative Commons, 171 Second Street, Suite 300, San Francisco,
% California, 94105, USA.
%
% THE SOFTWARE IS PROVIDED "AS IS", WITHOUT WARRANTY OF ANY KIND, EXPRESS
% OR IMPLIED, INCLUDING BUT NOT LIMITED TO THE WARRANTIES OF
% MERCHANTABILITY, FITNESS FOR A PARTICULAR PURPOSE AND NONINFRINGEMENT.
% IN NO EVENT SHALL THE AUTHORS OR COPYRIGHT HOLDERS BE LIABLE FOR ANY
% CLAIM, DAMAGES OR OTHER LIABILITY, WHETHER IN AN ACTION OF CONTRACT,
% TORT OR OTHERWISE, ARISING FROM, OUT OF OR IN CONNECTION WITH THE
% SOFTWARE OR THE USE OR OTHER DEALINGS IN THE SOFTWARE.
%----------------------------------------------------------------------%

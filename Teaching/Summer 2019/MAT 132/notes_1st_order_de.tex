\documentclass[11pt]{amsart}
\usepackage{amssymb,amsfonts,amsthm,amsmath}
\usepackage[english]{babel}
\usepackage[all,cmtip]{xy}
\usepackage{hyperref}
\usepackage{url}
%\usepackage{mathrsfs}
%\usepackage[notcite,notref]{showkeys}
\def\limproj{\mathop{\oalign{lim\cr\hidewidth$\longleftarrow$\hidewidth\cr}}}

%--------------
% macro perso
%--------------
\newcommand{\inputc}[1]{ \raisebox{-0.5\height}{\input{#1}} }
\newtheorem{theorem}{Theorem}[section]
\newtheorem{lemma}[theorem]{Lemma}
\newtheorem{corollary}[theorem]{Corollary}
\newtheorem{definition}[theorem]{Definition}
\newtheorem{proposition}[theorem]{Proposition}
\newtheorem{remark}[theorem]{Remark}
\newtheorem{example}{Example}[section]
%\newtheorem{theorem*}{Theorem}

\newcommand{\func}[3]{{#1} : {#2} \longrightarrow {#3}}
\numberwithin{equation}{section}

\newenvironment{myproof}{\noindent{it Proof}
\setlength{\parindent}{0mm}}
{$\hfill \bs$}

\title[Notes on linear, first-order equations.]{Notes on linear, first-order differential equations.}
\author[M. Gomes]{Marlon Gomes}
\date{07/29/2019}
%

\begin{document}
\maketitle

\section{Introduction}
A linear, first-order differential equation is an equation of type 
\begin{equation}
\label{model}
y'(x)+P(x)y(x)=Q(x),
\end{equation}
where the variable is the function $y(x)$. Throughout this course, we have seen how to solve the simplest types of such equations, as the examples below show. 

\begin{example}
Consider the case when $P(x)=0$. Such equations were studied in the first week of the course. Here are a few examples, 
\begin{align*}
y'(x) & = 0, \\
y'(x) & = 2, \\
y'(x) & = x^3, \\
y'(x) & = \cos(x), \\
y'(x) & =  e^x, \\
y'(x) & = \frac{1}{1+x^2}.
\end{align*}
These equations can all be solved by direct integration, according to the fundamental theorem of Calculus. 
\end{example}

\begin{example}
Consider the case when $Q(x)=0$. Such equations fall into the category of separable equations, whose solutions we learned this week. Here are a few examples, 
\begin{align*}
y'(x) + xy(x) & = 0, \\
y'(x)+ e^{-x} y(x) & = 0,\\
y'(x) - \frac{\ln(x)}{x}y(x) & = 0, \\
y'(x) - \frac{y(x)}{1+x^2} & = 0.
\end{align*}
\end{example}


\section{Integrating Factors}
Our goal is to solve a general linear first-order equation, that is, without the assumption that either $P(x)$ or $Q(x)$ is zero. The method we use to solve such equations is called the \textit{Method of Integrating Factors}. 

Let's examine the structure of a general equation, 
\begin{equation*}
y'(x)+P(x)y(x)=Q(x),
\end{equation*}
The left-hand side contains two terms, one involving $y$, the other involving $y'$. This resembles the shape of the product rule, but it is not quite the same. The integrating factor of the equation is a function $\mu(x)$ that, when multiplied by the left-hand side, yields the derivative of the product $\mu(x)y(x)$, that is
\begin{equation*}
(\mu(x)y(x))'=\mu(x)y'(x)+\mu(x)P(x)y(x).
\end{equation*}
See the examples below.

\begin{example}
Consider the differential equation 
\begin{equation}
\label{example1}
y'(x)+y(x)=3.
\end{equation}
We wish to find a function $\mu(x)$ satifying the equation
\begin{equation*}
(\mu(x) y(x))'= \mu(x)y'(x)+\mu(x)y(x).
\end{equation*}
Applying the product rule to the left-hand side, this turns into
\begin{equation*}
\mu(x)y'(x)+\mu'(x)y(x)= \mu(x)y'(x) + \mu(x)y(x),
\end{equation*}
which we can simplify to 
\begin{equation*}
\mu'(x)y(x)=\mu(x)y(x).
\end{equation*}
One way to solve this last equation is to find a function $\mu(x)$ such that 
\begin{equation*}
\mu'(x)=\mu(x).
\end{equation*}
This is the \textit{Integrating Factor Equation} for this problem. A simple solution is well-known, $\mu(x)=e^x$. This is an \textit{integrating factor} for this differential equation. 
\end{example}

\begin{example}
Consider the equation
\begin{equation}
\label{example2}
y'(x)+3y(x)=x.
\end{equation}
We wish to find a function $\mu(x)$ satifying the equation
\begin{equation*}
(\mu(x) y(x))'= \mu(x)y'(x)+3\mu(x)y(x).
\end{equation*}
Applying the product rule to the left-hand side, this turns into
\begin{equation*}
\mu(x)y'(x)+\mu'(x)y(x)= \mu(x)y'(x) + 3\mu(x)y(x),
\end{equation*}
which we can simplify to 
\begin{equation*}
\mu'(x)y(x)=3\mu(x)y(x).
\end{equation*}
One way to solve this last equation is to find a function $\mu(x)$ such that 
\begin{equation*}
\mu'(x)=3\mu(x).
\end{equation*}
This is the \textit{Integrating Factor Equation} for this problem. One solution is $\mu(x)=e^{3x}$. This is an integrating factor for this problem. 
\end{example}

\begin{example}
Consider the equation
\begin{equation}
\label{example3}
y'(x)+2y(x)=\cos(x).
\end{equation}
We wish to find a function $\mu(x)$ satifying the equation
\begin{equation*}
(\mu(x) y(x))'= \mu(x)y'(x)+2\mu(x)y(x).
\end{equation*}
Applying the product rule to the left-hand side, this turns into
\begin{equation*}
\mu(x)y'(x)+\mu'(x)y(x)= \mu(x)y'(x) + 2\mu(x)y(x),
\end{equation*}
which we can simplify to 
\begin{equation*}
\mu'(x)y(x)=2\mu(x)y(x).
\end{equation*}
One way to solve this last equation is to find a function $\mu(x)$ such that 
\begin{equation*}
\mu'(x)=2\mu(x).
\end{equation*}
An integrating factor is given by $\mu(x)=e^{2x}$. 
\end{example}

In all the examples we've seen so far, the function $P(x)$ was a constant. The next two examples are a bit more challeging. 

\begin{example}
Consider the equation
\begin{equation}
\label{example4}
y'(x)+2xy(x)=e^{-x^2}.
\end{equation}
We wish to find a function $\mu(x)$ satifying the equation
\begin{equation*}
(\mu(x) y(x))'= \mu(x)y'(x)+2x\mu(x)y(x).
\end{equation*}
Applying the product rule to the left-hand side, this turns into
\begin{equation*}
\mu(x)y'(x)+\mu'(x)y(x)= \mu(x)y'(x) + 2x\mu(x)y(x),
\end{equation*}
which we can simplify to 
\begin{equation*}
\mu'(x)y(x)=2x\mu(x)y(x).
\end{equation*}
One way to solve this last equation is to find a function $\mu(x)$ such that 
\begin{equation*}
\mu'(x)=2x\mu(x).
\end{equation*}
This is a separable equation, whose solution, the integrating factor for our problem, is $y(x)=e^{x^2}$.
\end{example}

\begin{example}
Consider the equation
\begin{equation}
\label{example5}
y'(x)+\frac{1}{x}y(x)=\ln(x),
\end{equation}
where $x>0$. 

We wish to find a function $\mu(x)$ satifying the equation
\begin{equation*}
(\mu(x) y(x))'= \mu(x)y'(x)+\frac{1}{x}\mu(x)y(x).
\end{equation*}
Applying the product rule to the left-hand side, this turns into
\begin{equation*}
\mu(x)y'(x)+\mu'(x)y(x)= \mu(x)y'(x) + \frac{1}{x}\mu(x)y(x),
\end{equation*}
which we can simplify to 
\begin{equation*}
\mu'(x)y(x)=\frac{1}{x}\mu(x)y(x).
\end{equation*}
One way to solve this last equation is to find a function $\mu(x)$ such that 
\begin{equation*}
\mu'(x)=\frac{1}{x}\mu(x).
\end{equation*}
This is a separable equation, whose solution, the integrating factor for our problem, is $y(x)=e^{\ln(x)}=x$.
\end{example}

By examining the examples above more closely, we find that for the standard equation
\begin{equation*}
y'(x)+P(x)y(x)=Q(x), 
\end{equation*}
the integrating factor is 
\begin{equation}
\label{integrating_factor}
\mu(x)=e^{\int P(x)dx}.
\end{equation}

\section{Solving linear, first-order equations}
In this section we return to the examples from section 2, to find solutions of the equations. 

\begin{example}
In example 2.1, 
\begin{equation*}
y'(x)+y(x)=3,
\end{equation*}
we found the integrating factor $\mu(x)=e^x$. Multiplying the equation by this factor, we obtain
\begin{equation*}
(e^xy(x))' = 3e^x.
\end{equation*}
Integrating both sides, we have 
\begin{equation}
\label{example1_integrated}
e^xy(x) = \int 3e^x \ dx = 3e^x + C.
\end{equation}
Solving this equation for $y(x)$, we have 
\begin{equation*}
y(x) = 3 + Ce^{-x}.
\end{equation*}
\end{example}

\begin{example}
In example 2.2, 
\begin{equation*}
y'(x)+3y(x)=x,
\end{equation*}
we found the integrating factor $\mu(x)=e^{3x}$. Multiplying the equation by this factor, we obtain 
\begin{equation*}
(e^{3x}y(x))'=xe^{3x}.
\end{equation*}
Integrating both sides, we have 
\begin{equation}
\label{example2_integrated}
e^{3x}y(x)= \int xe^{3x}\ dx.
\end{equation}
The integral in the right-hand side can be computed by integration by parts, 
\begin{align*}
\int xe^{3x} \ dx & = \frac{xe^{3x}}{3}-\frac{1}{3}\int e^{3x}\ dx \\
& = \frac{xe^{3x}}{3} -\frac{e^{3x}}{9} + C.
\end{align*}
It follows from equation \eqref{example2_integrated} that
\begin{align*}
e^{3x}y(x) & =  \frac{xe^{3x}}{3} -\frac{e^{3x}}{9} + C \\
y(x) & = \frac{x}{3} -\frac{1}{9} + Ce^{-3x}.
\end{align*}
\end{example}

\begin{example}
In example 2.3, 
\begin{equation*}
y'(x)+2y(x)=\cos(x),
\end{equation*}
we found the integrating factor $\mu(x)=e^{2x}$. Once we multiply the equation by this factor, it becomes
\begin{equation*}
(e^{2x}y(x))'=e^{2x}\cos(x).
\end{equation*}
Integrating both sides, we obtain
\begin{equation}
\label{example3_integrated}
e^{2x}y(x)=\int e^{2x}\cos(x)\ dx.
\end{equation}
The integral on the right-hand side can be computed by Integration by parts, 
\begin{align*}
\int e^{2x}\cos(x)\ dx & = e^{2x}\sin(x)-\int 2e^{2x}\sin(x)\ dx \\
& = e^{2x}\sin(x)-2\left(-e^{2x}\cos(x)+2\int e^{2x} \cos(x) \ dx\right).
\end{align*}
Solving this equation for the desired integral, we have 
\begin{equation*}
\int e^{2x}\cos(x)\ dx = \frac{e^{2x}(\sin(x)+2\cos(x))}{5} + C.
\end{equation*}
Going back to equation \eqref{example3_integrated}, we find
\begin{align*}
e^{2x}y(x) & = \frac{e^{2x}(\sin(x)+2\cos(x))}{5} + C \\
y(x) & = \frac{\sin(x)+2\cos(x)}{5} + Ce^{-2x}.
\end{align*}
\end{example}

\begin{example}
In example 2.4, 
\begin{equation*}
y'(x)+2xy(x)=e^{-x^2}, 
\end{equation*}
we found the integrating factor $\mu(x)=e^{x^2}$. Multiplying the equation by this factor, we obtain
\begin{equation*}
(e^{x^2}y(x))'=1.
\end{equation*}
Integrating both sides, 
\begin{align*}
e^{x^2}y(x) & =x+C \\
y(x) &  =xe^{-x^2} + Ce^{-x^2}.
\end{align*}
\end{example}

\begin{example}
In example 2.5, 
\begin{equation*}
y'(x)+\frac{1}{x}y(x)=\ln(x),
\end{equation*}
$(x>0)$, we obtained the integrating factor $\mu(x)=x$. Multiplying the equation by this factor yields, 
\begin{equation*}
(xy(x))' = x\ln(x).
\end{equation*}
Integrating both sides, we have 
\begin{equation}
\label{example5_integrated}
xy(x) = \int x\ln(x) \ dx.
\end{equation}
The integral on the right-hand side can be computed by integration by parts, 
\begin{align*}
\int x\ln(x) \ dx & = \frac{x^2\ln(x)}{2}-\int \frac{x^2}{2}\frac{1}{x} \ dx \\
& = \frac{x^2\ln(x)}{2} -\int \frac{x}{2} \ dx \\
& =  \frac{x^2\ln(x)}{2} -\frac{x^2}{4} + C.
\end{align*}
It follows that 
\begin{align*}
xy(x) & = \frac{x^2\ln(x)}{2} -\frac{x^2}{4} + C \\
y(x) & = \frac{x\ln(x)}{2} -\frac{x}{4} + \frac{C}{x},
\end{align*}
for $x>0$. 
\end{example}

In general, the solution to the standard equation 
\begin{equation*}
y'(x)+P(x)y(x)= Q(x)
\end{equation*}
can be described in terms of the integrating factor, 
\begin{equation}
\label{general_solution}
y(x)=\frac{\int \mu(x) Q(x) \ dx}{\mu(x)}.
\end{equation}

\end{document}

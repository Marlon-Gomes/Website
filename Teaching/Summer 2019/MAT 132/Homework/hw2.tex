\documentclass[12pt,oneside]{exam}

% This package simply sets the margins to be 1 inch.
\usepackage[margin=1in]{geometry}

% These packages include nice commands from AMS-LaTeX
\usepackage{amssymb,amsmath,amsthm,amsfonts,latexsym,verbatim,xspace,setspace}

% Make the space between lines slightly more
% generous than normal single spacing, but compensate
% so that the spacing between rows of matrices still
% looks normal.  Note that 1.1=1/.9090909...
\renewcommand{\baselinestretch}{1.1}
\renewcommand{\arraystretch}{.91}

% Define an environment for exercises.
\newenvironment{exercise}[1]{\vspace{.1in}\noindent\textbf{Exercise #1 \hspace{.05em}}}{}

% define shortcut commands for commonly used symbols
\newcommand{\R}{\mathbb{R}}
\newcommand{\C}{\mathbb{C}}
\newcommand{\Z}{\mathbb{Z}}
\newcommand{\Q}{\mathbb{Q}}
\newcommand{\N}{\mathbb{N}}
\newcommand{\calP}{\mathcal{P}}

\DeclareMathOperator{\vsspan}{span}

\title{Math 132 - Summer II 2019: Homework 1}

%%%%%%%%%%%%%%%%%%%%%%%%%%%%%%%%%%%%%%%%%%

\begin{document}

\begin{flushright}
\sc MAT 132 (lecture 1) - Summer II\\
July 15, 2019
\end{flushright}
\bigskip

\begin{center}
\textsf{Homework 2} 
\end{center}

%%%%%%%%%%%%%%%%%%%%%%%%%%%%%%%%%%%%%%%%

\begin{exercise}{1}
Section 5.9: 10, 18, 24, 38.
\end{exercise}

\begin{exercise}{2}
Section 5.10: 2, 6, 8, 20, 22, 34, 44, 46, 64. 
\end{exercise}

\begin{exercise}{3} 
Section 6.1: 6, 10, 14, 24. 
\end{exercise}

\begin{exercise}{4}
Section 6.2: 6, 12, 18, 40. 
\end{exercise}

\begin{exercise}{5}
Section 6.3: 4, 6, 14, 32. 
\end{exercise}

\begin{exercise}{6}
Section 6.4: 7, 8, 9, 20. 
\end{exercise}

\begin{exercise}{7}
Section 6.6: 46, 48. 
\end{exercise}

\end{document}


\documentclass[11pt]{amsart}
\usepackage{amssymb,amsfonts,amsthm,amsmath}
\usepackage[english]{babel}
\usepackage[all,cmtip]{xy}
\usepackage{hyperref}
\usepackage{url}
%\usepackage{mathrsfs}
%\usepackage[notcite,notref]{showkeys}
\def\limproj{\mathop{\oalign{lim\cr\hidewidth$\longleftarrow$\hidewidth\cr}}}

%--------------
% macro perso
%--------------
\newcommand{\inputc}[1]{ \raisebox{-0.5\height}{\input{#1}} }
\newtheorem{theorem}{Theorem}[section]
\newtheorem{lemma}[theorem]{Lemma}
\newtheorem{corollary}[theorem]{Corollary}
\newtheorem{definition}[theorem]{Definition}
\newtheorem{proposition}[theorem]{Proposition}
\newtheorem{remark}[theorem]{Remark}
\newtheorem{example}{Example}[section]
%\newtheorem{theorem*}{Theorem}

\newcommand{\func}[3]{{#1} : {#2} \longrightarrow {#3}}
\numberwithin{equation}{section}

\newenvironment{myproof}{\noindent{it Proof}
\setlength{\parindent}{0mm}}
{$\hfill \bs$}

\title[MAT 132 (lecture 1) Syllabus - Summer II 2019]{MAT 132 (lecture 1) Syllabus - Summer II 2019}

\author[M. Gomes]{Marlon Gomes}
%

\begin{document}
\maketitle

\section{Course Description}
\subsection{Course Goal}
A continuation of MAT 131, covering symbolic and numeric methods of integration; area under a curve; volume; applications such as work and probability; improper integrals and l'Hospital's rule; complex numbers; sequences; series; Taylor series; differential equations; and modelling. May not be taken for credit in addition to MAT 127, MAT 142, MAT 171, or AMS 161. This course has been designated as a High Demand/Controlled Access (HD/CA) course. Students registering for HD/CA courses for the first time will have priority to do so.
\subsection{Prerequisite}
C or higher in AMS 151, MAT 131, MAT 141, or level 7 on the Mathematics Placement Exam.

\subsection{Topics Covered}
Our starting point will be a review of Integration, discussing the notion of Riemann sums, anti-derivatives, and the Fundamental Theorem of Calculus. We will then move on to advanced methods of symbolic integration: integration by substitution; integration by parts; the method of partial fractions. We shall then discuss numeric integration, involving various approximation techniques. The last part of this section
will deal with Improper integrals.

In the second section of the course, we will study applications of integrals to Geometry. Topics will involve computation of areas between plane curves, volumes of solids of revolution, arc length of curves, and centroids. 

In the third section of the course, an Introduction to Differential Equations, we will begin with a qualitative study of first-order equations discussing integral curves, equilibrium solutions, and asymptotic behavior. Then we will move on to quantitative theory, focusing on first-order equations, and linear, second-order equations. In this section, we will introduce complex numbers and complex exponentials as a convenient tool for solving second-order equations with constant coefficients.

In the fourth topic, Sequences and Series, our motivating question is whether an infinite sum makes sense. We will begin with and elementary discussion of Sequences and their convergence, and then move on to a special kind of sequence: a series. We will discuss elementary examples of nfinite sums, whose values can be computed either directly or by integration, without discussing their convergence theory in this first moment. Then, we will move on to Convergence theory, and discuss many Convergence tests. Finally, we will discuss power series, including Taylor and Maclaurin Series.

\subsection{Textbook}
The required textbook is \textit{Single Variable Calculus}, by James Stewart, 4th Stony Brook edition. We will also use supplemental resources on differential equations, which will be available on the course webpage.

\subsection{Important Times and Dates}
\begin{itemize}
\item Lectures: Mondays, Wednesdays and Thursdays, 6:00 pm - 9:05 pm, in the Melville Library, room E4320.
\item Exam schedule:
\begin{itemize}
\item Quiz 1: 07/15, between 7:00pm and 7:50pm.
\item Quiz 2: 07/24, between 7:00pm and 7:50pm.
\item Quiz 3: 07/31, between 7:00pm and 7:50pm.
\item Quiz 4: 08/08, between 8:15pm and 9:05pm.
\item Final exam, part 1: 08/12, between 7:55pm and 9:05pm.
\item Final exam, part 1, retake: 8/14, between 7:55pm and 9:05pm.
\item Final exam, part 2: 8/15, between 6:00pm and 9:05pm.
\end{itemize}
\end{itemize}

A detailed course schedule will be available on the course webpage. 

\subsection{Assignments}
Your assignments are an important part of the course. They will serve as preparation for the quizzes and final exam. There will be four problem sets, according to the schedule above. The assignments will be available on the course webpage (link below).
	
Problems will range from simple manipulations of the concepts developed in class on that week to more involved applications of these concepts.
    
Homework will not be graded. All problems will be solved in class during the week, before the corresponding quiz. You are encouraged to try the problems on your own, before the problems are discussed in class.

\subsection{Exams}
This class will have four quizzes and a final exam on the dates and times listed above.

Your final exam will be divided into two parts. The first is a minimum competency test. It determines whether you pass or fail the course. Students who fail this test on 8/12 may try it again on 8/14.

\textbf{Missed exams:}There will not be retakes for the quizzes. Students who miss more than one quiz will not have their grades prorated. Students who miss part 1 of the final exam may take it on 8/14. Such students will not be entitled to a retake in case of failure, except if extraordinary circumstances prevented the student from attending the exam on 8/12. Such cases will be judged at the discretion of the
Instructor.
\subsection{Grades}
Part one of the final exam determines whether a student passes or fails the course (see the note below for exceptions). To obtain a passing grade (C), a student must score at least 15/20 on this test. Students who obtain this score are guaranteed a C, regardless of their numerical grade (see below).

A student's performance on the quizzes and part two of the final exam will determine the student's numerical grade, which will be computed as follows:
\begin{enumerate}
\item Three best quizzes: $20\% $ each.
\item Final exam, part 2: $40\% $.
\end{enumerate}

The numerical grade will serve to determine the student's letter grade, according to the cut-offs below: 

\begin{center}
\begin{tabular}{|c|c|}
\hline 
Grade threshold  & Letter grade \\ 
\hline 
94.3 & A \\ 
\hline 
88.6 & A- \\ 
\hline 
82.9 & B+ \\ 
\hline 
77.2 & B \\ 
\hline 
71.5 & B- \\ 
\hline 
65.8 & C+ \\ 
\hline 
60 & C \\ 
\hline 
54.3 & C- \\ 
\hline 
48.6 & D+ \\ 
\hline 
40 & D \\ 
\hline 
0 &  F  \\ 
\hline 
\end{tabular} 
\end{center}

\textbf{Note:} Students may decide to take part two of the final, regardless of whether they passed or failed part one. In the exceptional case a student attempts and fails part one, but the student's numerical grade amounts to at least 60/100, the student will be granted a passing grade.

\subsection{Course management}
Course content will be theavailable on the course webpage,
\begin{center}
\url{http://www.math.stonybrook.edu/~mgomes/mat132sum19.html}.
\end{center}
Blackboard (accessible with you netID), will be used for grade tracking and secure communication. 
\section{Contact}

My office hours are on Thursdays, 12:00 pm to 2:00 pm, in the Math Learning Center (Math S-235), and Thursdays, 4:30pm to 5:30pm, in my office (Math 3-101). E-mail is the best form of communication besides lectures and office hours. My address is
\begin{center}
\href{mailto: mgomes@math.stonybrook.edu}{ mgomes@math.stonybrook.edu}.
\end{center}

\section{DSS Notice}
If you have a physical, psychological, medical, or learning disability that may impact your course work, please contact Disability Support Services at (631) 632-6748 or at
\begin{center}
\url{http://studentaffairs.stonybrook.edu/dss/}.
\end{center}
They will determine with you what accommodations are necessary and appropriate. All information and documentation is confidential. Students who require assistance during emergency evacuation are encouraged to discuss their needs with their professors and Disability Support Services. For procedures and information go to the following website: http://www.sunysb.edu/

\section{Academic Integrity}
Each student must pursue his or her academic goals honestly and be personally accountable for all submitted work. Representing another person's work as your own is always wrong. Faculty are required to report any suspected instances of academic dishonesty to the Academic Judiciary. Faculty in the Health Sciences Center (School of Health Technology and Management, Nursing, Social Welfare, Dental Medicine) and School of Medicine are required to follow their school-specific procedures. For more comprehensive information on academic integrity, including categories of academic dishonesty, please refer to the academic judiciary website at
\begin{center}
\url{http://www.stonybrook.edu/uaa/academicjudiciary/}
\end{center}

\section{Critical Incident Management Statement}
Stony Brook University expects students to respect the rights, privileges, and property of other people. Faculty are required to report to the Office of Judicial Affairs any disruptive behavior that interrupts their ability to teach, compromises the safety of the learning environment, or inhibits students' ability to learn. Faculty in the HSC Schools and the School of Medicine are required to follow their school-specific procedures.
\end{document}

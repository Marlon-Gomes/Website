%% AMS-LaTeX Created with the Wolfram Language : www.wolfram.com

\documentclass{article}
\usepackage{amsmath, amssymb, graphics, setspace}

\newcommand{\mathsym}[1]{{}}
\newcommand{\unicode}[1]{{}}

\newcounter{mathematicapage}
\begin{document}

\title{MAT 132 - Calculus II}
\author{Summer II 2019\\
Mathematics Department\\
Stony Brook University}
\date{Instructor: Marlon de Oliveira Gomes}
\maketitle

\section*{Integration}

\subsection*{Motivation from geometry: the Area Problem.}

Area Problem: to find the area of a planar region bounded by the graph of a positive function.

Simple examples:

A constant function

\begin{doublespace}
\noindent\(\pmb{\text{Plot}[1,\{x,0,5\}, \text{Filling}\to \text{Bottom}, \text{PlotLegends}\to \text{Placed}[\{\text{{``}f(x)=1{''}}\},\text{Above}]]}\)
\end{doublespace}

\begin{doublespace}
\noindent\(\begin{array}{c}
  \\
  \\
\end{array}\)
\end{doublespace}

A first-degree polynomial

\begin{doublespace}
\noindent\(\pmb{\text{Plot}[2x+3, \{x,0,5\}, \text{Filling}\to \text{Bottom}, \text{PlotLegends}\to \text{Placed}[\{\text{{``}f(x)=2x+3{''}}\}, \text{Above}]]}\)
\end{doublespace}

\begin{doublespace}
\noindent\(\begin{array}{c}
  \\
  \\
\end{array}\)
\end{doublespace}

$\blacktriangleleft \thickspace \thickspace \thickspace $$\thickspace \thickspace |\thickspace \thickspace $$\thickspace \thickspace \thickspace
\blacktriangleright $



\section*{Integration}

\subsection*{Motivation from geometry: the Area Problem.}

How to estimate area under a the graph of a general positive function?

Idea: use rectangles as an elementary approximation. 

Archimedes{'} example: the area under a parabola. 

\begin{doublespace}
\noindent\(\pmb{\text{Manipulate}[}\\
\pmb{\text{Show}[}\\
\pmb{\text{DiscretePlot}[t{}^{\wedge}2,}\\
\pmb{\{t,0,1,1/2{}^{\wedge}n\},}\\
\pmb{\text{ExtentSize}\to \text{Right},}\\
\pmb{\text{Frame}\to \text{True},}\\
\pmb{\text{PlotRange}\to \{\{0,1\},\{0,1\}\},}\\
\pmb{\text{PlotRangeClipping}\to  \text{True}],}\\
\pmb{\text{Plot}[t{}^{\wedge}2,}\\
\pmb{\{t,0,1\},}\\
\pmb{\text{Frame}\to \text{True},}\\
\pmb{\text{PlotRange}\to \{\{0,1\},\{0,1\}\},}\\
\pmb{\text{PlotRangeClipping}\to  \text{True},}\\
\pmb{\text{PlotLegends}\to \text{Placed}[\{\text{{``}f(x)=x${}^{\wedge}$2{''}}\},\text{Above}]}\\
\pmb{]}\\
\pmb{],}\\
\pmb{\{n,1,8,1\},}\\
\pmb{\text{FrameLabel}\to \text{Style}[\text{{``}Left-endpoint approximations{''}},13]]}\)
\end{doublespace}

\begin{doublespace}
\noindent\(\)
\end{doublespace}

$\blacktriangleleft \thickspace \thickspace \thickspace $$\thickspace \thickspace |\thickspace \thickspace $$\thickspace \thickspace \thickspace
\blacktriangleright $



\section*{Integration}

\subsection*{Motivation from Geometry: the Area Problem. }

Archimedes{'} example (continued): the area under a parabola. 

\begin{doublespace}
\noindent\(\pmb{\text{Manipulate}[}\\
\pmb{\text{Show}[}\\
\pmb{\text{DiscretePlot}[t{}^{\wedge}2,}\\
\pmb{\{t,0,1,1/2{}^{\wedge}n\},}\\
\pmb{\text{ExtentSize}\to \text{Left},}\\
\pmb{\text{Frame}\to \text{True},}\\
\pmb{\text{PlotRange}\to \{\{0,1\},\{0,1\}\},}\\
\pmb{\text{PlotRangeClipping}\to  \text{True}],}\\
\pmb{\text{Plot}[t{}^{\wedge}2,}\\
\pmb{\{t,0,1\},}\\
\pmb{\text{Frame}\to \text{True},}\\
\pmb{\text{PlotRange}\to \{\{0,1\},\{0,1\}\},}\\
\pmb{\text{PlotRangeClipping}\to  \text{True},}\\
\pmb{\text{PlotLegends}\to \text{Placed}[\{\text{{``}f(x)=x${}^{\wedge}$2{''}}\},\text{Above}]}\\
\pmb{]}\\
\pmb{],}\\
\pmb{\{n,1,8,1\},}\\
\pmb{\text{FrameLabel}\to \text{Style}[\text{{``}Right-endpoint approximations{''}}, 13]]}\)
\end{doublespace}

\begin{doublespace}
\noindent\(\)
\end{doublespace}

$\blacktriangleleft \thickspace \thickspace \thickspace $$\thickspace \thickspace |\thickspace \thickspace $$\thickspace \thickspace \thickspace
\blacktriangleright $



\section*{Integration}

\subsection*{Motivation from Geometry: the Area Problem}

The idea of approximating the area under graphs was formalized by Riemann. To approximate the area underneath the graph of a positive function {``}f{''}
within the interval \([a,b]\), we: 

subdivide the interval into {``}n{''} equally sized subintervals. 

select a sample point in each subinterval, whose image is the height of the approximating rectangle.

add the areas of the approximating rectangles.

These sums take the following form:

\begin{doublespace}
\noindent\(\pmb{\text{TraditionalForm}\left[\text{HoldForm}\left[\text{Sum}\left[\text{$\texttt{"}$f(}x_i^*\text{)$\Delta $x$\texttt{"}$},\{i,1,n\}\right]\right]\right]}\)
\end{doublespace}

\begin{doublespace}
\noindent\(\sum _{i=1}^n \text{f(}x_i^*\text{)$\Delta $x}\)
\end{doublespace}

Notation:

{``}f{''} is the function whose graph bounds the region in question. 

{``}\(x_i^*\){''} are the \textit{ sample points}, one for each rectangle. 

{``}$\Delta $x{''} is the width of each rectangle.

Typical sampling points: left-endpoints, midpoints, right-endpoints.



$\blacktriangleleft \thickspace \thickspace \thickspace $$\thickspace \thickspace |\thickspace \thickspace $$\thickspace \thickspace \thickspace
\blacktriangleright $



\section*{Integration}

\subsection*{Motivation from Geometry: the Area Problem}

\subsubsection*{A comparison between samplings}

\begin{doublespace}
\noindent\(\pmb{\text{Manipulate}[}\\
\pmb{\text{Show}[}\\
\pmb{\text{DiscretePlot}[t{}^{\wedge}2,}\\
\pmb{\{t,0,1,1/n\},}\\
\pmb{\text{ExtentSize}\to \text{Right},}\\
\pmb{\text{Frame}\to \text{True},}\\
\pmb{\text{PlotRange}\to \{\{0,1\},\{0,1\}\},}\\
\pmb{\text{PlotRangeClipping}\to  \text{True}],}\\
\pmb{\text{DiscretePlot}[t{}^{\wedge}2,}\\
\pmb{\{t,0,1,1/n\},}\\
\pmb{\text{ExtentSize}\to \text{Left},}\\
\pmb{\text{Frame}\to \text{True},}\\
\pmb{\text{PlotRange}\to \{\{0,1\},\{0,1\}\},}\\
\pmb{\text{PlotRangeClipping}\to  \text{True}],}\\
\pmb{\text{Plot}[t{}^{\wedge}2,}\\
\pmb{\{t,0,1\},}\\
\pmb{\text{Frame}\to \text{True},}\\
\pmb{\text{PlotRange}\to \{\{0,1\},\{0,1\}\},}\\
\pmb{\text{PlotRangeClipping}\to  \text{True},}\\
\pmb{\text{PlotLegends}\to \text{Placed}[\{\text{{``}f(x)=x${}^{\wedge}$2{''}}\},\text{Above}]}\\
\pmb{]}\\
\pmb{],}\\
\pmb{\{n,1,20,1\}}\\
\pmb{]}\)
\end{doublespace}

\begin{doublespace}
\noindent\(\)
\end{doublespace}

This example illustrates the fact that the approximations by left-endpoints (more opaque) and right-endpoints (less opaque) seem to converge to each
other, as the number of rectangles grows. 

$\blacktriangleleft \thickspace \thickspace \thickspace $$\thickspace \thickspace |\thickspace \thickspace $$\thickspace \thickspace \thickspace
\blacktriangleright $



\section*{Integration}

\subsection*{Motivation from Geometry: the Area Problem}

One might wonder whether our intuitive idea that all of these approximations will \textit{ eventually} (as the number of rectangles grows) lead to
a number. After all, the number of summands grows infinitely large, and there is no telling where their sum will go. To test it, we compute the limits
of the approximations, as the number of rectangles goes to infinity: { } 

\begin{doublespace}
\noindent\(\pmb{\text{TraditionalForm}\left[\text{HoldForm}\left[\text{Limit}\left[\text{Sum}\left[\text{$\texttt{"}$f(}x_i^*\text{)$\Delta $x$\texttt{"}$},\{i,1,n\}\right],n\to
 \infty \right] \right]\right]}\)
\end{doublespace}

\begin{doublespace}
\noindent\(\underset{n\to \infty }{\text{lim}}\left(\sum _{i=1}^n \text{f(}x_i^*\text{)$\Delta $x}\right)\)
\end{doublespace}

Such limits are called \textit{ Riemann Sums. }They can also be defined for functions that change sign, although in this case they no longer represent
\textit{ areas, }but rather \textit{ signed areas, }in which case the area of regions below the axis is counted with a negative sign.

\subsubsection*{Definition}

A function is called integrable on an interval when \textit{ all of its Riemann sums }(that is, for all possibilities of sampling points) exist and
coincide. 

An immediate question is how to test integrability - given that testing each sampling is impractical. We will not answer this question in this course.
Instead, we shall contempt ourselves with the fact that \textit{ all continuous functions are integrable.}\textit{  }

$\blacktriangleleft \thickspace \thickspace \thickspace $$\thickspace \thickspace |\thickspace \thickspace $$\thickspace \thickspace \thickspace
\blacktriangleright $



\section*{Integration}

\subsection*{The Definite Integral}

\begin{doublespace}
\noindent\(\pmb{\text{Plot}[\text{Sin}[x+\text{Sin}[2x]],}\\
\pmb{ \{x,0,\text{Pi}\},}\\
\pmb{ \text{Filling}\to \text{Bottom},}\\
\pmb{\text{PlotLegends}\to \text{Placed}[\text{{``}y=f(x){''}},\text{Above}] ]}\)
\end{doublespace}

\begin{doublespace}
\noindent\(\begin{array}{c}
  \\
  \\
\end{array}\)
\end{doublespace}

If a function {``}f{''} is integrable in an interval \([a,b]\), we call the limit of its Riemann sums therein its \textit{ integral} on the interval
\([a,b]\), represented by 

\begin{doublespace}
\noindent\(\pmb{\text{TraditionalForm}[\text{Integrate}[f[x],\{x,a,b\}]]}\)
\end{doublespace}

\begin{doublespace}
\noindent\(\int_a^b f(x) \, dx\)
\end{doublespace}

The above integral is read {``}integral of {`}f{'}, from {`}a{'} to {`}b{'}, relative to {`}x{'}{''}.

$\blacktriangleleft \thickspace \thickspace \thickspace $$\thickspace \thickspace |\thickspace \thickspace $$\thickspace \thickspace \thickspace
\blacktriangleright $



\section*{Integration}

\subsection*{The Definite Integral}

\subsubsection*{Back to Archimedes: the area under a parabola, by Riemann sums.}

In Archimedes example, \(f(x)=x^2\), \(a=0\), \(b=1\). Subdividing this interval into \(n\) rectangles yields a width size 

\(\begin{array}{c}
 
\begin{array}{c}
 \text{$\Delta $x}=\frac{1}{n} \\
\end{array}
. \\
\end{array}\)

Since f is continuous, we can approximate it by any sampling (in the limit, the results are all the same). We use right-endpoints

\(\begin{array}{c}
 
\begin{array}{c}
 x_i=\frac{i}{n} \\
\end{array}
. \\
\end{array}\)

The Riemann sum is 

\begin{doublespace}
\noindent\(\pmb{\text{TraditionalForm}[}\\
\pmb{\text{HoldForm}[\text{Integrate}[x{}^{\wedge}2,\{x,0,1\}]==\text{Limit}[\text{Sum}[i{}^{\wedge}2/n{}^{\wedge}3,\{i,1,n\}],n\to  \infty ]== }\\
\pmb{\text{Limit}[\text{Sum}[(n*(n+1)*(2n+1))/(6n{}^{\wedge}3),\{i,1,n\}],n\to  \infty ]==1/3 ]]}\)
\end{doublespace}

\begin{doublespace}
\noindent\(\int_0^1 x^2 \, dx=\underset{n\to \infty }{\text{lim}}\left(\sum _{i=1}^n \frac{i^2}{n^3}\right)=\underset{n\to \infty }{\text{lim}}\left(\sum
_{i=1}^n \frac{n (n+1) (2 n+1)}{6 n^3}\right)=\frac{1}{3}\)
\end{doublespace}

$\blacktriangleleft \thickspace \thickspace \thickspace $$\thickspace \thickspace |\thickspace \thickspace $$\thickspace \thickspace \thickspace
\blacktriangleright $



\section*{Integration}

\subsection*{The Definite Integral}

Computing definite integrals from Riemann sums is hard. We{'}ll study other means of computing integrals in what follows. 

\subsubsection*{Properties of integration}

Linearity: {``}f{''} and {``}g{''} integrable functions, {``}t{''} a constant

\begin{doublespace}
\noindent\(\pmb{\text{TraditionalForm}[\text{Integrate}[(f[x]+\text{tg}[x]),\{x,a,b\}]== }\\
\pmb{(\text{Integrate}[f[x],\{x,a,b\}]) +t(\text{Integrate}[g[x],\{x,a,b\}])]}\)
\end{doublespace}

\begin{doublespace}
\noindent\(\int_a^b (f(x)+\text{tg}(x)) \, dx=\int_a^b f(x) \, dx+t \int_a^b g(x) \, dx\)
\end{doublespace}

Domain additivity: if \(a<c<b\), and the function {``}f{''} is integrable on \([a,b]\), then 

\begin{doublespace}
\noindent\(\pmb{\text{TraditionalForm}[\text{Integrate}[f[x],\{x,a,b\}]== }\\
\pmb{(\text{Integrate}[f[x],\{x,a,c\}]) +(\text{Integrate}[f[x],\{x,c,b\}])]}\)
\end{doublespace}

\begin{doublespace}
\noindent\(\int_a^b f(x) \, dx=\int_a^c f(x) \, dx+\int_c^b f(x) \, dx\)
\end{doublespace}

Comparison: if \(a<b\), and \(f(x) \leq g(x)\), for all \(a \leq x\leq b\), then 

\begin{doublespace}
\noindent\(\pmb{\text{TraditionalForm}[\text{Integrate}[f[x],\{x,a,b\}]\leq  (\text{Integrate}[g[x],\{x,a,b\}])]}\)
\end{doublespace}

\begin{doublespace}
\noindent\(\int_a^b f(x) \, dx\leq \int_a^b g(x) \, dx\)
\end{doublespace}

$\blacktriangleleft \thickspace \thickspace \thickspace $$\thickspace \thickspace |\thickspace \thickspace $$\thickspace \thickspace \thickspace
\blacktriangleright $



\section*{Integration}

\subsection*{The Definite Integral}

\subsubsection*{The Mean Value Theorem}

Suppose the integrand \(f(x)\) is a continuous function on the interval \([a,b]\). Then there exists a number \(c\), between \(a\) and \(b\), for
which 

\begin{doublespace}
\noindent\(\pmb{\text{TraditionalForm}[\text{Integrate}[f[x],\{x,a,b\}]== f[c](b-a)]}\)
\end{doublespace}

\begin{doublespace}
\noindent\(\int_a^b f(x) \, dx=(b-a) f(c)\)
\end{doublespace}

The idea behind this theorem is the \textit{ Intermediate Value Theorem} for continuous functions. 

\begin{doublespace}
\noindent\(\pmb{\text{Manipulate}[}\\
\pmb{\text{Show}[}\\
\pmb{\text{Plot}[\text{Sin}[x+\text{Sin}[2x]],}\\
\pmb{ \{x,0,\text{Pi}\},}\\
\pmb{\text{PlotLegends}\to \text{Placed}[\text{{``}y=f(x) vs y=$\pi $m{''}},\text{Above}],}\\
\pmb{\text{PlotRange}\to \{\{0,\text{Pi}\},\{0,\text{Pi}/2\}\},}\\
\pmb{\text{PlotRangeClipping}\to  \text{True} ],}\\
\pmb{\text{Plot}[\text{Pi}*m,}\\
\pmb{ \{x,0,\text{Pi}\}, }\\
\pmb{\text{Filling}\to \text{Bottom},}\\
\pmb{\text{PlotRange}\to \{\{0,\text{Pi}\},\{0,\text{Pi}/2\}\},}\\
\pmb{\text{PlotRangeClipping}\to  \text{True}}\\
\pmb{]], }\\
\pmb{\{m,0,1/\text{Pi}\}]}\\
\pmb{}\)
\end{doublespace}

\begin{doublespace}
\noindent\(\)
\end{doublespace}



\section*{Integration}

\subsection*{The Indefinite Integral}

Having seen a few examples and properties of integrals, we turn to study \textit{ Integration}, on its own right. 

Given a function \(f(x)\), integrable on the interval \([a,b]\), we may consider intervals in intermediate regions.

\begin{doublespace}
\noindent\(\pmb{\text{DynamicModule}[\{\text{pts}=\{\{0,0\},\{\text{Pi},0\}\}\},\text{LocatorPane}[\text{Dynamic}[\text{pts},(\text{pts}[[1]]=\{\#[[1,1]],0\};}\\
\pmb{\text{pts}[[2]]=\{\#[[2,1]],0\})\&],}\\
\pmb{\text{Dynamic}[}\\
\pmb{\text{Framed}@}\\
\pmb{\text{Show}@}\\
\pmb{\{\text{Plot}[\text{Sin}@x,\{x,0,2 \text{Pi}\},}\\
\pmb{\text{PlotLabel}\to \text{ToString}@\text{TraditionalForm}[\text{Integrate}[\sin [x],\{x,\text{pts}[[1,1]],\text{pts}[[2,1]]\}]]<>}\\
\pmb{\text{{``} = {''}}<>\text{ToString}[\text{Integrate}[\text{Sin}@x,\{x,\text{pts}[[1,1]],\text{pts}[[2,1]]\}]]],}\\
\pmb{\text{Plot}[\text{Sin}@x,\{x,\text{pts}[[1,1]],\text{pts}[[2,1]]\},\text{PlotRange}\to  \{-1,1\},\text{Filling}\to \text{Axis}]\}]]]}\)
\end{doublespace}

\begin{doublespace}
\noindent\(\)
\end{doublespace}

That is, we may define a function by integrating another:

\begin{doublespace}
\noindent\(\pmb{\text{TraditionalForm}[g[y]==\text{Integrate}[f[x],\{x,a,y\}]]}\)
\end{doublespace}

\begin{doublespace}
\noindent\(g(y)=\int_a^y f(x) \, dx\)
\end{doublespace}

The function \(g\) is called an \textit{ indefinite integral} of f. 

$\blacktriangleleft \thickspace \thickspace \thickspace $$\thickspace \thickspace |\thickspace \thickspace $$\thickspace \thickspace \thickspace
\blacktriangleright $



\section*{Integration}

\subsection*{The indefinite integral}



\end{document}

% Exam Template for UMTYMP and Math Department courses
%
% Using Philip Hirschhorn's exam.cls: http://www-math.mit.edu/~psh/#ExamCls
%
% run pdflatex on a finished exam at least three times to do the grading table on front page.
%
%%%%%%%%%%%%%%%%%%%%%%%%%%%%%%%%%%%%%%%%%%%%%%%%%%%%%%%%%%%%%%%%%%%%%%%%%%%%%%%%%%%%%%%%%%%%%%

% These lines can probably stay unchanged, although you can remove the last
% two packages if you're not making pictures with tikz.
\documentclass[11pt]{exam}
\RequirePackage{amssymb, amsfonts, amsmath, latexsym, verbatim, xspace, setspace}


% By default LaTeX uses large margins.  This doesn't work well on exams; problems
% end up in the "middle" of the page, reducing the amount of space for students
% to work on them.
\usepackage[margin=1in]{geometry}
\usepackage[english]{babel}
\usepackage[autostyle]{csquotes} %%%% This package allows Tex to recognize quotation marks with the \enquote command. 


% Here's where you edit the Class, Exam, Date, etc.
\newcommand{\class}{MAT 132}
\newcommand{\term}{Summer II 2019}
\newcommand{\examnum}{Final exam - part I}
\newcommand{\examdate}{08/12/19}
\newcommand{\timelimit}{80 minutes}

% For an exam, single spacing is most appropriate
\singlespacing
% \onehalfspacing
% \doublespacing

% For an exam, we generally want to turn off paragraph indentation
\parindent 0ex
\title{MAT 132 - Summer II 2019: Final Exam - part I}
\begin{document} 

% These commands set up the running header on the top of the exam pages
\pagestyle{head}
\firstpageheader{}{}{}\textbf{}
\runningheader{\class}{\examnum\ - Page \thepage\ of \numpages}{\examdate}
\runningheadrule

\begin{flushright}
\begin{tabular}{p{2.8in} r l}
\textbf{\class} & \textbf{Name (Print):} & \makebox[2in]{\hrulefill}\\
\textbf{\term} &&\\
\textbf{\examnum} &&\\
\textbf{\examdate} &&\\
\textbf{Time Limit: \timelimit} & ID number & \makebox[2in]{\hrulefill}
\end{tabular}\\
\end{flushright}
\rule[1ex]{\textwidth}{.1pt}

\begin{center}
\large{\textbf{Instructions}}
\end{center}

\begin{minipage}[t]{3.7in}
\vspace{0pt}
\begin{itemize}

\item This exam contains \numpages\ pages (including this cover page) and
\numquestions\ problems.  Check to see if any pages are missing.  Enter
all requested information on the top of this page, and put your initials
on the top of every page, in case the pages become separated.

\item You may \textit{not} use your books, notes, or any device that is capable of accessing the internet on this exam (e.g., smartphones, smartwatches, tablets). You may not use a calculator.

\item \textbf{Organize your work}, in a reasonably neat and coherent way, in
the space provided. Work scattered all over the page without a clear ordering will 
receive very little credit.  

\item \textbf{Mysterious or unsupported answers will not receive full
credit}.

\end{itemize}

\end{minipage}
\hfill
\begin{minipage}[t]{2.3in}
\vspace{0pt}
%\cellwidth{3em}
\gradetablestretch{2}
\vqword{Problem}
\addpoints % required here by exam.cls, even though questions haven't started yet.	
\gradetable[v]%[pages]  % Use [pages] to have grading table by page instead of question

\end{minipage}
\newpage % End of cover page

%%%%%%%%%%%%%%%%%%%%%%%%%%%%%%%%%%%%%%%%%%%%%%%%%%%%%%%%%%%%%%%%%%%%%%%%%%%%%%%%%%%%%
%
% See http://www-math.mit.edu/~psh/#ExamCls for full documentation, but the questions
% below give an idea of how to write questions [with parts] and have the points
% tracked automatically on the cover page.
%
%
%%%%%%%%%%%%%%%%%%%%%%%%%%%%%%%%%%%%%%%%%%%%%%%%%%%%%%%%%%%%%%%%%%%%%%%%%%%%%%%%%%%%%

\begin{questions}

% Basic question
%%%%%%%%%%%%%%

%%%%%%%%%%%%%%%%% 

\addpoints
\question Compute the following integrals
\begin{parts}
\part[1] 
\begin{equation*}
\int xe^{x^2} \ dx
\end{equation*}

\textbf{Solution:} Use the substitution $u=x^2$. Then $du=2xdx$, and the integral becomes
\begin{align*}
\int xe^{x^2} \ dx & = \int \frac{e^u}{2} \ du \\
& = \frac{e^u}{2} + C \\
& = \frac{e^{x^2}}{2}+C.
\end{align*}
\vfill 
\part[1]
\begin{equation*}
\int x\sin(x) \ dx
\end{equation*}

\textbf{Solution:} Use integration by parts, with 
\begin{equation*}
\begin{array}{rl}
f(x)=x, & f'(x)=1, \\
g(x)=-\cos(x), & g'(x)=\sin(x).
\end{array}
\end{equation*}
The integral is thus 
\begin{align*}
\int x\sin(x) \ dx & = -x\cos(x) + \int \cos(x) \ dx \\
& = -x\cos(x) +\sin(x) + C.
\end{align*}
\vfill 
\newpage
\part[2] 
\begin{equation*}
\int_{0}^{1} \frac{1}{(x+2)(x+3)}\ dx
\end{equation*}

\textbf{Solution:} The integrand is a compound fraction. We will decompose into partial fractions before integrating, 
\begin{equation*}
\frac{1}{(x+2)(x+3)} = \frac{A}{x+2}+\frac{B}{x+3}.
\end{equation*}
The constants $A$ and $B$ need to satisfy
\begin{equation*}
\left\{
\begin{array}{rl}
A+B & = 0 \\
3A+2B & =1,
\end{array}
\right.
\end{equation*}
The solution to this system is $A=1$, $B=-1$. Thus, the integral is 
\begin{align*}
\int_{0}^{1} \frac{1}{(x+2)(x+3)} \ dx & = \int_{0}^{1} \frac{1}{x+2} - \frac{1}{x+3} \ dx \\
& = \ln(x+2) -\ln(x+3) \Big|_{0}^{1} \\
& = 2\ln(3)-\ln(2)-\ln(4).
\end{align*}
\vfill 
\part[1] 
\begin{equation*}
\int_{0}^{2} \frac{1}{x^2+4} \ dx
\end{equation*}

\textbf{Solution:} This integral can be solved by means of the trigonometric substitution $x=2\tan(\theta)$. The relation between the differentials is 
\begin{equation*}
dx = 2\sec^{2}(\theta) d\theta.
\end{equation*}
In terms of $\theta$, the integral becomes 
\begin{align*}
\int_{0}^{2} \frac{1}{x^2+4} \ dx & = \int_{0}^{\frac{\pi}{4}} \frac{2\sec^{2}(\theta)}{4+4\tan^2(\theta)} \ d\theta \\
& = \int_{0}^{\frac{\pi}{4}} \frac{1}{2} \ d \theta \\
& = \frac{\theta}{2}\Big|_{0}^{\frac{\pi}{4}} \\
& = \frac{\pi}{8}.
\end{align*}
\vfill
\end{parts}
%%%%%%%%%%%%%%%%

\newpage
\addpoints
\question[1] Find the area of the region bounded by the curves $y=\cos(x)$ and $y=\sin(x)$, for $0 \leq x \leq \frac{\pi}{4}$. 

\textbf{Solution:} In the domain of integration, $0 \leq x \leq \frac{\pi}{4}$, the graph of the cosine function is above the sine. Thus the area is 
\begin{align*}
A & = \int_{0}^{\frac{\pi}{4}} \cos(x)-\sin(x) \ dx \\
& = \sin(x)+\cos(x) \Big|_{0}^{\frac{\pi}{4}} \\
& = \sqrt{2}-1.
\end{align*}
%%%%%%%%%%%%%%%%%

\newpage
\addpoints
\question[2] The region bounded by the curves $y=\sqrt{x}, \ x>0$ and $y=x^2$ is rotated about the the $x$-axis. Find the volume of the resulting solid. 

\textbf{Solution:} The two curves intersect when $x=0$ or $x=1$, so the domain of integration is $0 \leq x \leq 1$. The curve $y=\sqrt{x}$ is farther away from the axis of rotation than $y=x^2$, for the values of $x$ referenced above. We compute the volume by means of the washer method, 
\begin{align*}
V & = \int_{0}^{1} \pi [(\sqrt{x})^2-(x^2)^2] \ dx \\
& = \pi \int_{0}^{1} (x-x^4) \ dx \\
& = \pi \left[\frac{x^2}{2}-\frac{x^5}{5} \right] \Big|_{0}^{1} \\
& = \frac{3\pi}{10}.
\end{align*}

%%%%%%%%%%%%%%%%%

\newpage
\addpoints
\question[2] Use the arclength formula to find the length of the curve $y=\sqrt{4-x^2}$, for $-2\leq x \leq 2$.

\textbf{Solution:} The speed of the curve is 
\begin{equation*}
\sqrt{1+\left(\frac{dy}{dx}\right)^2} = \sqrt{1+ \left(\frac{-x}{\sqrt{4-x^2}}\right)^2} = \sqrt{1+\frac{x^2}{4-x^2}} =\frac{2}{\sqrt{4-x^2}}.
\end{equation*}

The length of the curve is given by the integral of its speed, 
\begin{equation*}
L  = \int_{-2}^{2} \frac{2}{\sqrt{4-x^2}} \ dx.
\end{equation*}
This integral can be solved by means of the trigonometric substitution $x=2\sin(\theta)$, $-\frac{\pi}{2} \leq \theta \leq \frac{\pi}{2}$. Its value is 
\begin{align*}
L & = \int_{-\frac{\pi}{2}}^{\frac{\pi}{2}} \frac{4\cos(\theta)}{\sqrt{4-4\sin^2(\theta)}} \ d\theta \\
& = 2\pi. 
\end{align*}

%%%%%%%%%%%%%%%%%
\newpage
\addpoints
\question Find the general solutions of the following differential equations
\begin{parts}
\part[1] 
\begin{equation*}
\frac{dy}{dx} = xy^2
\end{equation*}

\textbf{Solution:} By separating variables and integrating, we have 
\begin{align*}
\int \frac{1}{y^2} \ dy & = \int x \ dx \\
-\frac{1}{y} &= \frac{x^2}{2}+C.
\end{align*}
The general solution is thus 
\begin{equation*}
y=-\frac{1}{\frac{x^2}{2}+C}.
\end{equation*}
\vfill 
\part[1]
\begin{equation*}
y''+3y'+2y=0
\end{equation*}

\textbf{Solution:} The characteristic equation is 
\begin{equation*}
r^2+3r+2=0,
\end{equation*}
and its solutions are $r=-1$ and $r=-2$. The general solution to the differential equation is 
\begin{equation*}
y(x)=De^{-x} + Ee^{-2x}
\end{equation*}

\vfill 
\newpage
\part[1] 
\begin{equation*}
y''+6y+9=0
\end{equation*}

\textbf{Solution:} This problem contains a typo. The equation was meant to be $y''+6y+9y=0$. All students were granted credit for this problem. 

\vfill
\part[1]
\begin{equation*}
y''+16y=0
\end{equation*}

\textbf{Solution:} The characteristic equation is 
\begin{equation*}
r^2+16=0.
\end{equation*}
This equation does not have real solutions. 

Since the above quadratic does not contain a linear term ($b=0$), there is no exponential component in the solution. Meanwhile, the discriminant is 
\begin{equation*}
\Delta =-64,
\end{equation*}
so the trigonometric (and only) component of the solution is 
\begin{equation*}
y(x)= D \cos{4x} + E\sin(4x).
\end{equation*}

\vfill
\end{parts}
%%%%%%%%%%%%%%%
\newpage
\addpoints
\question[1] Use the Comparison Test to determine whether the following series converges. 
\begin{equation*}
\sum_{n=1}^{\infty} \frac{\ln(n+3)}{n}
\end{equation*}

\textbf{Solution:} We use the comparison $\ln(n+3) >1$, for all $n$, which leads to 
\begin{equation*}
\frac{\ln(n+3)}{n} > \frac{1}{n}.
\end{equation*}
Since the series $\sum \frac{1}{n}$ diverges, the series $\sum \frac{\ln(n+3)}{n}$ diverges, by comparison. 

%%%%%%%%%%%%%%%
\newpage 
\addpoints
\question[1] Use the Integral Test to determine whether the following series converges. 
\begin{equation*}
\sum_{n=1}^{\infty} \frac{1}{n^3}
\end{equation*}

\textbf{Solution:} We use the auxilliary function $f(x)=\frac{1}{x^3}$, $x \geq 1$. This function is continuous and positive in the domain $x \geq 1$. It is decreasing, as 
\begin{equation*}
f'(x)=-\frac{3}{x^4} <0.
\end{equation*}
Its improper integral is 
\begin{align*}
\int_{1}^{\infty} \frac{1}{x^3} \ dx & = \lim_{t \to \infty} \int_{1}^{t} \frac{1}{x^3} \ dx \\
& = \lim_{t \to \infty} -\frac{1}{2x^2} \Big|_{1}^{t} \\
& = \frac{1}{2}.
\end{align*}
Since the integral converges, so does the series. 

%%%%%%%%%%%%%%%

\newpage
\addpoints
\question[1] Use the Limit Comparison Test to determine whether the following series converges. 
\begin{equation*}
\sum_{n=1}^{\infty} \frac{n^2-2n+1}{n^3+1}
\end{equation*}

\textbf{Solution:} The series is comparable to $\sum \frac{1}{n}$, for 
\begin{align*}
\lim_{n \to \infty} \frac{\frac{n^2-2n+1}{n^3+1}}{\frac{1}{n}} & = \lim_{n \to \infty} \frac{n^3-2n^2+n}{n^3+1} \\
& = 1.
\end{align*}
Since the series $\sum \frac{1}{n}$ diverges, so does $\sum_{n=1}^{\infty} \frac{n^2-2n+1}{n^3+1}$, by the Limit Comparison Test. 

%%%%%%%%%%%%%%%

\newpage
\addpoints 
\question[1] Use the Alternating Series Test to determine whether the following series converges. 
\begin{equation*}
\sum_{n=1}^{\infty} \frac{(-1)^n}{n}
\end{equation*}

\textbf{Solution:} The positive part of the series is $a_n=\frac{1}{n}$. This sequence is decreasing, as $n$ increases, and its limit is 
\begin{equation*}
\lim_{n \to \infty} \frac{1}{n} = 0.
\end{equation*}
It follows from the Alternating Series Test that $\sum_{n=1}^{\infty} \frac{(-1)^n}{n}$ converges. 
%%%%%%%%%%%%%%%

\newpage
\addpoints
\question[2] Use the Ratio Test to determine the radius of convergence of the following power series
\begin{equation*}
\sum_{n=1}^{\infty} \frac{(3x+1)^n}{n^2+1}.
\end{equation*}

\textbf{Solution:} We compute the limitting ratio, 
\begin{align*}
\lim_{n \to \infty} \left| \frac{\frac{(3n+1)^{n+1}}{(n+1)^2+1}}{\frac{(3x+1)^n}{n^2+1}}\right| & = \lim_{n \to \infty} \left| \frac{(3x+1)(n^2+1)}{n^2+2n+2} \right| \\ 
& = |3x+1|.
\end{align*}|
The series is guaranteed to converge so long as $|3x+1|<1$, that is 
\begin{equation*}
-\frac{2}{3} < x < 0, 
\end{equation*}
thus its radius of convergence is $R=\frac{1}{3}$. 
\end{questions}
\end{document}

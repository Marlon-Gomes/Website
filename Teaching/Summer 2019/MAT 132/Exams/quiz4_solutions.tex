% Exam Template for UMTYMP and Math Department courses
%
% Using Philip Hirschhorn's exam.cls: http://www-math.mit.edu/~psh/#ExamCls
%
% run pdflatex on a finished exam at least three times to do the grading table on front page.
%
%%%%%%%%%%%%%%%%%%%%%%%%%%%%%%%%%%%%%%%%%%%%%%%%%%%%%%%%%%%%%%%%%%%%%%%%%%%%%%%%%%%%%%%%%%%%%%

% These lines can probably stay unchanged, although you can remove the last
% two packages if you're not making pictures with tikz.
\documentclass[11pt]{exam}
\RequirePackage{amssymb, amsfonts, amsmath, latexsym, verbatim, xspace, setspace, graphicx, caption}


% By default LaTeX uses large margins.  This doesn't work well on exams; problems
% end up in the "middle" of the page, reducing the amount of space for students
% to work on them.
\usepackage[margin=1in]{geometry}
\usepackage[english]{babel}
\usepackage[autostyle]{csquotes} %%%% This package allows Tex to recognize quotation marks with the \enquote command. 


% Here's where you edit the Class, Exam, Date, etc.
\newcommand{\class}{MAT 132}
\newcommand{\term}{Summer II 2019}
\newcommand{\examnum}{Quiz 4}
\newcommand{\examdate}{08/08/19}
\newcommand{\timelimit}{50 minutes}

% For an exam, single spacing is most appropriate
\singlespacing
% \onehalfspacing
% \doublespacing

% For an exam, we generally want to turn off paragraph indentation
\parindent 0ex
\title{MAT 132 - Summer II 2019: Quiz 4 solutions}
\begin{document} 

% These commands set up the running header on the top of the exam pages
\pagestyle{head}
\firstpageheader{}{}{}\textbf{}
\runningheader{\class}{\examnum\ - Page \thepage\ of \numpages}{\examdate}
\runningheadrule

\begin{flushright}
\begin{tabular}{p{2.8in} r l}
\textbf{\class} & \textbf{Name (Print):} & \makebox[2in]{\hrulefill}\\
\textbf{\term} &&\\
\textbf{\examnum} &&\\
\textbf{\examdate} &&\\
\textbf{Time Limit: \timelimit} & ID number & \makebox[2in]{\hrulefill}
\end{tabular}\\
\end{flushright}
\rule[1ex]{\textwidth}{.1pt}

\begin{center}
\large{\textbf{Instructions}}
\end{center}

\begin{minipage}[t]{3.7in}
\vspace{0pt}
\begin{itemize}

\item This exam contains \numpages\ pages (including this cover page) and
\numquestions\ problems.  Check to see if any pages are missing.  Enter
all requested information on the top of this page, and put your initials
on the top of every page, in case the pages become separated.

\item You may \textit{not} use your books, notes, or any device that is capable of accessing the internet on this exam (e.g., smartphones, smartwatches, tablets). You may use a calculator.

\item \textbf{Organize your work}, in a reasonably neat and coherent way, in
the space provided. Work scattered all over the page without a clear ordering will 
receive very little credit.  

\item \textbf{Mysterious or unsupported answers will not receive full
credit}.

\end{itemize}

\end{minipage}
\hfill
\begin{minipage}[t]{2.3in}
\vspace{0pt}
%\cellwidth{3em}
\gradetablestretch{2}
\vqword{Problem}
\addpoints % required here by exam.cls, even though questions haven't started yet.	
\gradetable[v]%[pages]  % Use [pages] to have grading table by page instead of question

\end{minipage}
\newpage % End of cover page

%%%%%%%%%%%%%%%%%%%%%%%%%%%%%%%%%%%%%%%%%%%%%%%%%%%%%%%%%%%%%%%%%%%%%%%%%%%%%%%%%%%%%
%
% See http://www-math.mit.edu/~psh/#ExamCls for full documentation, but the questions
% below give an idea of how to write questions [with parts] and have the points
% tracked automatically on the cover page.
%
%
%%%%%%%%%%%%%%%%%%%%%%%%%%%%%%%%%%%%%%%%%%%%%%%%%%%%%%%%%%%%%%%%%%%%%%%%%%%%%%%%%%%%%

\begin{questions}

% Basic question
%%%%%%%%%%%%%%

%%%%%%%%%%%%%%%%% 

\addpoints
\question[2] Determine whether the sequence 
\begin{equation*}
a_n = \frac{n\sin(n)}{n^2+1}
\end{equation*}
converges or diverges. If it converges, finds its limit. 

\textbf{Solution:} The sequence is bounded as follows 
\begin{equation*}
\frac{-n}{n^2+1} \leq a_n \leq \frac{n}{n^2+1}. 
\end{equation*}
The bounds have the same limit, as $n \to \infty$, 
\begin{equation*}
\lim_{n\to \infty} \frac{-n}{n^2+1}=\lim_{n\to \infty} \frac{n}{n^2+1} = 0,
\end{equation*}
hence the limit of the sequence is 
\begin{equation*}
\lim_{n \to \infty} a_n=0.
\end{equation*}
%%%%%%%%%%%%%%%%%

\newpage
\addpoints
\question Determine the value of the following series: 
\begin{parts}
\part[2] 
\begin{equation*}
\sum_{n=1}^{\infty} \frac{3^n}{4^{n+2}}
\end{equation*}

\textbf{Solution:} This is a geometric series. It can be written in standard form as follows, 
\begin{equation*}
\sum_{n=1}^{\infty} \frac{3^n}{4^{n+2}} = \sum_{n=1}^{\infty} \frac{1}{16}\left(\frac{3}{4}\right)^n.
\end{equation*}
We recognize the ratio as $\frac{3}{4}$. Since the ratio is, in absolute value, less than 1, the series converges, and its value is given by 
\begin{equation*}
 \sum_{n=1}^{\infty} \frac{1}{16}\left(\frac{3}{4}\right)^n = \frac{\frac{1}{16}\left(\frac{3}{4}\right)}{1-\frac{3}{4}} = \frac{3}{16}.
 \end{equation*}
 
\newpage
\part[2] 
\begin{equation*}
\sum_{n=1}^{\infty} \frac{1}{n^2+7n+10}
\end{equation*}
\end{parts}

\textbf{Solution:} The denominator of the fractions can be factored as 
\begin{equation*}
n^2+7n+10 =(n+2)(n+5).
\end{equation*}
The compound fraction can be written in terms of partial fractions as 
\begin{align*}
\frac{1}{n^2+7n+10} & =\frac{A}{n+2} + \frac{B}{n+5}\\
& = \frac{A(n+5)+B)n+2)}{(n+2)(n+5)}\\
& = \frac{(A+B)n+(5A+2B)}{(n+2)(n+5)},
\end{align*}
for constants $A$ and $B$ to be determined. These constants satisfy the following system of equations
\begin{equation*}
\left\{
\begin{array}{rl}
A + B & = 0\\
5A+2B & = 1
\end{array}
\right.
\end{equation*}
The solutions of the system are $A=\frac{1}{3}$ and $B=-\frac{1}{3}$. 

Now we can rewritte the series as 
\begin{equation*}
\sum_{n=1}^{\infty} \frac{1}{n^2+7n+10} = \sum_{n=1}^{\infty} \frac{1}{3}\left[ \frac{1}{n+2} - \frac{1}{n+5} \right].
\end{equation*}
We recognize the series as a telescopic series, with cancellations every three terms. The value of the series is therefore the sum of its three elements, 
\begin{equation*}
\sum_{n=1}^{\infty} \frac{1}{n^2+7n+10} = \frac{1}{9}+\frac{1}{12}+\frac{1}{15} = \frac{47}{180}.
\end{equation*}

%%%%%%%%%

\newpage
\addpoints
\question[2] Use the Integral Test to determine whether the following series converges
\begin{equation*}
\sum_{n=1}^{\infty} \frac{1}{n^2+1}
\end{equation*}

\textbf{Solution:} The function we use for comparison is 
\begin{equation*}
f(x)=\frac{1}{x^2+1}.
\end{equation*}
This function is decreasing, for $x>0$, and the improper integral
\begin{equation*}
\int_{1}^{\infty} \frac{1}{x^2+1} \ dx. 
\end{equation*}
converges, as the computation below shows, 
\begin{align*}
\int_{1}^{\infty} \frac{1}{x^2+1} \ dx & = \lim_{t \infty} \int_{1}^{t} \frac{1}{x^2+1} \ dx \\
& = \lim_{t \to \infty} \left( \arctan(x) \Big|_{1}^{t} \right) \\
& = \lim_{t \to \infty} (\arctan(t)-\arctan(1))\\
& = \frac{\pi}{2} -\frac{\pi}{4}\\ 
& = \frac{\pi}{4}.
\end{align*}
Therefore, by the Integral Comparison Test, the series 
\begin{equation*}
\sum_{n=1}^{\infty} \frac{1}{n^2+1}
\end{equation*}
converges.

%%%%%%%%%%%%
%%%%%%%%%
\newpage
\addpoints
\question[2] Use the Comparison Test to determine whether the following series converges
\begin{equation*}
\sum_{n=1}^{\infty} \frac{1+\sin(n)}{3^n}
\end{equation*}

\textbf{Solution:} The terms of the series are bounded as follows, 
\begin{equation*}
0 \leq \frac{1+ \sin(n)}{3^n} \leq \frac{2}{3^n}.
\end{equation*}
The series formed by the upper bounds, 
\begin{equation*}
\sum_{n=1}^{\infty} \frac{2}{3^n},
\end{equation*}
is a geometric series whose ratio is $\frac{1}{3}$, thus it converges. Therefore, by comparison, the series 
\begin{equation*}
\sum_{n=1}^{\infty} \frac{1+\sin(n)}{3^n}
\end{equation*}
converges. 

%%%%%%%%%%%%

\newpage
\addpoints
\question[2] Use the Limit Comparison Test to determine whether the following series converges
\begin{equation*}
\sum_{n=1}^{\infty} \frac{n^2-1}{3n^4+1}
\end{equation*}

\textbf{Solution:} This series is comparable to 
\begin{equation*}
\sum_{n=1}^{\infty} \frac{1}{n^2}.
\end{equation*}
Indeed, the limit 
\begin{equation*}
\lim_{n \to \infty} \frac{\frac{n^2-1}{3n^4+1}}{\frac{1}{n^2}} = \lim_{n \to \infty} \frac{n^4-n^2}{3n^4+1} =\frac{1}{3}
\end{equation*}
exists and is a positive number. By the Limit Comparison Test, the two series have the same behavior relative to convergence. Since $\sum \frac{1}{n^2}$ is known to converge, the series 
\begin{equation*}
\sum_{n=1}^{\infty} \frac{n^2-1}{3n^4+1}
\end{equation*}
also converges. 
%%%%%%%%%%

\newpage
\addpoints 
\question[2] Use the Alternating Series Test to determine whether the following series converges
\begin{equation*}
\sum_{n=1}^{\infty} \frac{(-1)^n(2n+1)}{3n-1}
\end{equation*}

\textbf{Solution:} The sequence of absolute values
\begin{equation*}
a_n=\frac{2n+1}{3n-1}
\end{equation*} 
is decreasing. Indeed, the auxilliary function 
\begin{equation*}
f(x)=\frac{2x+1}{3x-1},
\end{equation*}
whose values coincide with the sequence $a_n$ for $x=n$, has derivative
\begin{equation*}
f'(x)=\frac{-5}{(3x-1)^2},
\end{equation*}
which is negative for all values of $x$ for which it is defined ($x\neq \frac{1}{3}$). In particular, this function decreases for $x\geq1$. 

However, the limit of the sequence is not zero:
\begin{equation*}
\lim_{n \to \infty} a_n = \lim_{n \to 0} \frac{2n+1}{3n-1}=\frac{2}{3},
\end{equation*}
hence the alternating series diverges. 
%%%%%%%%%%%

\newpage
\addpoints
\question[3] Find the radius and interval of convergence of the following series
\begin{equation*}
\sum_{n=1}^{\infty} \frac{n(x+1)^n}{4^n}
\end{equation*}

\textbf{Solution:} We shall use the Ratio Test to determine the radius of convergence. The limitting ratio is 
\begin{align*}
\lim_{n \to \infty} \left| \frac{\frac{(n+1)(x+1)^{n+1}}{4^{n+1}}}{\frac{n(x+1)^n}{4^n}} \right| & = \lim_{n \to \infty} \left|\frac{(n+1)(x+1)}{4n} \right| \\
& = \frac{|x+1|}{4}.
\end{align*}
This limit is less than 1 if 
\begin{align*}
\frac{|x+1|}{4} & < 1 \\
|x+1| & < 4,
\end{align*}
so the radius of convergence of the series is 4. 

To determine the interval of convergence, we test for convergence at the endpoints.

When $x=-5$, the series becomes
\begin{equation*}
\sum_{n=1}^{\infty}  n\left(\frac{-4}{4}\right)^n = \sum_{n=1}^{\infty} (-1)^n n
\end{equation*}
This is a divergent alternating series, as the sequence of absolute values,
\begin{equation*}
a_n=n
\end{equation*}
diverges to infinity. 

When $x=3$, the series becomes 
\begin{equation*}
\sum_{n=1}^{\infty}  n\left(\frac{4}{4}\right)^n = \sum_{n=1}^{\infty} n
\end{equation*}
This series diverges, as the general term, $b_n=n$, diverges to infinity. 

The interval of convergence of the power series is $(-5,3)$. 


%%%%%%%%%%%

\newpage
\addpoints
\question[3] Find a power series representation of the following function
\begin{equation*}
f(x)=\frac{x}{9+x^2}.
\end{equation*}

\textbf{Solution:} We can rewrite this function as 
\begin{equation*}
\frac{x}{9+x^2}=\frac{x}{9}\left[\frac{1}{1-\left(\frac{-x^2}{9}\right)}\right].
\end{equation*}
In this way, we recognize the expression on the rightmost fraction as a a geometric sum (for $|x|<3$), hence
\begin{align*}
f(x) & = \frac{x}{9}\sum_{n=0}^{\infty} \left(\frac{-x^2}{9}\right)^n \\
& = \frac{x}{9}\sum_{n=0}^{\infty} (-1)^n \frac{x^{2n}}{9^{n}} \\
& = \sum_{n=0}^{\infty} (-1)^n\frac{x^{2n+1}}{9^{n+1}}, 
\end{align*}
for $|x|<3$. 
\end{questions}
\end{document}

% Exam Template for UMTYMP and Math Department courses
%
% Using Philip Hirschhorn's exam.cls: http://www-math.mit.edu/~psh/#ExamCls
%
% run pdflatex on a finished exam at least three times to do the grading table on front page.
%
%%%%%%%%%%%%%%%%%%%%%%%%%%%%%%%%%%%%%%%%%%%%%%%%%%%%%%%%%%%%%%%%%%%%%%%%%%%%%%%%%%%%%%%%%%%%%%

% These lines can probably stay unchanged, although you can remove the last
% two packages if you're not making pictures with tikz.
\documentclass[11pt]{exam}
\RequirePackage{amssymb, amsfonts, amsmath, latexsym, verbatim, xspace, setspace}


% By default LaTeX uses large margins.  This doesn't work well on exams; problems
% end up in the "middle" of the page, reducing the amount of space for students
% to work on them.
\usepackage[margin=1in]{geometry}
\usepackage[english]{babel}
\usepackage[autostyle]{csquotes} %%%% This package allows Tex to recognize quotation marks with the \enquote command. 


% Here's where you edit the Class, Exam, Date, etc.
\newcommand{\class}{MAT 132}
\newcommand{\term}{Summer II 2019}
\newcommand{\examnum}{Final exam - part I (retake)}
\newcommand{\examdate}{08/14/19}
\newcommand{\timelimit}{80 minutes}

% For an exam, single spacing is most appropriate
\singlespacing
% \onehalfspacing
% \doublespacing

% For an exam, we generally want to turn off paragraph indentation
\parindent 0ex
\title{MAT 132 - Summer II 2019: Final Exam - part I (retake)}
\begin{document} 

% These commands set up the running header on the top of the exam pages
\pagestyle{head}
\firstpageheader{}{}{}\textbf{}
\runningheader{\class}{\examnum\ - Page \thepage\ of \numpages}{\examdate}
\runningheadrule

\begin{flushright}
\begin{tabular}{p{2.8in} r l}
\textbf{\class} & \textbf{Name (Print):} & \makebox[2in]{\hrulefill}\\
\textbf{\term} &&\\
\textbf{\examnum} &&\\
\textbf{\examdate} &&\\
\textbf{Time Limit: \timelimit} & ID number & \makebox[2in]{\hrulefill}
\end{tabular}\\
\end{flushright}
\rule[1ex]{\textwidth}{.1pt}

\begin{center}
\large{\textbf{Instructions}}
\end{center}

\begin{minipage}[t]{3.7in}
\vspace{0pt}
\begin{itemize}

\item This exam contains \numpages\ pages (including this cover page) and
\numquestions\ problems.  Check to see if any pages are missing.  Enter
all requested information on the top of this page, and put your initials
on the top of every page, in case the pages become separated.

\item You may \textit{not} use your books, notes, or any device that is capable of accessing the internet on this exam (e.g., smartphones, smartwatches, tablets). You may not use a calculator.

\item \textbf{Organize your work}, in a reasonably neat and coherent way, in
the space provided. Work scattered all over the page without a clear ordering will 
receive very little credit.  

\item \textbf{Mysterious or unsupported answers will not receive full
credit}.

\end{itemize}

\end{minipage}
\hfill
\begin{minipage}[t]{2.3in}
\vspace{0pt}
%\cellwidth{3em}
\gradetablestretch{2}
\vqword{Problem}
\addpoints % required here by exam.cls, even though questions haven't started yet.	
\gradetable[v]%[pages]  % Use [pages] to have grading table by page instead of question

\end{minipage}
\newpage % End of cover page

%%%%%%%%%%%%%%%%%%%%%%%%%%%%%%%%%%%%%%%%%%%%%%%%%%%%%%%%%%%%%%%%%%%%%%%%%%%%%%%%%%%%%
%
% See http://www-math.mit.edu/~psh/#ExamCls for full documentation, but the questions
% below give an idea of how to write questions [with parts] and have the points
% tracked automatically on the cover page.
%
%
%%%%%%%%%%%%%%%%%%%%%%%%%%%%%%%%%%%%%%%%%%%%%%%%%%%%%%%%%%%%%%%%%%%%%%%%%%%%%%%%%%%%%

\begin{questions}

% Basic question
%%%%%%%%%%%%%%

%%%%%%%%%%%%%%%%% 

\addpoints
\question Compute the following integrals
\begin{parts}
\part[1] 
\begin{equation*}
\int x(1+x^2)^{15} \ dx
\end{equation*}
\vfill 
\part[1]
\begin{equation*}
\int xe^x \ dx
\end{equation*}
\vfill 
\newpage
\part[2] 
\begin{equation*}
\int_{0}^{1} \frac{1}{(x+1)(x+4)}\ dx
\end{equation*}
\vfill 
\part[1] 
\begin{equation*}
\int_{0}^{\frac{1}{2}} \frac{1}{\sqrt{1-x^2}} \ dx
\end{equation*}
\vfill
\end{parts}
%%%%%%%%%%%%%%%%

\newpage
\addpoints
\question[1] Find the area of the region bounded by the curves $y=x^3$ and $y=x$, for $x>0$. 
%%%%%%%%%%%%%%%%%

\newpage
\addpoints
\question[2] The region bounded by the curves $y=\sqrt{x}, \ x>0$ and $y=x^2$ is rotated about the the $y$-axis. Find the volume of the resulting solid. 

%%%%%%%%%%%%%%%%%

\newpage
\addpoints
\question[2] Use the arclength formula to find the length of the curve $y=x^\frac{3}{2}$, for $0 \leq x \leq 1$.

%%%%%%%%%%%%%%%%%
\newpage
\addpoints
\question Find the general solutions of the following differential equations
\begin{parts}
\part[1] 
\begin{equation*}
\frac{dy}{dx} = \frac{y}{x}, \ x>0
\end{equation*}
\vfill 
\part[1]
\begin{equation*}
y''+5y'+6y=0
\end{equation*}
\vfill 
\newpage
\part[1] 
\begin{equation*}
y''+4y'+4y=0
\end{equation*}
\vfill
\part[1]
\begin{equation*}
y''+9y=0
\end{equation*}
\vfill
\end{parts}
%%%%%%%%%%%%%%%
\newpage
\addpoints
\question[1] Use the Comparison Test to determine whether the following series converges. 
\begin{equation*}
\sum_{n=1}^{\infty} \frac{e^n}{n}
\end{equation*}

%%%%%%%%%%%%%%%
\newpage 
\addpoints
\question[1] Use the Integral Test to determine whether the following series converges. 
\begin{equation*}
\sum_{n=1}^{\infty} e^{-n}
\end{equation*}

%%%%%%%%%%%%%%%

\newpage
\addpoints
\question[1] Use the Limit Comparison Test to determine whether the following series converges. 
\begin{equation*}
\sum_{n=1}^{\infty} \frac{n^2-6n+1}{n^4+n^2}
\end{equation*}

%%%%%%%%%%%%%%%

\newpage
\addpoints 
\question[1] Use the Alternating Series Test to determine whether the following series converges. 
\begin{equation*}
\sum_{n=1}^{\infty} \frac{(-1)^n}{\sqrt{n}}
\end{equation*}

%%%%%%%%%%%%%%%

\newpage
\addpoints
\question[2] Use the Ratio Test to determine the radius of convergence of the following power series
\begin{equation*}
\sum_{n=1}^{\infty} \frac{(x+2)^n}{2n^2+1}.
\end{equation*}

\end{questions}
\end{document}

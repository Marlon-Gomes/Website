% Exam Template for UMTYMP and Math Department courses
%
% Using Philip Hirschhorn's exam.cls: http://www-math.mit.edu/~psh/#ExamCls
%
% run pdflatex on a finished exam at least three times to do the grading table on front page.
%
%%%%%%%%%%%%%%%%%%%%%%%%%%%%%%%%%%%%%%%%%%%%%%%%%%%%%%%%%%%%%%%%%%%%%%%%%%%%%%%%%%%%%%%%%%%%%%

% These lines can probably stay unchanged, although you can remove the last
% two packages if you're not making pictures with tikz.
\documentclass[11pt]{exam}
\RequirePackage{amssymb, amsfonts, amsmath, latexsym, verbatim, xspace, setspace}


% By default LaTeX uses large margins.  This doesn't work well on exams; problems
% end up in the "middle" of the page, reducing the amount of space for students
% to work on them.
\usepackage[margin=1in]{geometry}
\usepackage[english]{babel}
\usepackage[autostyle]{csquotes} %%%% This package allows Tex to recognize quotation marks with the \enquote command. 


% Here's where you edit the Class, Exam, Date, etc.
\newcommand{\class}{MAT 132}
\newcommand{\term}{Summer II 2019}
\newcommand{\examnum}{Quiz 2}
\newcommand{\examdate}{07/24/19}
\newcommand{\timelimit}{50 minutes}

% For an exam, single spacing is most appropriate
\singlespacing
% \onehalfspacing
% \doublespacing

% For an exam, we generally want to turn off paragraph indentation
\parindent 0ex
\title{MAT 132 - Summer II 2019: Quiz 2 solutions}
\begin{document} 

% These commands set up the running header on the top of the exam pages
\pagestyle{head}
\firstpageheader{}{}{}\textbf{}
\runningheader{\class}{\examnum\ - Page \thepage\ of \numpages}{\examdate}
\runningheadrule

\begin{flushright}
\begin{tabular}{p{2.8in} r l}
\textbf{\class} & \textbf{Name (Print):} & \makebox[2in]{\hrulefill}\\
\textbf{\term} &&\\
\textbf{\examnum} &&\\
\textbf{\examdate} &&\\
\textbf{Time Limit: \timelimit} & ID number & \makebox[2in]{\hrulefill}
\end{tabular}\\
\end{flushright}
\rule[1ex]{\textwidth}{.1pt}

\begin{center}
\large{\textbf{Instructions}}
\end{center}

\begin{minipage}[t]{3.7in}
\vspace{0pt}
\begin{itemize}

\item This exam contains \numpages\ pages (including this cover page) and
\numquestions\ problems.  Check to see if any pages are missing.  Enter
all requested information on the top of this page, and put your initials
on the top of every page, in case the pages become separated.

\item You may \textit{not} use your books, notes, or any device that is capable of accessing the internet on this exam (e.g., smartphones, smartwatches, tablets). You may use a calculator.

\item \textbf{Organize your work}, in a reasonably neat and coherent way, in
the space provided. Work scattered all over the page without a clear ordering will 
receive very little credit.  

\item \textbf{Mysterious or unsupported answers will not receive full
credit}.

\end{itemize}

\end{minipage}
\hfill
\begin{minipage}[t]{2.3in}
\vspace{0pt}
%\cellwidth{3em}
\gradetablestretch{2}
\vqword{Problem}
\addpoints % required here by exam.cls, even though questions haven't started yet.	
\gradetable[v]%[pages]  % Use [pages] to have grading table by page instead of question

\end{minipage}
\newpage % End of cover page

%%%%%%%%%%%%%%%%%%%%%%%%%%%%%%%%%%%%%%%%%%%%%%%%%%%%%%%%%%%%%%%%%%%%%%%%%%%%%%%%%%%%%
%
% See http://www-math.mit.edu/~psh/#ExamCls for full documentation, but the questions
% below give an idea of how to write questions [with parts] and have the points
% tracked automatically on the cover page.
%
%
%%%%%%%%%%%%%%%%%%%%%%%%%%%%%%%%%%%%%%%%%%%%%%%%%%%%%%%%%%%%%%%%%%%%%%%%%%%%%%%%%%%%%

\begin{questions}

% Basic question
%%%%%%%%%%%%%%

%%%%%%%%%%%%%%%%% 

\addpoints
\question This problem aims to estimate the area under the curve $y=e^x/x$, for $1 \leq x \leq 4$. 
\begin{parts}
\part[2] Estimate using the midpoint rule with $n=4$.

\textbf{Solution:} The range of integration is $3$ units long. The width of the approximating rectangles is 
\begin{equation*}
\Delta x = \frac{3}{4}.
\end{equation*}
The sampling points for the midpoint rule are 
\begin{align*}
\overline{x}_i & = 1+ \frac{2i-1}{2}\Delta x \\
& = 1+ \frac{2i-1}{2} \frac{3}{4}\\
& \frac{6i+5}{8},
\end{align*}
for $1 \leq i \leq 4$. The table below contains the sampling points and the respective values of the function.
\begin{center}
\begin{tabular}{|c|c|c|c|c|}
\hline 
$\overline{x}_{i}$ & $\frac{11}{8}$ & $\frac{17}{8}$ & $\frac{23}{8}$ & $\frac{29}{8}$ \\ 
\hline 
$f(\overline{x}_i)$ & $\frac{8}{11}e^{\frac{11}{8}}$ & $\frac{8}{17}e^{\frac{17}{8}}$ & $\frac{8}{23}e^{\frac{23}{8}}$  & $\frac{8}{29}e^{\frac{29}{8}}$  \\ 
\hline 
\end{tabular} 
\end{center}
The approximation is 
\begin{equation*}
M_4 = \frac{3}{4}\left(\frac{8}{11}e^{\frac{11}{8}}+\frac{8}{17}e^{\frac{17}{8}}+\frac{8}{23}e^{\frac{23}{8}} +\frac{8}{29}e^{\frac{29}{8}}\right) \approx 17.500.
\end{equation*}

\part[2] Estimate using the trapezoidal rule with $n=4$.

\textbf{Solution:} The width of the trapezoids is the same as in part (a). The smapling points are 
\begin{equation*}
x_i = 1+ i\Delta x =1+\frac{3i}{4} = \frac{4+3i}{4},
\end{equation*}
for $0 \leq i \leq 4$. The table below contains the sampling points and the corresponding values of the function. 
\begin{center}
\begin{tabular}{|c|c|c|c|c|c|}
\hline 
$x_i$ & $1$ & $\frac{7}{4}$ & $\frac{5}{2}$ & $\frac{13}{4}$ & $4$ \\ 
\hline 
$f(x_i)$ & $e$ & $\frac{4}{7}e^{\frac{7}{4}}$ & $\frac{2}{5}e^{\frac{5}{2}}$ & $\frac{4}{13}e^{\frac{13}{4}}$& $\frac{1}{4}e^4$ \\ 
\hline 
\end{tabular} 
\end{center}
The approximation is 
\begin{equation*}
T_4=\frac{3}{8}\left(e + \frac{8}{7}e^{\frac{7}{4}}+\frac{4}{5}e^{\frac{5}{2}}+\frac{8}{13}e^{\frac{13}{4}}+\frac{1}{4}e^4\right)\approx 18.211
\end{equation*}
\vfill 
\newpage
\part[2] Estimate using Simpson's rule with $n=4$.

\textbf{Solution:} The width of the parabolic arcs is the same as in parts $(a)$ and $(b)$. The sampling points and values of the function are the same as those in part $(b)$. The approximation is 
\begin{equation*}
S_4=\frac{1}{4}\left(e + \frac{4}{7}e^{\frac{7}{4}}+\frac{4}{5}e^{\frac{5}{2}}+\frac{16}{13}e^{\frac{13}{4}}+\frac{1}{4}e^4\right)\approx 17.752
\end{equation*}
\end{parts}
%%%%%%%%%%%%%%%%%

\newpage
\addpoints
\question[3] Determine whether the integral is divergent or convergent. If the integral converges, compute its value. 
\begin{equation*}
\int_{2}^{3}\frac{1}{\sqrt{3-x}} \ dx
\end{equation*}

\textbf{Solution:} This is an improper integral of type 2 (the integrand has a discontinuity at $x=3$). It is convergent, as we can see by approximating via definite integrals, 
\begin{align*}
\int_{2}^{3} \frac{1}{\sqrt{3-x}} \ dx & = \lim_{t \to 3-} \int_{2}^{t} \frac{1}{\sqrt{3-x}} \ dx \\ 
& = \lim_{t \to 3-} \int_{2}^{t} \left( -2\sqrt{3-x} \Big|_{2}^{t} \right) \\
& = \lim_{t \to 3-} (-2\sqrt{3-t}+2\sqrt{3-2})\\
& = 2.
\end{align*}

\newpage

%%%%%%%%%%%%%%%%%%%
\addpoints
\question[4] Find the area of the region enclosed by the curves 
\begin{equation*}
f(x)=\cos(x)
\end{equation*}
and 
\begin{equation*}
g(x)=2-\cos(x),
\end{equation*}
for $0 \leq x \leq 2\pi$. 

\textbf{Solution:} The graph of $f(x)$ sits below the graph of $g(x)$, for all $0 \leq x \leq 2\pi$, thus the area is given by 
\begin{align*}
A & = \int_{0}^{2\pi} (2-\cos(x)) - (\cos(x)) \ dx \\
& = \int_{0}^{2\pi} 2-2\cos(x) \ dx\\
& = 4\pi - 2\int_{0}^{2\pi} \cos(x) \ dx\\
& = 4\pi - \left( 2\sin(x) \Big|_{0}^{2\pi} \right) \\
& = 4\pi.
\end{align*}
%%%%%%%%%
\newpage
\addpoints

\question[4] The region bounded by the curves
\begin{align*}
y & =\sqrt{x}\\
y & =x^2,
\end{align*}
is rotated about the $y$-axis. Find the volume of the resulting solid.

\textbf{Solution:} The curves intersect when $y=0$ and $y=1$. The outermost curve is $y=x^2$, while the innermost curve is $y=\sqrt{x}$.  In terms of $y$, the outer and inner radii are respectively $R_{out}(y)=\sqrt{y}$ and $R_{in}(y)=y^2$. Using the washer method, we obtain 
\begin{align*}
V & = \int_{0}^{1} \pi[(\sqrt{y})^2-(y^2)^2] \ dy \\
& = \int_{0}^{1} \pi [y-y^4] \ dy\\
& = \pi\left(\frac{y^2}{2}-\frac{y^5}{5} \Big|_{0}^{1} \right)\\
& = \pi \left( \frac{1}{2}-\frac{1}{5}\right)\\
& =\frac{3\pi}{10}.
\end{align*}
\addpoints 
\newpage

\question[3] Find the exact length of the arc of the curve
\begin{equation*}
y=x^\frac{3}{2}
\end{equation*}
bounded between $1 \leq x \leq 2$. 

\textbf{Solution:}
The slope of the graph is given by 
\begin{equation*}
\frac{dy}{dx}=\frac{3\sqrt{x}}{2}.
\end{equation*}
The length of the curve is given by the arclength formula, 
\begin{align*}
L & = \int_{1}^{2}\sqrt{1+\left(\frac{dy}{dx}\right)^2} \ dx \\
& = \int_{1}^{2} \sqrt{1+\left(\frac{3\sqrt{x}}{2}\right)^2} \ dx\\
& = \int_{1}^{2} \sqrt{1+\frac{9x}{4}} \ dx \\
& =  \frac{8}{27}\left[\left(1+\frac{9x}{4}\right)^{\frac{3}{2}} \Big|_{1}^{2} \right] \\ 
& = \frac{8}{27}\left[\left(\frac{11}{2}\right)^{\frac{3}{2}} - \left(\frac{13}{4}\right)^{\frac{3}{2}}\right].
\end{align*}
\addpoints
\end{questions}
\end{document}

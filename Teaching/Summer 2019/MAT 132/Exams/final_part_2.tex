% Exam Template for UMTYMP and Math Department courses
%
% Using Philip Hirschhorn's exam.cls: http://www-math.mit.edu/~psh/#ExamCls
%
% run pdflatex on a finished exam at least three times to do the grading table on front page.
%
%%%%%%%%%%%%%%%%%%%%%%%%%%%%%%%%%%%%%%%%%%%%%%%%%%%%%%%%%%%%%%%%%%%%%%%%%%%%%%%%%%%%%%%%%%%%%%

% These lines can probably stay unchanged, although you can remove the last
% two packages if you're not making pictures with tikz.
\documentclass[11pt]{exam}
\RequirePackage{amssymb, amsfonts, amsmath, latexsym, verbatim, xspace, setspace}

%%%%%%%%% User's commands
\newcommand{\func}[3]{{#1} : {#2} \longrightarrow {#3}}

%%%%%%%%%%%%%%%

% By default LaTeX uses large margins.  This doesn't work well on exams; problems
% end up in the "middle" of the page, reducing the amount of space for students
% to work on them.
\usepackage[margin=1in]{geometry}
\usepackage[english]{babel}
\usepackage[autostyle]{csquotes} %%%% This package allows Tex to recognize quotation marks with the \enquote command. 


% Here's where you edit the Class, Exam, Date, etc.
\newcommand{\class}{MAT 132}
\newcommand{\term}{Summer II 2019}
\newcommand{\examnum}{Final exam - part II}
\newcommand{\examdate}{08/15/19}
\newcommand{\timelimit}{3 hours and 5 minutes}


% For an exam, single spacing is most appropriate
\singlespacing
% \onehalfspacing
% \doublespacing

% For an exam, we generally want to turn off paragraph indentation
\parindent 0ex
\title{MAT 132 - Summer II 2019: Final Exam - part I}
\begin{document} 

% These commands set up the running header on the top of the exam pages
\pagestyle{head}
\firstpageheader{}{}{}\textbf{}
\runningheader{\class}{\examnum\ - Page \thepage\ of \numpages}{\examdate}
\runningheadrule

\begin{flushright}
\begin{tabular}{p{2.8in} r l}
\textbf{\class} & \textbf{Name (Print):} & \makebox[2in]{\hrulefill}\\
\textbf{\term} &&\\
\textbf{\examnum} &&\\
\textbf{\examdate} &&\\
\textbf{Time Limit: \timelimit} & ID number & \makebox[2in]{\hrulefill}
\end{tabular}\\
\end{flushright}
\rule[1ex]{\textwidth}{.1pt}

\begin{center}
\large{\textbf{Instructions}}
\end{center}

\begin{minipage}[t]{3.7in}
\vspace{0pt}
\begin{itemize}

\item This exam contains \numpages\ pages (including this cover page) and
\numquestions\ problems.  Check to see if any pages are missing.  Enter
all requested information on the top of this page, and put your initials
on the top of every page, in case the pages become separated.

\item You may \textit{not} use your books, notes, or any device that is capable of accessing the internet on this exam (e.g., smartphones, smartwatches, tablets). You may not use a calculator.

\item \textbf{Organize your work}, in a reasonably neat and coherent way, in
the space provided. Work scattered all over the page without a clear ordering will 
receive very little credit.  

\item \textbf{Mysterious or unsupported answers will not receive full
credit}.

\end{itemize}

\end{minipage}
\hfill
\begin{minipage}[t]{2.3in}
\vspace{0pt}
%\cellwidth{3em}
\gradetablestretch{2}
\vqword{Problem}
\addpoints % required here by exam.cls, even though questions haven't started yet.	
\gradetable[v]%[pages]  % Use [pages] to have grading table by page instead of question

\end{minipage}
\newpage % End of cover page

%%%%%%%%%%%%%%%%%%%%%%%%%%%%%%%%%%%%%%%%%%%%%%%%%%%%%%%%%%%%%%%%%%%%%%%%%%%%%%%%%%%%%
%
% See http://www-math.mit.edu/~psh/#ExamCls for full documentation, but the questions
% below give an idea of how to write questions [with parts] and have the points
% tracked automatically on the cover page.
%
%
%%%%%%%%%%%%%%%%%%%%%%%%%%%%%%%%%%%%%%%%%%%%%%%%%%%%%%%%%%%%%%%%%%%%%%%%%%%%%%%%%%%%%

\begin{questions}

% Basic question
%%%%%%%%%%%%%%

%%%%%%%%%%%%%%%%% 

\addpoints
\question Compute the following integrals
\begin{parts}
\part[2] 
\begin{equation*}
\int \frac{\sin(\sqrt{x})}{\sqrt{x}} \ dx
\end{equation*}
\vfill 
\part[2]
\begin{equation*}
\int \cos^2(x)\sin^2(x) \ dx
\end{equation*}
\vfill 
\newpage
\part[3] 
\begin{equation*}
\int_{1}^{2} \ln(x) \ dx
\end{equation*}
\vfill 
\part[3] 
\begin{equation*}
\int_{0}^{2} \frac{1}{x^2+5x+6} \ dx
\end{equation*}
\vfill
\end{parts}
%%%%%%%%%%%%%%%%

\newpage
\addpoints
\question Consider the function 
\begin{equation*}
\func{f}{[1,\infty)}{\mathbb{R}},
\end{equation*}
given by 
\begin{equation*}
f(x)=\frac{1}{x}.
\end{equation*}
The graph of this function is rotated about the $x$-axis. 
\begin{parts}
\part[2] Find the surface area of the resulting surface by using the following formula:
\begin{equation*}
A=\int_{1}^{\infty} f(x)\sqrt{1+(f'(x))^2} \ dx
\end{equation*}
\newpage
\part[2] Find the volume of the solid enclosed by this surface by using the disk method. 
\end{parts}
%%%%%%%%%%%%%%%%%

\newpage
\addpoints
\question[3] The region bounded by the curves $x=0$, $y=x^2$,  and $y=1$ is rotated about the the $y$-axis. Find the volume of the resulting solid by using the cylindrical shells method.

%%%%%%%%%%%%%%%%%

\newpage
\addpoints
\question[3] Find the length of the curve given by the equation 
\begin{equation*}
9y^2=4(x+1)^3,
\end{equation*}
and the constraints $0 \leq x \leq 1$, $y \geq 0$. 

%%%%%%%%%%%%%%%%%
\newpage
\addpoints
\question Find the general solutions of the following differential equations
\begin{parts}
\part[2] 
\begin{equation*}
xy^2-y'x^2=0
\end{equation*}
\vfill 
\part[3]
\begin{equation*}
y'+3x^2y=e^{-x^3}
\end{equation*}
\vfill 
\newpage
\part[3] 
\begin{equation*}
y''+8y+16=0
\end{equation*}
\vfill
\part[2]
\begin{equation*}
y''+4y=0
\end{equation*}
\vfill
\end{parts}
%%%%%%%%%%%%%%%
\newpage
\addpoints
\question Determine whether the following series converges or diverges (clearly indicate the convergence test used). In case the series converges, you \textbf{do not} need to find its value. 
\begin{parts}
\part[2]
\begin{equation*}
\sum_{n=1}^{\infty} \frac{2^{2n}+(-1)^n)}{5^n}
\end{equation*}
\vfill 
\part[2]
\begin{equation*}
\sum_{n=1}^{\infty} \frac{n}{n^3+4n-3}
\end{equation*}
\vfill
\end{parts}

%%%%%%%%%%%%%%%

\newpage
\addpoints
\question[3] Write the following function as a power series in $x$, and determine its radius and interval of convergence. 
\begin{equation*}
\sum_{n=1}^{\infty} \frac{x}{x^2+9}
\end{equation*}

%%%%%%%%%%%%%%%
\newpage
\addpoints
\question[3] Use the Taylor series of the cosine to evaluate the following integral as a numerical series (that is, the end result should not contain references to the variable $x$). 
\begin{equation*}
\int_{0}^{1}x\cos(x)^3\ dx.
\end{equation*}

\end{questions}
\end{document}

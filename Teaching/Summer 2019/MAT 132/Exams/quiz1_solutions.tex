% Exam Template for UMTYMP and Math Department courses
%
% Using Philip Hirschhorn's exam.cls: http://www-math.mit.edu/~psh/#ExamCls
%
% run pdflatex on a finished exam at least three times to do the grading table on front page.
%
%%%%%%%%%%%%%%%%%%%%%%%%%%%%%%%%%%%%%%%%%%%%%%%%%%%%%%%%%%%%%%%%%%%%%%%%%%%%%%%%%%%%%%%%%%%%%%

% These lines can probably stay unchanged, although you can remove the last
% two packages if you're not making pictures with tikz.
\documentclass[11pt]{exam}
\RequirePackage{amssymb, amsfonts, amsmath, latexsym, verbatim, xspace, setspace}


% By default LaTeX uses large margins.  This doesn't work well on exams; problems
% end up in the "middle" of the page, reducing the amount of space for students
% to work on them.
\usepackage[margin=1in]{geometry}
\usepackage[english]{babel}
\usepackage[autostyle]{csquotes} %%%% This package allows Tex to recognize quotation marks with the \enquote command. 


% Here's where you edit the Class, Exam, Date, etc.
\newcommand{\class}{MAT 132}
\newcommand{\term}{Summer II 2019}
\newcommand{\examnum}{Quiz 1}
\newcommand{\examdate}{07/15/19}
\newcommand{\timelimit}{50 minutes}

% For an exam, single spacing is most appropriate
\singlespacing
% \onehalfspacing
% \doublespacing

% For an exam, we generally want to turn off paragraph indentation
\parindent 0ex
\title{MAT 132 - Summer II 2019: Quiz 1 solutions}
\begin{document} 

% These commands set up the running header on the top of the exam pages
\pagestyle{head}
\firstpageheader{}{}{}\textbf{}
\runningheader{\class}{\examnum\ - Page \thepage\ of \numpages}{\examdate}
\runningheadrule

\begin{flushright}
\begin{tabular}{p{2.8in} r l}
\textbf{\class} & \textbf{Name (Print):} & \makebox[2in]{\hrulefill}\\
\textbf{\term} &&\\
\textbf{\examnum} &&\\
\textbf{\examdate} &&\\
\textbf{Time Limit: \timelimit} & ID number & \makebox[2in]{\hrulefill}
\end{tabular}\\
\end{flushright}
\rule[1ex]{\textwidth}{.1pt}

\begin{center}
\large{\textbf{Instructions}}
\end{center}

\begin{minipage}[t]{3.7in}
\vspace{0pt}
\begin{itemize}

\item This exam contains \numpages\ pages (including this cover page) and
\numquestions\ problems.  Check to see if any pages are missing.  Enter
all requested information on the top of this page, and put your initials
on the top of every page, in case the pages become separated.

\item You may \textit{not} use your books, notes, or any device that is capable of accessing the internet on this exam (e.g., smartphones, smartwatches, tablets). You may not use a calculator.

\item \textbf{Organize your work}, in a reasonably neat and coherent way, in
the space provided. Work scattered all over the page without a clear ordering will 
receive very little credit.  

\item \textbf{Mysterious or unsupported answers will not receive full
credit}.

\end{itemize}

\end{minipage}
\hfill
\begin{minipage}[t]{2.3in}
\vspace{0pt}
%\cellwidth{3em}
\gradetablestretch{2}
\vqword{Problem}
\addpoints % required here by exam.cls, even though questions haven't started yet.	
\gradetable[v]%[pages]  % Use [pages] to have grading table by page instead of question

\end{minipage}
\newpage % End of cover page

%%%%%%%%%%%%%%%%%%%%%%%%%%%%%%%%%%%%%%%%%%%%%%%%%%%%%%%%%%%%%%%%%%%%%%%%%%%%%%%%%%%%%
%
% See http://www-math.mit.edu/~psh/#ExamCls for full documentation, but the questions
% below give an idea of how to write questions [with parts] and have the points
% tracked automatically on the cover page.
%
%
%%%%%%%%%%%%%%%%%%%%%%%%%%%%%%%%%%%%%%%%%%%%%%%%%%%%%%%%%%%%%%%%%%%%%%%%%%%%%%%%%%%%%

\begin{questions}

% Basic question
%%%%%%%%%%%%%%

%%%%%%%%%%%%%%%%% 

\addpoints
\question[2] Express the area under the graph of the function $f(x)=e^{3x}$, in the region $0 \leq x \leq 2$, as a limit of Riemann sums. Clearly indicate the choice of sampling points, and the width of the subintervals (that is, do not simply write $x_{i}^{*}$ and $\Delta x$, respectively). 

\textbf{Solution:} We assume a subdivision with $n$ rectangles, and will use right-endpoints as the sampling points. The range of integration has lentgh $2$, so the width of each rectangle is 
\begin{equation*}
\Delta x = \frac{2}{n}.
\end{equation*}
The sampling points are 
\begin{equation*}
x_i=\frac{2i}{n},
\end{equation*}
for $1 \leq i \leq n$. 

The area can be expressed as the following Riemann sum
\begin{equation*}
\lim_{n \to \infty} \sum_{i=1}^{n} f(x_i)\Delta x = \lim_{n \to \infty} \sum_{i=1}^{n} e^{\left(\frac{6i}{n}\right)} \frac{2}{n}.
\end{equation*}
\newpage

%%%%%%%%%%%%%%%%%
\addpoints
\question Compute the indefinite integrals below:
\begin{parts}
\part[3] 
\begin{equation*}
\int \sin(x)\cos(\cos(x))\ dx
\end{equation*}

\textbf{Solution:} Make the substitution 
\begin{align*}
u & = \cos(x),\\
du & = -\sin(x) dx.
\end{align*}
 
The integral can be computed in terms of the new variable as 
\begin{align*}
\int \sin(x)\cos(\cos(x))\ dx & = -\int \cos(u) \ du \\ 
& = -\sin(u) + C\\
& = -\sin(\cos(x))+C.
\end{align*}

\newpage

\part[4]
\begin{equation*}
\int \frac{x-3}{x(x^2+4)} \ dx
\end{equation*}

\textbf{Solution:} The partial fractions decomposition of the integrand is of the form 
\begin{equation*}
\frac{x-3}{x(x^2+4)} = \frac{A}{x} + \frac{Bx+C}{x^2+4}.
\end{equation*}
Comparing the fractions, we get the following system of equations
\begin{equation*}
\left\{
\begin{array}{rl}
A + B & = 0\\
C & = 1\\ 
4A & = -3,
\end{array}
\right.
\end{equation*}
whose solution is $A=-\frac{3}{4}$, $B=1$, $C=\frac{3}{4}$. 

It follows that 
\begin{align}
\nonumber
\int \frac{x-3}{x(x^2+4)} \ dx  & = -\left(\int \frac{3}{4x} \ dx\right) + \left( \int \frac{\frac{3x}{4}+1}{x^2+4} \ dx\right) \\
\nonumber
& = -\frac{3\ln(|x|)}{4} + \frac{1}{4}\int \frac{3x+4}{x^2+4} \ dx\\ 
\label{last step}
& = -\frac{3\ln(|x|)}{4} + \frac{3}{4} \left(\int \frac{x}{x^2+4} \ dx \right) + \left(\int \frac{1}{x^2+4}\ dx \right)
\end{align}
To solve the remaining integrals, we use the substitutions $u=x^2+4$, and $x=2\tan(\theta)$, respectively:
\begin{align*}
\int \frac{x}{x^2+4} \ dx & = \int \frac{1}{2u} \ du \\
& = \frac{\ln(|u|)}{2} \\ 
& = \frac{\ln(x^2+4)}{2};
\end{align*}
\begin{align*}
\int \frac{1}{x^2+4} \ dx & = \int \frac{2\sec^2(\theta)}{4\tan^2(\theta) + 4} \ d\theta\\ 
& = \int \frac{1}{2} \ d\theta \\
& = \frac{\theta}{2}\\
& = \frac{1}{2}\arctan(\frac{x}{2}).
\end{align*}
Combining it all into equation \eqref{last step}, we obtain 
\begin{equation*}
\int \frac{x-3}{x(x^2+4)} \ dx = -\frac{3\ln(|x|)}{4} + \frac{3\ln(x^2+4)}{8} + \frac{\arctan(\frac{x}{2})}{2} + C
\end{equation*}
\end{parts}
\newpage

%%%%%%%%%%%%%%%%%%%
\addpoints
\question Compute the definite integrals below: 
\begin{parts}
\part[3] 
\begin{equation*}
\int_{1}^{2} \frac{\ln(x)}{x^2} \ dx 
\end{equation*}

\textbf{Solution:}
By integration by parts, an antiderivative of the integrand is given by 
\begin{align*}
\int \frac{\ln(x)}{x^2} \ dx & = -\frac{\ln(x)}{x} + \int \frac{1}{x^2} \ dx \\
& = \frac{\ln(x)}{x}-\frac{1}{x}.
\end{align*}
Applying the Net Change Theorem we get, 
\begin{align*}
\int_{1}^{2} \int \frac{\ln(x)}{x^2} \ dx & = -\left(\frac{\ln(x)}{x}-\frac{1}{x}\right)\Big|_{1}^{2} \\
& = -\frac{\ln(2)}{2}+\frac{1}{2}.
\end{align*}
\newpage
\part[4] 
\begin{equation*}
\int_{0}^{\frac{\pi}{4}} (\sin(x))^3(\cos(x))^2\  dx 
\end{equation*}

\textbf{Solution:}
In order to simplify the integrand, we use the identity
\begin{equation*}
\sin^2(x)=1-\cos^2(x).
\end{equation*}
An antiderivative can be found by the substitution $u=\cos(x)$, 
\begin{align*}
\int (\sin(x))^3(\cos(x))^2\  dx  & = \int \sin(x) (1-\cos^2(x))\cos^2(x) \ dx \\
& = -\int (1-u^2)u^2 \ du \\
& = \int u^4-u^2 \ du \\
& = \frac{u^5}{5} -\frac{u^3}{3}\\
& =\frac{\cos^5(x)}{5} - \frac{\cos^3(x)}{3}.
\end{align*}
Using the Net Change Theorem, we get 
\begin{equation*}
\int_{0}^{\frac{\pi}{4}} (\sin(x))^3(\cos(x))^2\  dx = \left(\frac{\sqrt{2}}{40}-\frac{\sqrt{2}}{12}\right)-\left(\frac{1}{5}-\frac{1}{3}\right).
\end{equation*}

\end{parts}
%%%%%%%%%5
\newpage
\addpoints

\question[4] Find the derivative of the function $f(x)$, defined by the integral below
\begin{equation*}
f(x)=\int_{0}^{x^2+2x+1} \sqrt{1+3s} \ ds
\end{equation*}

\textbf{Solution:}
We use the chain rule and the fundamental theorem of Calculus, 
\begin{equation*}
f'(x)=\sqrt{1+3(x^2+2x+1)}(2x+2).
\end{equation*}
\addpoints
\end{questions}
\end{document}

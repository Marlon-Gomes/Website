\documentclass[12pt,oneside]{exam}

% This package simply sets the margins to be 1 inch.
\usepackage[margin=1in]{geometry}

% These packages include nice commands from AMS-LaTeX
\usepackage{amssymb,amsmath,amsthm,amsfonts,latexsym,verbatim,xspace,setspace}
\usepackage{tikz}

% Make the space between lines slightly more
% generous than normal single spacing, but compensate
% so that the spacing between rows of matrices still
% looks normal.  Note that 1.1=1/.9090909...
\renewcommand{\baselinestretch}{1.1}
\renewcommand{\arraystretch}{.91}

% Define an environment for exercises.
\newenvironment{exercise}[1]{\vspace{.1in}\noindent\textbf{Exercise #1 \hspace{.05em}}}{}
\newenvironment{newsolution}{\vspace{.1in}\noindent\textbf{Solution: \hspace{.05em}}}{}

% define shortcut commands for commonly used symbols
\newcommand{\R}{\mathbb{R}}
\newcommand{\C}{\mathbb{C}}
\newcommand{\Z}{\mathbb{Z}}
\newcommand{\Q}{\mathbb{Q}}
\newcommand{\N}{\mathbb{N}}
\newcommand{\calP}{\mathcal{P}}

\DeclareMathOperator{\vsspan}{span}

\title{Math 511 - Summer I 2019: Solutions to Homework 3}

%%%%%%%%%%%%%%%%%%%%%%%%%%%%%%%%%%%%%%%%%%

\begin{document}

\begin{flushright}
\sc MAT 511 - Summer 1\\
June 20, 2019
\end{flushright}
\bigskip

\begin{center}
\textsf{Homework 3: solutions to selected problems.} 
\end{center}

%%%%%%%%%%%%%%%%%%%%%%%%%%%%%%%%%%%%%%%%

\begin{exercise}{1}
Let $A=\{1,2,3\}$. Construct a relation on $A \times A$ satisfying the following properties:
\begin{parts}
\part It is not reflexive, not symmetric, and not transitive.
\part It is reflexive, not symmetric, and not transitive.
\part It is not reflexive, symmetric, and not transitive.
\part It is reflexive, symmetric, and not transitive.
\part It is not reflexive, not symmetric, and transitive.
\part It is reflexive, not symmetric, and transitive.
\part It is not reflexive, symmetric, and transitive.
\part It is reflexive, symmetric, and transitive.
\end{parts}
\end{exercise}

\begin{newsolution}
Many of these problems have multiple correct solutions. In such cases, what is presented below is just one example. 
\begin{parts}
\part $R=\{(1,3),(2,1)\}$.
\part $R=\{(1,1), (2,1), (2,2), (3,2), (3,3)\}$.
\part $R=\{(1,2), (2,1)\}$.
\part $R=\{(1,1), (1,2), (1,3), (2,1), (2,2), (3,1), (3,3)\}$.
\part $R=\{(1,2), (1,3), (2,3)\}$.
\part $R=\{(1,1), (1,2), (1,3), (2,2), (2,3), (3,3)\}$.
\part $R=\{(1,2),(2,1), (2,2)\}$.
\part $R=\{(1,1),(2,2),(3,3)\}$.
\end{parts}
\end{newsolution}

\begin{exercise}{2}
Define the following relation on $\mathbb{R} \times \mathbb{R}$: a point $(a,b)$ is related to $(x,y)$ if 
\begin{equation*}
y-b=x-a
\end{equation*}
Show that this relation is an equivalence, i.e., it is reflexive, symmetric, and transitive. What are the equivalence classes?
\end{exercise}

\begin{newsolution}
To show that the relation is reflexive is to show that every element of $\mathbb{R} \times \mathbb{R}$ is related to itself. Consider such an element, say $(a,b) \in \mathbb{R} \times \mathbb{R}$. Its coordinates satisfy the defining equation of the relation:
\begin{equation*}
b-b=a-a,
\end{equation*}
hence $(a,b)R(a,b)$, for all $(a,b) \in \mathbb{R}\times \mathbb{R}$, i.e., the relation is reflexive. 

To show symmetry, we consider a pair of elements in $\mathbb{R}\times \mathbb{R}$, say $(a,b)$ and $(c,d)$, such that $(a,b)R(c,d)$, i.e., 
\begin{equation*}
d-b=c-a.
\end{equation*}
Multipying both sides of the equation by $(-1)$ yields
\begin{equation*}
b-d=a-c,
\end{equation*}
which, according to the definition of $R$, means $(c,d) R (a,b)$, i.e., the relation is symmetric. 

Finally, we consider the question of transitivity. Let $(a,b), (c,d), (e,f)$ be points in $\mathbb{R} \times \mathbb{R}$, satisfying 
\begin{align*} 
(a,b) & R (c,d),\\
(c,d) & R (e,f).
\end{align*}
Then we have 
\begin{align*}
d-b & = c- a\\
f-d & = e-c.
\end{align*}
Combining the two equations, we obtain 
\begin{equation*}
f-b=e-a,
\end{equation*}
that is, $(a,b)R(e,f)$, so the relation is transitive. 

One recognizes the condition defining this relation as saying that the points $(a,b)$ and $(x,y)$ are related if they lie on a line with slope 1, so the equivalence classes on the plane are all the lines of slope 1. 
\end{newsolution}

\begin{exercise}{3}
Let $A = \mathbb{Z} \times (\mathbb{Z}-\{0\})$. This is the set of pairs of integers, in which the second entry is non-zero. On this set, we consider the following relation, 
\begin{equation*}
\mathbb{Q}=\{ ((a,b), (c,d)) \in A \times A| \ ad=bc\}.
\end{equation*}
Show that this relation is an equivalence relation, that is:
\begin{parts}
\part it is reflexive: $(a,b) \in A$ is related to itself.
\part it is symmetric: if $((a,b),(c,d)) \in Q$, then $((c,d), (a,b)) \in Q$. 
\part it is transitive: given $((a,b),(c,d)) \in Q$, and $((c,d),(e,f)) \in Q$, then $((a,b),(e,f))$. 
\end{parts}
Furthermore, describe the equivalence classes of this relation. 
\end{exercise}


\begin{newsolution}
\begin{parts}
\part This is clear from the defining equation: $ab=ab$. 
\part This follows from commutativity, and symmetry of the defining equation, if $ad=bc$, then $da=cb$, or equivalently $cb=da$. The latter means, by the definition of the relation $\mathbb{Q}$, that $(c,d)\mathbb{Q}(a,b)$.
\part Suppose that $(a,b)\mathbb{Q}(c,d)$, and $(c,d)\mathbb{Q}(e,f)$, that is, 
\begin{align*}
ad & = bc \\
cf & = de.
\end{align*}
Then, multiplying the first equation by $f$ and using the second equation, we have 
\begin{align*}
adf & = bcf  \Rightarrow\\
adf & = bde.
\end{align*}
Finally, we can divide the latter by $d$ (since $d$ is non-zero), to obtain 
\begin{equation*}
af=be,
\end{equation*}
which, according to the definition of the relation, means $(a,b)\mathbb{Q}(e,f)$.
\end{parts}
As the notation suggests, this relation defines the rational numbers. A pair $(a,b)$ is to be though of as the fraction $a/b$. The above relation can be reinterpreted in this language to mean 
\begin{equation*}
(a,b)\mathbb{Q}(c,d) \Leftrightarrow \ \frac{a}{b}=\frac{c}{d},
\end{equation*}
that is, the relation defines equivalent fractions by simplification of common factors. The set of equivalence classes can be ideitified with the set of irreducible fractions $a/b$, in which $b \in \mathbb{N}\setminus\{0\}$ (the last condition is necessary to rule out having double representation, such as $\frac{-2}{-1}=\frac{2}{1}$).
\end{newsolution}

\begin{exercise}{4}
In each of the problems below, you are given a set $A$ and a collection of subsets. Determine if this collection is a partition. Explain your reasoning. 
\begin{parts}
\part $A =\mathbb{N}$, $\mathcal{P}=\{ \{0\}, \{n \in \mathbb{N}| \ n \ \mbox{is even}\}, \{n \in \mathbb{N}| \ n \ \mbox{is a prime number}\}\}$.
\part $A= \mathbb{N}$, $\mathcal{P}=\{ \{0,1\}, \{n\in \mathbb{N}| \ n \ \mbox{has a prime factor} \}\}$.
\end{parts}
\end{exercise}

\begin{newsolution}
\begin{parts}
\part The collection $\mathcal{P}$ does not form a partition of $\mathbb{N}$, for a couple of reasons: sets within $\mathcal{P}$ intersect ($2$ is both even and prime); not all elements in $\mathbb{N}$ belong to one of the subsets of the partition (for instance, $9$ is an odd, non-prime number).
\part This collection is a partition. The sets on subsets on the partition do not intersect (recall that we made a convention that $0$ has no prime factors). Furthermore, if $n$ is any natural number other than $0$ and $1$, then it has prime factors. 
\end{parts}
\end{newsolution}

\begin{exercise}{5}
Let $A=\{a,b,c\}$. Give an example of a relation on $A$ that is 
\begin{parts}
\part antisymmetric and symmetric.
\part antisymmetric, reflexive, and not symmetric.
\part antisymmetric, not reflexive, and not symmetric. 
\part symmetric and not antisymmetric. 
\part not symmetric and not antisymmetric.
\part irreflexive and not symmetric.
\part irreflexive and not antisymmetric.
\part antisymmetric, not reflexive, and not irreflexive.
\part transitive, antisymmetric, and irreflexive.
\end{parts}
\end{exercise}


\begin{newsolution}
Many of these problems have multiple correct solutions. In such cases, what is presented below is just one example.
\begin{parts}
\part $R=\{(a,a)\}$.
\part $R=\{(a,a), (b,a), (b,b), (c,c)\}$.
\part $R=\{(a,b)\}$.
\part $R=\{(a,b), (b,a)\}$.
\part $R=\{(a,b), (b,a), (b,c)\}$.
\part $R=\{(b,a)\}$.
\part $R = \{(a,b), (b,a)\}$.
\part $R=\{(a,b), (b,b)\}$.
\part $R=\{(b,a), (c,a), (c,b)\}$.
\end{parts}
\end{newsolution}

\begin{exercise}{6}
Define the relation $R$ on $\mathbb{N}$ by the following property: $a$ is related to $b$ by $R$ if there exists a non-negative integer $k$ such that $b=2^ka$. Show that this is a partial ordering. Does this relation have the comparability property?
\end{exercise}

\begin{newsolution}
We have to show that this relation is reflexive, antisymmetric, and transitive. 

First, consider the question of reflexivity. If $a \in \mathbb{N}$, then $a=2^0a$, so $aRa$. 

Second, consider a related pair $aRb$. We will show that $b$ is not related to $a$, unless $b=a$. Our hypothesis is that there exists $k \in \mathbb{N}$ for which 
\begin{equation*}
b=2^ka.
\end{equation*}
If we assume, that there exists a second natural number $l$ for which $a=2^lb$, then we have 
\begin{equation*}
b=2^{k+l}b,
\end{equation*}
from which we infer $2^{k+l}=1$, i.e., $k+l=0$. The only pair of natural numbers satisfying this equation is $k=l=0$, so $b=a$. It follows that the relation is antisymmetric. 

Finally, we consider the question of transitivity. Let $a,b,c$ be three natural numbers satisfying $aRb$, $bRc$, that is
\begin{align*}
b& =2^ka\\
c & =2^l b,
\end{align*}
for appropriate choices of natural numbers $k$ and $l$. It follows that
\begin{equation*}
c=2^{k+l}a,
\end{equation*}
form which we infer $aRc$. Thus, the relation is transitive. 

Satisfying all the desired properties, $R$ is a partial order on the set of natural numbers. It does not, however, have the comparability property, as, for instance, the numbers $3$ and $5$ are not related by this order. 
\end{newsolution}

\begin{exercise}{7}
Consider a partially ordered set $A$, with order relation $R$. Assume that $C \subset B \subset A$. Determine whether the statements below are true or false.
\begin{parts}
\part Every upper bound for $C$ is an upper bound for $B$.
\part Every upper bound for $B$ is an upper bound for $C$.
\end{parts}
\end{exercise}

\begin{newsolution}
\begin{parts}
\part This is false, in general. For instance, consider the usual order on the set $A = \mathbb{N}$, and the subsets $C=\{0\}$, $B=\{0,1,2\}$. Then $1$ is an upper bound for $C$, but not an upper bound for $B$. 
\part This is true. Consider an upper bound $a \in A$ for $B$. This means that for every $b \in B$, $bRa$. In particular, if $c \in C$, then $c \in B$, by inclusion, hence we have $cRa$. This is true for all elements of $C$, thus $a$ is an upper bound for $C$. 
\end{parts}
\end{newsolution}

\begin{exercise}{8}
Consider an order relation on the set $A=\{a,b,c,d,e,f,g,h\}$, given by the following properties: 
\begin{itemize}
\item $g \leq  f$
\item $h \leq f$
\item $h \leq d$
\item $f \leq c$
\item $f \leq b$
\item $e \leq b$
\item $e \leq a$.
\end{itemize}

Construct its Hasse diagram. In addition, find the following bounds:
\begin{parts}
\part all upper bounds for the set $\{b,f,g,h\}$.
\part all lower bounds for the set $\{a,d\}$.
\part the supremum, if it exists, for the set $\{e,g,h\}$.
\part the infimum, if it exists, for the set $\{b,f,g\}$.
\part the smallest element, if it exists, for the set $\{b,c,d\}$.
\end{parts}
\end{exercise}

\begin{newsolution}
The Hasse Diagram for this relation is presented below.

\begin{center}
\begin{tikzpicture}
\node (a) at (0,0) {a};
\node (e) at (1,-1) {e};
\node (b) at (2,0) {b};
\node (g) at (2,-2) {g};
\node (f) at (3,-1) {f};
\node (c) at (4,0) {c};
\node (h) at (4,-2) {h};
\node (d) at (5,-1) {d};
\draw (a) -- (e);
\draw (e) -- (b);
\draw (b) -- (f);
\draw (f) --(c);
\draw (f) -- (g);
\draw (f) -- (h);
\draw (h) -- (d);
\end{tikzpicture}
\end{center}

From this pictorial representation, the desired bounds are easy to find:
\begin{parts}
\part The only upper bound for this set is $b$. 
\part There is no lower bound for this set, as there exists no element which is less than $a$ and $d$ simultaneously. 
\part The set of upper bounds for $\{e,g,h\}$ consists of a single element, $b$, which is therefore its supremum. 
\part The set of lower bounds for $\{b,f,g\}$ consists of a single element, $g$, which is therefore its infimum.
\part The set $\{b,c,d\}$ has no smallest element, as one cannot compare $d$ to the other elements. In other terms, the set $\{b,c,d\}$ has an infimum, $h$, which does not belong to it, so $\{b,c,d\}$ does not have a smallest element. 
\end{parts}
\end{newsolution}
\end{document}
\documentclass[12pt,oneside]{exam}

% This package simply sets the margins to be 1 inch.
\usepackage[margin=1in]{geometry}

% These packages include nice commands from AMS-LaTeX
\usepackage{amssymb,amsmath,amsthm,amsfonts,latexsym,verbatim,xspace,setspace}

% Make the space between lines slightly more
% generous than normal single spacing, but compensate
% so that the spacing between rows of matrices still
% looks normal.  Note that 1.1=1/.9090909...
\renewcommand{\baselinestretch}{1.1}
\renewcommand{\arraystretch}{.91}

% Define an environment for exercises.
\newenvironment{exercise}[1]{\vspace{.1in}\noindent\textbf{Exercise #1 \hspace{.05em}}}{}
\newenvironment{newsolution}{\vspace{.1in}\noindent\textbf{Solution: \hspace{.05em}}}{}

% define shortcut commands for commonly used symbols
\newcommand{\R}{\mathbb{R}}
\newcommand{\C}{\mathbb{C}}
\newcommand{\Z}{\mathbb{Z}}
\newcommand{\Q}{\mathbb{Q}}
\newcommand{\N}{\mathbb{N}}
\newcommand{\calP}{\mathcal{P}}

\DeclareMathOperator{\vsspan}{span}

\title{Math 511 - Summer I 2019: Solutions to Homework 2}

%%%%%%%%%%%%%%%%%%%%%%%%%%%%%%%%%%%%%%%%%%

\begin{document}

\begin{flushright}
\sc MAT 511 - Summer 1\\
June 4, 2019
\end{flushright}
\bigskip

\begin{center}
\textsf{Homework 2: solutions to selected problems.} 
\end{center}

%%%%%%%%%%%%%%%%%%%%%%%%%%%%%%%%%%%%%%%%

\begin{exercise}{1}
Prove the uniqueness of the empty set, that is, if $A$ and $B$ are empty sets, show that $A = B$. 
\end{exercise}

\begin{newsolution}
Solved in class.
\end{newsolution}

\begin{exercise}{2}
The following claim is true, but the argument presented as its proof contains a mistake. Find the mistake, and fix it. 

\textbf{Claim}: If $A$ is a set, then $A \in \mathcal{P}(A)$. 

\begin{proof}
Suppose $x \in A$. Then $\{x\} \subset A$. Thus $\{x\} \in \mathcal{P}(A)$. Therefore, $A \in \mathcal{P}(A)$.
\end{proof}
\end{exercise}

\begin{newsolution}
Solved in class.
\end{newsolution}

\begin{exercise}{3}
Let $A$ and $B$ be sets. Prove that $A = B$ if and only if $\mathcal{P}(A)=\mathcal{P}(B)$.
\end{exercise}

\begin{exercise}{4}
Let $A$, $B$, $C$ and $D$ be sets. Prove that
\begin{parts}
\part $A \subset B$ if and only if $A-B = \emptyset$.
\part If $A \subset B \cup C$ and $A\cap B = \emptyset$, then $A \subset C$. 
\part If $A \subset C$ and $B \subset C$, then $A \cup B \subset C$. 
\part If $C \subset A$ and $D \subset B$, then $C\cap D \subset A \cap B$. 
\part If $A \cup B \subset C \cup D$, $A\cap B = \emptyset$, and $C \subset A$, then $B \subset D$. 
\end{parts}
\end{exercise}

\begin{newsolution}
In each part, we may take as the Universe of Discourse the union of the sets $A$, $B$, $C$ and $D$, according to their appearance in the corresponding statement.
\begin{parts}
\part Universe: $A \cup B$. 

Consider the predicates
\begin{itemize}
\item $P$:``it belongs to $A$";
\item  $Q$: ``it belongs to $B$".
\end{itemize}
In terms of these predicates, what we wish to prove is that 
\begin{equation*}
[\forall x \ (P(x) \Rightarrow Q(x))] \Leftrightarrow [\neg \exists x \ (P(x) \land \neg Q(x))]
\end{equation*}
By Material Implication, 
\begin{equation*}
P(x) \Rightarrow Q(x) \Leftrightarrow \neg P(x) \lor Q(x).
\end{equation*} 
The negation of $[\forall x \ (P(x) \Rightarrow Q(x))]$ can thus be written as 
\begin{align*}
[\exists x \ \neg (P(x) \Rightarrow Q(x))] & \Leftrightarrow [\exists x \ \neg( \neg P(x) \lor Q(x))]\\
& \Leftrightarrow [\exists x \ (P(x) \land \neg Q(x))],
\end{align*}
thus the desired equivalence follows from the double negation law, 
\begin{align*}
[\forall x \ (P(x) \Rightarrow Q(x))] & \Leftrightarrow \neg [\neg [\forall x \ (P(x) \Rightarrow Q(x))]]\\ 
& \Leftrightarrow \neg [\exists x \ \neg (P(x) \Rightarrow Q(x))] \\
& \Leftrightarrow \neg [\exists x \ (P(x) \land \neg Q(x))].
\end{align*}
\part Universe: $A \cup B \cup C$.

Consider the predicates
\begin{itemize}
\item $P$:``it belongs to $A$";
\item  $Q$: ``it belongs to $B$";
\item $R$: ``it belongs to $C$.
\end{itemize}
In terms of these predicates, the statement we seek to prove is that if
\begin{equation*}
[\forall x \ (P(x) \Rightarrow Q(x) \lor R(x))] \land [\forall x \ (P(x) \Rightarrow \neg Q(x)) \land (Q(x) \Rightarrow \neg P(x))],
\end{equation*}
then 
\begin{equation*}
\forall x \ (P(x) \Rightarrow R(x)).
\end{equation*}
In doing so, the first step is to simplify the premise, extracting the quantifier
\begin{equation*}
\forall x \ [(P(x) \Rightarrow Q(x) \lor R(x)) \land (P(x) \Rightarrow \neg Q(x)) \land (Q(x) \Rightarrow \neg P(x))].
\end{equation*}
From 
\begin{equation*}
(P(x) \Rightarrow Q(x) \lor R(x)) \land (P(x) \Rightarrow \neg Q(x)),
\end{equation*}
we may infer 
\begin{equation*}
(P(x) \Rightarrow R(x)),
\end{equation*}
thus
\begin{equation*}
\forall x \ [(P(x) \Rightarrow R(x)) \land (Q(x) \Rightarrow \neg P(x))],
\end{equation*}
from which we infer the statement on the left by simplification, 
\begin{equation*}
\forall x (P(x) \Rightarrow R(x)),
\end{equation*}
as desired.
\part Universe: $A \cup B \cup C$.

Consider the predicates
\begin{itemize}
\item $P$:``it belongs to $A$";
\item  $Q$: ``it belongs to $B$";
\item $R$: ``it belongs to $C$.
\end{itemize}
In terms of these predicates, the statement we wish to prove is: if 
\begin{equation*}
\forall x \ [(P(x) \Rightarrow R(x))\land (Q(x)\Rightarrow R(x))],
\end{equation*}
then 
\begin{equation*}
\forall x \ [(P(x) \lor Q(x)) \Rightarrow  R(x)].
\end{equation*}
This is easily achived by using material implication (applied three times), distributivity, and DeMorgan's laws:
\begin{align*}
\forall x \ [(P(x) \Rightarrow R(x))\land (Q(x)\Rightarrow R(x)) & \Leftrightarrow \forall x \ [(\neg(x) \lor R(x)) \land (\neg Q(x) \lor R(x))],\\
& \Leftrightarrow \forall x \ [(\neg P(x)) \land (\neg Q(x))]\lor R(x)\\
& \Leftrightarrow \forall x \ [\neg (P(x) \lor Q(x))] \lor R(x)\\
& \Leftrightarrow \forall x \ (P(x) \lor Q(x)) \Rightarrow R(x).
\end{align*}
\part Suppose $x \in C \cap D$. Then, in particular, $x \in C$, hence $x \in A$, by inclusion. Likewise, $x \in D$, thus $x \in B$, by inclusion. It follows that $x \in A \cap B$. This shows that $C \cap D \subset A \cap B$. 
\part Solved in class.
\end{parts}
\end{newsolution}

\begin{exercise}{5}
Let $A$ and $B$ be sets. Prove that 
\begin{equation*}
\mathcal{P}(A) \cup \mathcal{P}(B) \subset \mathcal{P}(A \cup B).
\end{equation*}
Show, by means of an example, that the equality 
\begin{equation*}
\mathcal{P}(A) \cup \mathcal{P}(B) = \mathcal{P}(A \cup B)
\end{equation*}
need not be true. 
\end{exercise}

\begin{newsolution}
Solved in class.
\end{newsolution}

\begin{exercise}{6}
Let $A, B, C$ and $D$ be sets. 
\begin{parts}
\part Prove that $(A \times B) \cap (C \times D) = (A\cap C) \times (B \cap D)$.
\part Find an example that show thats the equality 
\begin{equation*}
(A\times B) \cup (C \times D) = (A\cup C) \times (B \cup D)
\end{equation*}
is, in general, false.   
\end{parts}
\end{exercise}

\begin{newsolution}
Solved in office hours.
\end{newsolution}

\begin{exercise}{7}
Use Mathematical Induction to verify the following statements:
\begin{parts}
\part For every $n \in \mathbb{N}$, the number 
\begin{equation*}
\frac{n(n+1)}{2}
\end{equation*}
is an integer.
\part For every $n \in \mathbb{N}$, the number 
\begin{equation*}
n(n+1)(2n+1)
\end{equation*}
is divisible by $6$.
\part For all $n \in \mathbb{N}$, the sum of the interior angles of a convex polygon with $(n+2)$-sides is $180 \cdot n$ degrees.
\part For all $n \in \mathbb{N}$, given $n$ points in a plane, no three of which are collinear, there are exactly 
\begin{equation*}
\frac{n^2-n}{2}
\end{equation*}
line segments joining pairs all pairs of points. 
\end{parts}
\end{exercise}

\begin{newsolution}
\begin{parts}
\part Solved in class.
\part Solved in class.
\part As remarked in class, there was a typo in this problem: the polygon is meant to have $(n+3)$ sides, and the sum of angles is correspondently $180\cdot(n+1)$ degrees. This is for compatibility with our assumption that $0 \in P$. 

Consider the following subset of $\mathbb{N}$, 
\begin{equation*}
P = \{ n \in \mathbb{N} | \ \mbox{the sum of angles of any convex polygon with} \ (n+3) \ \mbox{sides is} \ 180\cdot (n+1)\}.
\end{equation*} 
When $n=0$, this is the statement that the sum of angles of any triangle is $180$ degrees, which we know to be true, from Euclidean Geometry. This is a fundamental fact, which we will use during the inductive step. 

Next, we assume that $n \in P$, and consider the question of whether $(n+1) \in P$. Let $A_0A_1A_2A_3\cdots A_{n+2}A_{n+3}$ be a convex $(n+4$)sided polygon, with $A_i$ being the vertices. Consider the line segment $\overline{A_1A_{n+3}}$. It splits the polygon into two others, a triangle: $A_{n+3}A_0A_1$, and a polygon with $(n+3)$-sides, 
$A_1A_2\cdots A_{n+2}A_{n+3}$. By the induction hypothesis, the sum of angles in the latter is $180(n+1)$, while the sum of angles in the former is $180$. Overall, the sum of angles in the original polygon is 
\begin{equation*}
180(n+1)+180=180(n+2). 
\end{equation*}
Since this procedure can be carried out for any $(n+4)$-sided polygon, we conclude that $(n+1)\in P$, as we wanted to show.
\part Solved in class.
\end{parts}
\end{newsolution}

\begin{exercise}{8}
The Fibonacci numbers are recursively defined by the relations
\begin{align*}
f_1 & =1,\\
f_2 & = 1,\\
f_{n+2} &=f_{n+1}+f_{n}.
\end{align*}
In this problem, you are required to use Induction (or its variants) to show the following:
\begin{parts}
\part Two consecutive terms of this sequence have no common divisors, other than $\pm 1$.
\part $f_{3n}$ is always even. 
\part $f_{4n}$ is divisible by $3$, for all $n \in \mathbb{N}$.
\end{parts}
\end{exercise}

\begin{newsolution}
First, a clarification: the rules above imply that $f_0=0$. 
\begin{parts}
\part Consider the following subset of $\mathbb{N}$:
\begin{equation*}
P=\{n \in \mathbb{N} |\  f_n \ \mbox{and} \ f_{n+1} \ \mbox{have no common divisors, other than} \pm 1\}.
\end{equation*}
We observe that $0 \in P$, as $f_0=0$ and $f_1=1$ have no common divisors, other than $\pm 1$. 

Assume that $n \in P$, that is, $f_n$ and $f_{n+1}$ have no common divisors. Then, we consider the question of whether $(n+1) \in P$. We will prove this is the case by means of contradiction. Suppose that the induction hypothesis holds, but $(n+1) \notin P$, that is, suppose that $f_{n+1}$ and $f_{n+2}$ have a common divisor, say $k \in \mathbb{Z}$, $k \neq \pm 1$. Then, 
\begin{align*}
f_{n+1} & = ka, \ \mbox{and},\\ 
f_{n+2} &=  kb, 
\end{align*}
for certain integers $a$, $b$. It follows from the recursion relating $f_{n+2}$, $f_{n+1}$ and $f_n$, that
\begin{equation*}
f_n=f_{n+2}-f_{n+1} = kb-ka=k(b-a),
\end{equation*}
that is, $k$ also divides $f_n$. This contradicts our induction hypothesis, so it must be the case that $(n+1) \in P$.
\part Consider the following subset of $\mathbb{N}$:
\begin{equation*}
Q = \{n \in \mathbb{N} | f_{3n} \ \mbox{is even}\}.
\end{equation*}
As usual, we begin by verifying that $0$ belongs to $Q$, for $f_0=0$ is even. 

Next we assume that $f_{3n}$ is even. Then, we use the recursion defining the sequence to relate $f_{3(n+1)}=f_{3n+3}$ and $f_{3n}$:
\begin{align*}
f_{3n+3} & = f_{3n+2} + f_{3n+1} \\
& = (f_{3n+1}+f_{3n})+f_{3n+1}\\
& = 2f_{3n+1} +f_{3n}.
\end{align*}
Since $f_{3n}$ is even, and $2f_{3n+1}$ is even, their sum, $f_{3n+3}$, is also even, as we wanted to prove.
\part Consider the following subset of $\mathbb{N}$:
\begin{equation*}
R=\{n \in \mathbb{N} | \ f_{4n} \ \mbox{is divisible by} \ 3\}.
\end{equation*}
It is easy to verify that $0 \in R$, as $f_0=0$ is divisible by $3$. 

Assume that $n \in R$, that is, $f_{4n}$ is divisible by $3$. Again, we use the recursion to relate $f_{4(n+1)}=f_{4n+4}$ and $f_n$, 
\begin{align*}
f_{4n+4} & = f_{4n+3} + f_{4n+2} \\
& = (f_{4n+2}+f_{4n+1}) + (f_{4n+1}+f_{4n}) \\
& =f_{4n+2} +2f_{4n+1} + f_{4n}\\
& = (f_{4n+1}+f_{4n}) + 2f_{4n+1}+f_{4n}\\
& = 3f_{4n+1}+2f_{4n}.
\end{align*}
Since $f_{4n}$ is divisible by $3$, so is $2f_{4n}$. Clearly, $3f_{4n+1}$ is divisible by $3$, so the sum 
\begin{equation*}
f_{4n+4} = 3f_{4n+1}+2f_{4n}
\end{equation*}
is divisible by $3$, as we wanted to show. 
\end{parts}
\end{newsolution}

\begin{exercise}{9}
In a certain kind of tournament, every player plays every other player exactly once, and either wins or losess. There are no ties. Define a top player to be a player who, for every other player $x$, either beats $x$ or beats a player $y$ who beats $x$. 
\begin{parts}
\part Show, by means of an example, that there can be more than one top player.
\part Use Induction to show that every such tournament with $n$ players has a top player. 
\part Use the Well-Ordering Principle to show that every such tournament with $n$ players has a top player.
\end{parts}
\end{exercise}

\begin{newsolution}
Solved in class.
\end{newsolution}
\end{document}
\documentclass[12pt,oneside]{exam}

% This package simply sets the margins to be 1 inch.
\usepackage[margin=1in]{geometry}

% These packages include nice commands from AMS-LaTeX
\usepackage{amssymb,amsmath,amsthm,amsfonts,latexsym,verbatim,xspace,setspace}

% Make the space between lines slightly more
% generous than normal single spacing, but compensate
% so that the spacing between rows of matrices still
% looks normal.  Note that 1.1=1/.9090909...
\renewcommand{\baselinestretch}{1.1}
\renewcommand{\arraystretch}{.91}

% Define an environment for exercises.
\newenvironment{exercise}[1]{\vspace{.1in}\noindent\textbf{Exercise #1 \hspace{.05em}}}{}

% define shortcut commands for commonly used symbols
\newcommand{\R}{\mathbb{R}}
\newcommand{\C}{\mathbb{C}}
\newcommand{\Z}{\mathbb{Z}}
\newcommand{\Q}{\mathbb{Q}}
\newcommand{\N}{\mathbb{N}}
\newcommand{\calP}{\mathcal{P}}

\DeclareMathOperator{\vsspan}{span}

\title{Math 511 - Summer I 2019: Homework 2}

%%%%%%%%%%%%%%%%%%%%%%%%%%%%%%%%%%%%%%%%%%

\begin{document}

\begin{flushright}
\sc MAT 511 - Summer 1\\
June 4, 2019
\end{flushright}
\bigskip

This homework is due on Tuesday, June 11, by 7:00 pm. 
\begin{center}
\textsf{Homework 2} 
\end{center}

%%%%%%%%%%%%%%%%%%%%%%%%%%%%%%%%%%%%%%%%

\begin{exercise}{1}
Prove the uniqueness of the empty set, that is, if $A$ and $B$ are empty sets, show that $A = B$. 
\end{exercise}

\begin{exercise}{2}
The following claim is true, but the argument presented as its proof contains a mistake. Find the mistake, and fix it. 

\textbf{Claim}: If $A$ is a set, then $A \in \mathcal{P}(A)$. 

\begin{proof}
Suppose $x \in A$. Then $\{x\} \subset A$. Thus $\{x\} \in \mathcal{P}(A)$. Therefore, $A \in \mathcal{P}(A)$.
\end{proof}
\end{exercise}

\begin{exercise}{3}
Let $A$ and $B$ be sets. Prove that $A = B$ if and only if $\mathcal{P}(A)=\mathcal{P}(B)$.
\end{exercise}

\begin{exercise}{4}
Let $A$, $B$, $C$ and $D$ be sets. Prove that
\begin{parts}
\part $A \subset B$ if and only if $A-B = \emptyset$.
\part If $A \subset B \cup C$ and $A\cap B = \emptyset$, then $A \subset C$. 
\part If $A \subset C$ and $B \subset C$, then $A \cup B \subset C$. 
\part If $C \subset A$ and $D \subset B$, then $C\cap D \subset A \cap B$. 
\part If $A \cup B \subset C \cup D$, $A\cap B = \emptyset$, and $C \subset A$, then $B \subset D$. 
\end{parts}
\end{exercise}

\begin{exercise}{5}
Let $A$ and $B$ be sets. Prove that 
\begin{equation*}
\mathcal{P}(A) \cup \mathcal{P}(B) \subset \mathcal{P}(A \cup B).
\end{equation*}
Show, by means of an example, that the equality 
\begin{equation*}
\mathcal{P}(A) \cup \mathcal{P}(B) = \mathcal{P}(A \cup B)
\end{equation*}
need not be true. 
\end{exercise}

\begin{exercise}{6}
Let $A, B, C$ and $D$ be sets. 
\begin{parts}
\part Prove that $(A \times B) \cap (C \times D) = (A\cap C) \times (B \cap D)$.
\part Find an example that show thats the equality 
\begin{equation*}
(A\times B) \cup (C \times D) = (A\cup C) \times (B \cup D)
\end{equation*}
is, in general, false.   
\end{parts}
\end{exercise}

\begin{exercise}{7}
Use Mathematical Induction to verify the following statements:
\begin{parts}
\part For every $n \in \mathbb{N}$, the number 
\begin{equation*}
\frac{n(n+1)}{2}
\end{equation*}
is an integer.
\part For every $n \in \mathbb{N}$, the number 
\begin{equation*}
n(n+1)(2n+1)
\end{equation*}
is divisible by $6$.
\part For all $n \in \mathbb{N}$, the sum of the interior angles of a convex polygon with $(n+2)$-sides is $180 \cdot n$ degrees.
\part For all $n \in \mathbb{N}$, given $n$ points in a plane, no three of which are collinear, there are exactly 
\begin{equation*}
\frac{n^2-n}{2}
\end{equation*}
line segments joining pairs all pairs of points. 
\end{parts}
\end{exercise}

\begin{exercise}{8}
The Fibonacci numbers are recursively defined by the relations
\begin{align*}
f_1 & =1,\\
f_2 & = 1,\\
f_{n+2} &=f_{n+1}+f_{n}.
\end{align*}
In this problem, you are required to use Induction (or its variants) to show the following:
\begin{parts}
\part Two consecutive terms of this sequence have no common divisors, other than $\pm 1$.
\part $f_{3n}$ is always even. 
\part $f_{4n}$ is divisible by $3$, for all $n \in \mathbb{N}$.
\end{parts}
\end{exercise}

\begin{exercise}{9}
In a certain kind of tournament, every player plays every other player exactly once, and either wins or losess. There are no ties. Define a top player to be a player who, for every other player $x$, either beats $x$ or beats a player $y$ who beats $x$. 
\begin{parts}
\part Show, by means of an example, that there can be more than one top player.
\part Use Induction to show that every such tournament with $n$ players has a top player. 
\part Use the Well-Ordering Principle to show that every such tournament with $n$ players has a top player.
\end{parts}
\end{exercise}
\end{document}
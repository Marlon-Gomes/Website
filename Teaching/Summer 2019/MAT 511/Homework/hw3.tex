\documentclass[12pt,oneside]{exam}

% This package simply sets the margins to be 1 inch.
\usepackage[margin=1in]{geometry}

% These packages include nice commands from AMS-LaTeX
\usepackage{amssymb,amsmath,amsthm,amsfonts,latexsym,verbatim,xspace,setspace}

% Make the space between lines slightly more
% generous than normal single spacing, but compensate
% so that the spacing between rows of matrices still
% looks normal.  Note that 1.1=1/.9090909...
\renewcommand{\baselinestretch}{1.1}
\renewcommand{\arraystretch}{.91}

% Define an environment for exercises.
\newenvironment{exercise}[1]{\vspace{.1in}\noindent\textbf{Exercise #1 \hspace{.05em}}}{}

% define shortcut commands for commonly used symbols
\newcommand{\R}{\mathbb{R}}
\newcommand{\C}{\mathbb{C}}
\newcommand{\Z}{\mathbb{Z}}
\newcommand{\Q}{\mathbb{Q}}
\newcommand{\N}{\mathbb{N}}
\newcommand{\calP}{\mathcal{P}}

\DeclareMathOperator{\vsspan}{span}

\title{Math 511 - Summer I 2019: Homework 3}

%%%%%%%%%%%%%%%%%%%%%%%%%%%%%%%%%%%%%%%%%%

\begin{document}

\begin{flushright}
\sc MAT 511 - Summer 1\\
June 14, 2019
\end{flushright}
\bigskip

This homework is due on Thursday, June 20, by 7:00 pm. 
\begin{center}
\textsf{Homework 3} 
\end{center}

%%%%%%%%%%%%%%%%%%%%%%%%%%%%%%%%%%%%%%%%

\begin{exercise}{1}
Let $A=\{1,2,3\}$. Construct a relation on $A \times A$ satisfying the following properties:
\begin{parts}
\part It is not reflexive, not symmetric, and not transitive.
\part It is reflexive, not symmetric, and not transitive.
\part It is not reflexive, symmetric, and not transitive.
\part It is reflexive, symmetric, and not transitive.
\part It is not reflexive, not symmetric, and transitive.
\part It is reflexive, not symmetric, and transitive.
\part It is not reflexive, symmetric, and transitive.
\part It is reflexive, symmetric, and transitive.
\end{parts}
\end{exercise}

\begin{exercise}{2}
Define the following relation on $\mathbb{R} \times \mathbb{R}$: a point $(a,b)$ is related to $(x,y)$ if 
\begin{equation*}
y-b=x-a
\end{equation*}
Show that this relation is an equivalence, i.e., it is reflexive, symmetric, and transitive. What are the equivalence classes?
\end{exercise}

\begin{exercise}{3}
Let $A = \mathbb{Z} \times (\mathbb{Z}-\{0\})$. This is the set of pairs of integers, in which the second entry is non-zero. On this set, we consider the following relation, 
\begin{equation*}
\mathbb{Q}=\{ ((a,b), (c,d)) \in A \times A| \ ad=bc\}.
\end{equation*}
Show that this relation is an equivalence relation, that is:
\begin{parts}
\part it is reflexive: $(a,b) \in A$ is related to itself.
\part it is symmetric: if $((a,b),(c,d)) \in Q$, then $((c,d), (a,b)) \in Q$. 
\part it is transitive: given $((a,b),(c,d)) \in Q$, and $((c,d),(e,f)) \in Q$, then $((a,b),(e,f))$. 
\end{parts}

Furthermore, describe the equivalence classes of this relation. 
\end{exercise}

\begin{exercise}{4}
In each of the problems below, you are given a set $A$ and a collection of subsets. Determine if this collection is a partition. Explain your reasoning. 
\begin{parts}
\part $A =\mathbb{N}$, $\mathcal{P}=\{ \{0\}, \{n \in \mathbb{N}| \ n \ \mbox{is even}\}, \{n \in \mathbb{N}| \ n \ \mbox{is a prime number}\}\}$.
\part $A= \mathbb{N}$, $\mathcal{P}=\{ \{0,1\}, \{n\in \mathbb{N}| \ n \ \mbox{has a prime factor} \}\}$.
\end{parts}
\end{exercise}

\begin{exercise}{5}
Let $A=\{a,b,c\}$. Give an example of a relation on $A$ that is 
\begin{parts}
\part antisymmetric and symmetric.
\part antisymmetric, reflexive, and not symmetric.
\part antisymmetric, not reflexive, and not symmetric. 
\part symmetric and not antisymmetric. 
\part not symmetric and not antisymmetric.
\part irreflexive and not symmetric.
\part irreflexive and not antisymmetric.
\part antisymmetric, not reflexive, and not irreflexive.
\part transitive, antisymmetric, and irreflexive.
\end{parts}
\end{exercise}

\begin{exercise}{6}
Define the relation $R$ on $\mathbb{N}$ by the following property: $a$ is related to $b$ by $R$ if there exists a non-negative integer $k$ such that $b=2^ka$. Show that this is a partial ordering. Does this relation have the comparability property?
\end{exercise}

\begin{exercise}{7}
Consider a partially ordered set $A$, with order relation $R$. Assume that $C \subset B \subset A$. Determine whether the statements below are true or false.
\begin{parts}
\part Every upper bound for $C$ is an upper bound for $B$.
\part Every upper bound for $B$ is an upper bound for $C$.
\end{parts}
\end{exercise}

\begin{exercise}{8}
Consider an order relation on the set $A=\{a,b,c,d,e,f,g,h\}$, given by the following properties: 
\begin{itemize}
\item $g \leq  f$
\item $h \leq f$
\item $h \leq d$
\item $f \leq c$
\item $f \leq b$
\item $e \leq b$
\item $e \leq a$.
\end{itemize}

Construct its Hasse diagram. In addition, find the following bounds:
\begin{parts}
\part all upper bounds for the set $\{b,f,g,h\}$.
\part all lower bounds for the set $\{a,d\}$.
\part the supremum, if it exists, for the set $\{e,g,h\}$.
\part the infimum, if it exists, for the set $\{b,f,g\}$.
\part the smallest element, if it exists, for the set $\{b,c,d\}$.
\end{parts}
\end{exercise}

\end{document}
% Exam Template for UMTYMP and Math Department courses
%
% Using Philip Hirschhorn's exam.cls: http://www-math.mit.edu/~psh/#ExamCls
%
% run pdflatex on a finished exam at least three times to do the grading table on front page.
%
%%%%%%%%%%%%%%%%%%%%%%%%%%%%%%%%%%%%%%%%%%%%%%%%%%%%%%%%%%%%%%%%%%%%%%%%%%%%%%%%%%%%%%%%%%%%%%

% These lines can probably stay unchanged, although you can remove the last
% two packages if you're not making pictures with tikz.
\documentclass[11pt]{exam}
\RequirePackage{amssymb, amsfonts, amsmath, latexsym, verbatim, xspace, setspace}


% By default LaTeX uses large margins.  This doesn't work well on exams; problems
% end up in the "middle" of the page, reducing the amount of space for students
% to work on them.
\usepackage[margin=1in]{geometry}
\usepackage[english]{babel}
\usepackage[autostyle]{csquotes} %%%% This package allows Tex to recognize quotation marks with the \enquote command. 


% Here's where you edit the Class, Exam, Date, etc.
\newcommand{\class}{MAT 511}
\newcommand{\term}{Summer I 2019}
\newcommand{\examnum}{Midterm}
\newcommand{\examdate}{06/11/19}
\newcommand{\timelimit}{1 hour and 55 minutes}

% For an exam, single spacing is most appropriate
\singlespacing
% \onehalfspacing
% \doublespacing

% For an exam, we generally want to turn off paragraph indentation
\parindent 0ex
\title{MAT 511 - Summer I 2019: Midterm}
\begin{document} 

% These commands set up the running header on the top of the exam pages
\pagestyle{head}
\firstpageheader{}{}{}\textbf{}
\runningheader{\class}{\examnum\ - Page \thepage\ of \numpages}{\examdate}
\runningheadrule

\begin{flushright}
\begin{tabular}{p{2.8in} r l}
\textbf{\class} & \textbf{Name (Print):} & \makebox[2in]{\hrulefill}\\
\textbf{\term} &&\\
\textbf{\examnum} &&\\
\textbf{\examdate} &&\\
\textbf{Time Limit: \timelimit} & ID number & \makebox[2in]{\hrulefill}
\end{tabular}\\
\end{flushright}
\rule[1ex]{\textwidth}{.1pt}

\begin{center}
\large{\textbf{Instructions}}
\end{center}

\begin{minipage}[t]{3.7in}
\vspace{0pt}
\begin{itemize}

\item This exam contains \numpages\ pages (including this cover page) and
\numquestions\ problems.  Check to see if any pages are missing.  Enter
all requested information on the top of this page, and put your initials
on the top of every page, in case the pages become separated.

\item You may \textit{not} use your books, notes, or any device that is capable of accessing the internet on this exam (e.g., smartphones, smartwatches, tablets). You may not use a calculator.

\item \textbf{Organize your work}, in a reasonably neat and coherent way, in
the space provided. Work scattered all over the page without a clear ordering will 
receive very little credit.  

\item \textbf{Mysterious or unsupported answers will not receive full
credit}.

\end{itemize}

\end{minipage}
\hfill
\begin{minipage}[t]{2.3in}
\vspace{0pt}
%\cellwidth{3em}
\gradetablestretch{2}
\vqword{Problem}
\addpoints % required here by exam.cls, even though questions haven't started yet.	
\gradetable[v]%[pages]  % Use [pages] to have grading table by page instead of question

\end{minipage}
\newpage % End of cover page

%%%%%%%%%%%%%%%%%%%%%%%%%%%%%%%%%%%%%%%%%%%%%%%%%%%%%%%%%%%%%%%%%%%%%%%%%%%%%%%%%%%%%
%
% See http://www-math.mit.edu/~psh/#ExamCls for full documentation, but the questions
% below give an idea of how to write questions [with parts] and have the points
% tracked automatically on the cover page.
%
%
%%%%%%%%%%%%%%%%%%%%%%%%%%%%%%%%%%%%%%%%%%%%%%%%%%%%%%%%%%%%%%%%%%%%%%%%%%%%%%%%%%%%%

\begin{questions}

% Basic question
%%%%%%%%%%%%%%

\addpoints 
\question[4] In the following riddle, identify the universe of discourse, the propositions and conditional statement involving them, and solve the riddle by syllogisms and contrapositives. 
\begin{center}
``Animals, that do not kick, are always unexcitable.\\
Donkeys have no horns.\\
A buffalo can always toss one over a gate.\\
No animals that kick are easy to swallow.\\
No hornless animal can toss one over a gate.\\
All animals are excitable, except buffaloes.."
\end{center}
\vfill
\newpage

%%%%%%%%%%%%%%%%% 

\addpoints
\question[4] Use a truth-table to show that the following propositional forms
\begin{equation*}
(p \Rightarrow q) \Rightarrow r,
\end{equation*}
and 
\begin{equation*}
(p \land \neg q) \lor r
\end{equation*}
are equivalent. 
\newpage 


%%%%%%%%%%%%%%%%%%%%%%%%%%%%%%%%%%%
\addpoints 
\question The goal of this problem is to show that $\sqrt{2}$ is an irrational number. We will use the language of predicate logic, and our domain of discourse is that of the Real numbers (with the usual operations of algebra). You may assume the following facts. 

\begin{enumerate}
\item (Definition of even number) An integer number $n$ is called even if there exists an integer $k$ for which $n=2k$.
\item (Definition of rational number) A real number $x$ is called \textit{rational} if it can be written as a quotient of two integers, $a$ and $b$, such that $b \neq 0$, and $a,b$ have no common factors. 
\end{enumerate}

\begin{parts}
\part[2] Show that if $m$ is an integer number such that $m^2$ is even, then $m$ is even.
\vfill
\part[2] Show, by contradiction, that $\sqrt{2}$ is an irrational number. 
\vfill
\end{parts}
\newpage

%%%%%%%%%%%%%%%%%
\addpoints
\question[4] Prove that for any triple of sets $A,B$ and $C$, 
\begin{equation*}
(A-B)-C=(A-C)-(B-C),
\end{equation*}
by using properties of set membership or reduction to propositional logic.
\newpage

%%%%%%%%%%%%%%%%%%%
\addpoints
\question[4] Recall that a (non-zero) polynomial $d$ divides another polynomial $D$, if there exists a third polynomial $q$ for which the relation 
\begin{equation*}
D(x)=d(x)q(x)
\end{equation*}
holds for all real numbers $x$. In this exercise, use Mathematical Induction to show that $d(x)=x-1$ divides the polynomials $D_n(x)=x^n-1$, for all $n \in \mathbb{N}$.



\end{questions}
\end{document}

% Exam Template for UMTYMP and Math Department courses
%
% Using Philip Hirschhorn's exam.cls: http://www-math.mit.edu/~psh/#ExamCls
%
% run pdflatex on a finished exam at least three times to do the grading table on front page.
%
%%%%%%%%%%%%%%%%%%%%%%%%%%%%%%%%%%%%%%%%%%%%%%%%%%%%%%%%%%%%%%%%%%%%%%%%%%%%%%%%%%%%%%%%%%%%%%

% These lines can probably stay unchanged, although you can remove the last
% two packages if you're not making pictures with tikz.
\documentclass[11pt]{exam}
\RequirePackage{amssymb, amsfonts, amsmath, latexsym, verbatim, xspace, setspace}


% By default LaTeX uses large margins.  This doesn't work well on exams; problems
% end up in the "middle" of the page, reducing the amount of space for students
% to work on them.
\usepackage[margin=1in]{geometry}
\usepackage[english]{babel}
\usepackage[autostyle]{csquotes} %%%% This package allows Tex to recognize quotation marks with the \enquote command. 


% Here's where you edit the Class, Exam, Date, etc.
\newcommand{\class}{MAT 511}
\newcommand{\term}{Summer I 2019}
\newcommand{\examnum}{Final exam}
\newcommand{\examdate}{07/02/19}
\newcommand{\timelimit}{3 hours and 25 minutes}

% For an exam, single spacing is most appropriate
\singlespacing
% \onehalfspacing
% \doublespacing

% For an exam, we generally want to turn off paragraph indentation
\parindent 0ex
\title{MAT 511 - Summer I 2019: Final Exam}
\begin{document} 

% These commands set up the running header on the top of the exam pages
\pagestyle{head}
\firstpageheader{}{}{}\textbf{}
\runningheader{\class}{\examnum\ - Page \thepage\ of \numpages}{\examdate}
\runningheadrule

\begin{flushright}
\begin{tabular}{p{2.8in} r l}
\textbf{\class} & \textbf{Name (Print):} & \makebox[2in]{\hrulefill}\\
\textbf{\term} &&\\
\textbf{\examnum} &&\\
\textbf{\examdate} &&\\
\textbf{Time Limit: \timelimit} & ID number & \makebox[2in]{\hrulefill}
\end{tabular}\\
\end{flushright}
\rule[1ex]{\textwidth}{.1pt}

\begin{center}
\large{\textbf{Instructions}}
\end{center}

\begin{minipage}[t]{3.7in}
\vspace{0pt}
\begin{itemize}

\item This exam contains \numpages\ pages (including this cover page) and
\numquestions\ problems.  Check to see if any pages are missing.  Enter
all requested information on the top of this page, and put your initials
on the top of every page, in case the pages become separated.

\item You may \textit{not} use your books, notes, or any device that is capable of accessing the internet on this exam (e.g., smartphones, smartwatches, tablets). You may not use a calculator.

\item \textbf{Organize your work}, in a reasonably neat and coherent way, in
the space provided. Work scattered all over the page without a clear ordering will 
receive very little credit.  

\item \textbf{Mysterious or unsupported answers will not receive full
credit}.

\end{itemize}

\end{minipage}
\hfill
\begin{minipage}[t]{2.3in}
\vspace{0pt}
%\cellwidth{3em}
\gradetablestretch{2}
\vqword{Problem}
\addpoints % required here by exam.cls, even though questions haven't started yet.	
\gradetable[v]%[pages]  % Use [pages] to have grading table by page instead of question

\end{minipage}
\newpage % End of cover page

%%%%%%%%%%%%%%%%%%%%%%%%%%%%%%%%%%%%%%%%%%%%%%%%%%%%%%%%%%%%%%%%%%%%%%%%%%%%%%%%%%%%%
%
% See http://www-math.mit.edu/~psh/#ExamCls for full documentation, but the questions
% below give an idea of how to write questions [with parts] and have the points
% tracked automatically on the cover page.
%
%
%%%%%%%%%%%%%%%%%%%%%%%%%%%%%%%%%%%%%%%%%%%%%%%%%%%%%%%%%%%%%%%%%%%%%%%%%%%%%%%%%%%%%

\begin{questions}

% Basic question
%%%%%%%%%%%%%%

\addpoints 
\question Construct relations on the set $A=\{1,2,3\}$ satisfying the following properties:
\begin{parts}
\part [1] It is not reflexive, not symmetric, and transitive.
\vfill
\part [1] It is reflexive, not symmetric, and not transitive.
\vfill
\newpage
\part[1] It is anti-symmetric, not reflexive, and not irreflexive.
\vfill 
\part[1] It is not symmetric and not anti-symmetric.
\vfill
\end{parts}
\newpage
%%%%%%%%%%%%%%%%% 

\addpoints
\question 
\begin{parts}
\part[2]  Give an example of a surjective function which does not admit a left-inverse. 
\vfill
\part[2] Prove that if an injective function has a right-inverse, then it is a bijection. 
\vfill
\end{parts}
\newpage 


%%%%%%%%%%%%%%%%%%%%%%%%%%%%%%%%%%%
\addpoints 
\question
\begin{parts}
\part[2] Using the definitions of finite and infinite sets given in class, prove that the union of two finite sets is a finite set.
\vfill
\part[2] Prove, by induction, that 
\begin{equation*}
A_0, A_1, \cdots, A_{n+1}
\end{equation*}
are finite sets, then their union is also a finite set. 
\end{parts}
\vfill
\newpage

%%%%%%%%%%%%%%%%%
\addpoints
\question Let $\mathbb{N}^{*}$ denote the set of non-zero natural numbers. Define a relation $D \subset \mathbb{N}^{*} \times \mathbb{N}^{*}$ as follows:
\begin{equation*}
D =\{(a,b) \in \mathbb{N}^{*} \times \mathbb{N}^{*}| \ \exists \ k \in \mathbb{N}^{*}, b=ka\},
\end{equation*}
that is, $aDb$ if and only if a divides $b$. 
\begin{parts}
\part[1] Prove that this relation is reflexive. 
\vfill 
\part[1] Prove that this relation is anti-symmetric. 
\vfill
\newpage
\part[1] Prove that this relation is transitive. 
\vfill
\part[1] Does this ordering have the comparability property? Justify your answer. 
\vfill
\end{parts}
\newpage
%%%%%%%%%%%%%%%%%%%
\addpoints
\question In each of the following problems, a statement about sets is given. Determine whether the statement is true or false. If true, prove it, if false, give a counter example.
\begin{parts}
\part[1] If $A$ and $B$ are infinite sets, then $A\cap B$ is an infinite set. 
\vfill
\part[1] If $A$ is infinite and $B$ is finite, then $A\cap B$ is finite.
\vfill
\part[1] If $A$ is infinite and $B$ is finite, then $A - B$ is infinite.
\vfill
\part[1] If $A$ is infinite, and $B$ is finite, $A\cup B$ is infinite.
\vfill
\end{parts}
\end{questions}
\end{document}

\documentclass[11pt]{amsart}
\usepackage{amssymb,amsfonts,amsthm,amsmath}
\usepackage[english]{babel}
\usepackage[all,cmtip]{xy}
\usepackage{hyperref}
\usepackage{url}
%\usepackage{mathrsfs}
%\usepackage[notcite,notref]{showkeys}
\def\limproj{\mathop{\oalign{lim\cr\hidewidth$\longleftarrow$\hidewidth\cr}}}

%--------------
% macro perso
%--------------
\newcommand{\inputc}[1]{ \raisebox{-0.5\height}{\input{#1}} }
\newtheorem{theorem}{Theorem}[section]
\newtheorem{lemma}[theorem]{Lemma}
\newtheorem{corollary}[theorem]{Corollary}
\newtheorem{definition}[theorem]{Definition}
\newtheorem{proposition}[theorem]{Proposition}
\newtheorem{remark}[theorem]{Remark}
\newtheorem{example}{Example}[section]
%\newtheorem{theorem*}{Theorem}

\newcommand{\func}[3]{{#1} : {#2} \longrightarrow {#3}}
\numberwithin{equation}{section}

\newenvironment{myproof}{\noindent{it Proof}
\setlength{\parindent}{0mm}}
{$\hfill \bs$}

\title[MAT 511 Syllabus - Summer 2019]{MAT 511 Syllabus - Summer 2019}

\author[M. Gomes]{Marlon Gomes}
%

\begin{document}
\maketitle

\section{Course Description}
\subsection{Course Goal}
Brief history of mathematics; sets, functions and logic; constructions of number systems, including their historical development; mathematical induction. The main focus of the course will be on the construction and writing of mathematical proofs. 

\subsection{Topics Covered}
Our course will cover five main topics: Logic and Proofs; Elements of Set Theory; Relations between sets; Cardinality; Elementary Number Systems. Below is a brief description of each topic. 

In the first section, we will introduce the student to mathematics as a formal language, discussing its grammar (what elements go into a mathematical sentence), syntax (how to structure such sentences), and semantics (what is the meaning of a sentence). Then we shall move on to discuss aspects of this language's \textit{deductive apparatus}, examine many common types of inferences used in modern mathematical reasoning, and work on many exercises on proof writing and analysis.

In the second section, we will discuss general notions of set theory, such as membership, containment and set operations. We will then move on to work on a specific example, the Natural numbers, as construced by Peano, and discuss Mathematical Induction in its many forms, as well as basic principles of counting.

The third topic is concerned with relations between sets. We will study the general notion of a relation, constructions of new relations from old, and properties of such relations. After this introduction, we will study special relations: equivalences, orderings, functions.

In the fourth section, we will introduce the notion of equivalent sets, characterize infinite sets, and discuss the many layers of infinity, as well as the Continuum Hypothesis.

Finally, we will study elementary number systems. We will discuss the Integer Numbers and Modular Arithmetic, the Rational Numbers, and finish with a construction of the Real numbers.

\subsection{Textbook}
The required textbook is \textit{A Transition to Advanced Mathematics}, by Douglas Smith, Maurice Eggen and Richard St. Andre, 8th edition.

\subsection{Important Times and Dates}
\begin{itemize}
\item Lectures: Tuesdays and Thursdays, 6:00 pm - 9:25 pm, at Earth and Space Sciences (ESS) 177.
\item Midterm: June 11th, 7:30pm to 9:25pm, at EES 177.
\item Final Exam: [July 2nd, 6:0 pm to 9:25 pm, at EES 177.
\end{itemize}

\subsection{Assignments}
	There will be four problem sets, which will be available on the course webpage (link below) and Blackboard.
	
    Not all problems will be graded, but you should attempt to solve them all anyway, as they will serve as your foundation for the problems you will see on your exams. Graded problems will be discussed in class. Solutions to the remaining problems will be posted on the course webpage weekly.
    
    Due dates will be enforced: homework is due by 7:00 pm on the dates described on the schedule. If you cannot make it to class (in time or at all), send me a scanned copy of your homework via e-mail by the deadline (a follow-up physical copy would be appreciated). Late homework will be penalized by 10\% of its total value, for each day of tardiness.

\subsection{Grades}
Your grade will be calculated in the following way:
\begin{enumerate}
\item Homework assignments: $15\%$ each.
\item Midterm: $20\%$.
\item Final Exam: $20\%$.
\end{enumerate}

\section{Contact}
The course webpage is
\begin{center}
\url{http://www.math.stonybrook.edu/~mgomes/mat511sum19.html}.
\end{center}
My office hours are on Tuesdays and Thursdays, 1:00pm to 2:00pm, in the Math Learning Center (Math S-235), and Tuesdays, 4:30pm to 5:30pm, in my office (Math 3-101). E-mail is the best form of communication besides lectures and office hours. My address is
\begin{center}
\href{mailto: mgomes@math.stonybrook.edu}{ mgomes@math.stonybrook.edu}.
\end{center}

\section{DSS Notice}
If you have a physical, psychological, medical, or learning disability that may impact your course work, please contact Disability Support Services at (631) 632-6748 or at
\begin{center}
\url{http://studentaffairs.stonybrook.edu/dss/}.
\end{center}
They will determine with you what accommodations are necessary and appropriate. All information and documentation is confidential. Students who require assistance during emergency evacuation are encouraged to discuss their needs with their professors and Disability Support Services. For procedures and information go to the following website: http://www.sunysb.edu/

\section{Academic Integrity}
Each student must pursue his or her academic goals honestly and be personally accountable for all submitted work. Representing another person's work as your own is always wrong. Faculty are required to report any suspected instances of academic dishonesty to the Academic Judiciary. Faculty in the Health Sciences Center (School of Health Technology and Management, Nursing, Social Welfare, Dental Medicine) and School of Medicine are required to follow their school-specific procedures. For more comprehensive information on academic integrity, including categories of academic dishonesty, please refer to the academic judiciary website at
\begin{center}
\url{http://www.stonybrook.edu/uaa/academicjudiciary/}
\end{center}

\section{Critical Incident Management Statement}
Stony Brook University expects students to respect the rights, privileges, and property of other people. Faculty are required to report to the Office of Judicial Affairs any disruptive behavior that interrupts their ability to teach, compromises the safety of the learning environment, or inhibits students' ability to learn. Faculty in the HSC Schools and the School of Medicine are required to follow their school-specific procedures.
\end{document}

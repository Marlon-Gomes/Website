\chapter{Elements of Mathematical Logic}

\epigraph{``\textit{Just as the easist bodies to see are those that are neither very near nor very far, neither very small not very great, so the easiest conceptions to grasp are those that are neigher very complex nor very simple(...). And as we need two sorts of instruments, the telescope and the microscope, for the enlargment of our visual powers, so we need two sorts of instruments for the enlagement of our logical powers, one to take us forward to the higher mathematics, the other to take us backward to the logical foundations of the things that we are inclined to take for granted in mathematics. We shall find that by analysing our ordinary mathematics notions, we acquire fresh insight, new powers, and the means of reaching whole new mathematics subjects(...).}"}{\textit{Bertrand Russel}}

Mathematics is, in many ways, a language. Its words are ideas, and its grammar is Logic. In this lecture, we shall study aspects of Logic by means of this analogy. 

A logical system consists of the following components: a collection of symbols, the Alphabet; a Syntax; its Semantics. Elements of the Alphabet can be concatenated into finite strings, called \textit{Sentences}. Among such sentences, are those in the \textit{Syntax}, whose elements are called well-formed sentences. Naively, we think of the Syntax as a collection of rules, describing what combinations of symbols are allowed in the language. The Semantics are the interpretative component of the language. In its simplest form, it will consist of a collection of \textit{meanings}, and and assignment of a meaning to each well-formed sentence of the language. In more complex languages, the Semantics may also entail relations between components of the Language. 

In the sequel, we will describe three types of increasingly complex logical systems: Zeroth-order Logic, also known as \textit{Propositional Logic}; First-order Logic, or \textit{Predicate Logic}; and Second-Order Logic. 

\section{Propositional Logic}

The intuitive model for Propositional Logic is derived from natural languages. In English, we may express sentences in terms of 

In Propositional Logic, the \textit{Alphabet} is divided into two components: the atomic formulas and the logical connectives. Each logical connective has a \textit{valence} - the number of atomic formulas it is allowed to connect, according to the rules of the Grammar (more on this below). The Syntax of the Language is defined by declaring the following rules:
\begin{enumerate}
\item Every atomic formula is well-formed. 
\item If $\omega$ is a connective of valence $n$, and $a_1, a_2, \cdots, a_n$ are well-formed formulas, then $\omega a_1a_2\cdots a_n$ is a well-formed formula. 
\item (Closedness) No other sentences are well-formed.
\end{enumerate}
Well-formed sentences in this language are called \textit{propositions}.

The semantics of a zeroth-order system consists of a pair of \textit{meanings} \textbf{T}, and \textbf{F} (short for true and false, respectively), and an assignment of a meaning to each proposition. 

The main example we are concerned with is a Natural Deductive System. In this system, the atomic formulas and their truth-values can be prescribed at will. There are five logical connectives: 
\begin{enumerate}
\item Negation, denoted by $\neg$;  
\item Conjunction, denoted by $\land$ ; 
\item Disjunction, denoted by $\lor$; 
\item Implication, denoted by $\Rightarrow$; 
\item Equivalence, denoted by $\Leftrightarrow$.
\end{enumerate}
In what follows, we shall describe each of these logical connectives, and their interaction with the Semantics of the Language. 

\begin{definition}
The negation is a logical connective of valence 1. Given a proposition $P$, $\neg P$ is is to be though of as ``not P", and its meaning is defined as the opposite of that of $P$. 
\end{definition}


\begin{definition}
The conjunction is a logical connective of valence 2. Given two propositions, $P$ and $Q$, their conjunction is denoted by $P\land Q$, and read ``$P$ and $Q$".  It is true exactly when both $P$ and $Q$ are true. 
\end{definition}

\begin{definition}
The disjunction is a logical connective of valence 2. Given two propositions, $P$ and $Q$, their disjunction is denoted by $P \lor Q$, and read ``$P$ or $Q$". It is true when at least one of $P$ or $Q$ is true.
\end{definition}

\begin{definition}
Implication is a logical connective of valence 2. Given two propositions, $P$ and $Q$, the string $P\Rightarrow Q$ is denotes the implication ``if $P$, then $Q$". The sentence $P\Rightarrow Q$ is true when $P$ is false or $Q$ is true.
\end{definition}

We should pause and reflect on this definition. Another way to state that an implication is true, is if $Q$ is true, whenever $P$ is. 
The presents a conundrum in Logic: an implication may be true, even if the premise is false. This should serve as a warning for future lessons, that one should always double-check that the premises are sound, before applying them to an argument. 

\begin{example}
Let $P$ and $Q$ be the propositions
\begin{itemize}
\item $P$: Triangles have three sides.
\item $Q$: Squares have four sides .
\end{itemize}

Both of which are assumed to be true, in our logical system (hence, their negations are false). 
We shall consider the truth-value of the four implications relating $P$ and $\neg P$ as premises, and $Q$ and $\neg Q$ as conclusions. 

The implication $P \Rightarrow Q$, read ``If triangles have three sides, then squares have four sides", is true. The two sentences are not related, in any way, but the point it that the outcome - squares have four sides - is true, regardless of the premise. 

The implication $\neg P \Rightarrow Q$, read ``If triangles do not have three sides, then squares have four sides", is also true. Again, the fact that the premise is false, does not invalidate the fact that the conclusion is true. 

Now consider the implication $\neg P \Rightarrow \neg Q$, read ``If triangles do not have three sides, then squares do not have four sides". This implication, in our logical system, is true. The key here is that the conclusion is false, but since the premise is also false, one cannot deny the implication. 

Finally, we consider the implication $P \Rightarrow \neg Q$, read ``If triangles have three sides, then squares do not have four sides". This is false, because although the premise is true, the conclusion is false. 
\end{example}

\begin{definition}
Equivalence is a logical connective of valence 2. Given two propositions, $P$ and $Q$, the string $P \Leftrightarrow Q$ stands for ``P is equivalent to Q", or more commonly, ``P if and only if Q". This sentence is true when both $P$ and $Q$ have the same truth values. 
\end{definition}

This is, again, a definition which requires some thought, as it may be counterintuitive at first. The word \textit{equivalence} is employed here in its literal meaning: the two propositions have the same value in our logical system. 

These simple logical connectives can be combined into logical connectives of higher valence. For example, given three propositions $P$, $Q$ and $R$, we can form the proposition
\begin{equation*}
P \land (Q \lor R).
\end{equation*}
In dealing with compound connectives, we will often refer to them by means of their action on propositions, such as $(P\lor Q) \land R$, without assigning specific propositions $P,Q,R$ (that is, we treat the treat the propositions are \textit{variables}). To distinguish betweeen specific compound propositions and propositional forms, we will use lower-case letters to write the latter, i.e., the above form will be written $(p \lor q) \land r$. 

We remark that the use of parenthesis in such expressions is important, as it allows one to correctly discern the order in which the basic connectives are used.
For instance, given propositions $P$ and $Q$, the compound form ``Not $P$ or $Q$", $\neg P \lor Q$, is ambiguous. It could mean $\neg(P \lor Q)$, which in English would translate to ``Neither $P$ nor $Q$", or $(\neg P) \lor Q$ which would translate to ``Either not $P$, or else $Q$".  

Certain compound connectives generate propositions whose truth-values that do not depend on those of its components. These deserve special attention, and will be given appropriate names. 

\begin{definition}
A compound connective whose truth-value is \textbf{True}, regardless of its components, is called a tautology. A compound connective whose truth-value is \textbf{False}, regardless of its components,  is called a contradiction. 
\end{definition}

\begin{example}
The propositional forms $p \lor \neg p$ and $p \Leftrightarrow \neg(\neg p)$ are tautologies, called the \textit{Law of Excluded Middle} and the \textit{Double Negation Law}.
\end{example}

\begin{example}
The propositional forms $p \land \neg p$ and 

\end{example}

\begin{definition}
Let $P$ and $Q$ be propositions, and consider the proposition $P \Rightarrow Q$. We define two derived forms from it: the \textbf{converse}
\begin{equation*}
Q \Rightarrow P,
\end{equation*}
and the \textbf{contrapositive}
\begin{equation*}
(\neg Q) \Rightarrow (\neg P)
\end{equation*}
\end{definition}

Let us compare the how the truth-values of the propositional form $p \Rightarrow q$ and its derived forms, $q \rightarrow p$ and $\neg q \Rightarrow \neg p$, depend on the truth-assignments of $p$ and $q$. 
\begin{center}
\begin{tabular}{|c|c|c|c|c|}
\hline 
\rule[-1ex]{0pt}{2.5ex} p & T & T & F & F \\ 
\hline 
\rule[-1ex]{0pt}{2.5ex} q & T & F & T & F \\ 
\hline 
\rule[-1ex]{0pt}{2.5ex} $p \Rightarrow q$ & T & F & T & T \\ 
\hline 
\rule[-1ex]{0pt}{2.5ex} $q \Rightarrow p$ & T & T & F & T \\ 
\hline 
\rule[-1ex]{0pt}{2.5ex} $ \neg q \Rightarrow \neg p$ & T & F & T & T \\ 
\hline 
\end{tabular} 
\end{center}

We observe from the table that an implication $p \Rightarrow q$ and its converse $q \rightarrow p$ do not necessarily have the same truth-values. In fact, these are only equivalent when $p$ and $q$ themselves are equivalent. Meanwhile, the implication $p \Rightarrow q$ and its contrapositive $\neg q \Rightarrow \neg p$ always have the same truth-value, so these two propositional forms are equivalent. 

\section{Predicate Languages}
A natural language, such as English, has more to it than simple propositions with immutable truth values, like the ones we use in Propositional Logic. First-order logic adds two new layers of complexity: predicates and quantifiers. 

Consider the following examples of sentences in a natural language:
\begin{center}
``Euclid was a geometer."
\end{center}
\begin{center}
``Pappus was a geometer."
\end{center}

Both would be regarded as propositions, according to the previous section, since to each of them we can assign an unambigous truth-value. These sentences contain something in commom: they qualify the subjects, Euclid and Pappus, as geometers. We may add a \textit{variable} to the language, say $x$, and create a new sentence, 
\begin{center}
``x is a geometer,"
\end{center}
which is not a proposition (i.e., does not have a well-defined truth-value) until $x$ is assigned an object.

More complicated sentences, such as 
\begin{center}
``Gauss taught Riemann"
\end{center}
have subjects (Gauss) and objects (Riemann). To create such a distinction between words in a logical alphabet would add an unnecessary hurdle. We shall instead interpret 
\begin{center}
``x taught y"
\end{center}
where now $x$ and $y$ are variables, as a two-argument predicate, and treat both Gauss and Riemann as objects in the sentence above. Extrapolating this idea, one may think of many-argument predicates in English, and we shall allow then in our Predicate Logic as well. 

This poses a question: what are the objects of the language? In the exampe above, a simple substitution of objects, 
\begin{center}
``Gauss taught Mathematics"
\end{center}
whould affect the meaning of the predicate. Instead of referring to a Teacher/Student pair, it now refers to a Teacher/Subject pair. Avoiding such ambiguities is essential when studying Logic, hence we shall make a convention to always name the \textit{Domain of Discourse}, also called \textit{Universe of Discourse}, in which our variables take place. Having such a notion, one is lead to the notions of \textit{quantifiers}, which we describe next.

\begin{definition}
The existencial quantifier is the logical symbol $\exists$, read ``there exists". The universal quantifier is the symbol $\forall$, read ``for all".
\end{definition}
\chapter{Elements of Inference}

A proof is the successive application of a sequence of tautologies. In the following two sections, we present a list of basic argument forms, \textit{rules of inference} and \textit{rules of replacement}. Both types of rules are tautologies, their distincition being based on the types of conditionals involved. Rules of inference involve conditionals, while rules of replacement involve biconditionals. Once these forms have been established, we will study various examples of proofs using these rules. 

\section{Rules of Inference}
A rule of inference is a relation between conditionals. There are infinitely many such rules in Natural Deduction, but 
Below is a list of eleven basic rules of inference of Natural Deduction. 

\begin{enumerate}
\item \textit{Modus Ponens} (Conditional Elimination): From $P$ and $(P \Rightarrow Q)$, infer $Q$.
\item \textit{Modus Tollens}: From $\neg Q$ and $(P \Rightarrow Q)$, infer $\neg P$. 
\item \textit{Hypothetical Syllogism}: From $(P \Rightarrow Q)$ and $(Q \Rightarrow R)$, infer $P \Rightarrow R$. 
\item \textit{Disjunctive Syllogism}: From $(P \lor Q)$ and $\neg Q$ infer $P$. 
\item \textit{Constructive Dillema}: From $(P \Rightarrow Q) \land (R \Rightarrow S)$ and  $P \lor R$ infer  $Q \lor S$. 
\item \textit{Destructive Dillema}: From $(P \Rightarrow Q) \land (R \Rightarrow S)$ and $\neg Q \lor \neg S$, infer $\neg P \lor \neg R$. 
\item \textit{Simplification}: From $(P \land Q)$ infer $P$. 
\item \textit{Conjunction}: From $P$ and $Q$, infer $(P \land Q)$. 
\item \textit{Addition}: From $P$, infer $(P \lor Q)$. 
\end{enumerate}

\begin{example}{A logical puzzle}
Charles Dodgson, more commonly known for his pen name, Lewis Carroll, was a Professor of Logic at Oxford University. He created many puzzles, whose goal was both to amuse and illustrate concepts in logic. The following is an example. 
\begin{center}
``My saucepans are the only things I have that are made of tin. \\
I find all your presents very useful. \\
None of my saucepans are of the slightest use." 
\end{center}

In this riddle, we are given three propositions, are our goal is to connect them, logically, to derive a conclusion. The domain of discourse, to which all objects belong, is that of ``the things I have". There are four predicates in the riddle:
\begin{enumerate}
\item $S$: it is my saucepan;
\item $T$: it is made of tin;
\item $P$: it is your present; 
\item $U$: it is useful.
\end{enumerate}

The statements can thus be represented by
\begin{align*}
\neg S & \Rightarrow \neg T \\
P & \Rightarrow U \\
S & \Rightarrow \neg U.
\end{align*}
This riddle has two loose ends, the predicates $P$ and $T$, which appear only once in the statements above. A solution to the riddle is given by connecting these via tautologies, 
\begin{align*}
P & \Rightarrow U, & \ \mbox{given,} \\
& \Rightarrow \neg S, & \ \mbox{by Transposition and Modus Ponens},\\
& \Rightarrow \neg T, & \ \mbox{by Modus Ponens.}
\end{align*}

\end{example}


\section{Rules of Replacement}
A rule of replacement is a tautology involving biconditionals. 

\begin{enumerate}
\item \textit{Double Negation Law}: $p \Leftrightarrow \neg \neg p$. 
\item \textit{Commutative Laws}: $p \lor q \Leftrightarrow q \lor p$, and $p \land q \Leftrightarrow q \land p$. 
\item \textit{Associative Laws}: $[(p \lor q) \lor r] \Leftrightarrow [p \lor (q \lor r)]$, and $[(p \land q) \land r] \Leftrightarrow [p \land (q \land r)]$.
\item \textit{Distributive Laws}: $[p \land (q \lor r)] \Leftrightarrow (p \land q) \lor (p \land r)$, and $[p \lor (q \land r)] \Leftrightarrow (p \land q) \lor (p \land r)$. 
\item \textit{DeMorgan's Laws}: $\neg (p \land q) \Leftrightarrow (\neg p) \lor (\neg q)$, and $\neg (p \lor q) \Leftrightarrow (\neg p) \land (\neg q)$. 
\end{enumerate}

\section{Compound tautologies}
Combinations of the basic tautotogies studied above lead to \textit{compound tautologies}. In this section, we will study a few such tautologies which are used often enough to warrant notice. 

\begin{enumerate}
\item \textit{Transposition}: From $(p \Rightarrow q)$, infer $(\neg q \Rightarrow \neg p)$. 
\item \textit{Material Implication}: From $(p \Rightarrow q)$, infer $(\neg p \lor q)$. 
\item \textit{Material Equivalence(1)}: From $(p\Leftrightarrow q)$, infer $[(p \Rightarrow q) \land (q \Rightarrow p)]$.
\item \textit{Material Equivalence (2)}: From $(p \Leftrightarrow q)$, infer $[(p \land q) \lor (\neg p \land \neg q)]$.
\item \textit{Material Equavalence (3)}: From $(p \Leftrightarrow q)$, infer $[([p\lor \neg q) \land (\neg p \lor q)]$.
\item \textit{Exportation}: From $((p\land q) \Rightarrow r)$, infer $(p \Rightarrow (q \Rightarrow r))$.
\item \textit{Importation}: From $(p \Rightarrow (q \Rightarrow r))$, infer $((p\land q) \Rightarrow r)$
\end{enumerate}

\begin{example}
Let us derive \textit{Material Implication} from the basic rules. We are given $p \Rightarrow q$, and may take from granted the \textit{Law of Excluded Middle}, the tautology $\neg p \lor p$. Thus, we infer 
\begin{align*}
(\neg p \lor p) \land (p \Rightarrow q),  & \ \mbox{by Conjunction},\\
[\neg p \land (p \Rightarrow q)] \lor [p \land (p \Rightarrow q)], & \ \mbox{by Distributivity},\\
\neg p \lor q, & \ \mbox{by Simplification and Modus Ponens}.
\end{align*}
\end{example}

\begin{example}
Let us derive the \textit{Law of Transpotion} from the basic forms, and Material Equivalence. The assumption is 
\begin{equation*}
p \Rightarrow q,
\end{equation*}
from which we infer
\begin{align*}
\neg p \lor q, & \ \mbox{by Material Implication},\\
q \lor (\neg p), & \ \mbox{by Commutativity}, \\
\neg (\neg q) \lor (\neg p), & \ \mbox{by the Double Negation Law}
\end{align*}
\end{example}
\section{Proofs}


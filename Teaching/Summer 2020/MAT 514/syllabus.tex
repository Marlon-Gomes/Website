\documentclass[11pt]{amsart}
\usepackage{amssymb,amsfonts,amsthm,amsmath}
\usepackage[english]{babel}
\usepackage[all,cmtip]{xy}
\usepackage{hyperref}
\usepackage{url}
%\usepackage{mathrsfs}
%\usepackage[notcite,notref]{showkeys}
\def\limproj{\mathop{\oalign{lim\cr\hidewidth$\longleftarrow$\hidewidth\cr}}}

%--------------
% macro perso
%--------------
\newcommand{\inputc}[1]{ \raisebox{-0.5\height}{\input{#1}} }
\newtheorem{theorem}{Theorem}[section]
\newtheorem{lemma}[theorem]{Lemma}
\newtheorem{corollary}[theorem]{Corollary}
\newtheorem{definition}[theorem]{Definition}
\newtheorem{proposition}[theorem]{Proposition}
\newtheorem{remark}[theorem]{Remark}
\newtheorem{example}{Example}[section]
%\newtheorem{theorem*}{Theorem}

\newcommand{\func}[3]{{#1} : {#2} \longrightarrow {#3}}
\numberwithin{equation}{section}

\newenvironment{myproof}{\noindent{it Proof}
\setlength{\parindent}{0mm}}
{$\hfill \bs$}

\title[MAT 514 - Summer 2020 Syllabus]{MAT 514 Syllabus - Summer Session II 2020}
\author[M. Gomes]{Marlon Gomes}
%

\begin{document}
\maketitle


\section{Course Description}
\subsection{Course goal}
MAT 514 is a course in complex analysis for teachers. In this course, we will study the theory of functions of a single complex variable, particularly those which admit a complex derivative. We will contrast the behavior of such functions with that of functions of one and two real variables, and we will develop techniques that illustrate the surplus power of calculus in the complex variable setting when it is applied to certain problems in the real variable setting. By way of preparation for the main results of the course, we will review the algebra and topology of the complex plane, contrasting where possible to the real setting in one and two variables. The fundamental theorem of line integrals and Green's theorem for functions of two real variables will be important ingredients for this course - ideally, these will be treated as assumed knowledge but we will also recall them.\\

\subsection{Prerequisite}
MAT 511 or equivalent. Students can find the syllabus for this pre-requisite course at 
\begin{center}
\url{https://www.stonybrook.edu/sb/graduatebulletin/current/courses/mat/#511}
\end{center}

\subsection{Textbook}
\textbf{Textbooks}: M. Beck, G. Marchesi, D. Pixton, L. Sabalka, \textit{A First Course in Analysis} (Required). The textbook is available for free at 
\begin{center}
\url{http://math.sfsu.edu/beck/complex.html}.
\end{center}

\subsection{Important Times and Dates}
\begin{itemize}
\item Synchronous lectures: Tuesdays and Thursdays, 1:30pm - 3:15 pm, via Zoom.
\item Asynchronous lectures: accesible via Zoom on Wednesdays. Scheduled to be released at noon.
\item Long-form quizzes: July 16, 23, 30 and August 6, between 4:00pm and 4:55pm. 
\item Term paper due: August 13, at 11:59pm.
\end{itemize}

\subsection{Lecture format}
\noindent This usual format for this summer course is to have two 3.5 hour classroom lectures per week for 6 weeks. Since this summer's iteration will not be delivered in a classroom, we instead adopt the following format:
\begin{itemize}
\item 7 hours of lecture will be delivered online via \textbf{Zoom}. Half of this will be sychronous and half will be asychronous.
\begin{itemize}
\item The sychronous half will consist of two live and interactive Zoom lectures, each 1 hour and 45 minutes, which the students will be expected to attend and participate in.
\item The asychronous part will consist of two 1.5 hour recorded lectures, made available to be watched on the day in between the live lectures, which the students will be expected to make notes on to refer to during the live sessions.
\end{itemize}
\item The expectation will be that with the recorded lectures, students will have the ability to pause and/or rewind, so this will be the appropriate medium to deliver more of the theoretical aspect of the course.
\item In the live lectures, the plan will be to begin with an easy 5 minute live quiz on the content of the most recent recording, and then to proceed with a recap of the theory covered in that recording followed by a problem solving oriented session with discussion.
\end{itemize}

\subsection{Course management}

This course will be taught online, using Blackboard as the primary course management toll. Additionally, students may find information on the course webpage
\begin{center}
\url{http://www.math.stonybrook.edu/~mgomes/mat514sum20.html}.
\end{center}

A detailed course schedule, homework management, exams, will be primarily managed via Blackboard. We will use Zoom for lectures and office hours. Instructions on how to set up your Zoom account, as well as accessing the lectures will be posted to the course Blackboard page. E-mail will be an important tool, and students are expected to check for updates regularly.

\section{Office Hours and Instructor Contact}

The students will be able to reach the instructor during two weekly office hours held in the instructor's Personal Meeting Room on Zoom. Meeting ID and password to be provided separately via Blackboard announcement. For direct contact with the Instructor, students may use Blackboard communication tools of my e-mail address,
\begin{center}
\href{mailto: mgomes@math.stonybrook.edu}{ mgomes@math.stonybrook.edu}.
\end{center}

\section{Grading, Homework, Quizzes and Exams}

\noindent Credit for the course will be divided into the following components:
\begin{itemize}
\item 12 live ``in-class" quizzes (20\%)
\item 5 written homework sets (30\%)
\item Term paper (10\%)
\item 4 long-form quizzes (40\%)
\end{itemize}
\textbf{The term paper} will be expected to be an 8-10 page mathematical-historical paper on a topic in (the development of) complex analysis. The precise topic will be the result of a consultation between the instructor and each student individually. Independently of the topic, the paper will be expected to contain a precise statement of a mathematical theorem not covered in the course material as well as a sound mathematical argument for it.\\
\\
\textbf{The long-form quizzes} will each be take-home and open-book format. They will each consist of roughly 3 exam-level problems and students should take about 40 to 45 minutes to complete them. This will be achieved by posting them online at a set time, and requesting digital submission of solutions within 55 minutes (the additional time to account for expected digital processing).

\section{Course content - weekly schedule}
The following is an approximate schedule for weeks 1-6:

\begin{center}
\begin{tabular}{p{0.15\textwidth}|p{0.75\textwidth}}
\hline 
Week of & Subject \\ 
\hline 
07/06 & The fundamental theorem of line integrals and Green's theorem for functions of two real variables. Algebra, topology of the complex plane.  \\ 
\hline 
07/13 & Limits, continuity, holomorphicity and the Cauchy-Riemann equations. Examples of holomorphic functions and their derivatives.  \\ 
\hline 
07/20 & Antiderivatives, Cauchy's Theorem and Cauchy's integral formula. \\ 
\hline 
07/27 & Liouville's theorem, fundamental theorem of algebra, harmonic functions and their mean-value and minimum/maximum principles. \\ 
\hline 
08/03 & Power series, Taylor series and Laurent series. Isolated singularities.  \\ 
\hline 
08/10 & The Residue Theorem and applications.\\ 
\hline 
\end{tabular} 
\end{center}

\section{Student Accessibility Support Center Statement}

If you have a physical, psychological, medical, or learning disability that may impact your course work, please contact the Student Accessibility Support Center, 128 ECC Building, (631) 632-6748, or via e-mail at: 
\begin{center}
\href{mailto: sasc@stonybrook.edu}{sasc@stonybrook.edu}.
\end{center} 
They will determine with you what accommodations are necessary and appropriate. All information and documentation is confidential.

\section{Academic Integrity Statement}

Each student must pursue his or her academic goals honestly and be personally accountable for all submitted work. Representing another person's work as your own is always wrong. Faculty is required to report any suspected instances of academic dishonesty to the Academic Judiciary. Faculty in the Health Sciences Center (School of Health Technology \& Management, Nursing, Social Welfare, Dental Medicine) and School of Medicine are required to follow their school-specific procedures. For more comprehensive information on academic integrity, including categories of academic dishonesty please refer to the academic judiciary website at
\begin{center}
\url{http://www.stonybrook.edu/commcms/academic_integrity/index.html}
\end{center}

\section{Critical Incident Management}

Stony Brook University expects students to respect the rights, privileges, and property of other people. Faculty are required to report to the Office of University Community Standards any disruptive behavior that interrupts their ability to teach, compromises the safety of the learning environment, or inhibits students' ability to learn. Faculty in the HSC Schools and the School of Medicine are required to follow their school-specific procedures. Further information about most academic matters can be found in the Undergraduate Bulletin, the Undergraduate Class Schedule, and the Faculty-Employee Handbook.
\end{document}
\documentclass[12pt,oneside]{exam}

% This package simply sets the margins to be 1 inch.
\usepackage[margin=1in]{geometry}

% These packages include nice commands from AMS-LaTeX
\usepackage{amssymb,amsmath,amsthm,amsfonts,latexsym,verbatim,xspace,setspace}
\usepackage{hyperref}
\usepackage{graphicx}

% Make the space between lines slightly more
% generous than normal single spacing, but compensate
% so that the spacing between rows of matrices still
% looks normal.  Note that 1.1=1/.9090909...
\renewcommand{\baselinestretch}{1.1}
\renewcommand{\arraystretch}{.91}

% Define environments for exercises.
\newenvironment{exercise}[1]{\vspace{.1in}\noindent\textbf{Problem #1 \hspace{.05em}}}{}
\newenvironment{newsolution}{\vspace{.1in}\noindent\textbf{Solution: \hspace{.05em}}}{}

% define shortcut commands for commonly used symbols
\newcommand{\R}{\mathbb{R}}
\newcommand{\C}{\mathbb{C}}
\newcommand{\Z}{\mathbb{Z}}
\newcommand{\Q}{\mathbb{Q}}
\newcommand{\N}{\mathbb{N}}
\newcommand{\calP}{\mathcal{P}}
\DeclareMathOperator{\sech}{sech}
\DeclareMathOperator{\csch}{csch}
\DeclareMathOperator{\vsspan}{span}
\newcommand{\func}[3]{{#1} \colon {#2} \longrightarrow {#3}} 

\title{Math 514 - Summer II 2020: Homework 1}

%%%%%%%%%%%%%%%%%%%%%%%%%%%%%%%%%%%%%%%%%%

\begin{document}

\begin{flushright}
\sc MAT 514 - Summer II 2020\\
July 28, 2020
\end{flushright}
\bigskip
 
\begin{center}
\textsf{Solutions to Homework 3} 
\end{center}

%%%%%%%%%%%%%%%%%%%%%%%%%%%%%%%%%%%%%%%%

\begin{exercise}{3.44(d)}
Determine all complex numbers for which the function $f(z)=\exp(\overline{z})$ is holomorphic.
\end{exercise}

\vspace{0.5cm}

\noindent \textbf{Solution:} Let us write this function in terms of the real and imaginary parts of $z$, 
\begin{align*}
f(x+iy) & = \exp(x-iy) \\
& = e^{x}[\cos(-y)+i\sin(-y)]\\
& = e^{x}[\cos(y)-i\sin(y)],
\end{align*}
thus $f$ has real and imaginary parts 
\begin{equation*}
u(x,y)=e^{x}\cos(y), \, \, v(x,y) = -e^{x}\sin(y),
\end{equation*}
respecively. The Cauchy-Riemann equations are thus
\begin{align*}
e^{x}\cos(y) & = -e^{x}\cos(y) \\
-e^{x}\sin(y) & = e^{x}\cos(y).
\end{align*}
This set of equations has no solutions, thus $f$ is not complex-differentiable anywhere. 

\vspace{1cm}

\begin{exercise}{3.45(b)}
Find all solutions to the equation 
\begin{equation*}
\mathrm{Log}(z)=\frac{3\pi i}{2}
\end{equation*}
\end{exercise}

\vspace{0.5cm}
\noindent \textbf{Solution:} This equation has no solutions, as the argument of the principal branch of the logarithm is constrained between $-\pi$ and $\pi$. 
\vspace{1cm}

\begin{exercise}{4.3}
Integrate the function $f(z)=\overline{z}$ over the paths given below
\begin{itemize}
\item[(a)] $\gamma(t) = t+it$, $0\leq t \leq 1$. 
\item[(b)] $\gamma(t) = t + it^2$, $0 \leq t \leq 1$. 
\item[(c)] The path $\xi$, juxtaposition of the paths $\gamma_1(t) =t$, $0\leq t \leq 1$, and $\gamma_2(t) = 1+it$, $0 \leq t \leq 1$. 
\end{itemize}
\end{exercise}

\vspace{0.5cm}
\noindent \textbf{Solution:} 
\begin{itemize}
\item[(a)] The velocity vector of the path is $\gamma^{'}(t)=1+i$, hence the integral is 
\begin{align*}
\int_{\gamma} \overline{z} \, dz & = \int_{0}^{1} (t-it)(1+i) \, dt \\
& = \int_{0}^{1}(t+it-it-i^2t) \, dt \\
& = \int_{0}^{1} 2t \, dt\\
& = 1
\end{align*}
\item[(b)] The velocity vector of the path is $\gamma^{'}(t) = 1 + 2it$, thus the integral is 
\begin{align*}
\int_{\gamma} \overline{z} \, dz & = \int_{0}^{1} (t-it^2)(1+2it) \, dt \\
& = \int_{0}^{1} (t+2it^2-it^2-2i^2t^3) \, dt \\
& = \frac{t^2}{2} + \frac{it^3}{3}+\frac{t^4}{2} \Big|_{t=0}^{t=1} \\
& = 1+\frac{i}{3}.
\end{align*}
\item[(c)] This integral is split the sum of the integrals in the two paths, whose respective derivatives are $\gamma_{1}^{'}(t) = 1$ and $\gamma_{2}^{'}(t) = i$. 
\begin{align*}
\int_{\xi} \overline{z} \, dz & = \int_{\gamma_{1}} \overline{z} \, dz  + \int_{\gamma_{2}} \overline{z} \, dz \\
& = \int_{0}^{1} t \, dt + \int_{0}^{1} (1-it)i \, dt \\
& = \int_{0}^{1} (2t+i) \, dt \\
& = t^2+it \Big|_{t=0}^{t=1} \\
& = 1+i.
\end{align*}
\end{itemize}

\vspace{1cm}

\begin{exercise}{4.4}
Compute $\int_{\gamma} \frac{dz}{z}$, where $\gamma$ is the unit circle oriented counterclockwise. More generally, show that for any $w \in \mathbb{C}$ and $r>0$, 
\begin{equation*}
\int_{C[w,r]} \frac{dz}{w-z} = 2\pi i.
\end{equation*}
\end{exercise}

\vspace{0.5cm}

\noindent \textbf{Solution:} 
Let us parametrize the unit circle by $\gamma(t) = e^{it}$, $0 \leq t \leq 2\pi$, with velocity $\gamma'(t) = ie^{it}$. The first integral may be evaluated as 
\begin{equation*}
\int_{\gamma} \frac{dz}{z} = \int_{0}^{2\pi} \frac{ie^{it}}{e^{it}}  \, dt = \int_{0}^{1} i \, dt = 2\pi i.
\end{equation*}
More generally, consider the circle of center $w$ and radius $r$, oriented counterclockwise, parametrized by $\xi(t) = w + re^{it}$, for $0 \leq t \leq 2\pi$, with velocity $\xi'(t) = rie^{it}$. The second integral may be evaluated as
\begin{equation*}
\int_{C[w,r]} \frac{dz}{w-z} = \int_{0}^{2\pi} \frac{rie^{it}}{w+re^{it}-w}= \, dt = \int_{0}^{2\pi} i \, dt = 2\pi i.
\end{equation*}

\vspace{1cm}

\begin{exercise}{4.10}
Prove the following integration by parts statement: let $f$ and $g$ be holomorphic in $G$, and suppose $\gamma \subset G$ is a piecewise smooth path from $\gamma(a)$ to $\gamma(b)$. Then 
\begin{equation*}
\int_{\gamma} fg' = f(\gamma(b))g(\gamma(b))-f(\gamma(a))g(\gamma(a)) - \int_{\gamma}f'g 
\end{equation*}
\end{exercise}

\vspace{0.5cm}

\noindent \textbf{Solution:} If $f$ and $g$ are holomorphic, so is their product $fg$, whose derivative is 
\begin{equation*}
(fg)'=f'g+fg'.
\end{equation*}
Said it another way, $(fg)$ is an antiderivative of $(f'g+fg')$. Applying theorem $4.11$ to $(f'g+fg')$, we obtain the desired conclusion,
\begin{align*}
\int_{\gamma} f'g+fg' & = (fg)(\gamma(b))-(fg)(\gamma(a)) \\
\int_{\gamma} fg' & = f(\gamma(b))g(\gamma(b))-f(\gamma(a))g(\gamma(a))-\int_{\gamma} f'g.
\end{align*}
\end{document}


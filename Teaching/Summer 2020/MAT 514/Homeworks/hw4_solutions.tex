\documentclass[12pt,oneside]{exam}

% This package simply sets the margins to be 1 inch.
\usepackage[margin=1in]{geometry}

% These packages include nice commands from AMS-LaTeX
\usepackage{amssymb,amsmath,amsthm,amsfonts,latexsym,verbatim,xspace,setspace}
\usepackage{hyperref}
\usepackage{graphicx}

% Make the space between lines slightly more
% generous than normal single spacing, but compensate
% so that the spacing between rows of matrices still
% looks normal.  Note that 1.1=1/.9090909...
\renewcommand{\baselinestretch}{1.1}
\renewcommand{\arraystretch}{.91}

% Define environments for exercises.
\newenvironment{exercise}[1]{\vspace{.1in}\noindent\textbf{Problem #1 \hspace{.05em}}}{}
\newenvironment{newsolution}{\vspace{.1in}\noindent\textbf{Solution: \hspace{.05em}}}{}

% define shortcut commands for commonly used symbols
\newcommand{\R}{\mathbb{R}}
\newcommand{\C}{\mathbb{C}}
\newcommand{\Z}{\mathbb{Z}}
\newcommand{\Q}{\mathbb{Q}}
\newcommand{\N}{\mathbb{N}}
\newcommand{\calP}{\mathcal{P}}
\DeclareMathOperator{\sech}{sech}
\DeclareMathOperator{\csch}{csch}
\DeclareMathOperator{\vsspan}{span}
\newcommand{\func}[3]{{#1} \colon {#2} \longrightarrow {#3}} 

\title{Math 514 - Summer II 2020: Homework 1}

%%%%%%%%%%%%%%%%%%%%%%%%%%%%%%%%%%%%%%%%%%

\begin{document}

\begin{flushright}
\sc MAT 514 - Summer II 2020\\
August 11, 2020
\end{flushright}
\bigskip
 
\begin{center}
\textsf{Solutions to Homework 4} 
\end{center}

%%%%%%%%%%%%%%%%%%%%%%%%%%%%%%%%%%%%%%%%

\begin{exercise}{5.1(d)}
Compute the integral
\begin{equation*}
\int_{\Box} \frac{\exp(z)\cos(z)}{(z-\pi)^3}\, dz
\end{equation*}
where $\Box$ is the boundary of the square with vertices at $\pm 4 \pm 4i$, positively oriented.
\end{exercise}

\vspace{0.5cm}

\noindent \textbf{Solution:} The path of integration $\Box$ is $\mathbb{C}\setminus \{\pi\}$-homotopic to a circle centered at $\pi$, $C(\pi, r)$, for any radius $r>0$. We may therefore move the integration to such circles and employ Cauchy's Formulas for Derivatives, 
\begin{align*}
\int_{\Box} \frac{\exp(z)\cos(z)}{(z-\pi)^3}\, dz & = \int_{C[\pi,r]} \frac{\exp(z)\cos(z)}{(z-\pi)^3} \\
& = \pi i [\exp(z)\cos(z)]^{''}(\pi) \\
& = -2\pi i e^{\pi}\sin(\pi) \\
& = 0.
\end{align*}

\vspace{1cm}

\begin{exercise}{5.3(h)}
Integrate 
\begin{equation*}
\frac{1}{(z+4)(z^2+1)}
\end{equation*}
over the circle $C[0,3]$. 
\end{exercise}

\vspace{0.5cm}
\noindent \textbf{Solution:} This problem may be solved in two ways: by splitting the curve of integration, or via a partial fractions decomposition. We'll take the second approach here. Decomposing the integrand into its partial components, 
\begin{equation*}
\frac{1}{(z+4)(z^2+1)} = \frac{A}{z+4} + \frac{B}{z+i} + \frac{C}{z-i},
\end{equation*}
we find a system of linear equations on $A,B,C$:
\begin{align*}
 A + B + C & = 0 \\
 (4-i)B + (4+i)C & = 0 \\
 -4Bi +4Ci & = 1,
\end{align*}
whose solution is $A=\frac{1}{16}$, $B=-\frac{1}{32} +\frac{i}{8}$, $C = -\frac{1}{32}-\frac{i}{8}$. It follows that 
\begin{align*}
\int_{C[0,3]} \frac{1}{(z+4)(z^2+1)} \, dz & = \frac{1}{16}\int_{C[0,3]} \frac{1}{z+4}\, dz + \left(-\frac{1}{32} +\frac{i}{8}\right)\int_{C[0,3]} \frac{1}{z+i}\, dz \\
& \ \ \ \  + \left( -\frac{1}{32}-\frac{i}{8}\right) \int_{C[0,3]} \frac{1}{z-i}\, dz \\
& = 0 + \left(-\frac{1}{32} +\frac{i}{8}\right)\int_{C[0,3]} 2\pi i + \left( -\frac{1}{32}-\frac{i}{8}\right) \int_{C[0,3]} 2\pi i \\
& = -\frac{\pi i}{8}.
\end{align*}
The values of the integrals on the right-hand side were computed by means of Cauchy's Theorem and Cauchy's Integral Formula: the first integrand, $\frac{1}{z+4}$, is holomorphic on the disk $\overline{D[0,3]}$, hence its integral along the boundary is $0$; the latter two integrands contain an isolated singularity in $D[0,3]$, and have numerators equal to 1, hence their integrals are equal to $2\pi i$. 
\vspace{1cm}

\begin{exercise}{5.12}
Show that a polynomial of odd degree with real conefficients must have a real zero.
\end{exercise}

\vspace{0.5cm}
\noindent \textbf{Solution:} 
Let $p(z)$ be such a polynomial, 
\begin{equation*}
p(z) = a_0 + a_1z + a_2z^2 + \cdots + a_nz^n,
\end{equation*}
where all the $a_i$ are real, $a_n \neq 0$, and $n$ is odd. Then, for any complex number $z$, 
\begin{align*}
p(\overline{z}) & = a_0 + a_1\overline{z} + a_2(\overline{z})^2+\cdots + a_n(\overline{z})^n \\
& = \overline{a_0} + \overline{a_1z} + \overline{a_2z^2} + \cdots + \overline{a_nz^n} \\
& = \overline{a_0 + a_1z + a_2z^2 + \cdots + a_nz^n} \\
& \overline{p(z)}.
\end{align*}
It follows that if $z$ is a root of $p$, then so is $\overline{z}$. This has implications when $z$ is a complex, non-real number, in which case $z$ and $\overline{z}$ are distinct. In summary, complex, non-real roots to a polynomial with real coefficients come in pairs. 

An odd degree polynomial has an odd number of complex roots, according to the Fundamental Theorem of Algebra. If such a polynomial has real coefficients, at least one such root must be real, so as to not contradict the conclusion from the last paragraph. 
\vspace{1cm}

\begin{exercise}{5.13}
Suppose f is entire and $|f(z)| \leq \sqrt{|z|}$ for all $z \in \mathbb{C}$. Prove that $f$ is identically $0$. 
\end{exercise}

\vspace{0.5cm}

\noindent \textbf{Solution:} 
We will follows a similar approach to the one employed in the proof of Liouville's Theorem. Let $w \in \mathbb{C}$ be an arbitrary point. By Cauchy's Integral Formulas for Derivatives, 
\begin{equation*}
f'(w) = \frac{1}{2\pi i}\int_{C[w,r]} \frac{f(z)}{(z-w)^2} \, dz,
\end{equation*}
for all $r>0$. We will use the estimate provided in the statement to bound the value of $f'(w)$, 
\begin{align*}
|f'(w)| & \leq \frac{1}{2\pi}\int_{C[w,r]} \frac{|f(z)|}{|z-w|^2}\, dz \\
& \leq \frac{1}{2\pi} \int_{C[w,r]} \frac{\sqrt{|z|}}{r^2}\, dz \\
& \leq \frac{1}{2\pi} \int_{C[w,r]} \frac{\sqrt{|w|+r}}{r^2}\, dz\\
& \leq \frac{\sqrt{|w|+r}}{r}\\
& \leq \sqrt{\frac{|w|+r}{r^2}}.
\end{align*}
Since this bound is valid for all $r>0$, and the expression on the right-hand side converges to $0$ (regardless of $w$) as $r \to \infty$, we conclude that $f'(w)=0$, for all $w\in \mathbb{C}$, i.e., $f$ is constant. In particular, $|f(0)| \leq \sqrt{0} = 0$, the value of $f$ is $0$ on the entire plane.  

\vspace{1cm}

\begin{exercise}{5.15}
Suppose $f$ is entire with bounded real part, i.e., writing
\begin{equation*}
f(z)=u(z)+iv(z),
\end{equation*}
there exists $M>0$ such that 
\begin{equation*}
|u(z)| \leq M
\end{equation*}
for all $z \in \mathbb{C}$. Prove that $f$ is constant. 
\end{exercise}

\vspace{0.5cm}

\noindent \textbf{Solution:} 
Consider the entire function $g(z)=\exp(f(z)) = e^{u(z)}\cdot e^{iv(z)}$. The bound on the real part of $f$ implies a bound on $g$, 
\begin{equation*}
|g(z)| = |e^{u(z)} \leq e^{|u(z)|} \leq e^M.
\end{equation*}
It follows that $g$ is a bounded entire function, hence a constant. It follows that its derivative is zero, so 
\begin{equation*}
g'(z)=f'(z)\exp(f(z)) = 0,
\end{equation*}
thus $f'(z)=0$. This is valid for all $z \in \mathbb{C}$, therefore $f$ too is a constant function. 

\vspace{1cm}

\begin{exercise}{6.8}
Is it possible to find a real function $v(x,y)$ so that 
\begin{equation*}
x^3+y^3+iv(x,y)
\end{equation*}
is holomorphic?
\end{exercise}

\vspace{0.5cm}

\noindent \textbf{Solution:}
If such a function existed, $u(x,y) = x^3+y^3$ would be a harmonic function, as the real part of a holomorphic function. As it happens, 
\begin{equation*}
\Delta u = \frac{\partial^2 u}{\partial x^2} + \frac{\partial^2 u}{\partial y^2} = 6x+6y,
\end{equation*}
therefore $u$ is not harmonic, hence no such a function $v$ exists. 

\vspace{1cm}

\begin{exercise}{6.11}
Prove that, if $u$ is harmonic and bounded on $\C$, then $u$ is constant. 
\end{exercise}

\vspace{0.5cm}

\noindent \textbf{Solution:} 
As a harmonic function defined on a contractible domain, $u$ has a harmonic conjugate $\func{v}{\C}{\C}$ such that 
\begin{equation*}
f(z)=u(z)+iv(z)
\end{equation*}
is an entire function with bounded real part. By problem 5.13, $f$ must be constant. In particular, its real part $u$ is also constant.

\end{document}


\documentclass[12pt,oneside]{exam}

% This package simply sets the margins to be 1 inch.
\usepackage[margin=1in]{geometry}

% These packages include nice commands from AMS-LaTeX
\usepackage{amssymb,amsmath,amsthm,amsfonts,latexsym,verbatim,xspace,setspace}
\usepackage{hyperref}
\usepackage{graphicx}

% Make the space between lines slightly more
% generous than normal single spacing, but compensate
% so that the spacing between rows of matrices still
% looks normal.  Note that 1.1=1/.9090909...
\renewcommand{\baselinestretch}{1.1}
\renewcommand{\arraystretch}{.91}

% Define environments for exercises.
\newenvironment{exercise}[1]{\vspace{.1in}\noindent\textbf{Problem #1 \hspace{.05em}}}{}
\newenvironment{newsolution}{\vspace{.1in}\noindent\textbf{Solution: \hspace{.05em}}}{}

% define shortcut commands for commonly used symbols
\newcommand{\R}{\mathbb{R}}
\newcommand{\C}{\mathbb{C}}
\newcommand{\Z}{\mathbb{Z}}
\newcommand{\Q}{\mathbb{Q}}
\newcommand{\N}{\mathbb{N}}
\newcommand{\calP}{\mathcal{P}}
\DeclareMathOperator{\sech}{sech}
\DeclareMathOperator{\csch}{csch}
\DeclareMathOperator{\vsspan}{span}
\newcommand{\func}[3]{{#1} \colon {#2} \longrightarrow {#3}} 

\title{Math 514 - Summer II 2020: Homework 1}

%%%%%%%%%%%%%%%%%%%%%%%%%%%%%%%%%%%%%%%%%%

\begin{document}

\begin{flushright}
\sc MAT 514 - Summer II 2020\\
July 21, 2020
\end{flushright}
\bigskip
 
\begin{center}
\textsf{Solutions to Homework 2} 
\end{center}

%%%%%%%%%%%%%%%%%%%%%%%%%%%%%%%%%%%%%%%%

\begin{exercise}{2.2.(a)}
Evaluate the limit
\begin{equation*}
\lim_{z \to i} \frac{iz^3-1}{z+i},
\end{equation*}
or explain why it does not exist.
\end{exercise}

\vspace{0.5cm}

\noindent \textbf{Solution:} This limit exists, and it may be computed by direct substitution, as the function in question is continuous at $i$
\begin{equation*}
\lim_{z \to i} \frac{iz^3-1}{z+i} = \frac{i^4-1}{2i} = 0.
\end{equation*}

\vspace{1cm}

\begin{exercise}{2.6}
Proposition 2.2. is useful for showing that limits \textbf{do not} exist, but it is not at all useful for showing that a limit \textbf{does} exist. For example, define
\begin{equation*}
f(z) = \frac{x^2y}{x^4+y^2},
\end{equation*}
where$z=x+iy \neq 0$. Show that the limits of $f$ at $0$ along all straight lines through the origin exist and are equal, but
\begin{equation*}
\lim_{z\to 0} f(z)
\end{equation*}
does not exist (Hint: consider the limit along the parabola $y=x^2$).
\end{exercise}

\vspace{0.5cm}
\noindent \textbf{Solution:} Let $k \in \mathbb{R}$, and consider the line through the origin with slope $k$, $y=kx$. Along this line, for $x\neq 0$, the function reduces to 
\begin{align*}
\frac{x^2y}{x^4+y^2} & = \frac{x^2(kx)}{x^4+(kx)^2} \\
& = \frac{kx^3}{x^4+k^2x^2}\\
& = \frac{kx}{x^2+k^2}.
\end{align*}
The limit along such a line as $x\to 0$ is
\begin{equation*}
\lim_{x \to 0} \frac{kx}{x^2+k^2} = 0,
\end{equation*}
irrespective of $k$. 

Meanwhile, the limit along the parabola $y=x^2$ is 
\begin{equation*}
\lim_{x \to 0} \frac{x^2(x^2)}{x^4+(x^2)^2}  = \lim_{x \to 0} \frac{x^4}{2x^4} = \frac{1}{2}.
\end{equation*}
It follows that the limit of $f$ at $0$ does not exist, as it depends on the curve of approach. 
\vspace{1cm}

\begin{exercise}{2.6}
Consider the function $\func{f}{\mathbb{C}\setminus \{0\}}{\mathbb{C}}$, given by $f(z)=\frac{1}{z}$. Apply the definition of the derivative to give a direct proof that $f'(z)=-\frac{1}{z^2}$. 
\end{exercise}

\vspace{0.5cm}
\noindent \textbf{Solution:} Applying the definition via Newton quotients, 
\begin{align*}
\lim_{h \to 0} \frac{f(z+h)-f(z)}{h} & = \lim_{h \to 0} \frac{\frac{1}{z+h}-\frac{1}{z}}{h} \\
& =\lim_{h \to 0} \frac{\frac{z-(z+h)}{z(z+h)}}{h} \\
& =\lim_{h \to 0} \frac{-h}{zh(z+h)}\\
& = \lim_{h \to 0} -\frac{1}{z(z+h)}\\
& =-\frac{1}{z^2}.
\end{align*}

\vspace{1cm}

\begin{exercise}{2.15}
Find the derivative of the function $T(z) = \frac{az+b}{cz+d}$, where $a,b,c,d \in \C$, with $ad-bc\neq 0$. when is $T'(z)=0$?
\end{exercise}

\vspace{0.5cm}
\noindent \textbf{Solution:} Throughout the solution, we assume $cz+d \neq 0$. Applying the quotient rule, we obtain
\begin{align*}
T'(z) & = \frac{(az+b)'(cz+d)-(az+b)(cz+d)'}{(cz+d)^2}\\
& = \frac{a(cz+d)-(az+b)c}{(cz+d)^2}\\
& = \frac{ad-bc}{(cz+d)^2},
\end{align*}
hence, by assumption, $T'(z) \neq 0$, for all $z$ within its domain of definition.

\vspace{1cm}

\begin{exercise}{2.18}
Where are the following funcitons differentiable? Where are they holomorphic? Determine their derivatives at points where thery are differentiable.
\begin{itemize}
\item[(b)] $f(z)=2x+ixy^2$
\item[(f)] $f(z) = \Im(z)$
\item[(h)] $f(z)=z\Im(z)$
\item[(l)] $f(z)=z^2-(\overline{z})^2$
\end{itemize}
\end{exercise}

\noindent \textbf{Solution:}
\begin{itemize}
\item[(b)] The real and imaginary components of this function are $u(x+iy)=2x$, $v(x+y)=xy^2$, respectively. The corresponding Cauchy-Riemann equations are
\begin{align*}
2 & = 2xy \\
0 & = -y^2,
\end{align*}
a system without solutions. It follows that the function $f(z)=2x+ixy^2$ is nowhere complex-differentiable. 
\item[(f)] The real and imaginary components of this function are $u(x+y) = y$ and $v(x+iy)=0$. The corresponding Cauchy-Riemann equations are 
\begin{align*}
0 & = 1 \\
0 & = 0,
\end{align*}
a system without solutions. It follows that the function $f(z) = \Im(z)$ is nowhere-complex-differentiable.
\item[(h)] The real and imaginary parts of this function are $u(x+iy)=xy$, $v(x+iy) = y^2$, respectively. The corresponding Cauchy-Riemann equations are 
\begin{align*}
y & = 2y \\
x & = -0,
\end{align*}
a system whose only solution is $x+iy=0$. It follows that the function $f(z)=z\Im(z)$ is complex-differentiable at $0$, but nowhere holomorphic. Next we apply Newton quotients to compute the derivative at $0$, 
\begin{equation*}
f'(0) = \lim_{h \to 0} \frac{f(h)-f(0)}{h}.
\end{equation*}
We will use the fact that the function is differentiable at $0$, i.e. this limit exists, to choose a suitable direction for computation, say along the real axis. With this restriction, the limit becomes
\begin{align*}
f'(0) & = \lim_{h \to 0} \frac{f(h)-f(0)}{h}\\
& = \lim_{h \to 0} \frac{0}{h}\\
& = 0.
\end{align*} 
\item[(l)] The real and imaginary parts of this function are $u(x+iy)=0$, $v(x+iy)=4xy$. The corresponding Cauchy-Riemann equations are 
\begin{align*}
0 & = 4x \\
0 & =-4y,
\end{align*}
a system whose only solution is $x+iy=0$. It follows that the function $f(z)=z^2-(\overline{z})^2$ is complex-differentiable at $0$, but nowhere holomorphic.Next we apply Newton quotients to compute the derivative at $0$, 
\begin{equation*}
f'(0) = \lim_{h \to 0} \frac{f(h)-f(0)}{h}.
\end{equation*}
We will use the fact that the function is differentiable at $0$, i.e. this limit exists, to choose a suitable direction for computation, say along the real axis. With this restriction, the limit becomes
\begin{align*}
f'(0) & = \lim_{h \to 0} \frac{f(h)-f(0)}{h}\\
& = \lim_{h \to 0} \frac{0}{h}\\
& = 0.
\end{align*}
\end{itemize}

\vspace{1cm}

\begin{exercise}{2.25}
For each of the following funcitons $u$, find a function $v$ such that $u+iv$ is holomorphic in some region. Maximize that region. 
\begin{itemize}
\item[(a)] $u(x,y)=x^2-y^2$.
\item[(d)] $u(x,y) = \frac{x}{x^2+y^2}.$
\end{itemize}
\end{exercise}

\vspace{0.5cm}

\noindent \textbf{Solution:} 
\begin{itemize}
\item[(a)] The function has derivatives
\begin{equation*}
\frac{\partial u}{\partial x} = 2x, \, \, \frac{\partial u}{\partial y} = -2y.
\end{equation*}
If $u+iv$ is to be holomorphic, it must satify the Cauchy-Riemann equations, 
\begin{align*}
2x & = \frac{\partial v}{\partial y}\\
-2y & = -\frac{\partial v}{\partial x}.
\end{align*}
A simple solution to such equations, defined on the entire plane, is $v(x,y)=2xy$. Other solutions can be obtained by adding a (real) constant. 
\item[(d)] The function $u$ has derivatives
\begin{equation*}
\frac{\partial u}{\partial x} = \frac{y^2-x^2}{(x^2+y^2)^2}, \, \, \frac{\partial u}{\partial y}  = -\frac{2xy}{(x^2+y^2)^2}.
\end{equation*}
A companion funciton $v$ such that $u=iv$ is holomorphic must satisfy the Cauchy-Riemann equations, 
\begin{align*}
\frac{y^2-x^2}{(x^2+y^2)^2} & = \frac{\partial v}{\partial y}\\
-\frac{2xy}{(x^2+y^2)^2} & = - \frac{\partial v}{\partial x}.
\end{align*}
The function 
\begin{equation*}
v(x,y) = -\frac{y}{x^2+y^2}
\end{equation*}
is a solution, defined on $\mathbb{C} \setminus \{0\}$.
\end{itemize}
\end{document}


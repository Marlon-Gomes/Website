\documentclass[12pt,oneside]{exam}

% This package simply sets the margins to be 1 inch.
\usepackage[margin=1in]{geometry}

% These packages include nice commands from AMS-LaTeX
\usepackage{amssymb,amsmath,amsthm,amsfonts,latexsym,verbatim,xspace,setspace}
\usepackage{hyperref}
\usepackage{graphicx}

% Make the space between lines slightly more
% generous than normal single spacing, but compensate
% so that the spacing between rows of matrices still
% looks normal.  Note that 1.1=1/.9090909...
\renewcommand{\baselinestretch}{1.1}
\renewcommand{\arraystretch}{.91}

% Define environments for exercises.
\newenvironment{exercise}[1]{\vspace{.1in}\noindent\textbf{Problem #1 \hspace{.05em}}}{}
\newenvironment{newsolution}{\vspace{.1in}\noindent\textbf{Solution: \hspace{.05em}}}{}

% define shortcut commands for commonly used symbols
\newcommand{\R}{\mathbb{R}}
\newcommand{\C}{\mathbb{C}}
\newcommand{\Z}{\mathbb{Z}}
\newcommand{\Q}{\mathbb{Q}}
\newcommand{\N}{\mathbb{N}}
\newcommand{\calP}{\mathcal{P}}
\DeclareMathOperator{\sech}{sech}
\DeclareMathOperator{\csch}{csch}
\DeclareMathOperator{\vsspan}{span}

\newcommand{\func}[3]{{#1} : {#2} \longrightarrow {#3}}

\title{Math 514 - Summer II 2020: Long Quiz 3}

%%%%%%%%%%%%%%%%%%%%%%%%%%%%%%%%%%%%%%%%%%

\begin{document}

\begin{flushright}
\sc MAT 514 - Summer II 2020\\
July 30, 2020
\end{flushright}
\bigskip
 
\begin{center}
\textsf{Long Quiz 3} 
\end{center}

%%%%%%%%%%%%%%%%%%%%%%%%%%%%%%%%%%%%%%%%
The objective of this quiz is to use methods of complex integration to solve a real integral, 
\begin{equation*}
\int_{0}^{2\pi} \frac{1}{2+\sin(\phi)}\, d\phi.
\end{equation*}
The problems below will guide you through the solution. 

\begin{exercise}{1}
By expressing the sine function as a combination of complex exponentials, rewrite the integrand as a function of $e^{i\phi}$. 
\end{exercise}

\vfill

\begin{exercise}{2}
Use the substitution $z=e^{i\phi}$ to turn the real-valued integral into a line integral of a rational function in the complex variable $z$. Your answer should take the form
\begin{equation*}
\int_{C[0,1]} \frac{A}{p(z)} \, dz,
\end{equation*}
where $A$ is a constant and $p(z)$ is a quadratic polynomial on $z$.
\end{exercise}

\vfill

\newpage

\begin{exercise}{3}
Factor the polynomial $p(z)$ found in problem 2. Write the integrand from problem 2 as a sum of partial fractions whose denominators have degree one. 
\end{exercise}

\vfill

\begin{exercise}{4}
Use Cauchy's Theorem and Cauchy's integral formula to solve the integral 
\begin{equation*}
\int_{0}^{2\pi} \frac{1}{2+\sin(\phi)}\, d\phi.
\end{equation*}
via the method developed in problems 1 through 3. 
\end{exercise}

\vfill

\end{document}


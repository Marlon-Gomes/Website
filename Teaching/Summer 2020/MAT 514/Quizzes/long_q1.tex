\documentclass[12pt,oneside]{exam}

% This package simply sets the margins to be 1 inch.
\usepackage[margin=1in]{geometry}

% These packages include nice commands from AMS-LaTeX
\usepackage{amssymb,amsmath,amsthm,amsfonts,latexsym,verbatim,xspace,setspace}
\usepackage{hyperref}
\usepackage{graphicx}

% Make the space between lines slightly more
% generous than normal single spacing, but compensate
% so that the spacing between rows of matrices still
% looks normal.  Note that 1.1=1/.9090909...
\renewcommand{\baselinestretch}{1.1}
\renewcommand{\arraystretch}{.91}

% Define environments for exercises.
\newenvironment{exercise}[1]{\vspace{.1in}\noindent\textbf{Problem #1 \hspace{.05em}}}{}
\newenvironment{newsolution}{\vspace{.1in}\noindent\textbf{Solution: \hspace{.05em}}}{}

% define shortcut commands for commonly used symbols
\newcommand{\R}{\mathbb{R}}
\newcommand{\C}{\mathbb{C}}
\newcommand{\Z}{\mathbb{Z}}
\newcommand{\Q}{\mathbb{Q}}
\newcommand{\N}{\mathbb{N}}
\newcommand{\calP}{\mathcal{P}}
\DeclareMathOperator{\sech}{sech}
\DeclareMathOperator{\csch}{csch}
\DeclareMathOperator{\vsspan}{span}

\newcommand{\func}[3]{{#1} : {#2} \longrightarrow {#3}}

\title{Math 514 - Summer II 2020: Long Quiz 1}

%%%%%%%%%%%%%%%%%%%%%%%%%%%%%%%%%%%%%%%%%%

\begin{document}

\begin{flushright}
\sc MAT 514 - Summer II 2020\\
July 16, 2020
\end{flushright}
\bigskip
 
\begin{center}
\textsf{Long Quiz 1} 
\end{center}

%%%%%%%%%%%%%%%%%%%%%%%%%%%%%%%%%%%%%%%%

\begin{exercise}{1}
Determine the real and imaginary parts\footnote{We will use the symbols $\Re(z), \Im(z)$ for the real and imaginary parts of a complex number $z$, respectively.} of the complex number 
\begin{equation*}
z = \frac{3i+2}{12+5i}
\end{equation*}
\end{exercise}

\vfill
\begin{exercise}{2}
Find the conjugate, norm, and polar angle of the complex number
\begin{equation*}
z = \frac{4}{\sqrt{3}-i}
\end{equation*}
\end{exercise}
\vfill

\newpage

\begin{exercise}{3}
Sketch the set 
\begin{equation*}
\{z \in \C | \Re(z^2) \leq 0 \}
\end{equation*}
in the complex plane. Determine its basic topological properties: is it open, closed, or neither; bounded; compact; connected?
\end{exercise}
\vfill 

\begin{exercise}{4}
Determine whether the limit below exists:
\begin{equation*}
\lim_{z \to (1-i)} \Re(z)+i(2\Re(z) + \Im(z)).
\end{equation*}
If it exists, compute it. Otherwise, explain why it does not exist. 
\end{exercise}

\vfill
\end{document}


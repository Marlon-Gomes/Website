\documentclass[12pt,oneside]{exam}

% This package simply sets the margins to be 1 inch.
\usepackage[margin=1in]{geometry}

% These packages include nice commands from AMS-LaTeX
\usepackage{amssymb,amsmath,amsthm,amsfonts,latexsym,verbatim,xspace,setspace}
\usepackage{hyperref}
\usepackage{graphicx}

% Make the space between lines slightly more
% generous than normal single spacing, but compensate
% so that the spacing between rows of matrices still
% looks normal.  Note that 1.1=1/.9090909...
\renewcommand{\baselinestretch}{1.1}
\renewcommand{\arraystretch}{.91}

% Define environments for exercises.
\newenvironment{exercise}[1]{\vspace{.1in}\noindent\textbf{Problem #1 \hspace{.05em}}}{}
\newenvironment{newsolution}{\vspace{.1in}\noindent\textbf{Solution: \hspace{.05em}}}{}

% define shortcut commands for commonly used symbols
\newcommand{\R}{\mathbb{R}}
\newcommand{\C}{\mathbb{C}}
\newcommand{\Z}{\mathbb{Z}}
\newcommand{\Q}{\mathbb{Q}}
\newcommand{\N}{\mathbb{N}}
\newcommand{\calP}{\mathcal{P}}
\DeclareMathOperator{\sech}{sech}
\DeclareMathOperator{\csch}{csch}
\DeclareMathOperator{\vsspan}{span}

\newcommand{\func}[3]{{#1} : {#2} \longrightarrow {#3}}

\title{Math 514 - Summer II 2020: Quiz 6}

%%%%%%%%%%%%%%%%%%%%%%%%%%%%%%%%%%%%%%%%%%

\begin{document}

\begin{flushright}
\sc MAT 514 - Summer II 2020\\
July 28, 2020
\end{flushright}
\bigskip
 
\begin{center}
\textsf{Solutions to Quiz 6} 
\end{center}

%%%%%%%%%%%%%%%%%%%%%%%%%%%%%%%%%%%%%%%%

\begin{exercise}{1}
Compute the integral
\begin{equation*}
\int_{\gamma} \frac{z+1}{z}	\, dz,
\end{equation*}
where $\gamma$ is the line segment joining $4$ to $4i$.
\end{exercise}

\vspace{0.5cm}

\noindent \textbf{Solution:} The function $f(z)=\frac{z+1}{z}$ is singular at the origin, but otherwise holomorphic. An antiderivative is 
\begin{equation*}
F(z) = z + \mathrm{Log}(z).
\end{equation*}
This anti-derivative is defined and holomorphic in $\mathbb{C}\setminus \mathbb{R}_{\leq 0}$, a domain which includes the path of integration. It follows from Theorem 4.11 in our textbook that 
\begin{align*}
\int_{\gamma} \frac{z+1}{z} \, dz & = 4i+\mathrm{Log}(4i)-4-\mathrm{Log}(4) \\
& = 4i+ \log(4)+\frac{i\pi}{2} - 4 -\log(4) \\
& = -4 + \left(4 + \frac{\pi}{2}\right) i.
\end{align*}

\vspace{1cm}

\begin{exercise}{2}
Compute the integral 
\begin{equation*}
\int_{C[-1,2]} \frac{z^2}{4-z^2} \, dz.
\end{equation*}
\end{exercise}

\vspace{0.5cm}

\noindent \textbf{Solution:} We begin by simplifying the integrand, 
\begin{equation*}
\frac{z^2}{4-z^2}  = -1 + \frac{4}{4-z^2}.
\end{equation*}
Next we write the remaining fraction in terms of partial fractions, 
\begin{equation*}
\frac{4}{4-z^2} = -\frac{1}{z+2} + \frac{1}{z-2}.
\end{equation*}
We can therefore decompose the original integral into 
\begin{equation*}
\int_{C[-1,2]} \frac{z^2}{4-z^2} \, dz = -\int_{C[-1,2]} 1\, dz - \int_{C[-1,2]} \frac{1}{z+2} \, dz+ \int_{C[-1,2]} \frac{1}{z-2} \, dz.
\end{equation*}
Among these integrands, two are holomorphic functions within the disk bounded by the circle $C[-1,2]$, namely $(-1)$ and $\frac{1}{z-2}$. By Cauchy's Theorem, the first and third integrals above vanish. The second integral remains, as its integrand has a singularity at $-2$, a point situated within the disk $D[-1,2]$. The circle $C[-1,2]$ is $(\mathbb{C}\setminus \{-2\})$-homotopic to the circle $C[-2,2]$, by means of translations, so by Theorem 4.18, 
\begin{equation*}
\int_{C[-1,2]} \frac{1}{z+2} \, dz = \int_{C[-2,2]} \frac{1}{z+2}\, dz
\end{equation*}
The latter is evaluated by substitution, using the parametrization $\gamma(t)=-2+2e^{it}$, $0 \leq t \leq 2\pi$.
\begin{align*}
\int_{C[-2,2]} \frac{1}{z+2}\, dz & = \int_{0}^{2\pi} \frac{2ie^{it}}{2+(-2+2e^{it})}\, dt\\
& = \int_{0}^{2\pi} \frac{2ie^{it}}{2e^{it}}\, dt\\
& = 2\pi i.
\end{align*}
\end{document}


\documentclass[12pt,oneside]{exam}

% This package simply sets the margins to be 1 inch.
\usepackage[margin=1in]{geometry}

% These packages include nice commands from AMS-LaTeX
\usepackage{amssymb,amsmath,amsthm,amsfonts,latexsym,verbatim,xspace,setspace}
\usepackage{hyperref}
\usepackage{graphicx}

% Make the space between lines slightly more
% generous than normal single spacing, but compensate
% so that the spacing between rows of matrices still
% looks normal.  Note that 1.1=1/.9090909...
\renewcommand{\baselinestretch}{1.1}
\renewcommand{\arraystretch}{.91}

% Define environments for exercises.
\newenvironment{exercise}[1]{\vspace{.1in}\noindent\textbf{Problem #1 \hspace{.05em}}}{}
\newenvironment{newsolution}{\vspace{.1in}\noindent\textbf{Solution: \hspace{.05em}}}{}

% define shortcut commands for commonly used symbols
\newcommand{\R}{\mathbb{R}}
\newcommand{\C}{\mathbb{C}}
\newcommand{\Z}{\mathbb{Z}}
\newcommand{\Q}{\mathbb{Q}}
\newcommand{\N}{\mathbb{N}}
\newcommand{\calP}{\mathcal{P}}
\DeclareMathOperator{\sech}{sech}
\DeclareMathOperator{\csch}{csch}
\DeclareMathOperator{\vsspan}{span}

\newcommand{\func}[3]{{#1} : {#2} \longrightarrow {#3}}

\title{Math 514 - Summer II 2020: Long Quiz 3}

%%%%%%%%%%%%%%%%%%%%%%%%%%%%%%%%%%%%%%%%%%

\begin{document}

\begin{flushright}
\sc MAT 514 - Summer II 2020\\
August 11, 2020
\end{flushright}
\bigskip
 
\begin{center}
\textsf{Solutions to Quiz 8} 
\end{center}

%%%%%%%%%%%%%%%%%%%%%%%%%%%%%%%%%%%%%%%%
\begin{exercise}{1}
Find a power series representation for 
\begin{equation*}
\frac{1}{1-z}
\end{equation*}
centered at $z=4$. 
\end{exercise}

\vspace{0.5cm}

\textbf{Solution:} To obtain this power series representation we will use a change of variables trick, by substituting $w=z-4$. In terms of $w$, this function can be written as 
\begin{equation*}
\frac{1}{1-(w+4)} = \frac{1}{-3-w} = -\frac{1}{3+w} = \frac{-\frac{1}{3}}{1-\left(\frac{-w}{3}\right)}.
\end{equation*}
The last expression can be represented as a geometric series, so long as $|\frac{w}{3}| < 1$, 
\begin{align*}
\frac{1}{1-z} & = \frac{-\frac{1}{3}}{1-\left(\frac{-w}{3}\right)}\\
& = \left(-\frac{1}{3}\right) \sum_{k=0}^{\infty} \left(-\frac{w}{3}\right)^k \\
& = \sum_{k=0}^{\infty} \frac{(-1)^{k+1}w^k}{3^{k+1}}\\
& = \sum_{k=0}^{\infty} \frac{(-1)^{k+1}(z-4)^k}{3^{k+1}}
\end{align*}
\vspace{1cm}

\begin{exercise}{2}
Determine the first three terms of the Taylor series for 
\begin{equation*}
\frac{e^z}{1-z}
\end{equation*}
centered at $z=0$. Determine the radius of convergence of this power series. 
\end{exercise}

\vspace{0.5cm}

\textbf{Solution:} The first three terms of this Taylor series may be determined by Cauchy's Integral Formulas or via derivatives. In what follows, we use the second approach. 
\begin{itemize}
\item $a_0 = \frac{e^0}{1-0}= 1$.
\item $a_1 = \left(\frac{e^z}{1-z}\right)'(0) = \frac{-e^{z}(z-2)}{1-z} (0)=2$
\item $a_2 = \left(\frac{1}{2}\right)\left(\frac{e^z}{1-z}\right)^{''}(0) =  \left(\frac{1}{2}\right) \left[ \frac{e^z}{1-z}	+ \frac{2e^z}{(1-z)^2}+ \frac{2e^z}{(1-z)^3}\right](0) = \frac{5}{2}$.
\end{itemize}
The series converges for $|z|<1$, as this is the largest disk centered at $z=0$ not including the singularity at $z=1$. 

\end{document}


\documentclass[12pt,oneside]{exam}

% This package simply sets the margins to be 1 inch.
\usepackage[margin=1in]{geometry}

% These packages include nice commands from AMS-LaTeX
\usepackage{amssymb,amsmath,amsthm,amsfonts,latexsym,verbatim,xspace,setspace}
\usepackage{hyperref}
\usepackage{graphicx}

% Make the space between lines slightly more
% generous than normal single spacing, but compensate
% so that the spacing between rows of matrices still
% looks normal.  Note that 1.1=1/.9090909...
\renewcommand{\baselinestretch}{1.1}
\renewcommand{\arraystretch}{.91}

% Define environments for exercises.
\newenvironment{exercise}[1]{\vspace{.1in}\noindent\textbf{Problem #1 \hspace{.05em}}}{}
\newenvironment{newsolution}{\vspace{.1in}\noindent\textbf{Solution: \hspace{.05em}}}{}

% define shortcut commands for commonly used symbols
\newcommand{\R}{\mathbb{R}}
\newcommand{\C}{\mathbb{C}}
\newcommand{\Z}{\mathbb{Z}}
\newcommand{\Q}{\mathbb{Q}}
\newcommand{\N}{\mathbb{N}}
\newcommand{\calP}{\mathcal{P}}
\DeclareMathOperator{\sech}{sech}
\DeclareMathOperator{\csch}{csch}
\DeclareMathOperator{\vsspan}{span}

\newcommand{\func}[3]{{#1} : {#2} \longrightarrow {#3}}

\title{Math 514 - Summer II 2020: Long Quiz 2}

%%%%%%%%%%%%%%%%%%%%%%%%%%%%%%%%%%%%%%%%%%

\begin{document}

\begin{flushright}
\sc MAT 514 - Summer II 2020\\
July 23, 2020
\end{flushright}
\bigskip
 
\begin{center}
\textsf{Solutions to Long Quiz 2} 
\end{center}

%%%%%%%%%%%%%%%%%%%%%%%%%%%%%%%%%%%%%%%%

\begin{exercise}{1}
Determine at which points the function $f(z)=\frac{1}{\overline{z}}$, defined for $z\neq 0$, is complex-differentiable. 
\end{exercise}
\vspace{0.5cm}

\noindent\textbf{Solution:} Let us write this function in terms of the real and imaginary parts of $z= x+iy$, 
\begin{align*}
f(z) & = \frac{1}{\overline{z}} \\
& = \frac{1}{x-iy}\\
& = \left(\frac{1}{x-iy}\right)\left(\frac{x+iy}{x+iy}\right) \\
& = \left(\frac{x+iy}{x^2+y^2}\right).
\end{align*}
Its real and imaginary parts are thus 
\begin{equation*}
u(x,y) = \frac{x}{x^2+y^2}, \, \, v(x,y) = \frac{y}{x^2+y^2},
\end{equation*}
respectively. The Cauchy-Riemann equations for this function are
\begin{align*}
\frac{y^2-x^2}{(x^2+y^2)^2} & = \frac{x^2-y^2}{(x^2+y^2)^2} \\
-\frac{2xy}{(x^2+y^2)^2} & = \frac{2xy}{(x^2+y^2)^2},
\end{align*}
a system without solutions. It follows that $f$ is nowhere complex-differentiable.

\vspace{1cm}

\begin{exercise}{2}
Find a function $v(x,y)$ so that 
\begin{equation*}
f(x+iy) = (2x^2+x+1-2y^2) + iv(x,y)
\end{equation*}
safisties the Cauchy-Riemann equations.
\end{exercise}

\vspace{0.5cm}

\noindent \textbf{Solution:} The derivatives of the real part of $f$ are 
\begin{equation*}
\frac{\partial u}{\partial x} = 4x+1, \, \, \frac{\partial u}{\partial y} = -4y.
\end{equation*}
The Cauchy-Riemann equations for $v$ thus amount to 
\begin{align*}
\frac{\partial v}{\partial x} & = 4y \\
\frac{\partial v}{\partial y	} & = 4x+1.
\end{align*}
The function $v(x,y)=4xy+y$ is a solution, defined on the entire complex plane. 

\vspace{1cm}

\begin{exercise}{3}
Use properties of the exponential function to derive the following relation:
\begin{equation*}
\sin(2z) = 2\sin(z)\cos(z).
\end{equation*}
\end{exercise}

\vspace{0.5cm}

\noindent \textbf{Solution:} From the definition of the complex sine and cosine functions in terms of complex exponentials, 
\begin{align*}
\sin(2x) & = \frac{e^{2iz}-e^{-2iz}}{2i}\\
& = \frac{(e^{iz})^2-(e^{-iz})^2}{2i}\\
& = \frac{(e^{iz}+e^{-iz})(e^{iz}-e^{-iz})}{2i}\\
& = 2\left(\frac{e^{iz}-e^{-iz}}{2i}\right)\left(\frac{e^{iz}+e^{-iz}}{2}\right)\\
& = 2\sin(z)\cos(z).
\end{align*}
\end{document}


\documentclass[12pt,oneside]{exam}

% This package simply sets the margins to be 1 inch.
\usepackage[margin=1in]{geometry}

% These packages include nice commands from AMS-LaTeX
\usepackage{amssymb,amsmath,amsthm,amsfonts,latexsym,verbatim,xspace,setspace}
\usepackage{hyperref}
\usepackage{graphicx}

% Make the space between lines slightly more
% generous than normal single spacing, but compensate
% so that the spacing between rows of matrices still
% looks normal.  Note that 1.1=1/.9090909...
\renewcommand{\baselinestretch}{1.1}
\renewcommand{\arraystretch}{.91}

% Define environments for exercises.
\newenvironment{exercise}[1]{\vspace{.1in}\noindent\textbf{Problem #1 \hspace{.05em}}}{}
\newenvironment{newsolution}{\vspace{.1in}\noindent\textbf{Solution: \hspace{.05em}}}{}

% define shortcut commands for commonly used symbols
\newcommand{\R}{\mathbb{R}}
\newcommand{\C}{\mathbb{C}}
\newcommand{\Z}{\mathbb{Z}}
\newcommand{\Q}{\mathbb{Q}}
\newcommand{\N}{\mathbb{N}}
\newcommand{\calP}{\mathcal{P}}
\DeclareMathOperator{\sech}{sech}
\DeclareMathOperator{\csch}{csch}
\DeclareMathOperator{\vsspan}{span}

\newcommand{\func}[3]{{#1} : {#2} \longrightarrow {#3}}

\title{Math 514 - Summer II 2020: Quiz 2}

%%%%%%%%%%%%%%%%%%%%%%%%%%%%%%%%%%%%%%%%%%

\begin{document}

\begin{flushright}
\sc MAT 514 - Summer II 2020\\
July 21, 2020
\end{flushright}
\bigskip
 
\begin{center}
\textsf{Quiz 4} 
\end{center}

%%%%%%%%%%%%%%%%%%%%%%%%%%%%%%%%%%%%%%%%

\begin{exercise}{1}
Use the relation
\begin{equation*}
\sin(z) = \frac{e^{iz}-e^{-iz}}{2i}
\end{equation*}
to show that $\sin(z)$ is an antiderivative of $\cos(z)$. 
\end{exercise}

\noindent \textbf{Solution:} Using the chain rule
\begin{align*}
\sin'(z) & = \left( \frac{e^{iz}-e^{-iz}}{2}\right)^{'} \\
& = \frac{(e^{iz})'-(e^{-iz})'}{2} \\
& = \frac{ie^{iz}+ie^{-iz}}{2}\\
& = \cos(z).
\end{align*}
\vfill

\begin{exercise}{2}
Compute the derivative of 
\begin{equation*}
\cosh(z^2) 
\end{equation*}
by using properties of derivatives and complex exponentials.
\end{exercise}
\noindent \textbf{Solution:} Using the chain rule
\begin{align*}
(\cosh(z^2))' & = \sinh(z^2)(z^2)' \\
& = 2z\sinh(z^2).
\end{align*}
\vfill

\end{document}


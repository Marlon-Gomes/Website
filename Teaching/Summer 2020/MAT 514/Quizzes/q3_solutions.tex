\documentclass[12pt,oneside]{exam}

% This package simply sets the margins to be 1 inch.
\usepackage[margin=1in]{geometry}

% These packages include nice commands from AMS-LaTeX
\usepackage{amssymb,amsmath,amsthm,amsfonts,latexsym,verbatim,xspace,setspace}
\usepackage{hyperref}
\usepackage{graphicx}

% Make the space between lines slightly more
% generous than normal single spacing, but compensate
% so that the spacing between rows of matrices still
% looks normal.  Note that 1.1=1/.9090909...
\renewcommand{\baselinestretch}{1.1}
\renewcommand{\arraystretch}{.91}

% Define environments for exercises.
\newenvironment{exercise}[1]{\vspace{.1in}\noindent\textbf{Problem #1 \hspace{.05em}}}{}
\newenvironment{newsolution}{\vspace{.1in}\noindent\textbf{Solution: \hspace{.05em}}}{}

% define shortcut commands for commonly used symbols
\newcommand{\R}{\mathbb{R}}
\newcommand{\C}{\mathbb{C}}
\newcommand{\Z}{\mathbb{Z}}
\newcommand{\Q}{\mathbb{Q}}
\newcommand{\N}{\mathbb{N}}
\newcommand{\calP}{\mathcal{P}}
\DeclareMathOperator{\sech}{sech}
\DeclareMathOperator{\csch}{csch}
\DeclareMathOperator{\vsspan}{span}

\newcommand{\func}[3]{{#1} : {#2} \longrightarrow {#3}}

\title{Math 514 - Summer II 2020: Quiz 2}

%%%%%%%%%%%%%%%%%%%%%%%%%%%%%%%%%%%%%%%%%%

\begin{document}

\begin{flushright}
\sc MAT 514 - Summer II 2020\\
July 16, 2020
\end{flushright}
\bigskip
 
\begin{center}
\textsf{Solutions to Quiz 3} 
\end{center}

%%%%%%%%%%%%%%%%%%%%%%%%%%%%%%%%%%%%%%%%

\begin{exercise}{1}
Consider the function $\func{f}{\{z = x + iy \in \mathbb{C}| x,y \in \mathbb{R}, y \neq 0\}}{\mathbb{C}}$, defined by 
\begin{equation*}
f(x+iy) = \frac{ix+1}{y}.
\end{equation*}
Determine if this function has a limit at $0$. 
\end{exercise}

\textbf{Solution:}
This function does not have a complex number\footnote{Infinite limits will be dealt with in the following chapter.} as its limit at $0$. Indeed, as $z \to 0$, the numerator is bounded, whereas the denominator is unbounded. 
\vfill

\begin{exercise}{2}
Determine all poles (infinite discontinuities) of the function
\begin{equation*}
f(z) = \frac{z^2+z-2}{2z^2+z-3}.
\end{equation*}
\end{exercise}


\textbf{Solution:} As we learned in class, ratios of continuous functions are continuous, except perhaps at zeros of the denominator. In this case, the denominator has factorization
\begin{equation*}
2z^2+z-3 = (2z+3)(z-1),
\end{equation*}
hence its zeros are located at $z=-\frac{3}{2}$ and $z=1$. One of these points is an apparent discontinuity, since the expression of $f$ can be simplified to 
\begin{equation*}
f(z) = \frac{(z+2)(z-1)}{(2z+3)(z-1)} = \frac{z+2}{2z+3}.
\end{equation*}
It follows that the only pole of the function is located at $z=-\frac{3}{2}$. 
\vfill
\end{document}


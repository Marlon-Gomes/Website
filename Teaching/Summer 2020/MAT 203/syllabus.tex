\documentclass[11pt]{amsart}
\usepackage{amssymb,amsfonts,amsthm,amsmath}
\usepackage[english]{babel}
\usepackage[all,cmtip]{xy}
\usepackage{hyperref}
\usepackage{url}
%\usepackage{mathrsfs}
%\usepackage[notcite,notref]{showkeys}
\def\limproj{\mathop{\oalign{lim\cr\hidewidth$\longleftarrow$\hidewidth\cr}}}

%--------------
% macro perso
%--------------
\newcommand{\inputc}[1]{ \raisebox{-0.5\height}{\input{#1}} }
\newtheorem{theorem}{Theorem}[section]
\newtheorem{lemma}[theorem]{Lemma}
\newtheorem{corollary}[theorem]{Corollary}
\newtheorem{definition}[theorem]{Definition}
\newtheorem{proposition}[theorem]{Proposition}
\newtheorem{remark}[theorem]{Remark}
\newtheorem{example}{Example}[section]
%\newtheorem{theorem*}{Theorem}

\newcommand{\func}[3]{{#1} : {#2} \longrightarrow {#3}}
\numberwithin{equation}{section}

\newenvironment{myproof}{\noindent{it Proof}
\setlength{\parindent}{0mm}}
{$\hfill \bs$}

\title[MAT 203 Syllabus - Summer 2020]{MAT 203 Syllabus - Summer Session I 2020}

\author[M. Gomes]{Marlon Gomes}
%

\begin{document}
\maketitle

\section{Course Description}
\subsection{Course Goal}
This is a course about Calculus in higher dimensions. We will begin with a brief overview of linear algebra: the algebra of vectors in two and three-dimensions; linear transformations and determinants; dot and cross products; elementary analytic geometry (lines, planes, conics and quadrics).

The second part of the course is the study of continuity and differentiability of multivariable functions, with applications to geometry and optimization problems.
 
The third part is the study of higher-dimensional integral calculus: double and triple integrals in regions of the plane and space, respectively.
Finally, we will study the relationship between the differentiation and integration in vector calculus: line integrals and Green's theorem; curl and divergence of vector fields, and the theorems of Gauss and Stokes. 

\subsection{Requisites}
C or higher in MAT 127 or 132 or 142 or AMS 161 or level 9 on the mathematics placement examination. Students can find the syllabi for these other MAT courses at 
\begin{center}
\url{http://www.math.stonybrook.edu/mathematics-department-course-web-pages}
\end{center}

\subsection{Textbook}
The recommended textbook is \textit{Multivariable calculus}, by Ron Larson and Bruce H. Edwards, 11th edition. The textbook can be acquired at the campus bookstore. 

\subsection{Important Times and Dates}
\begin{itemize}
\item Synchronous lectures: Mondays, Wednesdays and Thursdays, 9:30am - 12:35 pm, via Zoom.
\item Make-up lecture: May 29th, 9:30am -12:35pm, via Zoom.
\item Midterm: June 11th, via Blackboard.
\item Final Exam: July 2nd, via Blackboard. 
\end{itemize}

\subsection{Weekly schedule}
Below is a tentative schedule for the course. A detailed version, including reading assignments and references will be posted on Blackboard. 

\begin{center}
\begin{tabular}{p{0.15\textwidth}|p{0.75\textwidth}}
\hline 
Week of & Subject \\ 
\hline 
05/25 & Vectors in Algebra and Geometry, multivariable and vector-valued functions.  \\ 
\hline 
06/01 & Continuity and Differentialbility in higher dimensions: directional derivatives, Jacobians, the Chain Rule, the Clairaut-Schwarz Theorem.  \\ 
\hline 
06/08 & Applications to optimization problems. \\ 
\hline 
06/15 & Multiple integration in the plane and in space. Iterated integrals, Fubini's Theorem and the Change of Variables Formula. \\ 
\hline 
06/21 & Vector Calculus: Vector fields, curls and divergences. Line integrals, integration along surfaces.  \\ 
\hline 
06/28 & The Theorems of Gauss, Green and Stokes. \\ 
\hline 
\end{tabular} 
\end{center}

\subsection{Assignments}
Your assignments are an important part of the course (and your grades). There will be four problem sets, due on Mondays of weeks 2, 3, 5 and 6. These assignments will be available on the course webpage on Blackboard.

 Problems will range from simple manipulations of the concepts developed in class on that week to more involved applications of these concepts. Not all problems will be graded, but you should attempt to solve them all anyway, as they will serve as your foundation for the problems you will see on your exams. Graded problems will be discussed in class. Solutions to selected non-graded problems will be posted on the course webpage weekly.  
 
 Students are expected to scan or type their assignments, as there will be no physical interaction between instructor and students. Students will be provided with information about a range of free, quality scanning softwares via Blackboard. 

\subsection{Grades}
Your grade will be calculated in the following way:
\begin{enumerate}
\item Three best homework assignments: 20$\%$ each.
\item Midterm: 20$\%$.
\item Final Exam: 20$\%$.
\end{enumerate}

\subsection{Course management}

This course will be taught online, using Blackboard as the primary course management toll. This includes a detailed course schedule, homework management, exams, and communication. We will use Zoom for lectures and office hours. Instructions on how to set up your Zoom account, as well as accessing the lectures will be posted to the course Blackboard page. E-mail will be an important tool, and students are expected to check for updates regularly.

\section{Contact}
My office hours will be announced on the course webpage, 
\begin{center}
\url{http://www.math.stonybrook.edu/~mgomes/mat203sum20.html},
\end{center}
as well as on Blackboard. E-mail is the best form of communication besides lectures and office hours. My address is 
\begin{center}
\href{mailto: mgomes@math.stonybrook.edu}{ mgomes@math.stonybrook.edu}.
\end{center}

\section{Student Accessibility Support Center Statement}

If you have a physical, psychological, medical, or learning disability that may impact your course work, please contact the Student Accessibility Support Center, 128 ECC Building, (631) 632-6748, or via e-mail at: \href{mailto: sasc@stonybrook.edu}{sasc@stonybrook.edu}. They will determine with you what accommodations are necessary and appropriate. All information and documentation is confidential.

\section{Academic Integrity Statement}

Each student must pursue his or her academic goals honestly and be personally accountable for all submitted work. Representing another person's work as your own is always wrong. Faculty is required to report any suspected instances of academic dishonesty to the Academic Judiciary. Faculty in the Health Sciences Center (School of Health Technology \& Management, Nursing, Social Welfare, Dental Medicine) and School of Medicine are required to follow their school-specific procedures. For more comprehensive information on academic integrity, including categories of academic dishonesty please refer to the academic judiciary website at
\begin{center}
\url{http://www.stonybrook.edu/commcms/academic_integrity/index.html}
\end{center}

\section{Critical Incident Management}

Stony Brook University expects students to respect the rights, privileges, and property of other people. Faculty are required to report to the Office of University Community Standards any disruptive behavior that interrupts their ability to teach, compromises the safety of the learning environment, or inhibits students' ability to learn. Faculty in the HSC Schools and the School of Medicine are required to follow their school-specific procedures. Further information about most academic matters can be found in the Undergraduate Bulletin, the Undergraduate Class Schedule, and the Faculty-Employee Handbook.
\end{document}
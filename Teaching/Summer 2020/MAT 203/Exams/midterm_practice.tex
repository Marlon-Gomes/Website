% Exam Template for UMTYMP and Math Department courses
%
% Using Philip Hirschhorn's exam.cls: http://www-math.mit.edu/~psh/#ExamCls
%
% run pdflatex on a finished exam at least three times to do the grading table on front page.
%
%%%%%%%%%%%%%%%%%%%%%%%%%%%%%%%%%%%%%%%%%%%%%%%%%%%%%%%%%%%%%%%%%%%%%%%%%%%%%%%%%%%%%%%%%%%%%%

% These lines can probably stay unchanged, although you can remove the last
% two packages if you're not making pictures with tikz.
\documentclass[11pt]{exam}
\RequirePackage{amssymb, amsfonts, amsmath, latexsym, verbatim, xspace, setspace}


% By default LaTeX uses large margins.  This doesn't work well on exams; problems
% end up in the "middle" of the page, reducing the amount of space for students
% to work on them.
\usepackage[margin=1in]{geometry}
\usepackage[english]{babel}
\usepackage[autostyle]{csquotes} %%%% This package allows Tex to recognize quotation marks with the \enquote command. 


% Here's where you edit the Class, Exam, Date, etc.
\newcommand{\class}{MAT 203}
\newcommand{\term}{Summer I 2020}
\newcommand{\examnum}{Midterm Practice}
\newcommand{\examdate}{06/11/20}
\newcommand{\timelimit}{3 hours and 25 minutes}

% For an exam, single spacing is most appropriate
\singlespacing
% \onehalfspacing
% \doublespacing

% For an exam, we generally want to turn off paragraph indentation
\parindent 0ex
\title{MAT 203 - Summer I 2020: Midterm practice}
\begin{document} 

% These commands set up the running header on the top of the exam pages
\pagestyle{head}
\firstpageheader{}{}{}\textbf{}
\runningheader{\class}{\examnum\ - Page \thepage\ of \numpages}{\examdate}
\runningheadrule

\begin{flushright}
\begin{tabular}{p{2.8in} r l}
\textbf{\class} & \textbf{Name (Print):} & \makebox[2in]{\hrulefill}\\
\textbf{\term} &&\\
\textbf{\examnum} &&\\
\textbf{\examdate} &&\\
\textbf{Time Limit: \timelimit} & ID number & \makebox[2in]{\hrulefill}
\end{tabular}\\
\end{flushright}
\rule[1ex]{\textwidth}{.1pt}

\begin{center}
\large{\textbf{Instructions}}
\end{center}

\begin{minipage}[t]{3.7in}
\vspace{0pt}
\begin{itemize}

\item This exam contains \numpages\ pages (including this cover page) and
\numquestions\ problems.  Check to see if any pages are missing.  Enter
all requested information on the top of this page, and put your initials
on the top of every page, in case the pages become separated.

\item This is an open-book exam. You may not use a calculator.

\item \textbf{Organize your work}, in a reasonably neat and coherent way, in
the space provided. Work scattered all over the page without a clear ordering will 
receive very little credit.  

\item \textbf{Mysterious or unsupported answers will not receive full
credit}.

\end{itemize}

\end{minipage}
\hfill
\begin{minipage}[t]{2.3in}
\vspace{0pt}
%\cellwidth{3em}
\gradetablestretch{2}
\vqword{Problem}
\addpoints % required here by exam.cls, even though questions haven't started yet.	
\gradetable[v]%[pages]  % Use [pages] to have grading table by page instead of question

\end{minipage}
\newpage % End of cover page

%%%%%%%%%%%%%%%%%%%%%%%%%%%%%%%%%%%%%%%%%%%%%%%%%%%%%%%%%%%%%%%%%%%%%%%%%%%%%%%%%%%%%
%
% See http://www-math.mit.edu/~psh/#ExamCls for full documentation, but the questions
% below give an idea of how to write questions [with parts] and have the points
% tracked automatically on the cover page.
%
%
%%%%%%%%%%%%%%%%%%%%%%%%%%%%%%%%%%%%%%%%%%%%%%%%%%%%%%%%%%%%%%%%%%%%%%%%%%%%%%%%%%%%%

\begin{questions}

% Basic question
%%%%%%%%%%%%%%

\addpoints
\question In each of the following problems, determine if the statements are true or false. Explain your reasoning (correct answers without an explanation will be worth only 2 points per statement). 
\begin{parts}
\part[5] Two vectors in space always determine a unique plane. 
\vfill
\part[5] Two planes in space always intersect. 
\vfill
\part[5] The dot product can be used to detect whether two non-zero vectors in space are aligned. 
\vfill
\part[5] Let $r(t)$ and $s(t)$ be curves in the plane, such that neither has a limit as $t$ converges to $0$. Then their cross product $r(t)\times s(t)$ does not have a limit at $0$ either. 
\vfill
\part[5] If all directed limits a scalar-valued, multivariable function at a point exist and coincide, then the function has a limit at the point, in the multivariable sense.
\vfill
\part[5] If a scalar-valued, multivariable function is separately continuous with respect to each variable, then it is continuous in the multivariable sense.
\vfill
\end{parts}

\newpage 


%%%%%%%%%%%%%%%%%%%%%%%%%%%%%%%%%%%
\addpoints 
\question Consider the lines whose parametric equations are given by 
\begin{align*}
L_1 \colon  & x=2t, y=4t, z=6t.\\
L_2 \colon & x=1-s, 4+s, -1+s.
\end{align*}

\begin{parts}
\part[5] Write symmetric equations for each line
\vfill
\part[5] Explain why these lines do not intersect. 
\vfill
\newpage
\part[5] Explain why these lines are not parallel. 
\vfill
\part[5]  Find the general equations of two parallel planes, $\Pi_1$ and $\Pi_2$, containing lines $L_1$ and $L_2$, respectively. 
\vfill
\end{parts}

\newpage

%%%%%%%%%%%%%%%%%
\addpoints
\question[10] Compute the trajectory of a curve whose velocity vector is given by 
\begin{equation*}
r'(t)= e^{-t}i + t^2j + \frac{1}{1+t^2} k,
\end{equation*}
and such that $r(0)=(1,1,1)$. 
\newpage
%%%%%%%%%%%%%%%%%

\addpoints

\addpoints
\question Consider the function
\begin{equation*}
f(x,y)=\left\{
\begin{array}{rc}
\frac{xy(x^2-y^2)}{x^2+y^2} & \mbox{if} \ (x,y) \neq (0,0)\\
0, & \mbox{if} \ (x,y)=(0,0).
\end{array}
\right.
\end{equation*}

\begin{parts}
\part[5] Where are the directed limits of the function along lines $y=kx$, as $x \to 0$?
\vfill
\part[5] Is this function continuous at the origin?
\vfill
\newpage
\part[5] What is the partial derivative of this function relative to $x$, at points other than the origin?
\vfill
\part[5] What is the partial derivative of this function relative to $y$, at points other than the origin?
\vfill
\end{parts}
\newpage

%%%%%%%%%%%%%%%%%%%%
\addpoints
\question Recall that the polar coordinate system in the plane is related to Cartesian coordinates by means of the equations
\begin{align}
x & = r\cos(\theta), \\
y & = r\sin(\theta).
\end{align}
\begin{parts}
\part[6] Differentiate equation (1) relative to $x$ to find a relation between $\frac{\partial r}{\partial x}$ and $\frac{\partial \theta}{\partial x}$.
\vfill 
\newpage 
\part[6] Differentiate equation (2) relative to $x$ to find a second relation between $\frac{\partial r}{\partial x}$ and $\frac{\partial \theta}{\partial x}$.
\vfill
\newpage
\part[8] By solving the system of equations obtained in the previous two steps, compute the derivatives $\frac{\partial r}{\partial x}$ and $\frac{\partial \theta}{\partial x}$.
\end{parts}

%%%%%%%%%%%%%%%%%%%%%%%%%%

\end{questions}
\end{document}

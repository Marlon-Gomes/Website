% Exam Template for UMTYMP and Math Department courses
%
% Using Philip Hirschhorn's exam.cls: http://www-math.mit.edu/~psh/#ExamCls
%
% run pdflatex on a finished exam at least three times to do the grading table on front page.
%
%%%%%%%%%%%%%%%%%%%%%%%%%%%%%%%%%%%%%%%%%%%%%%%%%%%%%%%%%%%%%%%%%%%%%%%%%%%%%%%%%%%%%%%%%%%%%%

% These lines can probably stay unchanged, although you can remove the last
% two packages if you're not making pictures with tikz.
\documentclass[11pt]{exam}
\RequirePackage{amssymb, amsfonts, amsmath, latexsym, verbatim, xspace, setspace, color}


% By default LaTeX uses large margins.  This doesn't work well on exams; problems
% end up in the "middle" of the page, reducing the amount of space for students
% to work on them.
\usepackage[margin=1in]{geometry}
\usepackage[english]{babel}
\usepackage[autostyle]{csquotes} %%%% This package allows Tex to recognize quotation marks with the \enquote command. 


% Here's where you edit the Class, Exam, Date, etc.
\newcommand{\class}{MAT 203}
\newcommand{\term}{Summer I 2020}
\newcommand{\examnum}{Midterm Solutions}
\newcommand{\examdate}{06/11/20}
\newcommand{\timelimit}{3 hours and 25 minutes}

% For an exam, single spacing is most appropriate
\singlespacing
% \onehalfspacing
% \doublespacing

% For an exam, we generally want to turn off paragraph indentation
\parindent 0ex
\title{MAT 203 - Summer II 2020: Midterm solutions}
\begin{document} 

% These commands set up the running header on the top of the exam pages
\pagestyle{head}
\firstpageheader{}{}{}\textbf{}
\runningheader{\class}{\examnum\ - Page \thepage\ of \numpages}{\examdate}
\runningheadrule

\begin{flushright}
\begin{tabular}{p{2.8in} r l}
\textbf{\class} & \textbf{Name (Print):} & \makebox[2in]{\hrulefill}\\
\textbf{\term} &&\\
\textbf{\examnum} &&\\
\textbf{\examdate} &&\\
\textbf{Time Limit: \timelimit} & ID number & \makebox[2in]{\hrulefill}
\end{tabular}\\
\end{flushright}
\rule[1ex]{\textwidth}{.1pt}

\begin{center}
\large{\textbf{Instructions}}
\end{center}

\begin{minipage}[t]{3.7in}
\vspace{0pt}
\begin{itemize}

\item This exam contains \numpages\ pages (including this cover page) and
\numquestions\ problems.  Check to see if any pages are missing.  Enter
all requested information on the top of this page, and put your initials
on the top of every page, in case the pages become separated.

\item You may \textit{not} use your books, notes, or any device that is capable of accessing the internet on this exam (e.g., smartphones, smartwatches, tablets). You may not use a calculator.

\item \textbf{Organize your work}, in a reasonably neat and coherent way, in
the space provided. Work scattered all over the page without a clear ordering will 
receive very little credit.  

\item \textbf{Mysterious or unsupported answers will not receive full
credit}.

\end{itemize}

\end{minipage}
\hfill
\begin{minipage}[t]{2.3in}
\vspace{0pt}
%\cellwidth{3em}
\gradetablestretch{2}
\vqword{Problem}
\addpoints % required here by exam.cls, even though questions haven't started yet.	
\gradetable[v]%[pages]  % Use [pages] to have grading table by page instead of question

\end{minipage}
\newpage % End of cover page

%%%%%%%%%%%%%%%%%%%%%%%%%%%%%%%%%%%%%%%%%%%%%%%%%%%%%%%%%%%%%%%%%%%%%%%%%%%%%%%%%%%%%
%
% See http://www-math.mit.edu/~psh/#ExamCls for full documentation, but the questions
% below give an idea of how to write questions [with parts] and have the points
% tracked automatically on the cover page.
%
%
%%%%%%%%%%%%%%%%%%%%%%%%%%%%%%%%%%%%%%%%%%%%%%%%%%%%%%%%%%%%%%%%%%%%%%%%%%%%%%%%%%%%%

\begin{questions}

% Basic question
%%%%%%%%%%%%%%

\question In each of the following problems, determine if the statements are true or false. Explain your reasoning (correct answers without an explanation will be worth only 2 points per statement). 
\begin{parts}
\part[5] Two planes in space always intersect along a line. 

\textit{Solution:} This is false. Parallel planes do not intersect. 
\vfill

\part[5] Non-parallel lines in space always intersect. 

\textit{Solution:} This is false. Non-parallel lines in parallel planes (called skew lines) do not intersect. 
\vfill
\part[5] The cross product can be used to detect whether two non-zero vectors in space are perpendicular. 

\textit{Solution:} This is true. The cross product attains its maximal norm when vectors are perpendicular, owing to the relation
\begin{equation*}
\| u \times x \| = \|u\| \|v\| \sin(\theta),
\end{equation*}
where $\theta$ is the angle between the vectors. 
\vfill
\part[5] Let $r(t)$ and $s(t)$ be curves in the plane, such that neither has a limit as $t$ converges to $0$. Then their dot product $r(t)\cdot s(t)$ does not have a limit at $0$ either. 

\textit{Solution:} This is false. An example is given by the curves $r(t)=(t^{-1},t)$ and $s(t)=(-t,t^{-1})$. These curves are orthogonal where defined, so the limit of $r(t) \cdot s(t)$ as $t$ goes to $0$ exists (it is equal to $0$), while neither $r$ not $s$ have a limit as $t$ goes to $0$. 
\vfill
\part[5] If the limit of a scalar-valued, multivariable function exists at a point, then all directed limits at that point exist and coincide.

\textit{Solution:} This is true. The existence of a multivariable limit implies that limits along all curves through the point exist and coincide, in particular the directed limits. 
\vfill
\part[5] If all the partial derivatives of a scalar-valued, multivariable function exist at a point, then all directional derivatives at that point exist.

\textit{Solution:} This is false. The existence of partial derivatives does not guarantee the existence of all directional derivatives. A simple counter-example is given by a piecewise defined function, given by
\begin{equation*}
f(x,y,z)=0, \, \mbox{if one of the coordinates is zero},
\end{equation*} 
and 
\begin{equation*}
f(x,y,z) = x, \, \mbox{otherwise}.
\end{equation*}
\end{parts}
\vfill
\newpage

%%%%%%%%%%%%%%%%%%%%%%%%%%%%%%%%%%%
\newpage 
\addpoints
\question Consider the lines whose parametric equations are given by 
\begin{align*}
L_1 \colon  & x=3t, y=2-t, z=-1+t.\\
L_2 \colon & x=1+4s, y=-2+s, z=-3-3s.
\end{align*}

\begin{parts}
\part[5] Write symmetric equations for each line.

\textit{Solution:} We obtain the symmetric equations for line $L_1$ by solving each of the parametric equations for $t$, 
\begin{align*}
t & = \frac{x}{3} \\
t & = 2-y\\
t & = z+1,
\end{align*}
hence 
\begin{equation*}
\frac{x}{3} = 2-y = z+1.
\end{equation*}
Likewise, solving the parametric equations of $L_2$ for $s$ yields,
\begin{align*}
s & = \frac{x-1}{4} \\
s & = y+2 \\
s& = \frac{-3-z}{3}
\end{align*}
thus 
\begin{equation*}
\frac{x-1}{4} =  y+2 =  \frac{-3-z}{3}.
\end{equation*}
\vfill
\part[5] Explain why these lines do not intersect. 

\textit{Solution:} At a presumptive intersection point the coordinates $x,y,z$ of both lines would be the same. We investigate this possibility with the parametric equations, obtaining a system of $3$ equations in $2$ variables, 
\begin{align*}
3t & = 1+4s \\
2-t & = -2+s \\
-1+t & = -3-3s.
\end{align*}
Solving the upper two equations leads to $t=\frac{17}{7}, s=\frac{11}{7}$, while solving the bottom two leads to $t=7, s=-3$. As these solutions are incompatible, the lines have no intersection.
\vfill

\newpage
\part[5] Explain why these lines are not parallel. 

\textit{Solution:} We can extract direction vectors for lines $L_1$ and $L_2$ from their parametric equations, 
\begin{align*}
v_1 & = \left(3, -1,1\right),\\
v_2 & = \left(4,1,-3\right),
\end{align*}
respectively. These vectors are not multiples of each other, hence the lines are not parallel. 

\vfill
\part[5]  Find the general equations of two parallel planes, $\Pi_1$ and $\Pi_2$, containing lines $L_1$ and $L_2$, respectively. 

\textit{Solution:} By finding the cross-product between the direction vectors found on part (c), we will obtain a vector perpendicular to both lines, whose entries will serve as coefficients for the equations of the desired planes. 
\begin{equation*}
v_1 \times v_2 = (2,13,7).
\end{equation*}

A plane parallel to both lines has equation
\begin{equation*}
2x+13y+7z = D,
\end{equation*}
for some constant $D$. Substituting the coordinates of a point in $L_1$, say $(0,2,-1)$, we obtain
\begin{equation*}
D_1  = 2(0)+13(2)+7(-1) = 19,
\end{equation*}
while substituting the coordinates of a point in $L_2$, say $(1,-2,-3)$, we get 
\begin{equation*}
D_2 = 2(1)+13(-2)+7(-3) = -45.
\end{equation*}
\vfill
\end{parts}
\newpage

%%%%%%%%%%%%%%%%%
\addpoints 
\question A line and a plane are said to be parallel if they do not intersect. Consider the line $L$ whose symmetric equations are
\begin{equation*}
x-1 = \frac{y-1}{-2} = z-1,
\end{equation*}
and the plane $\Pi$ with equation
\begin{equation*}
x+y+z=1.
\end{equation*}

\begin{parts}
\part[3] Find a vector perpendicular to the plane. 

\textit{Solution:} The coordinates of such a vector may be extracted from the coefficients of the equation of the plane, 
\begin{equation*}
v=(1,1,1)
\end{equation*}
\vfill
\part[3] Write parametric equations for the line. Clearly identify a direction vector.

\textit{Solution:} By setting the common value of the expressions in the symmetric equations to a parameter $t$, we obtain
\begin{align*}
x & = 1+t\\
y & = 1-2t\\
z & = 1+t
\end{align*}
This line may be expressed in terms of a base-point, parameter and direction vector as 
\begin{equation*}
(x,y,z) = (1,1,1)+t(1,-2,1),
\end{equation*}
respectively.
\vfill
\part[4] Check that the line $L$ and the plane $\Pi$ are parallel by verifying that the vector perpendicular to the plane you found on part (a) is also perpendicular to the direction vector you found in part (b).

\textit{Solution:} We compute the dot product between the reffered vectors, 
\begin{equation*}
(1,1,1)\cdot (1,-2,1) = 1-2+1 = 0,
\end{equation*}
thus concluding that the vectors are orthogonal. 
\vfill

\newpage

\part[5] The point $P = (1,1,1)$ belongs to the line $L$. Write an equation of a line perpendicular to the plane, passing through $P$. 

\textit{Solution:} A line perpendicular to $\Pi$ has direction vector $(1,1,1)$. By imposing the restriction that it passes through $P$, we obtain its equation in parametric form, 
\begin{equation*}
(x,y,z) = (1,1,1) + s(1,1,1).
\end{equation*}
\vfill
\part[5] The line you found on part (d) should intersect the plane at a point $Q$. Find the distance between $P$ and $Q$. 

\textit{Solution:} To find the intersection between the line found on part (d) and the plane $\Pi$, we plug in the parametric descriptions of the coordinates of the line into the equation of the plane, 
\begin{equation*}
(1+s)+(1+s)+(1+s) = 1 \Rightarrow (3+3s)=1 \Rightarrow s = -\frac{2}{3}
\end{equation*}
The point at which the intersection occurs is thus 
\begin{equation*}
Q = \left(1-\frac{2}{3},1-\frac{2}{3},1-\frac{2}{3}\right) = \left( \frac{1}{3}, \frac{1}{3},\frac{1}{3} \right)
\end{equation*}
We use the distance formula to find the distance between $P$ and $Q$: 
\begin{align*}
d & = \sqrt{\left( 1-\frac{1}{3}\right)^2 + \left( 1-\frac{1}{3}\right)^2+ \left( 1-\frac{1}{3}\right)^2} \\
& = \sqrt{\frac{12}{9}} \\
& = \frac{2\sqrt{3}}{3}.
\end{align*}
\vfill 
\end{parts}

%%%%%%%%%%%%%%%%%

\newpage
\addpoints
\question You are given a curve with velocity vector 
\begin{equation*}
r'(t)= \sin(t)\mathbf{i} + \cos(t)\mathbf{j} + t \mathbf{k},
\end{equation*}
and such that $r(0)=(1,0,1)$.
\begin{parts}
\part[5] Compute the trajectory of a curve, as a function of $t$. 

\textit{Solution:} The trajectory can be found by means of the Fundamental Theorem of Calculus for Curves, 
\begin{align*}
r(t) & = r(0) + \int_{0}^{t} r'(s) \, ds \\
& = \mathbf{i}+ \mathbf{k} + \int_{0}^{t}  \sin(s)\mathbf{i} + \cos(s)\mathbf{j} + s \mathbf{k} \, ds \\
& = \mathbf{i} + \mathbf{k} + (-\cos(t) +1) \mathbf{i} +\sin(t)\mathbf{j} + \frac{t^2}{2}\mathbf{k} \\
& = (2-\cos(t))\mathbf{i} + \sin(t)\mathbf{j} + \left( 1+ \frac{t^2}{2}\right)\mathbf{k}
\end{align*}
\vfill
\part[5] Compute the arclength of the curve for $0 \leq t \leq 2\pi$. 

\textit{Solution:} The arclength formula yields,
\begin{equation*}
l = \int_{0}^{2\pi} \|r(s)\| \, ds = \int_{0}^{2\pi} \sqrt{ \sin^2(s)+\cos^2(s)+s^2} \, ds = \int_{0}^{2\pi} \sqrt{ 1+s^2} \, ds.
\end{equation*}
This integral can be computed by a hyperbolic substitution, $s=\sinh(\theta)$, $ds=\cosh(\theta)d\theta$. Its value is 
\begin{equation*}
l = \pi\sqrt{1+4\pi^2} + \frac{\sinh^{-1}(2\pi)}{2}
\end{equation*}
\vfill
\end{parts}
\addpoints

%%%%%%%%%%%%%%%%%%%%

\newpage
\addpoints
\question Recall that the polar coordinate system in the plane is related to Cartesian coordinates by means of the equations
\begin{align}
x & = r\cos(\theta), \\
y & = r\sin(\theta).
\end{align}
\begin{parts}
\part[6] Differentiate equation (1) relative to $y$ to find a relation between $\frac{\partial r}{\partial y}$ and $\frac{\partial \theta}{\partial y}$.

\textit{Solution:} By the product rule, 
\begin{equation*} 
\nonumber 
0  = \frac{\partial r}{\partial y}\cos(\theta) -r\sin(\theta)\frac{\partial \theta}{\partial y}
\end{equation*}
\vfill
\part[6] Differentiate equation (2) relative to $y$ to find a second relation between $\frac{\partial r}{\partial y}$ and $\frac{\partial \theta}{\partial y}$.

\textit{Solution:} 
\begin{equation*}
1 =\frac{\partial r}{\partial y}\sin(\theta)+ r\cos(\theta)\frac{\partial \theta}{\partial y}
\end{equation*}
\vfill
\newpage
\part[8] By solving the system of equations obtained in the previous two steps, compute the derivatives $\frac{\partial r}{\partial x}$ and $\frac{\partial \theta}{\partial x}$.

\textit{Solution:} This problem can be solved by elimination. Multiplying equation 1 by $\sin(\theta)$ and equation 2 by $-\cos(\theta)$ and adding the resulting equations, we obtain
\begin{equation*}
\frac{\partial \theta}{\partial y} = \frac{\cos(\theta)}{r}
\end{equation*}
Substituting this value within one of the equations of the system yields
\begin{equation*}
\frac{\partial r}{\partial y} = \sin(\theta)
\end{equation*}
\end{parts}


%%%%%%%%%%%%%%%%%%%%%%%%%%

\end{questions}
\end{document}

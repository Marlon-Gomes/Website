% Exam Template for UMTYMP and Math Department courses
%
% Using Philip Hirschhorn's exam.cls: http://www-math.mit.edu/~psh/#ExamCls
%
% run pdflatex on a finished exam at least three times to do the grading table on front page.
%
%%%%%%%%%%%%%%%%%%%%%%%%%%%%%%%%%%%%%%%%%%%%%%%%%%%%%%%%%%%%%%%%%%%%%%%%%%%%%%%%%%%%%%%%%%%%%%

% These lines can probably stay unchanged, although you can remove the last
% two packages if you're not making pictures with tikz.
\documentclass[11pt]{exam}
\RequirePackage{amssymb, amsfonts, amsmath, latexsym, verbatim, xspace, setspace}


% By default LaTeX uses large margins.  This doesn't work well on exams; problems
% end up in the "middle" of the page, reducing the amount of space for students
% to work on them.
\usepackage[margin=1in]{geometry}
\usepackage[english]{babel}
\usepackage[autostyle]{csquotes} %%%% This package allows Tex to recognize quotation marks with the \enquote command. 


% Here's where you edit the Class, Exam, Date, etc.
\newcommand{\class}{MAT 203}
\newcommand{\term}{Summer I 2020}
\newcommand{\examnum}{Midterm}
\newcommand{\examdate}{06/11/20}
\newcommand{\timelimit}{3 hours and 5 minutes}

% For an exam, single spacing is most appropriate
\singlespacing
% \onehalfspacing
% \doublespacing

% For an exam, we generally want to turn off paragraph indentation
\parindent 0ex
\title{MAT 203 - Summer I 2020: Midterm}
\begin{document} 

% These commands set up the running header on the top of the exam pages
\pagestyle{head}
\firstpageheader{}{}{}\textbf{}
\runningheader{\class}{\examnum\ - Page \thepage\ of \numpages}{\examdate}
\runningheadrule

\begin{flushright}
\begin{tabular}{p{2.8in} r l}
\textbf{\class} & \textbf{Name (Print):} & \makebox[2in]{\hrulefill}\\
\textbf{\term} &&\\
\textbf{\examnum} &&\\
\textbf{\examdate} &&\\
\textbf{Time Limit: \timelimit} & ID number & \makebox[2in]{\hrulefill}
\end{tabular}\\
\end{flushright}
\rule[1ex]{\textwidth}{.1pt}

\begin{center}
\large{\textbf{Instructions}}
\end{center}

\begin{minipage}[t]{3.7in}
\vspace{0pt}
\begin{itemize}

%\item This exam contains \numpages\ pages (including this cover page) and
%\numquestions\ problems.  Check to see if any pages are missing.  Enter
%all requested information on the top of this page, and put your initials
%on the top of every page, in case the pages become separated.

\item You may use your textbook and class notes. You may not use calculators.

\item \textbf{Organize your work}, in a reasonably neat and coherent way, in
the space provided. Work scattered all over the page without a clear ordering will 
receive very little credit.  

\item \textbf{Mysterious or unsupported answers will not receive full
credit}.

\end{itemize}

\end{minipage}
\hfill
\begin{minipage}[t]{2.3in}
\vspace{0pt}
%\cellwidth{3em}
\gradetablestretch{2}
\vqword{Problem}
\addpoints % required here by exam.cls, even though questions haven't started yet.	
\gradetable[v]%[pages]  % Use [pages] to have grading table by page instead of question

\end{minipage}
\newpage % End of cover page

%%%%%%%%%%%%%%%%%%%%%%%%%%%%%%%%%%%%%%%%%%%%%%%%%%%%%%%%%%%%%%%%%%%%%%%%%%%%%%%%%%%%%
%
% See http://www-math.mit.edu/~psh/#ExamCls for full documentation, but the questions
% below give an idea of how to write questions [with parts] and have the points
% tracked automatically on the cover page.
%
%
%%%%%%%%%%%%%%%%%%%%%%%%%%%%%%%%%%%%%%%%%%%%%%%%%%%%%%%%%%%%%%%%%%%%%%%%%%%%%%%%%%%%%

\begin{questions}

% Basic question
%%%%%%%%%%%%%%

\addpoints
\question In each of the following problems, determine if the statements are true or false. Explain your reasoning (correct answers without an explanation will be worth only 2 points per statement). 
\begin{parts}
\part[5] Two planes in space always intersect along a line. 
\vfill
\part[5] Non-parallel lines in space always intersect. 
\vfill
\part[5] The cross product can be used to detect whether two non-zero vectors in space are perpendicular. 
\vfill
\part[5] Let $r(t)$ and $s(t)$ be curves in the plane, such that neither has a limit as $t$ converges to $0$. Then their dot product $r(t)\cdot s(t)$ does not have a limit at $0$ either. 
\vfill
\part[5] If the limit of a scalar-valued, multivariable function exists at a point, then all directed limits at that point exist and coincide.
\vfill
\part[5] If all the partial derivatives of a scalar-valued, multivariable function exist at a point, then all directional derivatives at that point exist.
\vfill
\end{parts}

\newpage 
\addpoints
\question Consider the lines whose parametric equations are given by 
\begin{align*}
L_1 \colon  & x=3t, y=2-t, z=-1+t.\\
L_2 \colon & x=1+4s, y=-2+s, z=-3-3s.
\end{align*}

\begin{parts}
\part[5] Write symmetric equations for each line
\vfill
\part[5] Explain why these lines do not intersect. 
\vfill
\newpage
\part[5] Explain why these lines are not parallel. 
\vfill
\part[5]  Find the general equations of two parallel planes, $\Pi_1$ and $\Pi_2$, containing lines $L_1$ and $L_2$, respectively. 
\vfill
\end{parts}
\newpage

%%%%%%%%%%%%%%%%%%%%%%%%%%%%%%%%%%%
\addpoints 
\question A line and a plane are said to be parallel if they do not intersect. Consider the line $L$ whose symmetric equations are
\begin{equation*}
x-1 = \frac{y-1}{-2} = z-1,
\end{equation*}
and the plane $\Pi$ with equation
\begin{equation*}
x+y+z=1.
\end{equation*}

\begin{parts}
\part[3] Find a vector perpendicular to the plane. 
\vfill
\part[3] Write parametric equations for the line. Clearly identify a direction vector.
\vfill
\part[4] Check that the line $L$ and the plane $\Pi$ are parallel by verifying that the vector perpendicular to the plane you found on part (a) is also perpendicular to the direction vector you found in part (b).
\vfill
\newpage
\part[5] The point $P = (1,1,1)$ belongs to the line $L$. Write an equation of a line perpendicular to the plane, passing through $P$. 
\vfill
\part[5] The line you found on part (d) should intersect the plane at a point $Q$. Find the distance between $P$ and $Q$. 
\vfill 
\end{parts}



%%%%%%%%%%%%%%%%%%%%%%%%%%
\newpage
\addpoints
\question You are given a curve with velocity vector 
\begin{equation*}
r'(t)= \sin(t)i + \cos(t)j + t k,
\end{equation*}
and such that $r(0)=(1,0,1)$.
\begin{parts}
\part[5] Compute the trajectory of a curve, as a function of $t$. 
\vfill
\part[5] Compute the arclength of the curve for $0 \leq t \leq 2\pi$. 
\vfill
\end{parts}
\addpoints
%%%%%%%%%%%
\newpage

\question Recall that the polar coordinate system in the plane is related to Cartesian coordinates by means of the equations
\begin{align}
x & = r\cos(\theta), \\
y & = r\sin(\theta).
\end{align}
\begin{parts}
\part[6] Differentiate equation (1) relative to $y$ to find a relation between $\frac{\partial r}{\partial y}$ and $\frac{\partial \theta}{\partial y}$.
\vfill 
\newpage 
\part[6] Differentiate equation (2) relative to $y$ to find a second relation between $\frac{\partial r}{\partial y}$ and $\frac{\partial \theta}{\partial y}$.
\vfill
\newpage
\part[8] By solving the system of equations obtained in the previous two steps, compute the derivatives $\frac{\partial r}{\partial y}$ and $\frac{\partial \theta}{\partial y}$.
\end{parts}
\addpoints

\end{questions}
\end{document}
\documentclass[12pt,oneside]{exam}

% This package simply sets the margins to be 1 inch.
\usepackage[margin=1in]{geometry}

% These packages include nice commands from AMS-LaTeX
\usepackage{amssymb,amsmath,amsthm,amsfonts,latexsym,verbatim,xspace,setspace}
\usepackage{hyperref}

% Make the space between lines slightly more
% generous than normal single spacing, but compensate
% so that the spacing between rows of matrices still
% looks normal.  Note that 1.1=1/.9090909...
\renewcommand{\baselinestretch}{1.1}
\renewcommand{\arraystretch}{.91}

% Define environments for exercises.
\newenvironment{exercise}[1]{\vspace{.1in}\noindent\textbf{Exercise #1 \hspace{.05em}}}{}
\newenvironment{newsolution}{\vspace{.1in}\noindent\textbf{Solution: \hspace{.05em}}}{}

% define shortcut commands for commonly used symbols
\newcommand{\R}{\mathbb{R}}
\newcommand{\C}{\mathbb{C}}
\newcommand{\Z}{\mathbb{Z}}
\newcommand{\Q}{\mathbb{Q}}
\newcommand{\N}{\mathbb{N}}
\newcommand{\calP}{\mathcal{P}}

\DeclareMathOperator{\vsspan}{span}

\title{Math 511 - Summer I 2019: Solutions to Homework 1}

%%%%%%%%%%%%%%%%%%%%%%%%%%%%%%%%%%%%%%%%%%

\begin{document}

\begin{flushright}
\sc MAT 203 - Summer I 2020\\
June 2, 2020
\end{flushright}
\bigskip
 
\begin{center}
\textsf{Homework 1 solutions} 
\end{center}

%%%%%%%%%%%%%%%%%%%%%%%%%%%%%%%%%%%%%%%%

\begin{exercise}{1}
Find the distance between the points $(-2,1,-5)$ and $(4,-1,-1)$.

\end{exercise}

\begin{newsolution} 
The distance formula yields
\begin{equation*}
d = \sqrt{(-2-4)^2+(1-(-1))^2+(-5-(-1))^2} = \sqrt{56}
\end{equation*}
\end{newsolution}

\begin{exercise}{2}
Initial and terminal points are given, (6,2,0), (3,-3,8), respectively. 
\begin{parts}
\part Sketch the directed line segment. 
\part Find the component form of the vector.
\part Write the vector using standard unit vector notation. 
\part Sketch the vector with its initial point at the origin.
\end{parts}
\end{exercise}

\begin{newsolution}
\begin{parts}
\part See class notes for similar problem.
\part $v=(3,-3,8)-(6,2,0) = (-3,-5,8)$.
\part $v = -3i -5j +8k$.
\part See class notes for similar problem.
\end{parts}
\end{newsolution}


\begin{exercise}{3}
Use vectors to determine whether the points (5,-4,7), (8,-5,5) and (11,6,3) are collinear.
\end{exercise}

\begin{newsolution}
Taking the point $(5,-4,7)$ as our base-point, we may define vectors 
\begin{align*}
u & = (8,-5,-5) - (5,-4,7) = (3,-1,-12) \\
v & = (11,6,3) - (5,-4,7) = (6,10,-4)
\end{align*}
A direct comparison between corresponding coordinates shows that these vectors are not scalar multiples of each other, hence the three given points are not collinear. 
\end{newsolution} 

\begin{exercise}{4}
You are given points P = (2,-1,3), Q = (0,5,1), R = (5,5,0). Let u be the vector from P to Q, v be the vector from P to R. Find
\begin{parts}
\part the component forms of u and v
\part $u \cdot v$
\part $v \cdot v$.
\end{parts}
\end{exercise}

\begin{newsolution}
\begin{parts}
\part The vectors $u$ and $v$ are 
\begin{align*}
u & = (0,5,1) - (2,-1,3) = (-2,6,-2)\\
v & = (5,5,0) - (2,-1,3) = (3,6,-3)
\end{align*}
\part 
\begin{equation*}
u \cdot v = (-2,6,-2) \cdot (3,6,-3) = (-2)\cdot 3 + 6 \cdot 6 + (-2)\cdot (-3) = 36.
\end{equation*}
\part 
\begin{equation*}
v \cdot v = (3,6,-3)\cdot (3,6,-3) = 3^2+6^2+(-3)^2 = 54
\end{equation*}
\end{parts}
\end{newsolution}

\begin{exercise}{5}
Find the angle between the vectors u=(1,0,-3) and v=(2,-2,1)
\begin{parts}
\part in radians;
\part in degrees.
\end{parts}
\end{exercise}

\begin{newsolution}
The angle between two vectors may be computed by means of the formula
\begin{equation*}
u \cdot v = \|u \| \|v\| \cos(\theta).
\end{equation*}
Using the given values of $u$ and $v$, we find
\begin{align*}
\cos(\theta)  & = \frac{1\cdot 2+ 0 \cdot (-2) + (-3) \cdot 1}{\sqrt{1^2+0^2+(-3)^2}\sqrt{2^2+(-2)^2+1^2}} \\
& = -\frac{1}{3\sqrt{10}}
\end{align*}
We find the values of the angle by using the arc-cosine function:
\begin{parts}
\part $1.676$ radians
\part$96.05^{\circ}$
\end{parts}
\end{newsolution}. 

\begin{exercise}{6}
You are given vectors u = (1,-1,1), v=(2,0,2). Find 
\begin{parts}
\part the projection of u onto v;
\part the vector component of u orthogonal to v.
\end{parts}
\end{exercise}

\begin{newsolution}
\begin{parts}
\part Recall that the norm of the projection vector is given by 
\begin{align*}
\| \mathrm{proj}(u,v)\| & = \frac{u \cdot v}{\|v\|} \\
& = \frac{1 \cdot 2 + (-1) \cdot 0 + 1 \cdot 2}{\sqrt{2^2+0^2+2^2}}\\
& = \frac{4}{2\sqrt{2}}\\
& = \sqrt{2}
\end{align*}
It follows that the projection vector is 
\begin{align*}
\mathrm{proj}(u,v) & = \frac{\sqrt{2}}{\|v\|}v\\
& = \frac{\sqrt{2}}{2\sqrt{2}}(2,0,2) \\
& = (1,0,1)
\end{align*}
\part The complement vector $w$ is defined by the equation 
\begin{equation*}
\mathrm{proj}(u,v) + w = u.
\end{equation*}
Using the given value of $u$ and the value we found for the projection on part $(a)$, we conclude
\begin{equation*}
w = (1,-1,1)-(1,0,1) = (0,-1,0).
\end{equation*}
\end{parts}
\end{newsolution}

\begin{exercise}{7}
You are given vectors u = (0,2,1), v=(1,-3,4). Find
\begin{parts}
\part $u \times v$;
\part $ v \times u$;
\part $v \times v$.
\end{parts}
\end{exercise}

\begin{newsolution}
\begin{parts}
\part We will use the determinants in this part:
\begin{align*}
u \times v & = 
\begin{vmatrix}
2 & 1 \\
1 & 4
\end{vmatrix}i + 
\begin{vmatrix}
0 & 1 \\
1 & 4
\end{vmatrix}j +
\begin{vmatrix}
0 & 2 \\
1 & -3
\end{vmatrix}k \\
& = (2\cdot 4 - 1\cdot(-3)) i - (0\cdot 4- 1\cdot 1)j + (0\cdot (-3) -1 \cdot 2)k \\
& = 11i +j -2k 
\end{align*}
\part In this case we will use multiplication rules for $i$, $j$ and $k$
\begin{align*}
v \times u & = (i-3j+4k)\times (2j+k) \\
& = 2 i \times j -6 j \times j + 8 k \times j + i \times k -3 j \times k + 4 k \times k \\
& = 2k - 8i-j-3i\\
& = -11i - j + 2k
\end{align*}
\part As we saw in class, the cross product between a vector and itself is $0$. 
\end{parts}
\end{newsolution}

\begin{exercise}{8}
You are given points (-1,4,3), (8,10,5). Find sets of 
\begin{parts}
\part parametric equations, and
\part symmetric equations.
\end{parts}
for the line that passes through the points (write the direction numbers as integers).
\end{exercise}

\begin{newsolution}
\begin{parts}
\part A parametric equation can be found by describing a base-point and a direction vector. We will choose $(-1,4,3)$ as our base point. The difference
\begin{equation*}
u = (8,10,5)-(-1,4,3) = (9,6,2)
\end{equation*}
serves a direction vector. The parametric equation with this data is 
\begin{align*}
(x,y,z) & = (-1,4,3) + \lambda (9,6,2) \\
& = (-1+9\lambda, 4 + 6\lambda, 3+2\lambda).
\end{align*}
\part Emphasizing the parameter $\lambda$ in the equations found on part (a), we find
\begin{align*}
\lambda & = \frac{x-1}{9}, \\
\lambda & = \frac{y-4}{6}, \\
\lambda & = \frac{z-3}{2}.
\end{align*}
This yields the symmetric equations:
\begin{equation*}
\frac{x-1}{9} = \frac{y-4}{6} = \frac{z-3}{2}
\end{equation*}
\end{parts}
\end{newsolution}

\begin{exercise}{9}
Find a set of parametric equations for the line that passes though the point $(1,2,3)$ and is parallel to the line given by $x=y=z$. 
\end{exercise}

\begin{newsolution}
A direction vector for the line $x=y=z$ can be found by choosing two points on the line and finding their difference, e.g.
\begin{equation*}
v = (1,1,1)-(0,0,0) = (1,1,1).
\end{equation*}
We can thus express the parametric equations for the desired line as 
\begin{equation*}
(x,y,z) = (1,2,3) + \lambda(1,1,1).
\end{equation*}
\end{newsolution}

\begin{exercise}{10} 
Find the equation of the plane that passes through the point $(-2,3,1)$ and is perpendicular to $n=3i - j + k$.
\end{exercise}

\begin{newsolution}
The general form of the equation of a plane perpendicular to $n = 3i - j +k$ takes the form
\begin{equation*}
3x-y+z = D.
\end{equation*}
To find $D$, we substitute the coordinates of the point $(-2,3,1)$:
\begin{equation*}
3\cdot (-2) - 3 + 1 = D,
\end{equation*}
hence the equation for the plane is 
\begin{equation*}
3x-y+z = -8.
\end{equation*}
\end{newsolution}

\end{document}


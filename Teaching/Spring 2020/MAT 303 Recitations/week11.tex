% This text is proprietary.
% It's a part of presentation made by myself.
% It may not used commercial.
% The noncommercial use such as private and study is free
% Sep. 2005 
% Author: Sascha Frank 
% University Freiburg 
% www.informatik.uni-freiburg.de/~frank/


\documentclass{beamer}
\usepackage{graphicx}
\usepackage{amsmath}
\usepackage{amssymb}
\usepackage{xcolor}
\definecolor{mygreen}{cmyk}{0.82,0.11,1,0.25}
\begin{document}
\title{Spring 2020 MAT303 Recitations}   
\author{Week of 4/6/20: Sections 3.4 and 3.5} 
\date{} 

\frame{\titlepage} 


\frame{\frametitle{Section 3.4: Mechanical Vibrations}

Problems in this section deal with second-order differential equations of type
\begin{equation*}
mx''(t)+cx'(t)+kx(t) = F(t),
\end{equation*}
\pause
where $m,c,k$ are constants ($m \neq 0$). \pause In today's recitation, we will discuss a few examples of such equations.
\vspace{.1in}
}

\frame{\frametitle{Section 3.4: Mechanical vibrations}

The first few homework problems deal with homogeneous equations. \pause We learned how to express solutions as combinations of sines, cosines, and exponentials in the last section, but now we adopt a different convention regarding trigonometric representations. \pause We will write 
\begin{equation*}
A\cos(\omega_1 t)+ B \sin(\omega_1 t) = C\cos(\omega_1 t -\alpha),
\end{equation*}
\pause
where 
\begin{itemize}
\item $C=\sqrt{A^2+B^2}$ the amplitude; \pause
\item $\omega_1=\frac{\sqrt{4cm-k^2}}{2m}$ is the relative frequency;\pause
\item $\alpha \in [0, 2\pi)$ is the phase, so that $\cos(\alpha)=\frac{A}{C}$ and $\sin(\alpha)=\frac{B}{C}$.
\end{itemize} 

}

\frame{\frametitle{Section 3.4: Mechanical Vibrations}
Problem 3.4.4 concerns the movement of a mass-spring system, with:\pause
\begin{itemize}
\item $m=250g=0.25kg$
\item $k=9N/0.25m = 36N/m$.
\item at $t=0s$, $x(0)=1m, x'(0)=-5m/s$.
\end{itemize}
\vspace{.1in}

The movement of the system is described by the equation
\begin{equation*}
0.25x^{''}+36x=0 \Leftrightarrow x^{''}+144x=0,
\end{equation*}
\pause
thus the relative frequency is $\omega=12$. \pause A solution can be written in form
\begin{equation*}
x(t)=C\cos(12t-\alpha),
\end{equation*}
for constants $C>0$, $\alpha\in [0,2\pi)$ to be determined.
}

\frame{\frametitle{Section 3.4: Mechanical Vibrations}
Finding these constants amounts to using the initial conditions. \pause From $x(0)=1$, $x'(0)=-5$, we obtain
\begin{equation*}
C\cos(\alpha)  =1, \  C\sin(\alpha)  = -\frac{5}{12},
\end{equation*}
\pause 
thus $C=\sqrt{1+\left(\frac{5}{12}\right)^2}=\frac{13}{12}$. \pause The value of $\alpha$ (in radians) that satisfies the constraints above is found by calculator:
\begin{equation*}
\alpha \approx 5.89.
\end{equation*}
\vspace{0.1in}
Thus 
\begin{equation*}
x(t)\approx \frac{13}{12}\cos(12t-5.89).
\end{equation*}
\pause
Amplitude: $A=\frac{13}{12}$ meters. Period: $T=\frac{2\pi}{12}=\frac{\pi}{6}$ seconds.
}

\frame{\frametitle{Section 3.4: Mechanical Vibrations}
In problem 3.4.14, we have a mass-spring-dashpot system with constants
\begin{equation*}
m=25, c=10, k=226,
\end{equation*}
and initial data $x(0)=20, x'(0)=41$. \pause Its characteristic polynomial is 
\begin{equation*}
25r^2+10r+226 = (5r+1)^2+15^2, 
\end{equation*}
whose roots are 
\begin{equation*}
r=-\frac{1}{5}\pm 3i.
\end{equation*}
\pause 
\vspace{0.1in} 
Solutions to the differential equation take the form
\begin{equation*}
x(t)=Ce^{-\frac{t}{5}}\cos(3t-\alpha).
\end{equation*}
}

\frame{\frametitle{Section 3.4: Mechanical Vibrations}
To use the initial data, we compute the first derivative,
\begin{equation*}
x'(t)=-Ce^{-\frac{t}{5}}\left[\frac{\cos(3t-\alpha)}{5}+3\sin(3t-\alpha)\right],
\end{equation*}
\pause
hence
\begin{align*}
C\cos(\alpha) & = 20 \\
C\left[ - \frac{\cos(\alpha)}{5} + 3\sin(\alpha) \right] & = 41,
\end{align*}
\pause
from which we infer $C=25$, $\alpha \approx 0.64$, so 
\begin{equation*}
x(t) \approx 25e^{-\frac{t}{5}}\cos(3t-0.64).
\end{equation*}
}

\frame{\frametitle{Section 3.4: Mechanical Vibrations}
In problem 3.4.17, we are meant to contrast the damped motion of a spring-mass-dashpot system with its undamped counterpart. \pause The differential equation modelling this problem is
\begin{equation*}
x^{''}+8x^{'}+16x=0,
\end{equation*}
\pause with initial data $x(0)=5, x'(0)=-10$.
\vspace{0.1in}

The characteristic polynomial of the problem is 
\begin{equation*}
r^2+8r+16=(r+4)^2.
\end{equation*}
Its root is $-4$, with multiplicity 2, thus this motion is critically damped. 
}

\frame{\frametitle{Section 3.4: Mechanical Vibrations}
A solution takes the form 
\begin{equation*}
    x(t)=e^{-4t}(At+B), 
\end{equation*}
\pause
and has derivative
\begin{equation*}
x'(t) = e^{-4t}(-4At+A-4B).
\end{equation*}
\pause
\vspace{0.1in}
Subject to initial conditions $x(0)=5, x'(0)=-10$, we find 
\begin{equation*}
x(t)=e^{-4t}(2t+1).
\end{equation*}
}

\frame{\frametitle{Section 3.4: Mechanical Vibrations}
The equation that models undamped motion is 
\begin{equation*}
u^{''}(t)+16u(t) = 0.
\end{equation*}
\pause
Its solutions take the form 
\begin{equation*}
u(t)=C\cos(4t-\alpha).
\end{equation*}
\pause
Assuming the same initial conditions $u(0)=5, u'(0)=-10$, we have 
\begin{equation*}
u(t)\approx \frac{5\sqrt{5}}{2}\cos(4t-5.82)
\end{equation*}
}

\frame{\frametitle{Section 3.4: Mechanical Vibrations}
In problem 3.4.31, we have a mass-spring-dashpot system with constants $m=1, c=10$ and $k=125$, subject to initial conditions $x(0)=6$, $x'(0)=50$. The characteristic polynomial of the system is 
\begin{equation*}
r^2+10r+125 = (r+5)^2+10^2,
\end{equation*}
whose roots are $r=-5 \pm 10i$, characterizing underdamped motion. The corresponding solution to the diferential equation is 
\begin{equation*}
x(t)=Ce^{-5t}\cos(10t-\alpha),
\end{equation*}
for constants $C>0$, $\alpha \in [0,2\pi)$ to be determined in what follows.
}
\frame{\frametitle{Section 3.4: Mechanical Vibrations}
The first derivative of the solution takes the form
\begin{equation*}
x^{'}(t)  = Ce^{-5t}[-5\cos(10t-\alpha)-10\sin(10t-\alpha)].
\end{equation*}
\pause
The initial conditions amount to
\begin{align*}
C\cos(\alpha) & =6, \\
C[-10\cos(\alpha)-5\sin(\alpha)] & = 50,
\end{align*}
\pause
from which we infer $C=10$, $\alpha\approx 0.93$. The solution is thus approximated by
\begin{equation*}
x(t)\approx 10e^{-5t}\cos(10t-0.93).
\end{equation*}
}

\frame{\frametitle{Section 3.4: Mechanical Vibrations}
We will constrast this to solutions of the undamped motion, modelled by the equation
\begin{equation*}
u''+125u=0.
\end{equation*}
\pause
The characteristic polynomial is $p(r)=r^2+125$, with roots $r=\pm5\sqrt{5}i$. Solutions to this equation can be expressed as
\begin{equation*}
u(t)=C_{*}\cos(5\sqrt{5}t-\alpha_{*}),
\end{equation*}
\pause
where the constants $C_{*}>0$ and $\alpha_{*} \in [0, 2\pi)$ are determined by the initial conditions $u(0)=6, u'(0)=50$.
}
\frame{\frametitle{Section 3.4: Mechanical Vibrations}
The derivative of a solution takes the form 
\begin{equation*}
u'(t)=-5\sqrt{5}C_{*}\sin(5\sqrt{5}t-\alpha_{*}), 
\end{equation*}
\pause
so using the initial date we obtain 
\begin{align*}
C\cos(\alpha_{*}) & =6  \\
5\sqrt{5}C_{*}\sin(\alpha_{*}) & = 50,
\end{align*}
\pause
and infer $C_{*} = 2\sqrt{14}$, $\alpha_{*} \approx 0.64$. Finally, we motion of the undamped system is 
\begin{equation*}
u(t)  \approx 2\sqrt{14}\cos(5\sqrt{5}t-0.64).
\end{equation*}
}


\frame{\frametitle{Section 3.5: Homogeneous Equations with Constant Coefficients}
In problem 3.5.3, we consider the equation 
\begin{equation}\label{3.5.3}
y^{''}-y^{'}-6y=2\sin(3x).
\end{equation}
\pause
The associated homogeneous equation has characteristic polynomial
\begin{equation*}
r^2-r-6=(r+2)(r-3),
\end{equation*}
\pause
with roots $r_1=-2, r_2=3$. The corresponding solutions to the complementary equation take the form 
\begin{equation*}
y_c=A_1e^{-2x}+A_2e^{3x}.
\end{equation*}
}

\frame{\frametitle{Section 3.5: Homogeneous Equations with Constant Coefficients}
As the solutions to the homogeneous problem and the inhomogeneous term of equation \eqref{3.5.3} are linearly independent (verify!), we can use the method of undetermined coefficients to guess a particular solution of the form 
\begin{equation*}
y_p=C_1\sin(3x)+C_2\cos(3x).
\end{equation*}
\pause
To find the coefficients $C_1,C_2$, we need to compute the derivatives of the particular solution, 
\begin{align*}
y_{p}^{'} & = 3C_1\cos(3x)-3C_2\sin(3x), \\
y_{p}^{''} & = -9C_1\sin(3x)-9C_2\cos(3x).
\end{align*}
}
\frame{\frametitle{Section 3.5: Homogeneous Equations with Constant Coefficients}
Combining all of this data into the equation we get
\begin{equation*}
(-15C_1+3C_2)\sin(3x) + (-3C_1-15C_2)\cos(3x) = 2\sin(3x), 
\end{equation*}
\pause
from which we infer $C_1=-\frac{5}{39}$, $C_2=\frac{1}{39}$. \pause A general solution to the differential equation takes the form
\begin{equation*}
y(x)=A_1e^{-2x}+A_2e^{3x}-\frac{5}{39}\sin(3x)+\frac{1}{39}\cos(3x),
\end{equation*}
\pause
for constants $A_1$ and $A_2$ to be determined based upon initial conditions. 
}

%
%\section{Section no.3} 
%\subsection{Tables}
%\frame{\frametitle{Tables}
%\begin{tabular}{|c|c|c|}
%\hline
%\textbf{Date} & \textbf{Instructor} & \textbf{Title} \\
%\hline
%WS 04/05 & Sascha Frank & First steps with  \LaTeX  \\
%\hline
%SS 05 & Sascha Frank & \LaTeX \ Course serial \\
%\hline
%\end{tabular}}
%
%
%\frame{\frametitle{Tables with pause}
%\begin{tabular}{c c c}
%A & B & C \\ 
%\pause 
%1 & 2 & 3 \\  
%\pause 
%A & B & C \\ 
%\end{tabular} }
%
%
%\section{Section no. 4}
%\subsection{blocs}
%\frame{\frametitle{blocs}
%
%\begin{block}{title of the bloc}
%bloc text
%\end{block}
%
%\begin{exampleblock}{title of the bloc}
%bloc text
%\end{exampleblock}
%
%
%\begin{alertblock}{title of the bloc}
%bloc text
%\end{alertblock}
%}
\end{document}



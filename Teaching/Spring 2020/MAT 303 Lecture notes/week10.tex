\documentclass[11pt]{amsart}
%%%%%%%%%%%%%% Packages
\usepackage{amssymb,amsfonts,amsthm,amsmath}
\usepackage[english]{babel}
\usepackage[all,cmtip]{xy}
\usepackage{tikz}
\usepackage{mathtools}
\usepackage{tensor}
\usepackage{csquotes}
\usepackage[vcentermath]{youngtab}
%\usepackage{stix}
\usetikzlibrary{arrows,chains,matrix,positioning,scopes}
%\usepackage{mathrsfs}
%\usepackage[notcite,notref]{showkeys}


%%%%%%%%%%% Tikz
\makeatletter
\tikzset{join/.code=\tikzset{after node path={%
\ifx\tikzchainprevious\pgfutil@empty\else(\tikzchainprevious)%
edge[every join]#1(\tikzchaincurrent)\fi}}}
\makeatother
\tikzset{>=stealth',every on chain/.append style={join},
         every join/.style={->}}

        

%%%%%%%%%%%% Personalized commands and environments

\newcommand{\inputc}[1]{ \raisebox{-0.5\height}{\input{#1}} }
\newtheorem{theorem}{Theorem}
\newtheorem{lemma}[theorem]{Lemma}
\newtheorem{corollary}[theorem]{Corollary}
\newtheorem{definition}[theorem]{Definition}
\newtheorem{proposition}[theorem]{Proposition}
\newtheorem{remark}[theorem]{Remark}
\newtheorem{example}{Example}
%\newtheorem{theorem*}{Theorem}

\newcommand{\func}[3]{{#1} : {#2} \longrightarrow {#3}}
\numberwithin{equation}{section}
\newcommand{\dbar}{\bar{\partial}}
\newenvironment{myproof}{\noindent{it Proof}
\setlength{\parindent}{0mm}}
{$\hfill \bs$}


%%%%%%%%%%%%%%%%%%%%%%%% Title and Author information

\title{MAT 303 Recitations: Week 10}


\author[M. Gomes]{Marlon de Oliveira Gomes}
\address{Mathematics Department 3-101, Stony Brook University,
100 Nicolls Road, Math Tower, 
Stony Brook, NY, 11794, USA} \email{mgomes@math.stonybrook.edu}



%%%%%%%%%%% Text
\begin{document}

\maketitle


\section*{Section 3.2: General solutions of linear equations}

In the first few problems of Homework 6 you're tasked with finding coefficients of a complementary solution to an inhomogeneous equation of type
\begin{equation*}
y^{(n)}+p_1(x)y^{(n-1)} + \cdots p_{n-1}(x)y^{'}+p_n(x) = f(x).
\end{equation*}

Solutions to such equations are obtained by combining a {\emph{particular solution}}, $y_p$, to solutions $y_c$ of the associated homogeneous equation,
\begin{equation*}
y(x)=y_c(x)+y_p(x).
\end{equation*}

The example below is extracted from problem 3.2.23 in our textbook. 
\begin{example}

\begin{equation*}
\left\{
\begin{array}{rl}
y''-2y'-3y & = 6, \\
y(0) & = 3,\\
y'(0) & =11.
\end{array}
\right.
\end{equation*}


You are given the forms of complementary and particular solutions,
\begin{equation*}
y_c = c_1e^{-x}+c_2e^{3x}; \ \ y_p=-2
\end{equation*}

Combining these with the initial conditions, a solution to the problem is obtained by solving the algebraic system
\begin{equation*}
\left\{
\begin{array}{rl}
c_1+c_2 & = 5, \\
-c_1+ 3c_2 & = 11.
\end{array}
\right.
\end{equation*}

The solutions are $c_1=1, c_2=4$, thus $y(x)=e^{-x}+4e^{3x}-2$.
\end{example}

Problem 3.2.33 concerns the Wronskian criterion, which we recall below:

\begin{theorem}
Let $y_1,y_2,\cdots, y_n$ be functions defined on an interval $I$, \underline{assumed to solve} an $n$-th order homogeneous, linear differential equation, with \underline{continuous coefficients}
\begin{equation*}
y^{(n)}+p_1(x)y^{(n-1)} + \cdots p_{n-1}(x)y^{'}+p_n(x) = 0.
\end{equation*}
\end{theorem}

Then either

\begin{enumerate}
    \item[(a)] Their Wronskian is identically zero on I, in which case the functions are L.D, or
    \item[(b)] Their Wronskian is nowhere zero on I, and the funcions are L.I.
\end{enumerate}

The following example, extracted from problem 3.2.30, illustrates well the subtleties of this criterion.  

\begin{example}
Consider the equation
\begin{equation*}
    x^2y^{''}-2xy^{'}+2y=0.
\end{equation*}
It is easy to verify that the functions $y_1(x)=x, y_2(x)=x^2$ are solutions on $\mathbb{R}$. These functions are independent, as if the combination $y(x)=ax+bx^2$ is null both coefficients must vanish. 
Meanwhile,
\begin{equation*}
    W(y_1,y_2)=\det\left(
    \begin{matrix}
    x & x^2 \\
    1 & 2x
    \end{matrix}
    \right) = 2x^2-x^2=x^2,
\end{equation*}
a function which vanishes when $x=0$. Why is this not a contradiction?

{\bf Warning:} the Wronskian criterion applies only to \underline{solutions} of equations with \underline{continuous coefficients},  when written in \underline{standard form}. The standard form of the previous equation is 
\begin{equation*}
    y^{''}-\frac{2}{x}y^{'}+\frac{2}{x^2}y=0,
\end{equation*}
discontinuous at $x=0$,  precisely where the criterion fails.

As a consequence there is no linear, homogeneous, second-order differential equation defined on $\mathbb{R}$ admitting $x$ and $x^2$ as solutions! Think about problem 3.2.33 in this context. Why is the Wronskian criterion applicable to the three functions involved? Can you think of a differential equation whose solutions are given by such functions?
\end{example}

\section*{Section 3.3: Homogeneous Equations with Constant Coefficients}

The method of characteristic equations assigns to an equation of type 
\begin{equation*}
    a_ny^{(n)}+a_{n-1}y^{(n-1)} + \cdots a_1y^{'}+a_0 = 0
\end{equation*}
its characteristic polynomial:
\begin{equation*}
    p(r)= a_nr^n+a_{n-1}r^{n-1}+\cdots +a_1r + a_0,
\end{equation*}
where each $a_i$ is a constant, $a_n \neq 0$,

To each (possibly complex) root $r$ of the polynomial, there corresponds a solution of the differential equation of the form $e^{rx}$. If the root has a multiplicity, then new solutions may be generated by monomial multiplication: $xe^{rx}, x^2e^{rx}, \cdots, x^{m-1}e^{rx}$, where $m$ is the multiplicity. 

The following example is extracted from problem 3.3.12 in our textbook.
\begin{example}
\begin{equation*}
    y^{(4)}-3y^{(3)}+3y^{''}-y^{'}=0.
\end{equation*}

Its characteristic polynomial is 
\begin{equation*}
     p(r)  = r^4-3r^3+3r^2-r = r(r-1)^3.
\end{equation*}
 The roots are $r_1=0$, with multiplicity one, and $r_2=1$, with multiplicity three.  The general solution of the equation takes the form
 \begin{equation*}
     y(x)=A + Be^x+Cxe^x+Dx^2e^x.
 \end{equation*}
\end{example}

The next example is a slight modification of problem 3.3.23.
\begin{example}
Consider the equation
\begin{equation*}
    y''-8y'+25y = 0,
\end{equation*}
with initial conditions $y(0)=1, y'(0)=2$.

The characteristic equation of the problem is 
\begin{equation*}
    r^2-8r+25=0, 
\end{equation*}
and its solutions are the complex numbers $r_1=4+3i$, $r_2=4-3i$.  The general real solution of the D.E. takes the form
\begin{equation*}
y(x)=e^{4x}(A\cos(3x)+B\sin(3x)).
\end{equation*}

To find the specific values of $A$ and $B$, we impose initial conditions.  Since  
\begin{equation*}
    y'(x)=4y(x)-3e^{4x}(A\sin(3x)-B\cos(3x)),
\end{equation*}
we have at $x=0$: 
\begin{align*}
    A  & = 1\\
    4A+3B & = 2,
\end{align*}
thus the desired solution is 
\begin{equation*}
    y(x)=e^{4x}\left(\cos(3x)-\frac{2}{3}\sin(3x)\right).
\end{equation*}
\end{example}

In problems 3.3.44 and 3.3.46, you are confronted with differential equations with {\emph{complex coefficients}}. In this case the roots of the characteristic polynomial need not be conjugates to each other, as the following example (problem 3.3.45) shows. 

\begin{example}
The differential equation
\begin{equation*}
    y''-2iy'+3y =0
\end{equation*}
has characteristic equation $r^2-2ir+3=0,$ whose solutions are $-i$ and $3i$.  The complex-valued solutions to the D.E. are:
\begin{equation*}
    y_1(x)= e^{-ix} = \cos(x)-i\sin(x),
\end{equation*}
and 
\begin{equation*}
    y_2(x)= e^{3ix} = \cos(3x) + i\sin(3x).
\end{equation*}
\end{example}

In problem 3.3.52 you are tasked with solving an Euler equation by means of a logarithmic substitution. Let's see this procedure in practice by solving problem 3.3.53.

\begin{example}
Consider the equation
\begin{equation*}
    x^2y^{''}+7xy^{'}+25y=0.
\end{equation*}
Using the substitution $v=\log(x), x>0$, we get 
\begin{align*}
    \frac{dy}{dx} & = \frac{1}{x}\frac{dy}{dx} \\ 
    \frac{d^2y}{dx^2} & = -\frac{1}{x^2}\frac{dy}{dv}+\frac{1}{x^2}\frac{d^2y}{dx^2}
\end{align*}

Rephrasing the original equation in terms of $v$ yields
\begin{equation*}
    \frac{d^2y}{dv^2}+6\frac{dy}{dv}+25y = 0,
\end{equation*}
whose solutions (obtained via characteristic equation) take the form
\begin{equation*}
    y(v)=e^{-3v}(A\cos(4v)+B\sin(4v)), 
\end{equation*}

or, in terms of $x$, 
\begin{equation*}
    y(x)=x^{-3}(A\cos(4\log(x))+B\sin(4\log(x))).
\end{equation*}
\end{example}

\begin{thebibliography}{0}

\bibitem{EdPeCa} C. Henry Edwards, David E. Penney and David T. Calvis, {\it Differential Equations and Boundary Value Problems: Computing and Modelling}, 5th edition, Pearson, 2014.

\end{thebibliography}
\end{document}



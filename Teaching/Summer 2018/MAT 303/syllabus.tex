\documentclass[11pt]{amsart}
\usepackage{amssymb,amsfonts,amsthm,amsmath}
\usepackage[english]{babel}
\usepackage[all,cmtip]{xy}
\usepackage{hyperref}
\usepackage{url}
%\usepackage{mathrsfs}
%\usepackage[notcite,notref]{showkeys}
\def\limproj{\mathop{\oalign{lim\cr\hidewidth$\longleftarrow$\hidewidth\cr}}}

%--------------
% macro perso
%--------------
\newcommand{\inputc}[1]{ \raisebox{-0.5\height}{\input{#1}} }
\newtheorem{theorem}{Theorem}[section]
\newtheorem{lemma}[theorem]{Lemma}
\newtheorem{corollary}[theorem]{Corollary}
\newtheorem{definition}[theorem]{Definition}
\newtheorem{proposition}[theorem]{Proposition}
\newtheorem{remark}[theorem]{Remark}
\newtheorem{example}{Example}[section]
%\newtheorem{theorem*}{Theorem}

\newcommand{\func}[3]{{#1} : {#2} \longrightarrow {#3}}
\numberwithin{equation}{section}

\newenvironment{myproof}{\noindent{it Proof}
\setlength{\parindent}{0mm}}
{$\hfill \bs$}

\title[MAT 303 Syllabus - Summer 2018]{MAT 303 Syllabus - Summer 2018}

\author[M. Gomes]{Marlon Gomes}
%

\begin{document}
\maketitle

\section{Course Description}
\subsection{Course overview}
 This is a course about ordinary differential equations (ODEs, for short). Our goal is threefold: to understand how such equations come about in Mathematics and other sciences; to understand quantitative aspects of elementary equations and systems; to learn how to predict the qualitative behavior of equations without solving them.
 
      The first part of our course will focus on understanding phenomena modeled by differential equations. We will then introduce simple qualitative tools to understand the behaviour of such phenomena (slope fields, integral curves, phase diagrams, critical points, analysis of stability). 
      
      The second (and longest) part of the course will focus on quantitative behavior of elementary ODEs. Topics will include:
\begin{itemize}
  \item Exact solutions of first-order initial-value problems; 
  \item Approximation methods for first-order equations; 
  \item Linear, higher-order initial-value problems with constant coefficients (the characteristic method, undetermined coefficients and variation of parameters), with applications; 
  \item Introduction to second-order boundary-value problems and eigenvalue problems; 
  \item Power series methods for second-order equations: classification of ordinary and singular points of differential equations; solutions near ordinary points; introduction to the Frobenius method. 
\end{itemize}    

The last part of the course is about systems of differential equations. Topics will include:
\begin{itemize}
\item first-order systems aas a way to encode higher-order differential equations; 
\item Linear, first-order, homegeneous systems with constant coefficients and the eigenvalue method; 
\item Generalized eigenvalues; 
\item Matrix exponentials; 
\item Phase portraits of 2D systems and stability.
\end{itemize}

\subsection{Requisites}
C or higher in MAT 127 or 132 or 142 or AMS 161 or level 9 on the mathematics placement examination. Students can find the syllabi for these other MAT courses at 
\begin{center}
\url{http://www.math.stonybrook.edu/mathematics-department-course-web-pages}.
\end{center}

Needless to say, being confortable with properties of derivatives, such as the product and chain rules, is a must. Integration techniques, specially the substitution method, will be used very frequently in the second part of the course. Students should have a basic understanding of operations with power series, and know how to compute their radii of convergence, as well as elementary Taylor series (notably for the sine, cosine and exponential functions).

      Linear algebra requirements will be kept to a minimum, as we will only study systems of equations with two or three variables. At the appropriate time, we will introduce/review the basic algebra of matrices and determinants (of small matrices). No prior knowlegde of complex numbers will be required.
      
\subsection{Textbook}
Differential equations and boundary-value problems, by C. Henry Edwards, David E. Penney and David T. Calvis (5th edition). 

\subsection{Important Times and Dates}
\begin{itemize}
\item Lectures: Mondays, Wednesdays, and Thursdays, from 6:00 pm to 9:05 pm, in Harriman Hall 111.
\item Midterm: July 26th, 6:00 pm to 9:05 pm, in Harriman Hall 111. 
\item Final Exam: August 16th, 6:00 pm to 9:05 pm, in Harriman Hall 111.
\end{itemize}

\subsection{Assignments}
 Your assignments are an important part of the course (and your grades). There will be four problem sets, due on Mondays of weeks 2, 3, 5 and 6. These assignments will be available on the course webpage (url below, in the Contact section) and Blackboard. 
 
    Problems will range from simple manipulations of the concepts developed in class on that week to more involved applications of these concepts. There will also be suggestions of problems that require the use of machine-assisted computation. These will be clearly marked as optional.
    
    Not all problems will be graded, but you should attempt to solve them all anyway, as they will serve as your foundation for the problems you will see on your exams. Graded problems will be discussed in class. Solutions to selected non-graded problems will be posted on the course webpage weekly.  The optional problems mentioned above will not be graded, and their solutions will not be posted the course webpage. 
    
    If you would like to know how to solve a problem which was not solved in class, whose solution is not posted on the webpage, or if you'd simply like to go over parts of the solution you don't understand, I'd be happy to discuss it with you during office hours, or via e-mail. 
    
    \textbf{Due dates will be strictly enforced}: homework is due by 6:00 pm on the dates described on the schedule. If you cannot make it to class (in time or at all), send me a scanned copy of your homework via e-mail by the deadline (a follow-up physical copy would be appreciated).
    
      You will also have reading assignments. These assignments will complement the lectures, and are an important part of the course. They will most often be examples, but from time to time may contain theory which is only tangentially related to the course (e.g. matrix algebra, complex numbers), or applications of the concepts studied in this course in science and technology. \textbf{The content of these reading assignments will be subject of homework problems and exams.}

\subsection{Grades}
Your grade will be calculated in the following way:
\begin{enumerate}
\item Homework assignments: 10$\%$ each.
\item Midterm: 30$\%$.
\item Final Exam: 30$\%$.
\end{enumerate}

\section{Contact}
My office hours will be announced on the course webpage, 
\begin{center}
\url{http://www.math.stonybrook.edu/~mgomes/mat303sum18.html}.
\end{center}
E-mail is the best form of communication besides lectures and office hours. My address is 
\begin{center}
\href{mailto: mgomes@math.stonybrook.edu}{mgomes@math.stonybrook.edu}.
\end{center}

\section{DSS Notice}
If you have a physical, psychological, medical, or learning disability that may impact your course work, please contact Disability Support Services at (631) 632-6748 or at 
\begin{center}
\url{http://studentaffairs.stonybrook.edu/dss/}.
\end{center}
They will determine with you what accommodations are necessary and appropriate. All information and documentation is confidential. Students who require assistance during emergency evacuation are encouraged to discuss their needs with their professors and Disability Support Services. For procedures and information go to the following website: http://www.sunysb.edu/

\section{Academic Integrity}
Each student must pursue his or her academic goals honestly and be personally accountable for all submitted work. Representing another person's work as your own is always wrong. Faculty are required to report any suspected instances of academic dishonesty to the Academic Judiciary. Faculty in the Health Sciences Center (School of Health Technology and Management, Nursing, Social Welfare, Dental Medicine) and School of Medicine are required to follow their school-specific procedures. For more comprehensive information on academic integrity, including categories of academic dishonesty, please refer to the academic judiciary website at 
\begin{center}
\url{http://www.stonybrook.edu/uaa/academicjudiciary/}
\end{center}

\section{Critical Incident Management Statement}
Stony Brook University expects students to respect the rights, privileges, and property of other people. Faculty are required to report to the Office of Judicial Affairs any disruptive behavior that interrupts their ability to teach, compromises the safety of the learning environment, or inhibits students' ability to learn. Faculty in the HSC Schools and the School of Medicine are required to follow their school-specific procedures.
\end{document}
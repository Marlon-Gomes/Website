\documentclass[12pt,oneside]{exam}

% This package simply sets the margins to be 1 inch.
\usepackage[margin=1in]{geometry}

% These packages include nice commands from AMS-LaTeX
\usepackage{amssymb,amsmath,amsthm,amsfonts,latexsym,verbatim,xspace,setspace}

% Make the space between lines slightly more
% generous than normal single spacing, but compensate
% so that the spacing between rows of matrices still
% looks normal.  Note that 1.1=1/.9090909...
\renewcommand{\baselinestretch}{1.1}
\renewcommand{\arraystretch}{.91}

% Define an environment for exercises.
\newenvironment{exercise}[1]{\vspace{.1in}\noindent\textbf{Exercise #1 \hspace{.05em}}}{}

% define shortcut commands for commonly used symbols
\newcommand{\R}{\mathbb{R}}
\newcommand{\C}{\mathbb{C}}
\newcommand{\Z}{\mathbb{Z}}
\newcommand{\Q}{\mathbb{Q}}
\newcommand{\N}{\mathbb{N}}
\newcommand{\calP}{\mathcal{P}}

\DeclareMathOperator{\vsspan}{span}

\title{Math 303 - Summer II 2018: Homework 2}

%%%%%%%%%%%%%%%%%%%%%%%%%%%%%%%%%%%%%%%%%%

\begin{document}

\begin{flushright}
\sc MAT 303 - Lecture 1\\
July 9, 2018.
\end{flushright}
\bigskip

This homework is due on Monday, 7/23, in class, by 6:00 pm. 
\begin{center}
\textsf{Homework 2} 
\end{center}

%%%%%%%%%%%%%%%%%%%%%%%%%%%%%%%%%%%%%%%%

Note: Warm-up problems do not need to be turned in. 

\begin{exercise}{1}
Warm-up by solving a couple of problems among problems 1-8, section 2.1. Don't simply apply the formulas from the textbook: try solving the equations by yourself! Also try drawing their slope fields, and compare your sketch to the answers in the textbook.
\end{exercise}

\begin{exercise}{2}
Solve problems 30 and 31, section 2.1. 
\end{exercise}

\begin{exercise}{3}
Read the \textit{Historical Note} in section 2.1 (examples 4 and 5). Then solve problems 36 and 37 in the same section (you may want to use a calculator on this one!). 
\end{exercise}

\begin{exercise}{4} 
Solve problem 39, section 2.1. When constructing the graph, you may choose specific values of $k$, $b$ and $P_0$ for your comparison (just keep it reasonably realistic: $b$ should be small relative to $k$ and $P_0$) . If you have acess to a graphing calculator, try plotting various graphs of the two populational models for different triples $(k,b,P_0)$ (you do not need to turn in many graphs, one is enough). 
\end{exercise}

\begin{exercise}{5}
Solve problem 23, section 2.2. Note: this problem is qualitative, and should be treated as such. You do not need to compute the limiting populations by hand (although you could certainly solve the problem in this way). 
\end{exercise}

\begin{exercise}{6} 
Section 2.3, problem 10. 
\end{exercise}

\begin{exercise}{7}
Section 2.3, problem 12.
\end{exercise}

\begin{exercise}{8}
Read the subsections on \textit{Variable Gravitational Acceleration} and \textit{Escape Velocity}, section 2.3. Then solve problem 30 in the same section. 
\end{exercise}

\begin{exercise}{9}
Read the subsection \textit{A Word of Caution}, in section 2.4. Then solve problem 29 in the same section (again, you may want to use a calculator).
\end{exercise}

\begin{exercise}{10} 
Read example 1, section 2.5. Then read the note after example 4. 
\end{exercise}

\begin{exercise}{11}
Read example 4, section 2.6. 
\end{exercise}

\begin{exercise}{12} 
Read Appendix A1, until example 2. Solve a few os the problems between 1 and 12 in the appendix, as a \textbf{warm-up} (and review of \textit{Power Series} and \textit{Taylor Series}).
\end{exercise}

\begin{exercise}{13} 
Solve problem 15 in appendinx A1. 
\end{exercise}

\begin{exercise}{14}
Solve problems 14 and 15, section 3.1. 
\end{exercise}

\begin{exercise}{15}
Read theorems 2 and 3 in section 3.1. Then solve problems problem 29 and 30. 
\end{exercise}

\begin{exercise}{16}
Problem 32, section section 3.1.
\end{exercise}

\begin{exercise}{17}
Section 3.2, problem 16. 
\end{exercise}

\begin{exercise}{18} 
Section 3.2, problem 24. 
\end{exercise}

\begin{exercise}{19}
Section 3.3., problem 34.
\end{exercise}

\begin{exercise}{20}
Section 3.3, problem 38. 
\end{exercise}

\begin{exercise}{21}
Section 3.3., problems 40 and 42. Justify your reasoning. 
\end{exercise}

\end{document}


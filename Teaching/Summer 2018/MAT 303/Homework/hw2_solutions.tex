\documentclass[12pt,oneside]{exam}

% This package simply sets the margins to be 1 inch.
\usepackage[margin=1in]{geometry}

% These packages include nice commands from AMS-LaTeX
\usepackage{amssymb,amsmath,amsthm,amsfonts,latexsym,verbatim,xspace,setspace}

% Make the space between lines slightly more
% generous than normal single spacing, but compensate
% so that the spacing between rows of matrices still
% looks normal.  Note that 1.1=1/.9090909...
\renewcommand{\baselinestretch}{1.1}
\renewcommand{\arraystretch}{.91}

% Define an environment for exercises.
\newenvironment{exercise}[1]{\vspace{.1in}\noindent\textbf{Exercise #1 \hspace{.05em}}}{}
\newenvironment{newsolution}{\vspace{.1in}\noindent\textbf{Solution \hspace{.05em}}}{}

% define shortcut commands for commonly used symbols
\newcommand{\R}{\mathbb{R}}
\newcommand{\C}{\mathbb{C}}
\newcommand{\Z}{\mathbb{Z}}
\newcommand{\Q}{\mathbb{Q}}
\newcommand{\N}{\mathbb{N}}
\newcommand{\calP}{\mathcal{P}}

\DeclareMathOperator{\vsspan}{span}

\title{Math 303 - Summer II 2018: Homework 2}

%%%%%%%%%%%%%%%%%%%%%%%%%%%%%%%%%%%%%%%%%%

\begin{document}

\begin{flushright}
\sc MAT 303 - Lecture 1\\
July 25, 2018.
\end{flushright}
\bigskip


\begin{center}
\textsf{Homework 2: solutions to selected problems} 
\end{center}

%%%%%%%%%%%%%%%%%%%%%%%%%%%%%%%%%%%%%%%%

Note: Warm-up problems do not need to be turned in. 

\begin{exercise}{1}
Warm-up by solving a couple of problems among problems 1-8, section 2.1. Don't simply apply the formulas from the textbook: try solving the equations by yourself! Also try drawing their slope fields, and compare your sketch to the answers in the textbook.
\end{exercise}

\begin{exercise}{2}
Solve problems 30 and 31, section 2.1. 
\end{exercise}

\begin{newsolution}
The differential equation from problem 30, 
\begin{equation*}
\frac{dP}{dt} = \beta_0 e^{-\alpha t}P,
\end{equation*}
is a separable differential equation.  Integrating and using the initial condition we obtain,   
\begin{align*}
\int_{0}^{t} \frac{1}{P} dP & = \int_{0}^{t} \beta_0 e^{-\alpha s} ds \\
\ln\left(\frac{P(t)}{P(0)}\right)& = \frac{-\beta_0}{\alpha}(e^{-\alpha t}-1)\\
P(t) & =P_0 \exp\left( \frac{-\beta_0}{\alpha}(1-e^{-\alpha t})\right).
\end{align*}
The observation about the limiting population follows easily from the fact that the limit of the exponential term is zero, since the coefficient $\alpha$ is positive. 
\end{newsolution}

\begin{exercise}{3}
Read the \textit{Historical Note} in section 2.1 (examples 4 and 5). Then solve problems 36 and 37 in the same section (you may want to use a calculator on this one!). 
\end{exercise}

\begin{exercise}{4} 
Solve problem 39, section 2.1. When constructing the graph, you may choose specific values of $k$, $b$ and $P_0$ for your comparison (just keep it reasonably realistic: $b$ should be small relative to $k$ and $P_0$) . If you have acess to a graphing calculator, try plotting various graphs of the two populational models for different triples $(k,b,P_0)$ (you do not need to turn in many graphs, one is enough). 
\end{exercise}

\begin{exercise}{5}
Solve problem 23, section 2.2. Note: this problem is qualitative, and should be treated as such. You do not need to compute the limiting populations by hand (although you could certainly solve the problem in this way). 
\end{exercise}

\begin{newsolution}
The new equation satisfied by the population is 
\begin{align*}
\frac{dx}{dt} & = kx(M-x) - hx \\
& =(kM-h)x - kx^2 \\
& = kx\left[ \left(M - \frac{h}{k}\right) - x \right],
\end{align*}
so the population still satisfies a logistic equation, albeit with a different environmental capacity (limiting population), as long as 
\begin{equation*}
M- \frac{h}{k} > 0, 
\end{equation*}
that is, $0<h<kM$. 

On the other hand, if $h > kM$, then for any positive value of $x(0)$, the slope at this point is negative, so solution curves with initial value $x(0)>0$ will converge to the equilibrium solution $x(t)=0$, as $t \to \infty$. 
\end{newsolution}

\begin{exercise}{6} 
Section 2.3, problem 10. 
\end{exercise}

\begin{exercise}{7}
Section 2.3, problem 12.
\end{exercise}

\begin{exercise}{8}
Read the subsections on \textit{Variable Gravitational Acceleration} and \textit{Escape Velocity}, section 2.3. Then solve problem 30 in the same section. 
\end{exercise}

\begin{newsolution}
The initial-value problem we have to solve is 
\begin{equation*}
v\frac{dv}{dr} = -\frac{GM_e}{r^2} + \frac{GM_m}{(S-r)^2}
\end{equation*}
with initial-condition $v(R)=v_0$ (this is obtained by combining the two initial conditions given in the problem). 

To reach the Moon, the projectile needs to enter the region where the lunar gravitational field overcomes the Earth's gravitational pull, and by Newton's second law that will happen when the acceleration of the projectile changes sign, hence equals zero. From the second-order differential equation given in the problem, we can compute the position $r_1$ at which this happens:
\begin{align*}
\frac{d^2r}{dt^2} & = 0   \\
\frac{M_e}{r_{1}^{2}} & = \frac{M_m}{(S-r_1)^2} \\
\frac{S}{r_1} & = 1+ \sqrt{\frac{M_m}{M_e}} \\
r_1 & = \frac{S}{1+ \sqrt{\frac{M_m}{M_e}}}  
\end{align*}

To find the minimal launch velocity that will make it so that the projectile reaches as far as $r_1$, we need to find the minimal launch velocity for which the velocity of the projectile at this position is still positive. In order to to this, we need to understand the solution of the initial-value problem above. The first-order equation is separable, and its solution is
\begin{equation*}
v(r)=v_0 + G \left[M_e\left(\frac{1}{r}-\frac{1}{R}\right) + M_m\left( \frac{1}{S-r} - \frac{1}{S-R}\right) \right].
\end{equation*} 
Therefore the least $v_0$ has to be so that $v(r_1)$ is still positive is 
\begin{equation*}
v_0 =  -G \left[M_e\left(\frac{1}{r_1}-\frac{1}{R}\right) + M_m\left( \frac{1}{S-r_1} - \frac{1}{S-R}\right) \right].
\end{equation*}
\end{newsolution}

\begin{exercise}{9}
Read the subsection \textit{A Word of Caution}, in section 2.4. Then solve problem 29 in the same section (again, you may want to use a calculator).
\end{exercise}


\begin{exercise}{10} 
Read example 1, section 2.5. Then read the note after example 4. 
\end{exercise}

\begin{exercise}{11}
Read example 4, section 2.6. 
\end{exercise}

\begin{exercise}{12} 
Read Appendix A1, until example 2. Solve a few os the problems between 1 and 12 in the appendix, as a \textbf{warm-up} (and review of \textit{Power Series} and \textit{Taylor Series}).
\end{exercise}

\begin{exercise}{13} 
Solve problem 15 in appendix A1. 
\end{exercise}

\begin{newsolution}
\textit{Note}: this problem does not require you to apply the Runge-Kutta method. Instead you are asked to compare the results of Picard's approximations with the values of the given Runge-Kutta approximations. 

The zeroth-order approximation of for this problem $y_0(x)=1$. Using this, we compute the first-order approximation, 
\begin{align*}
y_1(x) & =1+ \int_{1}^{x} (1+y_0^3(s)) ds \\
& =  1+ \int_{1}^{x} 2 \ ds \\
& = 1+2(x-1). 
\end{align*}
The second-order approximation is thus
\begin{align*}
y_2(x) & = 1+ \int_{1}^{x} (1+y_{1}^{3}(s)) ds \\
& = 1+ \int_{1}^{x} (1+ (2s-1)^3) ds
\end{align*}

\end{newsolution}

\begin{exercise}{14}
Solve problems 14 and 15, section 3.1. 
\end{exercise}

\begin{exercise}{15}
Read theorems 2 and 3 in section 3.1. Then solve problems problem 29 and 30. 
\end{exercise}

\begin{exercise}{16}
Problem 32, section section 3.1.
\end{exercise}

\begin{exercise}{17}
Section 3.2, problem 16. 
\end{exercise}

\begin{exercise}{18} 
Section 3.2, problem 24. 
\end{exercise}

\begin{exercise}{19}
Section 3.3., problem 34.
\end{exercise}

\begin{exercise}{20}
Section 3.3, problem 38. 
\end{exercise}

\begin{exercise}{21}
Section 3.3., problems 40 and 42. Justify your reasoning. 
\end{exercise}

\end{document}
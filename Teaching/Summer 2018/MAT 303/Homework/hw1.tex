\documentclass[12pt,oneside]{exam}

% This package simply sets the margins to be 1 inch.
\usepackage[margin=1in]{geometry}

% These packages include nice commands from AMS-LaTeX
\usepackage{amssymb,amsmath,amsthm,amsfonts,latexsym,verbatim,xspace,setspace}

% Make the space between lines slightly more
% generous than normal single spacing, but compensate
% so that the spacing between rows of matrices still
% looks normal.  Note that 1.1=1/.9090909...
\renewcommand{\baselinestretch}{1.1}
\renewcommand{\arraystretch}{.91}

% Define an environment for exercises.
\newenvironment{exercise}[1]{\vspace{.1in}\noindent\textbf{Exercise #1 \hspace{.05em}}}{}

% define shortcut commands for commonly used symbols
\newcommand{\R}{\mathbb{R}}
\newcommand{\C}{\mathbb{C}}
\newcommand{\Z}{\mathbb{Z}}
\newcommand{\Q}{\mathbb{Q}}
\newcommand{\N}{\mathbb{N}}
\newcommand{\calP}{\mathcal{P}}

\DeclareMathOperator{\vsspan}{span}

\title{Math 132 - Summer II 2016: Homework 4}

%%%%%%%%%%%%%%%%%%%%%%%%%%%%%%%%%%%%%%%%%%

\begin{document}

\begin{flushright}
\sc MAT 303 - Lecture 1\\
July 9, 2018.
\end{flushright}
\bigskip

This homework is due on Monday, 7/16, in class, by 6:00 pm. 
\begin{center}
\textsf{Homework 1} 
\end{center}

%%%%%%%%%%%%%%%%%%%%%%%%%%%%%%%%%%%%%%%%

Note: Warm-up problems do not need to be turned in. 

\begin{exercise}{1}
This is a warm-up problem. Solve problems 1, 2, 3, 6, 9 and 10 in section 1.1. 
\end{exercise}

\begin{exercise}{2}
Solve problems 20, 23 and 26 in section 1.1. Clearly explain how you found the values of the constants. Attempt to draw the integral curves by yourself (the various solutions are given by varying the value of $C$), then compare your  answers with the textbook. You do not need to turn your plots in. 
\end{exercise}

\begin{exercise}{3}
Solve problems 29, 30 and 31 in section 1.1. Clearly explain how you got to the equation from the geometric description of the graph. 
\end{exercise}

\begin{exercise}{4}
Read example 7 in section 1.1. Then solve problems 47 and 48. 
\end{exercise}

\begin{exercise}{5}
Warm-up: refresh your knowledge of integration by attempting a few problems among problems 1 to 10 in section 1.2. 
\end{exercise}

\begin{exercise}{6} 
Warm-up: Attempt to draw integral curves in problems 1 to 10 in section 1.3. Compare your answers to the textbook. 
\end{exercise}

\begin{exercise}{7}
Solve problem 22 in section 1.3. The slope field and solution curve should be turned in. 
\end{exercise}

\begin{exercise}{8}
Read the subsection on existence and uniqueness on section 1.3. Then solve problems 27 and 28. 
\end{exercise}

\begin{exercise}{9}
Solve problems 34 and 35. You may use the free online tool Wolfram Alpha to estimate the values of the solutions. This is an optional exercise which you do not need to turn in. You should attempt it at some point in this course, as this will be an important lesson to which we will return in the future. 
\end{exercise}

\begin{exercise}{10}
Warm-up: attempt a few problems among problems 1 to 12 in section 2.2., then compare your answers to the textbook. 
\end{exercise}

\begin{exercise}{11}
Solve problem 21 in section 1.3. 
\end{exercise}

\begin{exercise}{12}
Warm-up: solve a few among problems 1 to 8 in section 6.1. 
\end{exercise}

\begin{exercise}{13}
Section 1.4, problem 40. 
\end{exercise}

\begin{exercise}{14}
Section 1.4, problems 68 and 69. 
\end{exercise}

\begin{exercise}{15}
Section 1.5, problem 35. 
\end{exercise}

\begin{exercise}{16}
Section 1.5, problem 45. 
\end{exercise}

\begin{exercise}{17}
Warm-up: try to identify which method applies to each equation in problems 1 through 30, section 1.6 (but do not try to solve them all). Attempt to solve a few to confirm your expectations. 
\end{exercise}.

\begin{exercise}{18}
Solve problems 58, 60, and 64 in section 1.6. Use the hints given in the problems. 
\end{exercise}

\end{document}

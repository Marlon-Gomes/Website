\documentclass[12pt,oneside]{exam}

% This package simply sets the margins to be 1 inch.
\usepackage[margin=1in]{geometry}

% These packages include nice commands from AMS-LaTeX
\usepackage{amssymb,amsmath,amsthm,amsfonts,latexsym,verbatim,xspace,setspace}

% Make the space between lines slightly more
% generous than normal single spacing, but compensate
% so that the spacing between rows of matrices still
% looks normal.  Note that 1.1=1/.9090909...
\renewcommand{\baselinestretch}{1.1}
\renewcommand{\arraystretch}{.91}

% Define an environment for exercises.
\newenvironment{exercise}[1]{\vspace{.1in}\noindent\textbf{Exercise #1 \hspace{.05em}}}{}

% define shortcut commands for commonly used symbols
\newcommand{\R}{\mathbb{R}}
\newcommand{\C}{\mathbb{C}}
\newcommand{\Z}{\mathbb{Z}}
\newcommand{\Q}{\mathbb{Q}}
\newcommand{\N}{\mathbb{N}}
\newcommand{\calP}{\mathcal{P}}
\DeclareMathOperator{\vsspan}{span}

\title{Math 303 - Summer II 2018: Homework 4}

%%%%%%%%%%%%%%%%%%%%%%%%%%%%%%%%%%%%%%%%%%

\begin{document}

\begin{flushright}
\sc MAT 303 - Lecture 1\\
August 6, 2018.
\end{flushright}
\bigskip

This homework is due on Thursday, 8/16, in class, by 6:00 pm. 
\begin{center}
\textsf{Homework 4} 
\end{center}

%%%%%%%%%%%%%%%%%%%%%%%%%%%%%%%%%%%%%%%%

Note: You do not need to construct direction fields or solutions curves for the systems in sections 5.1 and 5.2.  

\begin{exercise}{1}
Solve problems 2, 6 and 8, section 4.1. 
\end{exercise}

\begin{exercise}{2}
Solve problems 4, 8, 10, 16, section 4.2.  
\end{exercise}

\begin{exercise}{3}
Solve problems 23 and 24, section 4.2.  
\end{exercise}

\begin{exercise}{4} 
Solve problems 26 and 28, section 4.2. 
\end{exercise}

\begin{exercise}{5}
Solve problems 12, 14 and 18, section 5.1. 
\end{exercise}

\begin{exercise}{6} 
Solve problems 2, 6 and 10, section 5.2. 
\end{exercise}

\begin{exercise}{7}
Solve problems 18, 20 and 24, section 5.2. 
\end{exercise}

\begin{exercise}{8}
Solve problems 17 to 28, section 5.3. You don't need to draw the direction fields. 
\end{exercise}

\begin{exercise}{9}
Solve problems 2, 8, and 14, section 5.5. 
\end{exercise}

\begin{exercise}{10} 
Problems 24, 26, 30, section 5.5.
\end{exercise}

\begin{exercise}{11}
Problems 2,  4 and 8, section 5.6.  
\end{exercise}

\begin{exercise}{12} 
Problems 10, 16, 20, section 5.6. 
\end{exercise}

\begin{exercise}{13} 
Problems 22, 24, section 5.6. 
\end{exercise}

\begin{exercise}{14}
Solve problems 26 and 28, section 5.6
\end{exercise}

\begin{exercise}{15}
Solve problems 1 through 8, section 6.1. 
\end{exercise}

\begin{exercise}{16} 
Solve problems 2, 12, 16, section 6.2.
\end{exercise}

\begin{exercise}{17} 
Solve problems 33 through 36, section 6.2.
\end{exercise}

\begin{exercise}{18}
Solve problems 14 through 17, section 6.3. 
\end{exercise}
\end{document}

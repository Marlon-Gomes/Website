\documentclass[12pt,oneside]{exam}

% This package simply sets the margins to be 1 inch.
\usepackage[margin=1in]{geometry}

% These packages include nice commands from AMS-LaTeX
\usepackage{amssymb,amsmath,amsthm,amsfonts,latexsym,verbatim,xspace,setspace}

% Make the space between lines slightly more
% generous than normal single spacing, but compensate
% so that the spacing between rows of matrices still
% looks normal.  Note that 1.1=1/.9090909...
\renewcommand{\baselinestretch}{1.1}
\renewcommand{\arraystretch}{.91}

% Define an environment for exercises.
\newenvironment{exercise}[1]{\vspace{.1in}\noindent\textbf{Exercise #1 \hspace{.05em}}}{}
\newenvironment{newsolution}{\vspace{.1in}\noindent\textbf{Solution \hspace{.05em}}}{}
% define shortcut commands for commonly used symbols
\newcommand{\R}{\mathbb{R}}
\newcommand{\C}{\mathbb{C}}
\newcommand{\Z}{\mathbb{Z}}
\newcommand{\Q}{\mathbb{Q}}
\newcommand{\N}{\mathbb{N}}
\newcommand{\calP}{\mathcal{P}}

\DeclareMathOperator{\vsspan}{span}

\title{Math 303 - Summer II 2018: Homework 1, solutions to selected problems. }

%%%%%%%%%%%%%%%%%%%%%%%%%%%%%%%%%%%%%%%%%%

\begin{document}

\begin{flushright}
\sc MAT 303 - Lecture 1\\
July 9, 2018.
\end{flushright}
\bigskip

\begin{center}
\textsf{Homework 1: solutions to selected problems.} 
\end{center}

%%%%%%%%%%%%%%%%%%%%%%%%%%%%%%%%%%%%%%%%

Note: Warm-up problems do not need to be turned in. 

\begin{exercise}{1}
This is a warm-up problem. Solve problems 1, 2, 3, 6, 9 and 10 in section 1.1. 
\end{exercise}

\begin{exercise}{2}
Solve problems 20, 23 and 26 in section 1.1. Clearly explain how you found the values of the constants. Attempt to draw the integral curves by yourself (the various solutions are given by varying the value of $C$), then compare your  answers with the textbook. You do not need to turn your plots in. 
\end{exercise}

\begin{exercise}{3}
Solve problems 29, 30 and 31 in section 1.1. Clearly explain how you got to the equation from the geometric description of the graph. 
\end{exercise}

\begin{newsolution}

Problem 29; Let $y(x)$ denote the desired function. The tangent line at a point $(x_0,y(x_0))=(x_0,y_0)$ must have slope $f(x_0,y_0)$. Assume that this slope is non-zero. Then, the slope of the normal line is $m=-\frac{1}{f(x_0,y_0}$ and its equation is 
\begin{equation*}
y=-\frac{(x-x_0)}{f(x_0,y_0)} + y_0
\end{equation*}
If this line passes through $(0,1)$, then 
\begin{equation*}
1=\frac{x_0}{f(x_0,y_0)} + y_0, 
\end{equation*}
hence $f(x,y)=\frac{x}{1-y}$, $y\neq 1$, is the function defining the equation we seek. 

Problem 30: Consider the curve $y=x^2+k$, whose tangent line at a point $(x_0,y_0)$ has slope $2x_0$. Assuming that $x_0 \neq 0$, the slope of the normal line is then given by $m=-\frac{1}{2x_0}$. Thus the function defining the desired differential equation is $f(x,y)=-\frac{1}{2x}$, for $x \neq 0$. 

Problem 31: Let $y(x)$ denote the desired function. It's tangent line at a point $(x_0,y(x_0))=(x_0,y_0)$ has equation
\begin{equation*}
y=f(x_0,y_0)(x-x_0) + y_0.
\end{equation*}
If this line passes through the point $(-y_0,x_0)$, then 
\begin{equation*}
x_0 = f(x_0,y_0)(-y_0-x_0)+y_0, 
\end{equation*}
thus the function defining the desired differential equation is 
\begin{equation*}
f(x,y)=\frac{y-x}{x+y},
\end{equation*}
for all $x \neq y$. 

\end{newsolution}

\begin{exercise}{4}
Read example 7 in section 1.1. Then solve problems 47 and 48. 
\end{exercise}

\begin{newsolution}
Problem 47: 
\begin{itemize}
\item[(a)] Straightforward.
\item[(b)] There is no value of $C$ such that $y_{C}(x)=\frac{1}{C-x}$ satisfies $y(0)=0$. Instead, the only solution of the differential equation with this property is the equilibrium solution $y(x)=0$, for all $x$. \textit{This is meant for you to notice a fact mentioned in class: it is not always possible to write all solutions of a differential equation in a single formula.}
\item[(c)] The curves $y_C$ do not fill the entire $xy$-plane by varying the values of $C$. As observed above, there is no value of $C$ that makes such curves pass through points of the form $(x,0)$. However, given any point $(x,y)$ in the plane with $y\neq 0$ there exists a unique curve $y_C$ through this point (in other words, one can solve the equation $y_C(x)=y$ uniquely for $C$). 
Nevertheless, solutions of the differential equation do span the whole plane, for points of the form $(x,0)$ are in the graph of the equilibrium solution $y(x)=0$. 
\end{itemize}

Problem 48: 
\begin{itemize}
\item[(a)] Straightforward; 
\item[(b)] That $y(x)$ solves the differential equation in each of the intervals $(-\infty, 0)$ and $(0,\infty)$ is an easy consequence of part $(a)$. To check that this function also satisfies the equation at $0$, one must compute the derivative from the definition, i.e., as a limit. Here is it important to note that the expressions for the function to the left and right of $0$ are different, thus one should verify that the lateral limits exist and coincide (both are $0$).  Trying to write this function in the form $y(x)=Cx^4$ would lead to ambiguity in the choice of $C$ (choose two values for $x$, one positive, one negative, to see this for yourself). 
\item[(c)] There is a mistake in the statement of this problem. This differential equation does not have solutions satisfying the initial condition $y(0)=b$, for $b\neq0$. 

Just like in problem 47, the curves $y_C(x)=Cx^4$ define solutions of the differential equation whose graphs do not intersect except at $(0,0)$. Moreover, such solutions span all of the $xy$-plane, except for points of the form $(0,b)$, $b\neq 0$.  To find infinitely many solutions passing through a point $(a,b)$ not on the $y$-axis, one must generalize the idea used to construct the soution $y(x)$ from the statement of the problem, i.e., to glue branches of the simple solutions $y_C$. 

Here is an example. Suppose that  $a \neq 0$. Then there exists a unique value of $C$ for which $y_C(a)=b$. Identify which half-plane the point $(a,b)$ belongs to. If it belongs to the left half-plane, then consider the functions $y_{C,D}$, defined by 
\begin{equation*}
y_{C,D}=
\left\{
\begin{array}{rl}
Cx^4 & \mbox{if} \ x <0, \\
Dx^4 & \mbox{if} \ x \geq 0. 
\end{array}
\right.
\end{equation*}
All of these functions are differentiable (again, one has to check differentiability at $0$ by computing the limit defining the derivative by hand), pass through $(a,b)$ (because the left part of the graph of $y_{C,D}$ is the same as that of $y_C$, and solve the equation, as it can be easily verified. By varying $D$, one obtains an infinite family of solutions through the point $(a,b)$. A similar argument would hold for points on the right half-plane: new solutions would be obtained by replacing the left part of the graph. 
\end{itemize}
\end{newsolution}

\begin{exercise}{5}
Warm-up: refresh your knowledge of integration by attempting a few problems among problems 1 to 10 in section 1.2. 
\end{exercise}

\begin{exercise}{6} 
Warm-up: Attempt to draw integral curves in problems 1 to 10 in section 1.3. Compare your answers to the textbook. 
\end{exercise}

\begin{exercise}{7}
Solve problem 22 in section 1.3. The slope field and solution curve should be turned in. 
\end{exercise}

\begin{exercise}{8}
Read the subsection on existence and uniqueness on section 1.3. Then solve problems 27 and 28. 
\end{exercise}

\begin{newsolution}
Problem 27: 

\begin{itemize}
\item[(a)] The computational part is similar to the argument used in problem 48. The graphs of the functions defined by this procedure look like a ray (i.e. half-line) to the left of $c$, and a half-parabola to the right of $c$. By varying the values of $c$ among non-negative numbers, all of these solutions pass through $(0,0)$. 
\item[(b)] Of course, for all negative values of $b$ the differential equation has no solutions, as those values do not have square roots. If $b$ is equal to $0$, the differential equation has infinitely many solutions defined for all time.

The remaining case is that of $b>0$. In this case, we claim that the differential equation a solution defined for all time, namely $y(x)=(x-\sqrt{b})^2$. That this solution is unique follows from the Existence and Uniqueness theorem, for the function $f(x,y)=2\sqrt{y}$ is continuously differentiable for all $y >0$. 
\end{itemize}

Problem 28 was solved in class. 

\end{newsolution}

\begin{exercise}{9}
Solve problems 34 and 35 in section 1.3. You may use the free online tool Wolfram Alpha to estimate the values of the solutions. This is an optional exercise which you do not need to turn in. You should attempt it at some point in this course, as this will be an important lesson to which we will return in the future. 
\end{exercise}

\begin{exercise}{10}
Warm-up: attempt a few problems among problems 1 to 12 in section 2.2., then compare your answers to the textbook. 
\end{exercise}

\begin{exercise}{11}
Solve problem 21 in section 1.3. 
\end{exercise}

\begin{exercise}{12}
Warm-up: solve a few among problems 1 to 8 in section 6.1. 
\end{exercise}

\begin{exercise}{13}
Section 1.4, problem 40. 
\end{exercise}

\begin{newsolution}
This problem was solved in class.
\end{newsolution}

\begin{exercise}{14}
Section 1.4, problems 68 and 69. 
\end{exercise}

\begin{newsolution}
Problem 68:

\begin{itemize}
\item[(a)] The problem states that 
\begin{equation}\label{Snells_law}
\frac{\sin(\alpha)}{v}=c,
\end{equation}
where $\alpha$ is the angle between the tangent line to the graph of $y$ and a vertical line, $v$ is the velocity of the moving particle, given by $v=\sqrt{2gy}$ (notice the orientation convention: the $y$-axis is oriented positively in the downwards direction), and $c$ is a constant. It is stated that $\cot(\alpha)=y'(x)$, which you should verufy for yourself (it is simple trigonometry, but again you need to take into consideration the inverted $y$-axis).  

We shall now derive the desired differential equation. The key point is to relate the cotangent and sine of the angle $\alpha$. ,This is done using the trigonometric identity
\begin{equation*}
\frac{1}{\sin^2(\alpha)} = \cot^2(\alpha) + 1.
\end{equation*}
Substituting the data of the problem, this yields
\begin{align*}
\frac{1}{(cv)^2} & =[y'(x)]^2 +1\\
\frac{1}{2gc^2y} & = [y'(x)]^2 +1,
\end{align*}
and a simple algebraic manipulation outputs
\begin{equation*}
y'(x)= \sqrt{\frac{2a-y}{y}}, 
\end{equation*}
where $a= (4gc^2)^{-1}$. Here it is important to notice that this derivation assumed (when taking square-roots) that $y'(x)>0$, i.e., that the bead is under descent. If the wire along which the bead travels turns upwards, this equation has to be modified. Similarly, when looking for solutions one must take into account the fact that the equation is only defined for as long as $2a-y >0$. 
\item[(b)] This step is a simple application of the chain rule. 
\end{itemize}

Problem 69: 
The problem amouts to the system of first-order equations
\begin{equation*}
\left\{  
\begin{array}{rl}
av'(x) & = \sqrt{1+v^2}, \\
y'(x) & = v(x),
\end{array}
\right.
\end{equation*}
which can be solved recursively for $y$. 

We recall from Calculus that the \textit{hiperbolic sine and cosine functions}, $\sinh(t),\cosh(t)$, are defined by 
\begin{align*}
\sinh(t) & = \frac{e^t -e^{-t}}{2}, \\
\cosh(t) & = \frac{e^t + e^{-t}}{2}.
\end{align*}
Is is easy to verify that these functions satisfy the properties
\begin{align*}
\sinh^2(t) - \cosh^2(t) & = 1, \\
\frac{d}{dt}\sinh(t) & = \cosh(t), \\
\frac{d}{dt} \cosh(t) & = \sinh(t).
\end{align*}
We use this to solve the equation for $v$ by direct integration, using the substitution $v=\sinh(t)$.
\begin{align*}
\int \dfrac{a}{\sqrt{1+v^2}} dv & = \int 1 dx \\
\int \dfrac{a}{\sqrt{1+\sinh^2(t)}}\cosh(t) dt & = x + D\\
a \sinh^{-1}(v) & = x + D,
\end{align*}
thus $v(x)=\sinh(\frac{x+D}{a})$. Since $v(0)=0$, $D=0$. 

Solving the equation for $y$ by direct integration is a direct application of the properties of the hiperbolic sine function listed above. 
\end{newsolution}

\begin{exercise}{15}
Section 1.5, problem 35. 
\end{exercise}

\begin{exercise}{16}
Section 1.5, problem 45. 
\end{exercise}

\begin{newsolution}
This is a mixture problem. You should read the corresponding subsection on section 1.5. to understand the derivation of the relevant differential equation. 

The total volume of the reservoir is assumed to be $V = (1km^2)(2m) = 2\cdot10^6 m^3$. The initial volume of polutant is $x(0)=0 10^6 L$ .  The  rate at which contaminated liquid flows into the reservoir is $r_i= 2\cdot 10^5 m^3/mo$, with a pollutant concentration of $c_i = 10L/m^3$. The outgoing  water flow is the same as the inflow $r_i$, but the concentration of pollutant is different (since the mixture is wll-mixed before leaving the tank), namely
\begin{equation*}
c_o(t) = \frac{x(t)}{V} L/m^3.
\end{equation*}
Thus the differential equation describing the volume of pullutant in this mixture is 
\begin{align}
\nonumber 
x'(t) & =\left(2\cdot 10^5\frac{m^3}{mo}\right) \left(10 \frac{L}{m^3}\right) - \left(2\cdot 10^5 \frac{m^3}{mo}\right)
\left(\frac{x(t)}{2\cdot 10^6} \frac{10^6 L}{m^3}\right) \\
\label{mixture_equation}
x'(t) & =\left( 2 - \frac{x(t)}{10}\right) \frac{10^6 L}{mo}.
\end{align}
This is a separable equation, whose solution (taking into account the initial value $x(0)=0$) is 
\begin{equation*}
x(t)=20 - 20e^{\frac{-t}{10}}
\end{equation*}
This confirms our expectation from the qualitative study of the equation: the solution curve passing through $(0,0)$ is a steadily rising curve (since $x'(t)>0$, for all $t>0$) whose limit as $t \to \infty$ is the equilibrium solution $x(t)=20$, for all $t$. 

Finding the time that it will take for the concentration of pollutant in the reservoir to reach $10L/m^3$ is a straightforward.
\end{newsolution}

\begin{exercise}{17}
Warm-up: try to identify which method applies to each equation in problems 1 through 30, section 1.6 (but do not try to solve them all). Attempt to solve a few to confirm your expectations. 
\end{exercise}.

\begin{exercise}{18}
Solve problems 58, 60, and 64 in section 1.6. Use the hints given in the problems. 
\end{exercise}

\begin{newsolution}
Problem 58:

To use the idea of problem 57 one first has to rewrite the equation as 
\begin{equation*}
\frac{dy}{dx} + (-4x)y = \frac{-2}{x}(y \ln y),  
\end{equation*}
for $x \neq 0$.
Thus we recognize that $P(x)=(-4x)$, while $Q(x)=-\frac{2}{x}$. Using the logarithmic substitution $v = \ln y$, the equation thus becomes (following problem 57)
\begin{equation*}
\frac{dv}{dx}  + \frac{2}{x}v = 4x.
\end{equation*}
This is a first-order, linear, inhomogeneous equation, so we will solve it using the integrating factor\footnote{Notice that in the integration step we used the assumption $x>0$ to avoid the singularity in the integral. It can be verified directly that the integrating factor we obtain also works for other values of $x$.}
\begin{equation*}
\mu(x)= \exp\left(\int_{1}^{x} \frac{2}{s} ds \right) = \exp\left(2\ln(s) \Big|_{1}^{x}\right) = \exp\left(2\ln(x)\right) = x^2.
\end{equation*}
Multiplying both sides of the equation by $\mu$ and integrating relative to $x$ yields
\begin{align*}
\int (x^2v(x))' dx & = \int 4x^3 dx \\
x^2v(x) & = x^4 + C, 
\end{align*}
thus $v(x)=\frac{x^4 +C}{x^2}$, and 
\begin{equation*}
y(x)=\exp\left(\frac{x^4+C}{x^2}\right), 
\end{equation*}
is a solution of the original equation.

Problem 60: We shall follow the suggestion of problem 59, i.e., to perform an affine substitution $x=u+h$ and $y=v+k$ to get rid of the constants and turn this equation into a homogeneous equation. It can be easily checked that the values of $h$ and $k$ that eliminate the constants are $h=3, k= -2$.  

With these substitutions, the problem becomes
\begin{equation*}
\frac{dv}{du}=\frac{-u+2v}{4u-3v}.
\end{equation*}
This is a homogeneous differential equation, and one can write it in standard form by dividing both numerator and denominator on the right-hand side by $u$. The solution then follows from standard techniques of homogeneous equations.  

Problem 64: In order to use the technique from problem 63, we first re-write the equation as 
\begin{equation*}
\frac{dy}{dx} = (-1)y^2 + 0y + (1+x^2).
\end{equation*}
Thus, using the notation from problem 63, we identify that $A(x)=-1$, $B(x)=0$ and $C(x)=(1+x^2)$. 

We use the suggested substitution, 
\begin{equation*}
y=y_1+\frac{1}{v},
\end{equation*}
with $y_1(x)=x$, as stated in the problem. Then Ricatti's equation becomes 
\begin{align*}
\frac{dv}{dx} + (0 + 2(-1)x)v & =1\\
\frac{dv}{dx} -2xv & = 1.
\end{align*}
The latter is a linear, first-order, inhomogeneous equation, which can be solved by the method of integrating factors. The integrating factor can be chosen to be $e^{-x^2}$. Thus the equation becomes, 
\begin{align*}
(e^{-x^2}v)' & =e^{-x^2}\\
e^{-x^2}v & = \int e^{-x^2}dx \\
v & = e^{x^2}\int e^{-x^2}dx.
\end{align*}
The integral on the right-hand side should not be computed, as it cannot be written in terms of elementary functions. The solution $y$ is then gotten from substitution, 
\begin{equation*}
y = x+ \frac{e^{-x^2}}{\int e^{-x^2}dx}
\end{equation*}
\end{newsolution}

\end{document}
% Exam Template for UMTYMP and Math Department courses
%
% Using Philip Hirschhorn's exam.cls: http://www-math.mit.edu/~psh/#ExamCls
%
% run pdflatex on a finished exam at least three times to do the grading table on front page.
%
%%%%%%%%%%%%%%%%%%%%%%%%%%%%%%%%%%%%%%%%%%%%%%%%%%%%%%%%%%%%%%%%%%%%%%%%%%%%%%%%%%%%%%%%%%%%%%

% These lines can probably stay unchanged, although you can remove the last
% two packages if you're not making pictures with tikz.
\documentclass[11pt]{exam}
\RequirePackage{amssymb, amsfonts, amsmath, latexsym, verbatim, xspace, setspace, graphicx, caption}


% By default LaTeX uses large margins.  This doesn't work well on exams; problems
% end up in the "middle" of the page, reducing the amount of space for students
% to work on them.
\usepackage[margin=1in]{geometry}
\usepackage[english]{babel}
\usepackage[autostyle]{csquotes} %%%% This package allows Tex to recognize quotation marks with the \enquote command. 


% Here's where you edit the Class, Exam, Date, etc.
\newcommand{\class}{MAT 303}
\newcommand{\term}{Summer II 2018}
\newcommand{\examnum}{Final}
\newcommand{\examdate}{08/16/18}
\newcommand{\timelimit}{3 hours and 25 minutes}

% For an exam, single spacing is most appropriate
\singlespacing
% \onehalfspacing
% \doublespacing

% For an exam, we generally want to turn off paragraph indentation
\parindent 0ex
\title{MAT 303 - Summer II 2018: Midterm}
\begin{document} 

% These commands set up the running header on the top of the exam pages
\pagestyle{head}
\firstpageheader{}{}{}\textbf{}
\runningheader{\class}{\examnum\ - Page \thepage\ of \numpages}{\examdate}
\runningheadrule

\begin{flushright}
\begin{tabular}{p{2.8in} r l}
\textbf{\class} & \textbf{Name (Print):} & \makebox[2in]{\hrulefill}\\
\textbf{\term} &&\\
\textbf{\examnum} &&\\
\textbf{\examdate} &&\\
\textbf{Time Limit: \timelimit} & ID number & \makebox[2in]{\hrulefill}
\end{tabular}\\
\end{flushright}
\rule[1ex]{\textwidth}{.1pt}

\begin{center}
\large{\textbf{Instructions}}
\end{center}

\begin{minipage}[t]{3.7in}
\vspace{0pt}
\begin{itemize}

\item This exam contains \numpages\ pages (including this cover page) and
\numquestions\ problems, one of which (problem 6) is an extra credit problem.  Check to see if any pages are missing.  Enter
all requested information on the top of this page, and put your initials
on the top of every page, in case the pages become separated.

\item You may \textit{not} use your books, notes, or any device that is capable of accessing the internet on this exam (e.g., smartphones, smartwatches, tablets). You may not use a calculator.

\item \textbf{Organize your work}, in a reasonably neat and coherent way, in
the space provided. Work scattered all over the page without a clear ordering will 
receive very little credit.  

\item \textbf{Mysterious or unsupported answers will not receive full
credit}.

\end{itemize}

\end{minipage}
\hfill
\begin{minipage}[t]{2.3in}
\vspace{0pt}
%\cellwidth{3em}
\gradetablestretch{2}
\vqword{Problem}
\addpoints % required here by exam.cls, even though questions haven't started yet.	
\gradetable[v]%[pages]  % Use [pages] to have grading table by page instead of question

\end{minipage}
\newpage % End of cover page

%%%%%%%%%%%%%%%%%%%%%%%%%%%%%%%%%%%%%%%%%%%%%%%%%%%%%%%%%%%%%%%%%%%%%%%%%%%%%%%%%%%%%
%
% See http://www-math.mit.edu/~psh/#ExamCls for full documentation, but the questions
% below give an idea of how to write questions [with parts] and have the points
% tracked automatically on the cover page.
%
%
%%%%%%%%%%%%%%%%%%%%%%%%%%%%%%%%%%%%%%%%%%%%%%%%%%%%%%%%%%%%%%%%%%%%%%%%%%%%%%%%%%%%%

\begin{questions}

% Basic question
%%%%%%%%%%%%%%

\addpoints
\question Find particular solutions of the differential equations below. 
\begin{parts}
\part[10] 
\begin{equation*}
y^{(4)}-5y''+4y=e^x-xe^{2x}
\end{equation*}
\vfill
\newpage
\part[10] 
\begin{equation*}
y''+9y=2\sec(3x)
\end{equation*}
\vfill
\end{parts}

\newpage

%%%%%%%%%%%%%%%%%
\addpoints
\question Consider the initial-value problem
\begin{equation*}
(2x-1)y'+2y=0,\  y(0)=1.
\end{equation*}
\begin{parts}
\part[10] Solve this equation via the power series method 
\vfill
\newpage
\part[5] Determine the radius of convergence of the power series solution you found. 
\vfill
\part[5] Write the power series solution in terms of elementary functions. 
\end{parts}
\newpage
%%%%%%%%%%%%%%%%%

\addpoints

\addpoints
\question[20] Find as many independent Frobenius series solutions as possible for the differential equation
\begin{equation*}
xy'' + 2y' +xy =0.
\end{equation*}
\newpage

%%%%%%%%%%%%%%%%%%%%
\addpoints
\question Find general solutions of the following linear systems by the eigenvalue method. 

\begin{parts}
\part[4]  
\begin{align*}
x' & = x - 5y \\
y' & = x + 3y
\end{align*} 
\vfill 
\newpage 
\part[6] 
\begin{align*}
x' & = -3x +5y +5z\\
y' & = 3x -y+3z \\
z' & = 9x - 8y +10z
\end{align*}
\vfill 
\newpage
\part[10] 
\begin{align*}
x' & =-x+y+z-2w \\
y' & = 7x-4y-6z+11w \\
z' & = 5z-y+z+3w\\
w' & = 6x-2y-2z+6w
\end{align*}
\vfill
\end{parts}

\newpage
%%%%%%%%%%%
\addpoints
\question Find general solutions of the following initial-value problems by the matrix exponential method. 

\begin{parts}
\part[6]  
\begin{align*}
x' & = 2x +y\\
y' & = x + 2y,
\end{align*} 
with $x(0)=3, y(0)=-2$. 
\vfill 
\newpage 
\part[7] 
\begin{align*}
x' & = 5x -6z\\
y' & = 2x -y-2z \\
z' & = 4x-2y-4z,
\end{align*}
with $x(0)=2, y(0)=1, z(0)=0$. 
\vfill 
\newpage
\part[7] 
\begin{align*}
x' & =x+2y+3z \\
y' & = y+2z\\
z' & = z,
\end{align*}
with $x(0)=1, y(0)=1, z(0)=0$. 
\vfill
\end{parts}

\newpage
%%%%%%%%%%%%%%%%%%%%%%%%%%
%%%%%%%%%%%%%%%%%%%%%%%%%%%%%%%%%%%
\addpoints 
\question[20] \textbf{Extra credit} Consider the predator-prey system consisting of two species coexisting in the same environment, whose populations at time $t$ are denoted by $x(t)$ and $y(t)$. The two species are assumed to naturally satisfy a logistic equation int he absence of the other. The two species are assumed to cooperate, in such a way that their interaction is described by the system
\begin{align*}
x' & =30x-3x^2+xy \\
y' & = 60y-3y^2+4xy.
\end{align*}
Find all four equilibrium solutions of the model (1 point for each equilibrium solution). Describe the local behaviour of the linearized system at each such solution (3 points for each equilibrium solution). Determine whether you expect the linearized behavior to be reflected by the non-linear system (1 point for each equilibrium solution).
\vfill 
\end{questions}
\end{document}
% Exam Template for UMTYMP and Math Department courses
%
% Using Philip Hirschhorn's exam.cls: http://www-math.mit.edu/~psh/#ExamCls
%
% run pdflatex on a finished exam at least three times to do the grading table on front page.
%
%%%%%%%%%%%%%%%%%%%%%%%%%%%%%%%%%%%%%%%%%%%%%%%%%%%%%%%%%%%%%%%%%%%%%%%%%%%%%%%%%%%%%%%%%%%%%%

% These lines can probably stay unchanged, although you can remove the last
% two packages if you're not making pictures with tikz.
\documentclass[11pt]{exam}
\RequirePackage{amssymb, amsfonts, amsmath, latexsym, verbatim, xspace, setspace, color}


% By default LaTeX uses large margins.  This doesn't work well on exams; problems
% end up in the "middle" of the page, reducing the amount of space for students
% to work on them.
\usepackage[margin=1in]{geometry}
\usepackage[english]{babel}
\usepackage[autostyle]{csquotes} %%%% This package allows Tex to recognize quotation marks with the \enquote command. 


% Here's where you edit the Class, Exam, Date, etc.
\newcommand{\class}{MAT 203}
\newcommand{\term}{Summer II 2018}
\newcommand{\examnum}{Midterm}
\newcommand{\examdate}{07/26/18}
\newcommand{\timelimit}{3 hours and 25 minutes}

% For an exam, single spacing is most appropriate
\singlespacing
% \onehalfspacing
% \doublespacing

% For an exam, we generally want to turn off paragraph indentation
\parindent 0ex
\title{MAT 203 - Summer II 2018: Midterm grading key}
\begin{document} 

% These commands set up the running header on the top of the exam pages
\pagestyle{head}
\firstpageheader{}{}{}\textbf{}
\runningheader{\class}{\examnum\ - Page \thepage\ of \numpages}{\examdate}
\runningheadrule

\begin{flushright}
\begin{tabular}{p{2.8in} r l}
\textbf{\class} & \textbf{Name (Print):} & \makebox[2in]{\hrulefill}\\
\textbf{\term} &&\\
\textbf{\examnum} &&\\
\textbf{\examdate} &&\\
\textbf{Time Limit: \timelimit} & ID number & \makebox[2in]{\hrulefill}
\end{tabular}\\
\end{flushright}
\rule[1ex]{\textwidth}{.1pt}

\begin{center}
\large{\textbf{Instructions}}
\end{center}

\begin{minipage}[t]{3.7in}
\vspace{0pt}
\begin{itemize}

\item This exam contains \numpages\ pages (including this cover page) and
\numquestions\ problems.  Check to see if any pages are missing.  Enter
all requested information on the top of this page, and put your initials
on the top of every page, in case the pages become separated.

\item You may \textit{not} use your books, notes, or any device that is capable of accessing the internet on this exam (e.g., smartphones, smartwatches, tablets). You may not use a calculator.

\item \textbf{Organize your work}, in a reasonably neat and coherent way, in
the space provided. Work scattered all over the page without a clear ordering will 
receive very little credit.  

\item \textbf{Mysterious or unsupported answers will not receive full
credit}.

\end{itemize}

\end{minipage}
\hfill
\begin{minipage}[t]{2.3in}
\vspace{0pt}
%\cellwidth{3em}
\gradetablestretch{2}
\vqword{Problem}
\addpoints % required here by exam.cls, even though questions haven't started yet.	
\gradetable[v]%[pages]  % Use [pages] to have grading table by page instead of question

\end{minipage}
\newpage % End of cover page

%%%%%%%%%%%%%%%%%%%%%%%%%%%%%%%%%%%%%%%%%%%%%%%%%%%%%%%%%%%%%%%%%%%%%%%%%%%%%%%%%%%%%
%
% See http://www-math.mit.edu/~psh/#ExamCls for full documentation, but the questions
% below give an idea of how to write questions [with parts] and have the points
% tracked automatically on the cover page.
%
%
%%%%%%%%%%%%%%%%%%%%%%%%%%%%%%%%%%%%%%%%%%%%%%%%%%%%%%%%%%%%%%%%%%%%%%%%%%%%%%%%%%%%%

\begin{questions}

% Basic question
%%%%%%%%%%%%%%

\addpoints
\question In each of the following problems, determine if the statements are true or false. Explain your reasoning (correct answers without an explanation will be worth only 2 points per statement). 
\begin{parts}
\part[5] Two vectors in space always determine a unique plane. 

\textit{Solution:} This is false. [2] Collinear vectors do not determine a unique plane. [3]

\vfill
\part[5] Two planes in space always intersect. 

\textit{Solution:} This is false. [2] Parallel planes do not intersect. [3]
\vfill
\part[5] The dot product can be used to detect whether two non-zero vectors in space are aligned. 

\textit{Solution:} This is true. [2] Two vectors are aligned if $(u\cdot v)^2 = (u\cdot u)(v\cdot v)$. [3]
\vfill
\part[5] Let $r(t)$ and $s(t)$ be curves in the plane, such that neither has a limit as $t$ converges to $0$. Then their cross-product $r(t)\times s(t)$ does not have a limit at $0$ either. 

\textit{Solution:} This is false. [2] Neither of the curves $r(t)=\frac{1}{t}i$ and $s(t)=\frac{2}{t}i$, defined for $t \neq 0$, has a limit as $t$ goes to $0$, but their cross-product $(r\times s)(t)$ is equal to zero for all $t \neq 0$, since the curves are collinear, hence has limit $0$ as $t$ goes to $0$. \textit{[3 points for any correct example] }
\vfill
\part[5] If all directed limits a scalar-valued, multivariable function at a point exist and coincide, then the function has a limit at the point, in the multivariable sense.

\textit{Solution:} This is false.[2] One counter-example, as seen in homework 2, is the function $f(x,y)=\frac{x^2y}{x^4+y^2}$, which has directed limits equal to $0$ at the origin, along any direction, but whose limit at the origin does not exist in the multivariable sense, since limits along other curves, such as the parabolae $y=kx^2$ yield inconsistent results, for different values of $k$. 

\textit{[3 points for a counter-example; 2 points for saying that the function can have limits along different curves, without explicitely writing an example; 1 point for mentioning such a function from homework 2, but not writing the function or the reason why it may fail to have a limit].}
\vfill
\part[5] If a scalar-valued, multivariable function is separately continuous with respect to each variable, then it is continuous in the multivariable sense.

\textit{Solution:} This is false [2]. One counter-example, as seen in homework 2, is the function defined by 
\begin{equation*}
f(x,y) = 
\left\{
\begin{array}{rl}
\frac{y}{x}-y & \mbox{ if} \ 0 \leq y < x \leq 1 \\
\frac{x}{y}-x & \mbox{ if} \ 0 \leq x < y \leq 1 \\
1-x & \mbox{if} \ 0 <x = y \leq 1 \\
0 & \mbox{ elsewhere},
\end{array}
\right.
\end{equation*}
which is separately continuous with respect to $x$ and $y$, but not continuous at the origin, for its directed limit along the ray $y=x, x>0$, as $x$ tends to $0$, does not coincide with the value of the function at the origin. 

\textit{[3 points for a counter-example; 2 points for saying that continuity along lines other than those parallel to the $x$ and $y$ axis is also a necessary condition, without writing a specific counter-example; 1 point for mentioning such a function from homework 2, but not writing the function or the reason why it may fail to be continuous].}
\vfill 
\end{parts}

\newpage 

%%%%%%%%%%%%%%%%%%%%%%%%%%%%%%%%%%%
\addpoints 
\question Consider the lines whose parametric equations are given by 
\begin{align*}
L_1 \colon  & x=2t, y=4t, z=6t.\\
L_2 \colon & x=1-s, y=4+s, z=-1+s.
\end{align*}

\begin{parts}
\part[5] Explain why these lines do not intersect. 
\textit{Solution:} If the lines intersect then the following system of equations has a solution, 
\begin{align*}
2t & =1-s \\
4t & = 4+s \\
6t & = -1+s.
\end{align*}

However, these equations are incompatible with each other: solving equations 1 and 2 for $t$ yields $t=\frac{5}{6}$; solving equations 1 and 3 for $t$ yields $t=0$. 

\textit{[2 points for stating that if lines intersect there must be a solution to the system; 3 points for arguing that the system is impossible, by exhibiting incompatible solutions.]}

\vfill
\part[5] Explain why these lines are not parallel. 

\textit{Solution:} Line $L_1$ is parallel to the vector $(2,4,6)$ \textit{[1 point]}. Line $L_2$ is parallel to the vector $(-1,1,1)$\textit{[1 point]}. These vectors are not multiples of each other, so the lines are not parallel \textit{[3 points]}. 

\vfill
\newpage
\part[5]  Find the general equations of two parallel planes, $\Pi_1$ and $\Pi_2$, containing lines $L_1$ and $L_2$, respectively. 


\textit{Solution:} The vector $(2,4,6) \times (-1,1,1) = (-2,-8,6)$ is normal to both lines, so it defines a plane parallel to both, given by equation
\begin{equation*}
\Pi_1 \colon \ -2x -8y + 6z =0.
\end{equation*}
Notice that this plane contains the line $L_1$. To find a parallel plane $\Pi_2$ containing the line $L_2$, we have to translate this plane. The amount of translation can be computed by inputing the coordinates of points in the line into the left-hand side of the equation of $\Pi_1$: since 
\begin{equation*}
-2(1) -8(4) + 6(-1) = -40,
\end{equation*}
the equation of the plane $\Pi_2$ is 
\begin{equation*}
\Pi_2 \colon \ -2x -8y + 6z =-40
\end{equation*}

\textit{[3 points for finding a normal vector; 1 point for writing an equation of each plane.] }

\vfill
\part[5] What is the distance between the lines $L_1$ and $L_2$. 

\textit{Solution:} The distance between the skew lines is the distance between the parallel planes that contain them \textit{[2 points]}. In this case, the distance is $40$. \textit{[3 points for identifying the distance by comparing the equations of the planes]}. 
\vfill
\end{parts}

\newpage

%%%%%%%%%%%%%%%%%
\addpoints
\question[10] Compute the trajectory of a curve whose velocity vector is given by 
\begin{equation*}
r'(t)= e^{-t}i + t^2j + \frac{1}{1+t^2} k,
\end{equation*}
and such that $r(0)=(1,1,1)$. 

\textit{Solution:} By the Fundamental Theorem of Calculus for Curves, 
\begin{align*}
r(t) & = r(0) +\int_{0}^{t} r'(s)\ ds\\
& = i + j +k + \left(\int_{0}^{t} e^{-s}ds \right)i + \left(\int_{0}^{t} s^2 ds\right)j + \left( \int_{0}^{t}  \frac{1}{1+s^2} ds \right) k \\
& = (2-e^{-t})i + \left( 1+ \frac{t^3}{3} \right)j + (1+\arctan(t))k
\end{align*}

\textit{[4 points for the correct use of the FTC (in definite integral form) \underline{or} 2 points for using the FTC in indefinite integral form, and another 2 points for finding the correct constant of integration; 2 points for each correct integral. ]}
\newpage


%%%%%%%%%%%%%%%%%

\addpoints

\addpoints
\question Consider the function
\begin{equation*}
f(x,y)=\left\{
\begin{array}{rc}
\frac{xy(x^2-y^2)}{x^2+y^2} & \mbox{if} \ (x,y) \neq (0,0)\\
0, & \mbox{if} \ (x,y)=(0,0).
\end{array}
\right.
\end{equation*}

\begin{parts}
\part[5] What are the directed limits of the function along lines $y=kx$, as $x \to 0$? 

\textit{Solution:} If $y=kx$, then 
\begin{equation*}
f(x,kx)= \frac{kx^2(x^2-k^2x^2)}{x^2+k^2x^2} = x^2\left(\frac{k(1-k^2)}{1+k^2}\right),
\end{equation*}
and the limit of this expression as $x \to 0$ is $0$, regardless of the value of $k$. \textit{[2 points for setting up the limit, 3 points for correct evaluation.]}
\vfill
\part[5] Is this function continuous at the origin?

\textit{Solution:} Yes, it is. The term 
\begin{equation*}
\frac{x^2-y^2}{x^2+y^2}
\end{equation*}
is bounded above and below, 
\begin{equation*}
-1 \leq \frac{x^2-y^2}{x^2+y^2} \leq 1. 
\end{equation*}
Thus the limit 
\begin{equation*}
\lim_{(x,y) \to (0,0)} \frac{xy(x^2-y^2)}{x^2+y^2}
\end{equation*}
exists, and equals the limit
\begin{equation*}
\lim_{(x,y) \to (0,0)} xy = 0,
\end{equation*}
the same as the value of the function at the origin. 

\textit{[1 point for saying function is continuous; 2 points for arguing that the fraction is bounded; 2 points for concluding that the limit is zero.]}

\vfill
\newpage
\part[5] What is the partial derivative of this function relative to $x$, at points other than the origin?
\begin{align*}
\frac{\partial f}{\partial x} & = \frac{(3x^2-y^3)(x^2+y^2) - (x^3y-xy^3)(2x)}{(x^2+y^2)^2}  \\
& = \frac{x^4y+4x^2y^3-y^5}{(x^2+y^2)^2}.
\end{align*}

\textit{[2 points for each of the partial derivatives in the quotient rule; 1 point for correct answer.]}
\vfill
\part[5] What is the partial derivative of this function relative to $y$, at points other than the origin?
\begin{align*}
\frac{\partial f}{\partial y} & = \frac{(3x^3-3y^2x)(x^2+y^2) - (x^3y-xy^3)(2y)}{(x^2+y^2)^2}  \\
& = \frac{x^5-4x^3y^2-xy^4}{(x^2+y^2)^2}.
\end{align*}

\textit{[2 points for each of the partial derivatives in the quotient rule; 1 point for correct answer.]]}
\vfill
\end{parts}
\newpage

%%%%%%%%%%%%%%%%%%%%
\addpoints
\question Consider a model of gravitation in which two bodies, A and B, exert gravitational pull on one another. Assume that the mass of A is much bigger than that of $B$, so that the acceleration felt by $A$ is negligible, and we can regard it as in rest (this is a good approximation for the orbital motion of artificial satelites around the Earth, for instance). We are going to consider our reference frame for 3-space as centered at $A$, and describe the position of $B$ by a vector-valued function of time, $r(t)$. 

Newton's Law of Universal Gravitation asserts that the gravitational force exerted by $A$ upon $B$ is of the form
\begin{equation*}
F=\frac{K}{||r(t)||^3} r(t),
\end{equation*}
where $K$ is a (negative) constant depending on the masses of $A$ and $B$. Meanwhile, Newton's Second Law of Motion says that the acceleration $r''(t)$ felt by $B$ as a result of the gravitational pull from $A$ can be computed by the well-known formula
\begin{equation*}
F=m r''(t),
\end{equation*}
where $m$ is the mass of $B$ (a constant). 

\begin{parts}
\part[10] By comparing the two expressions for the force, explain why the cross product of position and velocity vectors, $r(t) \times r'(t)$, is constant. 

\textit{Solution:} 
The comparison yields 
\begin{equation*}
\frac{K}{||r(t)||^3} r(t) = m r''(t).
\end{equation*}
This means that the position and acceleration vectors at time $t$ are collinear, hence their cross-product is zero, for all $t$. 

By the product rule for the cross-product, 
\begin{equation*}
\frac{d[r(t) \times r'(t)]}{dt} = r(t) \times r''(t) = 0,
\end{equation*}
so $r(t) \times r'(t)$ must be constant. 

\textit{[2 points for noticing the position and acceleration vectors are collinear; 3 points for concluding their cross product is 0; 5 points for using the product rule for the cross-product explain why $r(t) \times r'(t)$ must be constant; 3 points if student quotes homework 2 to conclude last step without explicit explanation.]}

\vfill 
\newpage 
\part[10] Use part (a) to explain why the position vector $r(t)$ belongs to a fixed plane, for all $t$.  

\textit{Solution:} The position $r(t)$ belongs to the plane orthogonal to $r(t)\times r'(t)$ \textit{[5 points]}, which does not depend on $t$, since the cross-product is constant \textit{[5 points]}. 

\vfill
\end{parts}

%%%%%%%%%%%%%%%%%%%%%%%%%%

\end{questions}
\end{document}

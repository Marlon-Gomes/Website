% Exam Template for UMTYMP and Math Department courses
%
% Using Philip Hirschhorn's exam.cls: http://www-math.mit.edu/~psh/#ExamCls
%
% run pdflatex on a finished exam at least three times to do the grading table on front page.
%
%%%%%%%%%%%%%%%%%%%%%%%%%%%%%%%%%%%%%%%%%%%%%%%%%%%%%%%%%%%%%%%%%%%%%%%%%%%%%%%%%%%%%%%%%%%%%%

% These lines can probably stay unchanged, although you can remove the last
% two packages if you're not making pictures with tikz.
\documentclass[11pt]{exam}
\RequirePackage{amssymb, amsfonts, amsmath, latexsym, verbatim, xspace, setspace}


% By default LaTeX uses large margins.  This doesn't work well on exams; problems
% end up in the "middle" of the page, reducing the amount of space for students
% to work on them.
\usepackage[margin=1in]{geometry}
\usepackage[english]{babel}
\usepackage[autostyle]{csquotes} %%%% This package allows Tex to recognize quotation marks with the \enquote command. 


% Here's where you edit the Class, Exam, Date, etc.
\newcommand{\class}{MAT 203}
\newcommand{\term}{Summer II 2018}
\newcommand{\examnum}{Make-up midterm}
\newcommand{\examdate}{07/23/18}
\newcommand{\timelimit}{3 hours and 25 minutes}

% For an exam, single spacing is most appropriate
\singlespacing
% \onehalfspacing
% \doublespacing

% For an exam, we generally want to turn off paragraph indentation
\parindent 0ex
\title{MAT 203 - Summer II 2018: Make-up midterm}
\begin{document} 

% These commands set up the running header on the top of the exam pages
\pagestyle{head}
\firstpageheader{}{}{}\textbf{}
\runningheader{\class}{\examnum\ - Page \thepage\ of \numpages}{\examdate}
\runningheadrule

\begin{flushright}
\begin{tabular}{p{2.8in} r l}
\textbf{\class} & \textbf{Name (Print):} & \makebox[2in]{\hrulefill}\\
\textbf{\term} &&\\
\textbf{\examnum} &&\\
\textbf{\examdate} &&\\
\textbf{Time Limit: \timelimit} & ID number & \makebox[2in]{\hrulefill}
\end{tabular}\\
\end{flushright}
\rule[1ex]{\textwidth}{.1pt}

\begin{center}
\large{\textbf{Instructions}}
\end{center}

\begin{minipage}[t]{3.7in}
\vspace{0pt}
\begin{itemize}

\item This exam contains \numpages\ pages (including this cover page) and
\numquestions\ problems.  Check to see if any pages are missing.  Enter
all requested information on the top of this page, and put your initials
on the top of every page, in case the pages become separated.

\item You may \textit{not} use your books, notes, or any device that is capable of accessing the internet on this exam (e.g., smartphones, smartwatches, tablets). You may not use a calculator.

\item \textbf{Organize your work}, in a reasonably neat and coherent way, in
the space provided. Work scattered all over the page without a clear ordering will 
receive very little credit.  

\item \textbf{Mysterious or unsupported answers will not receive full
credit}.

\end{itemize}

\end{minipage}
\hfill
\begin{minipage}[t]{2.3in}
\vspace{0pt}
%\cellwidth{3em}
\gradetablestretch{2}
\vqword{Problem}
\addpoints % required here by exam.cls, even though questions haven't started yet.	
\gradetable[v]%[pages]  % Use [pages] to have grading table by page instead of question

\end{minipage}
\newpage % End of cover page

%%%%%%%%%%%%%%%%%%%%%%%%%%%%%%%%%%%%%%%%%%%%%%%%%%%%%%%%%%%%%%%%%%%%%%%%%%%%%%%%%%%%%
%
% See http://www-math.mit.edu/~psh/#ExamCls for full documentation, but the questions
% below give an idea of how to write questions [with parts] and have the points
% tracked automatically on the cover page.
%
%
%%%%%%%%%%%%%%%%%%%%%%%%%%%%%%%%%%%%%%%%%%%%%%%%%%%%%%%%%%%%%%%%%%%%%%%%%%%%%%%%%%%%%

\begin{questions}

% Basic question
%%%%%%%%%%%%%%

\addpoints
\question In each of the following problems, determine if the statements are true or false. Explain your reasoning (correct answers without an explanation will be worth only 2 points per statement). 
\begin{parts}
\part[5] Two vectors in space always determine a plane. 
\vfill
\part[5] Two lines in space always intersect. 
\vfill
\part[5] The cross product can be used to detect whether two non-zero vectors in space are perpendicular. 
\vfill
\part[5] Let $r(t)$ and $s(t)$ be curves in the plane, such that neither has a limit as $t$ converges to $0$. Then their dot product $r(t)\cdot s(t)$ does not have a limit at $0$ either. 
\vfill
\part[5] If the limit of a scalar-valued, multivariable function exists at a point, then all directed limits at that point exist and coincide.
\vfill
\part[5] If all the partial derivatives of a scalar-valued, multivariable function exist at a point, then all directional derivatives at that point exist. 
\vfill
\end{parts}

\newpage 
\addpoints
\question The vectors $(1,0,0)$, $(1,1,0)$ and $(0,1,1)$, together with the origin, determine a unique parallelepiped in space. In this problem, you are supposed to use the dot and cross products to compute the volume of this solid. 

\begin{parts}
\part[5] The face of the parallepiped spanned by the vectors $(1,0,0)$, $(1,1,0)$ in the $xy$-plane is a parallelogram. Use an expression involving either the dot or the cross product to compute its area. 
\vfill
\part[5] Find a vector which is perpendicular to the parallelepiped generated by $(1,0,0)$, $(1,1,0)$.
\vfill 
\part[5] Compute the length of the orthogonal projection of $(0,1,1)$ onto the line determined by the cross product you obtained in part (b). This is the height of the parallelepiped (relative to the base spanned by $(1,0,0)$, $(1,1,0)$).
\vfill
\part[5] Use the results from parts (b) and (c) to compute the volume of the parallelepiped. 
\vfill
\end{parts}
\newpage

%%%%%%%%%%%%%%%%%%%%%%%%%%%%%%%%%%%
\addpoints 
\question A line and a plane are said to be parallel if they do not intersect. Consider the line whose symmetric equations are
\begin{equation*}
x-1 = \frac{y-1}{-2} = z-1,
\end{equation*}
and the plane with equation
\begin{equation*}
x+y+z=1.
\end{equation*}

\begin{parts}
\part[3] Find a vector perpendicular to the plane. 
\vfill
\part[3] Write a parametric equation of the line. Identity a vector parallel to the line. 
\vfill
\part[4] Check that the line and the plane are parallel by verifying that the vector perpendicular to the you found on part (a) is also perpendicular to the line. 
\vfill
\newpage
\part[5] Choose a point $P$ on the line, and write an equation of a line perpendicular to the plane, passing through $P$. 
\vfill
\part[5] The line you found on part (d) should intersect the plane at a point $Q$. Find the distance between $P$ and $Q$. That is the distance between the line and the plane.
\vfill 
\end{parts}

\newpage

%%%%%%%%%%%%%%%%%%%%
\addpoints
\question Consider the plane curve $r(t)=(\cos(t),\sin(t))$. 

\begin{parts}
\part[4] Sketch this curve in the plane. 
\vfill
\part[4] Compute its velocity vector $r'(t)$.
\vfill
\newpage 
\part[4] Show that its velocity vector has constant lentgh. 
\vfill
\part[4] Explain why $r''(t)$ is normal to the velocity vector. 
\vfill
\part[4] Compute the length of $r''(t)$. This function is called the curvature of $r$. 
\vfill
\end{parts}
%%%%%%%%%%%%%%%%%%%%%%%%%%
\newpage
\addpoints
\question[10] Compute the trajectory of a curve whose velocity vector is given by 
\begin{equation*}
r'(t)= e^{-t}i + t^2j + \frac{1}{1+t^2} k,
\end{equation*}
and such that $r(0)=(1,1,1)$. 
\addpoints


\end{questions}
\end{document}
% Exam Template for UMTYMP and Math Department courses
%
% Using Philip Hirschhorn's exam.cls: http://www-math.mit.edu/~psh/#ExamCls
%
% run pdflatex on a finished exam at least three times to do the grading table on front page.
%
%%%%%%%%%%%%%%%%%%%%%%%%%%%%%%%%%%%%%%%%%%%%%%%%%%%%%%%%%%%%%%%%%%%%%%%%%%%%%%%%%%%%%%%%%%%%%%

% These lines can probably stay unchanged, although you can remove the last
% two packages if you're not making pictures with tikz.
\documentclass[11pt]{exam}
\RequirePackage{amssymb, amsfonts, amsmath, latexsym, verbatim, xspace, setspace}


% By default LaTeX uses large margins.  This doesn't work well on exams; problems
% end up in the "middle" of the page, reducing the amount of space for students
% to work on them.
\usepackage[margin=1in]{geometry}
\usepackage[english]{babel}
\usepackage[autostyle]{csquotes} %%%% This package allows Tex to recognize quotation marks with the \enquote command. 


% Here's where you edit the Class, Exam, Date, etc.
\newcommand{\class}{MAT 203}
\newcommand{\term}{Summer II 2018}
\newcommand{\examnum}{Midterm}
\newcommand{\examdate}{08/16/18}
\newcommand{\timelimit}{2 hours and 30 minutes}

% For an exam, single spacing is most appropriate
\singlespacing
% \onehalfspacing
% \doublespacing

% For an exam, we generally want to turn off paragraph indentation
\parindent 0ex
\title{MAT 203 - Summer II 2018: Final Exam}
\begin{document} 

% These commands set up the running header on the top of the exam pages
\pagestyle{head}
\firstpageheader{}{}{}\textbf{}
\runningheader{\class}{\examnum\ - Page \thepage\ of \numpages}{\examdate}
\runningheadrule

\begin{flushright}
\begin{tabular}{p{2.8in} r l}
\textbf{\class} & \textbf{Name (Print):} & \makebox[2in]{\hrulefill}\\
\textbf{\term} &&\\
\textbf{\examnum} &&\\
\textbf{\examdate} &&\\
\textbf{Time Limit: \timelimit} & ID number & \makebox[2in]{\hrulefill}
\end{tabular}\\
\end{flushright}
\rule[1ex]{\textwidth}{.1pt}

\begin{center}
\large{\textbf{Instructions}}
\end{center}

\begin{minipage}[t]{3.7in}
\vspace{0pt}
\begin{itemize}

\item This exam contains \numpages\ pages (including this cover page) and
\numquestions\ problems.  Check to see if any pages are missing.  Enter
all requested information on the top of this page, and put your initials
on the top of every page, in case the pages become separated.

\item You may \textit{not} use your books, notes, or any device that is capable of accessing the internet on this exam (e.g., smartphones, smartwatches, tablets). You may not use a calculator.

\item \textbf{Organize your work}, in a reasonably neat and coherent way, in
the space provided. Work scattered all over the page without a clear ordering will 
receive very little credit.  

\item \textbf{Mysterious or unsupported answers will not receive full
credit}.

\end{itemize}

\end{minipage}
\hfill
\begin{minipage}[t]{2.3in}
\vspace{0pt}
%\cellwidth{3em}
\gradetablestretch{2}
\vqword{Problem}
\addpoints % required here by exam.cls, even though questions haven't started yet.	
\gradetable[v]%[pages]  % Use [pages] to have grading table by page instead of question

\end{minipage}
\newpage % End of cover page

%%%%%%%%%%%%%%%%%%%%%%%%%%%%%%%%%%%%%%%%%%%%%%%%%%%%%%%%%%%%%%%%%%%%%%%%%%%%%%%%%%%%%
%
% See http://www-math.mit.edu/~psh/#ExamCls for full documentation, but the questions
% below give an idea of how to write questions [with parts] and have the points
% tracked automatically on the cover page.
%
%
%%%%%%%%%%%%%%%%%%%%%%%%%%%%%%%%%%%%%%%%%%%%%%%%%%%%%%%%%%%%%%%%%%%%%%%%%%%%%%%%%%%%%

\begin{questions}

% Basic question
%%%%%%%%%%%%%%

\addpoints
\question Classify the statements below as true or false.  
\begin{parts}
\part[2] The gradient of a smooth function of two variables is tangent to the level curves. 
\vfill
\part[2] If the Laplacian of a smooth function of two variables is positive at an isolated critical point, then the critical point is a local minimum. 
\vfill
\part[2] The area element in polar coordinates $(r,\theta)$ is given by 
\begin{equation*}
dA = r\ dr d\theta.
\end{equation*}
\vfill
\part[2] The integrals below are equivalent (regardless of the function $f$)
\begin{equation*}
\int_{0}^{1} \int_{0}^{\sqrt{1-y^2}} f(x,y)  dx dy, \ \int_{0}^{1} \int_{0}^{\sqrt{1-x^2}} f(x,y) dy dx
\end{equation*}
\vfill
\part[2] The integral below is well-defined, 
\begin{equation*}
\int_{0}^{1} \int_{0}^{x} (x+y) dx dy.
\end{equation*}
\vfill
\part[2] The volume element in spherical coordinates $(r,\phi,\theta)$ is 
\begin{equation*}
dV = r\sin(\phi) dr d\phi d\theta
\end{equation*}
\vfill
\part[2] A vector field $F$ in the plane satisfying the closedness criterion on a region $R$ is the gradient of a function defined in $R$. 
\vfill
\part[2] The line integral of a gradient vector field depends only on initial and final position of the curve. 
\vfill
\part[2] The divergence of a curl is always zero. 
\vfill 
\part[2] The curl of a gradient vector field is always non-zero. 
\vfill 
\end{parts}

\newpage 


%%%%%%%%%%%%%%%%%%%%%%%%%%%%%%%%%%%
\addpoints 
\question[20] Find and classify all the critical points of the function 
\begin{equation*}
f(x,y)=2x^3-15x^2+36x-y^2-2y
\end{equation*}

\newpage

%%%%%%%%%%%%%%%%%
\addpoints
\question Compute the following multivariable integrals. 
\begin{parts}
\part[10] 
\begin{equation*}
\int_{0}^{4} \int_{\sqrt{\frac{y}{4}}}^{1} \int_{0}^{e^{-x^4}} x dz dx dy.
\end{equation*}
\vfill 
\newpage
\part[10] 
\begin{equation*}
\iiint_{R} \sqrt{(x^2+y^2+z^2)^3} dV,
\end{equation*}
where $R$ is the solid region bounded by the surfaces $x^2+y^2+z^2=1$ and $x^2+y^2+z^2 = 4$. 
\end{parts}
\vfill
\newpage
%%%%%%%%%%%%%%%%%

\addpoints

\addpoints
\question Compute the following line and surface integrals

\begin{parts}
\part[5] 
\begin{equation*}
\int_{C} (x^2+y^2+z^2) ds, 
\end{equation*}
where $C$ is the parametrized curve $C \colon r(t)=(\cos(t), \sin(t), t)$, $0 \leq t \leq 2\pi$. 
\vfill
\part[5] 
\begin{equation*}
\int_{C} (y^2 \textbf{i} + x^2 \textbf{j})\cdot dr, 
\end{equation*}
where $C$ is the plane curve bounding the region lying between the graphs of $y=x
$ and $y=\sqrt{x}$, $x \geq 0$, oriented counterclockwise.
\vfill
\newpage
\part[10] 
\begin{equation*}
\iint_{S} F \cdot N dS, 
\end{equation*}                  
where $F$ is the vector field 
\begin{equation*}
F(x,y,z)=\frac{x}{(x^2+y^2+z^2)^{\frac{3}{2}}}\textbf{i} +\frac{y}{(x^2+y^2+z^2)^{\frac{3}{2}}}\textbf{j} +\frac{z}{(x^2+y^2+z^2)^{\frac{3}{2}}}\textbf{k},
\end{equation*}
$S$ is the surface of the sphere of radius 1 centered at the origin, and $N$ denotes the outward pointing normal vector field along the sphere.
\vfill
\end{parts}
\newpage

%%%%%%%%%%%%%%%%%%%%
\addpoints
\question Consider the vector field 
\begin{equation*}
F(x,y,z)=\frac{-x}{(x^2+y^2+z^2)^{\frac{3}{2}}}\textbf{i} +\frac{-y}{(x^2+y^2+z^2)^{\frac{3}{2}}}\textbf{j} +\frac{-z}{(x^2+y^2+z^2)^{\frac{3}{2}}}\textbf{k},
\end{equation*}
defined for all points $(x,y,z)$ other than the origin. 

\begin{parts}
\part[10] Compute its line integral along the curve which bounds the surface 
\begin{equation*}
S \colon z=\sqrt{9-x^2-y^2}.
\end{equation*}
The orientation on the curve is so that when viewed from above, the curve is oriented counterclockwise. 
\vfill 
\newpage 
\part[10] Compute its flux integral (relative to outward pointing normal vectors) on the sphere of radius 1 centered at $(0,0,2)$. 
\vfill
\end{parts}

%%%%%%%%%%%%%%%%%%%%%%%%%%

\end{questions}
\end{document}

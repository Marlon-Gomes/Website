\documentclass[12pt,oneside]{exam}

% This package simply sets the margins to be 1 inch.
\usepackage[margin=1in]{geometry}

% These packages include nice commands from AMS-LaTeX
\usepackage{amssymb,amsmath,amsthm,amsfonts,latexsym,verbatim,xspace,setspace, graphicx}

% Make the space between lines slightly more
% generous than normal single spacing, but compensate
% so that the spacing between rows of matrices still
% looks normal.  Note that 1.1=1/.9090909...
\renewcommand{\baselinestretch}{1.1}
\renewcommand{\arraystretch}{.91}

% Define an environment for exercises.
\newenvironment{exercise}[1]{\vspace{.1in}\noindent\textbf{Exercise #1 \hspace{.05em}}}{}

% define shortcut commands for commonly used symbols
\newcommand{\R}{\mathbb{R}}
\newcommand{\C}{\mathbb{C}}
\newcommand{\Z}{\mathbb{Z}}
\newcommand{\Q}{\mathbb{Q}}
\newcommand{\N}{\mathbb{N}}
\newcommand{\calP}{\mathcal{P}}

\DeclareMathOperator{\vsspan}{span}

\title{Math 203 - Summer II 2018: Homework 3}

%%%%%%%%%%%%%%%%%%%%%%%%%%%%%%%%%%%%%%%%%%

\begin{document}

\begin{flushright}
\sc MAT 203 - Lecture 1\\
July 29, 2018.
\end{flushright}
\bigskip

This homework is due on Tuesday, 8/7, in class, by 1:30 pm. 
\begin{center}
\textsf{Homework 3} 
\end{center}

\begin{exercise}{1}
In class, we characterized the gradient of a differentiable function $f$ at a point $p$ geometrically as a vector which points in the directtion of maximum increase of the function. We also characterized it analytically as the vector encoding the best linear approximation of the function, 
\begin{equation*}
f(p)=f(p_0)+\nabla f(p_0) \cdot (p-p_0) + O(||p-p_0||), 
\end{equation*}
where $O(||p-p_0||)$ simbolizes an error term whose magnitude is small relative to that of $||p-p_0||$ ( a notion which can be made precise with multivariable limits). With this at hand, we saw in class that (for functions with continuous partial derivatives) the gradient can be expressed in a simple way using partial derivatives:
\begin{equation}\label{gradient}
\nabla f(p_0) = \left[\frac{\partial f}{\partial x} (p_0)\right]i + \left[\frac{\partial f}{\partial y} (p_0)\right] j  + \left[\frac{\partial f}{\partial z} (p_0)\right] k,
\end{equation}
the partial derivative relative to $z$ being absent when $f$ is a function of two variables only. 

In this exercise will see how the gradient relates to the notion of directional derivatives discussed earlier in the course. In the rest of this problem, $f$ denotes the function $f(x,y,z)=4x^2+3yz+2z$. 

\begin{itemize}
\item[(a)] Compute its gradient vector at a point $(x,y,z)$ using the formula given above. 
\item[(b)] Use the definition of directional derivative (as a directed limit) to compute the directional derivative of $f$ at $(0,0,0)$ in the direction 
\begin{equation*}
v=\left(\frac{1}{\sqrt 3}, \frac{1}{\sqrt 3},\frac{1}{\sqrt 3}\right).
\end{equation*}
\item[(c)] Compute the dot product between the gradient you found in part (a), at the point $(0,0,0)$, and the vector $v$ from part $b$. 
\end{itemize}

Your results from parts (b) and (c) should have been the same. In fact, this is the takeaway from this problem: \textit{for functions with continous partial derivatives, the gradient vector field can be used to compute directional derivatives}. The directional derivative of $f$ in the direction of $v$ at a point is given by the dot product between its gradient vector at that point and the direction $v$. 
\end{exercise}

\begin{exercise}{2} 
Consider the function $f(x,y)=x^2+y^2$. 
\begin{itemize}
\item[(a)] Describe its level curves $f(x,y)=c$, for all values of $c$. 
\item[(b)] Compute the gradient vector of $f$ at a point with coordinates $(x,y)$. 
\item[(c)] Describe a tangent vector of the level curve $f(x,y)=1$ at a point with coordinates $(x,y)$ on this curve. 
\item[(d)] Check that the gradient of the function at a point on the level curve $f(x,y)=1$ is orthogonal to this level curve, by computing the dot product of the vectors you found in parts (b) and (c). 
\item[(e)] The dot product you computed in part (d) can be interpretet as a directional derivative of the function. Use this interpretation to explain why this product is zero in a different way. 
\end{itemize}
\end{exercise}

\begin{exercise}{3}
Each of the surfaces below is the level set of a function of three variables. Find equations of the normal lines to these surfaces at the points indicated. 
\begin{parts}
\part $xy-z=0$, at $(-2, -3, -6)$. 
\part $xyz=10$, at $(-1,-1,10)$. 
\part $z=16-x^2-y^2$ at $(2,2,8)$. 
\end{parts}
\end{exercise}

\begin{exercise}{4}
The tangent plane to a level surface at a point can be computed as the plane orthogonal to the gradient vector of the function at the point.  For each of the level surfaces below, find the points $(x,y,z)$ at which their tangent planes are parallel to the $xy$-plane. 
\begin{itemize}
\item[(a)] $z=3x^2+2y^2-3x-4y-5$.
\item[(b)] $z=xy+\frac{1}{x} + \frac{1}{y}$, for $x,y \neq 0$.
\item[(c)] $z=5xy$.
\end{itemize}
\end{exercise}

\begin{exercise}{5}
Find and classify all the critical points of the functions below. 
\begin{itemize}
\item[(a)] $f(x,y)=-2x^4y^4$. 
\item[(b)] $g(x,y)=x^2-3xy-y^2$. 
\item[(c)] $h(x,y)=2xy-\frac{1}{2}(x^4+y^4) +1$.  
\end{itemize}
\end{exercise}

\begin{exercise}{6}
A rectangular box is to be constructed with volume equal to 100 units. In order to minimize the cost of production, you want to find the box with the lowest possible surface area. Find the measures of three adjacent sides $x,y$ and $z$ (all positive numbers) which achieve minimize surface area.
\end{exercise}

\begin{exercise}{7}
In each of the problems below, sketch the region of integration in the plane and change the order of integration. 
\begin{itemize}
\item[(a)] 
\begin{equation*}
\int_{0}^{4} \int_{0}^{y} f(x,y) dx dy
\end{equation*}
\item[(b)] 
\begin{equation*}
\int_{-2}^{2} \int_{0}^{\sqrt{4-y^2}} f(x,y) dx dy
\end{equation*}
\item[(c)] 
\begin{equation*}
\int_{-1}^{2} \int_{0}^{e^{-x}} f(x,y) dy dx
\end{equation*}
\item[(d)] 
\begin{equation*}
\int_{-\frac{\pi}{2}}^{\frac{\pi}{2}} \int_{0}^{\cos(x)} f(x,y) dy dx
\end{equation*}
\end{itemize}
\end{exercise}

\begin{exercise}{8}
Compute each of the iterated integrals below (note: you may need to switch the order of integration first). 
\begin{itemize}
\item[(a)] 
\begin{equation*}
\int_{1}^{4} \int_{1}^{\sqrt{x}} 2ye^{-x} dy dx
\end{equation*}
\item[(b)] 
\begin{equation*}
\int_{0}^{4} \int_{\sqrt{x}}^{2} \frac{3}{2+y^3}dy dx 
\end{equation*}
\item[(c)] 
\begin{equation*}
\int_{0}^{\frac{\pi}{2}} \int_{0}^{\sin(\theta)} \theta r dr d\theta
\end{equation*}
\item[(d)] 
\begin{equation*}
\int_{0}^{1} \int_{y}^{1} \sin(x^2) dx dy
\end{equation*}
\item[(e)] 
\begin{equation*}
\int_{1}^{\infty} \int_{0}^{\frac{1}{x}} y dy dx
\end{equation*}
\item[(f)] 
\begin{equation*}
\int_{0}^{\infty} \int_{0}^{\infty} xye^{-(x^2+y^2)} dx dy
\end{equation*}
\end{itemize}
\end{exercise}

\begin{exercise}{9}
For each of the following problems, sketch the region of integration, set up the double integral for both orders of integration, and choose the most convenient order to evaluate the integrals.

\begin{parts}
\part 
\begin{equation*}
\iint_{R} xy dA,
\end{equation*}
where $R$ is the rectangle with vertices $(0,0), (0,5), (3,5), (3,0)$. 
\part 
\begin{equation*}
\iint_{R} -2y dA,
\end{equation*}
where $R$ is the region bounded by $y=4-x^2$ and $y=4-x$. 
\part 
\begin{equation*}
\iint_{R} \frac{y}{1+x^2} dA,
\end{equation*}
where $R$ is the region bounded by $y=0, y=\sqrt{x}$ (for $x \geq 0$), and $x=4$. 
\part 
\begin{equation*}
\iint_{R} x^2+y^2 dA,
\end{equation*}
where $R$ is the semicircle bounded by $y=\sqrt{4-x^2}$, $y=0$. 
\end{parts}
\end{exercise}

\begin{exercise}{10}
Find the volume of the solid bounded by the relations. 
\begin{equation*}
z=\frac{1}{1+y^2}, x=0, x=2, y \geq 0.
\end{equation*}
\end{exercise}

\begin{exercise}{11}
Use polar coordinates to find the areas of the regions indicated below. 
\begin{parts}
\part The region inse the cardioid $r=2+\cos(\theta)$ and outside the circle $r=1$. 
\part The region inside the rose curve $r=4\sin(3\theta)$ and outside the circle $r=2$. 
\end{parts}
\end{exercise}

\begin{exercise}{12}
Set up the triple integrals whose values describe the volumes of the solids indicated below. Do not compute the integrals. 
\begin{itemize}
\item[(a)] The solid bounded by $z=9-x^2$, $z=0$, $y=0$ and $y=2x$. 
\item[(b)] The solid bounded above by the paraboloid $z=4-x^2$ and below by the paraboloid $z=x^2+3y^2$. 
\end{itemize}
\end{exercise}

\begin{exercise}{13}
Convert the integrals below from rectangular to both spherical and cylindrical coordinates. Then evaluate the integral in one of its forms. 
\begin{itemize}
\item[(a)]
\begin{equation*}
\int_{0}^{2} \int_{0}^{\sqrt{4-x^2}} \int_{0}^{\sqrt{16-x^2-y^2}} \sqrt{x^2+y^2} dz dy dx
\end{equation*}
\item[(b)]
\begin{equation*}
\int_{-1}^{1} \int_{-\sqrt{1-x^2}}^{\sqrt{1-x^2}} \int_{1}^{1+\sqrt{1-x^2-y^2}} x dz dy dx
\end{equation*}
\end{itemize}
\end{exercise}

\begin{exercise}{14}
Find the Jacobians of the changes of variables indicated below. 
\begin{itemize}
\item[(a)] $x=-\frac{1}{2}(u-v)$, $y=\frac{1}{2}(u+v)$
\item[(b)] $x=e^u\sin(v)$, $y=e^u\cos(v)$. 
\item[(c)] $x=4u-v$, $y=4v-w$, $z=u+w$. 
\item[(d)] $x=r\cos(\theta)$, $y=r\sin(\theta)$.
\item[(e)]  $x=\rho \sin(\phi) \cos(\theta)$, $y=\rho \sin(\phi) \sin(\theta)$, $z=\rho \cos(\phi)$.
\item[(f)] $x=r\cos(\theta)$, $y=r\sin(\theta)$, $z=z$. 
\end{itemize}
\end{exercise}

\begin{exercise}{15} 
In each of the problems below, use a change of variables to compute the volumes of the regions indicated. 
\begin{itemize}
\item[(a)] The region below the graph of $f(x,y)=9xy$, and above the square with vertices $(1,0)$, $(0,1)$, $(1,2)$ and $(2,1)$ in the $xy$-plane. 
\item[(b)] The region bounded by the graph of $f(x,y)=(3x+2y)(2y-x)^{\frac{3}{2}}$ and the parallelogram $R$ with vertices $(0,0), (-2,3), (2,5)$ and $(4,2)$ in the $xy$-plane. 
\item[(c)] The solid bounded above by the graph of $f(x,y)=\frac{xy}{1+x^2y^2}$, and bounded below by the planer region bounded by the curves $xy=1$, $xy=4$, $x=1$ and $x=4$ in the $xy$-plane. 
\end{itemize}
\end{exercise}
\end{document}

\documentclass[12pt,oneside]{exam}

% This package simply sets the margins to be 1 inch.
\usepackage[margin=1in]{geometry}

% These packages include nice commands from AMS-LaTeX
\usepackage{amssymb,amsmath,amsthm,amsfonts,latexsym,verbatim,xspace,setspace}

% Make the space between lines slightly more
% generous than normal single spacing, but compensate
% so that the spacing between rows of matrices still
% looks normal.  Note that 1.1=1/.9090909...
\renewcommand{\baselinestretch}{1.1}
\renewcommand{\arraystretch}{.91}

% Define an environment for exercises.
\newenvironment{exercise}[1]{\vspace{.1in}\noindent\textbf{Exercise #1 \hspace{.05em}}}{}

% define shortcut commands for commonly used symbols
\newcommand{\R}{\mathbb{R}}
\newcommand{\C}{\mathbb{C}}
\newcommand{\Z}{\mathbb{Z}}
\newcommand{\Q}{\mathbb{Q}}
\newcommand{\N}{\mathbb{N}}
\newcommand{\calP}{\mathcal{P}}

\DeclareMathOperator{\vsspan}{span}

\title{Math 132 - Summer II 2016: Homework 4}

%%%%%%%%%%%%%%%%%%%%%%%%%%%%%%%%%%%%%%%%%%

\begin{document}

\begin{flushright}
\sc MAT 203 - Lecture 1\\
July 17, 2018.
\end{flushright}
\bigskip

This homework is due on Tuesday, 7/24, in class, by 1:30 pm. 
\begin{center}
\textsf{Homework 2} 
\end{center}

\begin{exercise}{1}
Sketch the following plane curves:

\begin{parts}
\part $r(t) = (t,t^2)$.
\part $r(t)=(\cos(t), \sin(t))$.
\part $r(t)=(t^3-4t, t^2-4)$. 
\part $r(t)=(t^3,t^2)$. 
\part $r(t)=\left(\frac{3t}{1+t^3}, \frac{3t^2}{1+t^3}\right)$.
\end{parts}
\end{exercise}

\begin{exercise}{2} 
For each of the plane curves below, find the parametric equation of their tangent lines at the points indicated. 
\begin{parts}
\part $r(t)=(t,t^3)$ at the point $(2,8)$.
\part $r(t)=(\cos(t),\sin(t))$ at the point $\left(\frac{\sqrt{2}}{2}, \frac{\sqrt{2}}{2}\right)$. 
\end{parts}
\end{exercise}

\begin{exercise}{3}
Consider the curves $r(t)=(t,t,t^2)$ and $s(t)=\left(\frac{1}{t},\frac{1}{t},0\right)$, defined for $t \neq 0$. 
\begin{parts}
\part Does the curve $r$ have a limit as $t$ goes to $0$? If so, what is the limit?
\part Does the curve $s$ have a limit as $t$ goes to $0$? If so, what is the limit?
\part Compute the dot product of the curves, $r(t) \cdot s(t)$, for $t \neq 0$. 
\part Does the scalar function obtained in part $(c)$ have a limit as $t$ goes to $0$? If so, what is this limit?
\end{parts}
\end{exercise}

\begin{exercise}{4}
An object moves in the plane according to a trajectory described by a smooth curve $r(t)$. Assume that:
\begin{enumerate}
\item the curve never passes through the origin, i.e., $r(t) \neq 0$; 
\item the velocity vector is never zero, $r'(t) \neq 0$;
\item  at time $t=0$, the curve is at its closest point to the origin.
\end{enumerate}

Explain why the position and velocity vectors $r(0)$ and $r'(0)$ are perpendicular. 
\end{exercise}

\begin{exercise}{5}
Let $r(t)$ denote a spatial curve, and $r'(t)$, $r''(t)$ its first and second derivatives, respectively. Assume that $r''(t) \neq 0$. If the position $r(t)$ and acceleration $r''(t)$ are colinear, for all times, what can you say about the cross product $r(t) \times r'(t)$? 
\end{exercise}

\begin{exercise}{6}
As we saw in class, the Fundamental Theorem of Calculus for Curves can be used to compute the displacement vector, 
\begin{equation*}
\int_{a}^{b} r'(t) dt = r(b)-r(a).
\end{equation*}
Use this to describe the trajectory described by a curve with velocity vector 
\begin{equation*}
r'(t)=\frac{1}{1+t^2} i + tj + e^tk, 
\end{equation*}
and which satisfies $r(0)=(1,0,-1)$. 
\end{exercise}

\begin{exercise}{7}
Find two vector functions $r(t)$ and $s(t)$ for which 
\begin{equation*}
\int[ r(t) \times s(t)] dt  \neq \left(\int r(t) dt \right) \times \left( \int s(t) dt \right)
\end{equation*}
\end{exercise}

\begin{exercise}{8}
Consider the function 
\begin{equation*}
f(x,y)=\frac{x^2y}{x^4+y^2},
\end{equation*}
defined for all points $(x,y)$ in the plane except the origin $(0,0)$. This exercise will study the behavior of this function near the origin. 

\begin{parts}
\part Compute the directed limits of this function along the lines $y=kx$, in terms of the parameter $k$. 
\part Compute the limit of this function along the parabola $y=x^2$.
\end{parts}

The results of parts $a$ and $b$ should be different. This is to show you that unlike what you studied in single-variable Calculus and what we observed for vector-valued, single-variable functions, the existence and coincidence of directed limits no longer imply the existence of the limit of the function. This marks a sharp contrast between single-variable functions and multivariable functions.
\end{exercise}

\begin{exercise}{9}
This exercise is about the following function: 
\begin{equation*}
f(x,y) = 
\left\{
\begin{array}{rl}
\frac{y}{x}-y & \mbox{ if} \ 0 \leq y < x \leq 1 \\
\frac{x}{y}-x & \mbox{ if} \ 0 \leq x < y \leq 1 \\
1-x & \mbox{if} \ 0 <x = y \\
0 & \mbox{ elsewhere}.
\end{array}
\right.
\end{equation*}

\begin{parts}
\part Sketch the graph of this function on the square $[0,1] \times [0,1]$. 
\part What is the value of this function along the boundary of the square?
\part If the value of $x$ is kept constant, is $f$ a continuous function of $y$?
\part If the value of $y$ is kept constant, is $f$ a continuous function of $x$?
\part Compute the directed limit of the function as $(x,y)$ approaches the origin along the line $y=x$. 
\part Compare your answer of part $(e)$ with the value of the function at $(0,0)$ which you obtained in part $(b)$. Is this function continuous at $(0,0)$?
\end{parts}
\end{exercise}

\begin{exercise}{10}
Compute \textbf{all} the partial derivatives of the function $f(x,y,z)=ye^x+x\ln(z^2+1)$ up to order two. 
\end{exercise}

\begin{exercise}{11}
Verify that the functions given satisfy the corresponding equations. 
\begin{parts}
\part The function $f(t,x)=\sin(x-t)$ satisfies \textit{the wave equation}
\begin{equation*}
\frac{\partial^2 f}{\partial t^2} = \frac{\partial^2 f}{\partial x^2}.
\end{equation*}
\part The function $f(t,x)=e^{-t}\cos(x)$ satisfies \textit{the heat equation}
\begin{equation*}
\frac{\partial f}{\partial t} = \frac{\partial^2 f}{\partial x^2}
\end{equation*}
\part The function $f(x,y)=e^x\sin(y)$ satisfies \textit{Laplace's equation}
\begin{equation*}
\frac{\partial^2 f}{\partial x^2} + \frac{\partial^2 f}{\partial y^2} = 0. 
\end{equation*}
\end{parts}
\end{exercise}

\end{document}



\documentclass[11pt]{article}

%%%%%%%%%%%%%%%%%%%%%%%%%%%%%%
%
% THE FOLLOWING ARE FORMATTING COMMANDS.  Please do not delete or edit!
%
%%%%%%%%%%
\makeatletter
\newcommand{\spe@ker}{}
\newcommand{\t@lktitle}{}
\newcommand{\@ffili@tion}{}
\newcommand{\speaker}[1]{%
  \renewcommand{\spe@ker}{#1}}
\newcommand{\talktitle}[1]{%
  \renewcommand{\t@lktitle}{#1}}
\newcommand{\affiliation}[1]{%
  \renewcommand{\@ffili@tion}{#1}}
\renewenvironment{abstract}{%
  \begin{center}
    \large
    \textbf{\spe@ker} %(\@ffili@tion)
    : ``\t@lktitle''
  \end{center}
  \begin{trivlist}
  \item\textsc{Abstract:}}{%
  \end{trivlist}}
\makeatother
%%%%%%%%%%%%%%%%%%%%%%%%%%%%%%
\usepackage{amsmath,amssymb,latexsym}

%%%%%%%%%%%%%%%%%%%%%%%%%%%%%%%%%%%%%%%%
% IMPORTANT: PLEASE READ
% Do not define your own commands or load any other packages.  All
% talk submissions will be put together into a single document to
% produce a book of talk abstracts.  Defining your own commands could
% impact other speakers' abstracts!  Please do not do so.
%%%%%%%%%%%%%%%%%%%%%%%%%%%%%%%%%%%%%%%%

%%%%%%%%%%%%%%%%%%%%%%%%%%%%%%%%%%%%%%%%
%SEMINAR INFORMATION
\title{Student Differential Geometry Seminar}
\author{Stony Brook University}
\date{November 5, 2018}
%%%%%%%%%%%%%%%%%%%%%%%%%%%%%%%%%%%%%%%%
% Edit the SPEAKER, TALKTITLE, AFFILIATION and ABSTRACT in the
% following document body.
\begin{document}
\maketitle
\thispagestyle{empty}

\speaker{Marlon Gomes}
\talktitle{Joyce's construction of compact $G_2$-manifolds: part I.}
%\affiliation{Replace by your university, affiliation or organization}

\begin{abstract}
In the first part of the talk, I will prove that a Riemannian metric arising from a torsion-free $G_2$-structure on a compact manifold has reduced holonomy $G_2$ if and only if the fundamental group of the manifold is finite. 

In the second part, we will begin the study of Joyce's generalized Kummer construction of compact manifolds with holonomy $G_2$. I will describe the local structure near the singular points of quotients of the flat $G_2$ structure on the $7$-torus, as well as the geometry of their resultions. 
\end{abstract}

\end{document}
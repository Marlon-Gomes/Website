\documentclass[11pt]{article}

%%%%%%%%%%%%%%%%%%%%%%%%%%%%%%
%
% THE FOLLOWING ARE FORMATTING COMMANDS.  Please do not delete or edit!
%
%%%%%%%%%%
\makeatletter
\newcommand{\spe@ker}{}
\newcommand{\t@lktitle}{}
\newcommand{\@ffili@tion}{}
\newcommand{\speaker}[1]{%
  \renewcommand{\spe@ker}{#1}}
\newcommand{\talktitle}[1]{%
  \renewcommand{\t@lktitle}{#1}}
\newcommand{\affiliation}[1]{%
  \renewcommand{\@ffili@tion}{#1}}
\renewenvironment{abstract}{%
  \begin{center}
    \large
    \textbf{\spe@ker} %(\@ffili@tion)
    : ``\t@lktitle''
  \end{center}
  \begin{trivlist}
  \item\textsc{Abstract:}}{%
  \end{trivlist}}
\makeatother
%%%%%%%%%%%%%%%%%%%%%%%%%%%%%%
\usepackage{amsmath,amssymb,latexsym}

%%%%%%%%%%%%%%%%%%%%%%%%%%%%%%%%%%%%%%%%
% IMPORTANT: PLEASE READ
% Do not define your own commands or load any other packages.  All
% talk submissions will be put together into a single document to
% produce a book of talk abstracts.  Defining your own commands could
% impact other speakers' abstracts!  Please do not do so.
%%%%%%%%%%%%%%%%%%%%%%%%%%%%%%%%%%%%%%%%

%%%%%%%%%%%%%%%%%%%%%%%%%%%%%%%%%%%%%%%%
%SEMINAR INFORMATION
\title{Student Differential Geometry Seminar}
\author{Stony Brook University}
\date{October 22, 2018}
%%%%%%%%%%%%%%%%%%%%%%%%%%%%%%%%%%%%%%%%
% Edit the SPEAKER, TALKTITLE, AFFILIATION and ABSTRACT in the
% following document body.
\begin{document}
\maketitle
\thispagestyle{empty}

\speaker{Marlon Gomes}
\talktitle{Quaternion-K\"ahler geometry}
%\affiliation{Replace by your university, affiliation or organization}

\begin{abstract}
Quaternion-K\"ahler (QK) geometry is the most general quaternionic geometry, according to Berger's classification of holonomy groups. I will begin by discussing the elementary quaternionic geometry of quaternionic vector spaces and projective spaces, focusing on the constrast between the hyper-K\"ahler structures of the former and the strictly QK-structures on the latter. 

After this we shall move on to understand why QK metrics are Esintein, and discuss examples of such manifolds. If time permits, I will briefly discuss aspects of twistor theory and the LeBrun-Salamon conjecture. 
\end{abstract}

\end{document}
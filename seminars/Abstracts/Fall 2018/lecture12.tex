\documentclass[11pt]{article}

%%%%%%%%%%%%%%%%%%%%%%%%%%%%%%
%
% THE FOLLOWING ARE FORMATTING COMMANDS.  Please do not delete or edit!
%
%%%%%%%%%%
\makeatletter
\newcommand{\spe@ker}{}
\newcommand{\t@lktitle}{}
\newcommand{\@ffili@tion}{}
\newcommand{\speaker}[1]{%
  \renewcommand{\spe@ker}{#1}}
\newcommand{\talktitle}[1]{%
  \renewcommand{\t@lktitle}{#1}}
\newcommand{\affiliation}[1]{%
  \renewcommand{\@ffili@tion}{#1}}
\renewenvironment{abstract}{%
  \begin{center}
    \large
    \textbf{\spe@ker} %(\@ffili@tion)
    : ``\t@lktitle''
  \end{center}
  \begin{trivlist}
  \item\textsc{Abstract:}}{%
  \end{trivlist}}
\makeatother
%%%%%%%%%%%%%%%%%%%%%%%%%%%%%%
\usepackage{amsmath,amssymb,latexsym}

%%%%%%%%%%%%%%%%%%%%%%%%%%%%%%%%%%%%%%%%
% IMPORTANT: PLEASE READ
% Do not define your own commands or load any other packages.  All
% talk submissions will be put together into a single document to
% produce a book of talk abstracts.  Defining your own commands could
% impact other speakers' abstracts!  Please do not do so.
%%%%%%%%%%%%%%%%%%%%%%%%%%%%%%%%%%%%%%%%

%%%%%%%%%%%%%%%%%%%%%%%%%%%%%%%%%%%%%%%%
%SEMINAR INFORMATION
\title{Student Differential Geometry Seminar}
\author{Stony Brook University}
\date{December 3, 2018}
%%%%%%%%%%%%%%%%%%%%%%%%%%%%%%%%%%%%%%%%
% Edit the SPEAKER, TALKTITLE, AFFILIATION and ABSTRACT in the
% following document body.
\begin{document}
\maketitle
\thispagestyle{empty}

\speaker{Demetre Kazaras}
\talktitle{Aspects of Ricci-flat, ALE 4-manifolds}
%\affiliation{Replace by your university, affiliation or organization}

\begin{abstract}
In this talk, we will survey some existence and classification results for ALE (asymptotically locally Euclidian) 4-manifolds which are Ricci-flat (or scalar-flat). These objects arise in studying the moduli space of Einstein metrics on closed manifolds and are of interest in (mathematical) general relativity. We will sketch results due to LeBrun ’88 and Lock-Viaclovsky ‘16, indicating their relation to special holonomy and positive mass theorems.
\end{abstract}

\end{document}
